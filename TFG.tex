% Plantilla realizada por Alberto Brunete (UPM).

%Parametros de tipo libro
%\documentclass[12pt,a4paper]{book}
\documentclass[12pt,spanish,openany,letterpaper,pagesize]{scrbook}
\usepackage{textcomp}

\usepackage{booktabs}   % Para líneas de tabla profesionales (toprule, midrule, bottomrule)
\usepackage{siunitx}    % Para alinear números y manejar unidades (moneda)
\usepackage{multirow}   % Para celdas que ocupan múltiples filas
\usepackage{caption}    % Para captions personalizadas

\usepackage{siunitx}
\usepackage{xfp}
\usepackage{numprint}
\npthousandsep{,} % salida: coma
\npdecimalsign{.} % salida: punto




%======================
% PARÁMETROS Y CÁLCULOS
%======================
\def\ValorPunto{22358}

% Estudiante de Maestría
\def\HorasEst{1080} % 30h x 36 semanas
\def\PuntosHoraEst{2.5}
\def\CostoHoraEst{\fpeval{\PuntosHoraEst * \ValorPunto}}
\def\TotalEstudiante{\fpeval{\HorasEst * \CostoHoraEst}}

% Director
\def\HorasDir{144} % 2h x 36 semanas
\def\PuntosHoraDir{4}
\def\CostoHoraDir{\fpeval{\PuntosHoraDir * \ValorPunto}}
\def\TotalDirector{\fpeval{\HorasDir * \CostoHoraDir}}

% Subtotal Recursos Humanos
\def\TotalRH{\fpeval{\TotalEstudiante + \TotalDirector}}

% Hardware: depreciación PC
\def\ValPC{4250000}
\def\DepreciacionPC{\fpeval{0.25 * \ValPC}} % 9 meses de 36 (3 años)
\def\ValEquipos{2000000}
\def\DepreciacionUWB{\fpeval{0.008*\ValEquipos}}

\def\TotalHardware{\fpeval{\DepreciacionPC + \DepreciacionUWB}}



% Publicaciones
\def\TotalPublicaciones{3000000}

% Subtotal directo
\def\SubtotalDirecto{\fpeval{\TotalRH + \TotalHardware + \TotalPublicaciones}}

% Comunicaiones (2%)
\def\comunicacionesRubro{\fpeval{0.02 * \SubtotalDirecto}}

% AUI (20%)
\def\AUI{\fpeval{0.2 * (\SubtotalDirecto + \comunicacionesRubro)}}



% Total estudiante recurso
\def\TotalEstudianteRecurso{\fpeval{\TotalEstudiante + \DepreciacionPC + \TotalPublicaciones}}

% Total FIET
\def\TotalFIET{\fpeval{\TotalDirector + \AUI + \comunicacionesRubro + \DepreciacionUWB}}

% Total Proyecto
\def\TotalProyecto{\fpeval{\TotalFIET + \TotalEstudianteRecurso}}


%\setlength{\parindent}{0pt} quitar sangría
% --- Prefijo "Figura" en la lista de figuras ---
\renewcommand*\figurename{Figura}
\renewcommand*\listfigurename{\MakeUppercase{Lista de figuras}}

\makeatletter
\renewcommand*\l@figure[2]{%
  \@dottedtocline{1}{1.5em}{3.2em}{\figurename~#1}{#2}%
}
\makeatother

% --- Prefijo "Tabla" en la lista de tablas ---
\renewcommand*\tablename{Tabla}
\renewcommand*\listtablename{\MakeUppercase{Lista de tablas}}



\makeatletter
\renewcommand*\l@table[2]{%
  \@dottedtocline{1}{1.5em}{3.2em}{\tablename~#1}{#2}%
}
\makeatother


%Idioma español y acentos
\renewcommand{\baselinestretch}{1.2} % espacio entrelineas
\usepackage[spanish, es-tabla]{babel}
%\usepackage[latin1]{inputenc}
\usepackage[utf8]{inputenc}
\usepackage{nameref}
\usepackage[automark]{scrlayer-scrpage}
\pagestyle{scrheadings}
 \usepackage{listings}
\usepackage{cite}
\usepackage{listings} %codigos ros
\usepackage{rotating} % rotar hoja
\usepackage{pdflscape}
\usepackage[table]{xcolor}
\usepackage{colortbl}
\usepackage{gensymb}
\usepackage{float}
 %Paquete para que la cita no se muestre en la lista de referencias
\usepackage{notoccite}

%para centrar texto imagen pero no centra
\usepackage{wrapfig}
\usepackage{caption}

\usepackage{float}

%justificar texto sin guiones
\usepackage{microtype}
\microtypesetup{protrusion=false, expansion=false}

% Paquete para acrónimos y glosarios
\usepackage[acronym,nonumberlist]{glossaries}

% ===========================
% Macro para crear acrónimo con traducción
% ===========================
\makeatletter
\newcommand{\newdualacronym}[4]{%
  \newacronym{#1}{#2}{#3}%
  \expandafter\protected@edef\csname spanish@#1\endcsname{#4}%
}
\makeatother

% ===========================
% Estilo para imprimir acrónimos con traducción alineada
% ===========================
\newglossarystyle{spanishlist}{
  \setglossarystyle{long} % Se basa en el estilo 'long' que usa longtable
  
  % Aumenta el espacio antes y después de la tabla de acrónimos
  \setlength{\LTpre}{2.5em}
  \setlength{\LTpost}{2.5em}

  % Redefine el entorno del glosario para ajustar los anchos de columna
  \renewenvironment{theglossary}{%
    \begin{longtable}{p{0.2\textwidth}p{0.7\textwidth}}
  }{%
    \end{longtable}
  }

  \renewcommand*{\glossentry}[2]{%
    \glsentryitem{##1}\glstarget{##1}{\glsentryshort{##1}} &
    \textit{\glsentrylong{##1}}, \csname spanish@##1\endcsname
    ##2 \\ % Cambiado de \tabularnewline a \\
  }%
}

% Crear los glosarios
\makeglossaries
% --- INICIO: DEFINICIÓN DE ACRÓNIMOS DUALES ---
% Sintaxis: \newdualacronym{etiqueta}{SIGLA}{Significado en Inglés}{Significado en Español}

\newdualacronym{uwb}{UWB}{\textit{Ultra-Wideband}}{Banda Ultra Ancha}
\newdualacronym{ips}{IPS}{\textit{Indoor Positioning Systems}}{Sistemas de Posicionamiento en Interiores}
\newdualacronym{nlos}{NLOS}{\textit{Non-Line-of-Sight}}{Sin Línea de Vista}
\newdualacronym{bs}{BS}{\textit{Body Shadowing}}{Obstrucción Corporal}
\newdualacronym{imu}{IMU}{\textit{Inertial Measurement Unit}}{Unidad de Medición Inercial}

\newdualacronym{iot}{IoT}{\textit{Internet of Things}}{Internet de las Cosas}
\newdualacronym{lbs}{LBS}{\textit{Location Based Services}}{Servicios Basados en la Localización}
\newdualacronym{gps}{GPS}{\textit{Global Positioning System}}{Sistema de Posicionamiento Global}
\newdualacronym{gnss}{GNSS}{\textit{Global Navigation Satellite Systems}}{Sistemas Globales de Navegación por Satélite}
\newdualacronym{los}{LOS}{\textit{Line-of-Sight}}{Línea de Vista}
\newdualacronym{lps}{LPS}{\textit{Local Positioning Systems}}{Sistemas de Posicionamiento Local}
\newdualacronym{rtls}{RTLS}{\textit{Real-Time Location Systems}}{Sistemas de Localización en Tiempo Real}
\newdualacronym{rf}{RF}{\textit{Radio frequency}}{Radiofrecuencia}
\newdualacronym{rfid}{RFID}{\textit{Radio Frequency Identification}}{Identificación por Radiofrecuencia}
\newdualacronym{wifi}{WiFi}{\textit{Wireless Fidelity}}{Fidelidad Inalámbrica}
\newdualacronym{ble}{BLE}{\textit{Bluetooth Low Energy}}{Bluetooth de Baja Energía}
\newdualacronym{rss}{RSS}{\textit{Received Signal Strength}}{Intensidad de la Señal Recibida}
\newdualacronym{aoa}{AoA}{\textit{Angle of Arrival}}{Ángulo de Llegada}
\newdualacronym{toa}{ToA}{\textit{Time of Arrival}}{Tiempo de Llegada}
\newdualacronym{tof}{ToF}{\textit{Time of Flight}}{Tiempo de Vuelo}
\newdualacronym{twr}{TWR}{\textit{Two-Way Ranging}}{Medición de Distancia Bidireccional}
\newdualacronym{ds-twr}{DS-TWR}{\textit{Double-Sided Two-Way Ranging}}{Medición de Distancia Bidireccional de Doble Cara}
\newdualacronym{poa}{PoA}{\textit{Phase of Arrival}}{Fase de Llegada}
\newdualacronym{isi}{ISI}{\textit{Intersymbol Interference}}{Interferencia Intersimbólica}
\newdualacronym{fdtd}{FDTD}{\textit{Finite-Difference Time-Domain}}{Diferencias Finitas en el Dominio del Tiempo}
\newdualacronym{qlos}{QLOS}{\textit{Quasi-Line-of-Sight}}{Cuasi Línea de Vista}
\newdualacronym{em}{EM}{\textit{Expectation Maximization}}{Maximización de la Esperanza}
\newdualacronym{gmm}{GMM}{\textit{Gaussian Mixture Model}}{Modelo de Mezcla Gaussiana}
\newdualacronym{ml}{ML}{\textit{Machine Learning}}{Aprendizaje Automático}
\newdualacronym{per}{PER}{\textit{Packet Error Rate}}{Tasa de Error de Paquetes}
\newdualacronym{rha}{RHA}{\textit{Relative Heading Angle}}{Ángulo de Orientación Relativo}
\newdualacronym{vna}{VNA}{\textit{Vector Network Analyzer}}{Analizador Vectorial de Redes}
\newdualacronym{fcc}{FCC}{\textit{Federal Communications Commission}}{Comisión Federal de Comunicaciones}
\newdualacronym{ekf}{EKF}{\textit{Extended Kalman Filter}}{Filtro de Kalman Extendido}
\newdualacronym{snr}{SNR}{\textit{Signal-to-Noise Ratio}}{Relación Señal a Ruido}
\newdualacronym{mae}{MAE}{\textit{Mean Absolute Error}}{Error Absoluto Medio}
\newdualacronym{nlls}{NLLS}{\textit{Non-Linear Least Squares}}{Mínimos Cuadrados no Lineales}
\newdualacronym{tdoa}{TDoA}{\textit{Time Difference of Arrival}}{Diferencia de Tiempo de Llegada}
\newdualacronym{fft}{FFT}{\textit{Fast Fourier Transform}}{Transformada Rápida de Fourier}
\newdualacronym{tmuwb}{TM-UWB}{\textit{Time Modulated Ultra-Wideband}}{Ultra Banda Ancha Modulada en Tiempo}
\newdualacronym{ban}{BAN}{\textit{Body Area Network}}{Red de Área Corporal}
\newdualacronym{wls}{WLS}{\textit{Weighted Least Squares}}{Mínimos Cuadrados Ponderados}
\newdualacronym{pt}{PT}{\textit{Power Threshold}}{Umbral de Potencia}
\newdualacronym{svm}{SVM}{\textit{Support Vector Machine}}{Máquina de Vectores de Soporte}
\newdualacronym{knn}{K-NN}{\textit{k-Nearest Neighbors}}{k-Vecinos Más Cercanos}
\newdualacronym{rms}{RMS}{\textit{Root Mean Square}}{Raíz Cuadrática Media}
\newdualacronym{cir}{CIR}{\textit{Channel Impulse Response}}{Respuesta al Impulso del Canal}
\newdualacronym{ifft}{IFFT}{\textit{Inverse Fast Fourier Transform}}{Transformada Rápida Inversa de Fourier}
\newdualacronym{mse}{MSE}{Mean Squared Error}{Error Cuadrático Medio}
\newdualacronym{pf}{PF}{\textit{Particle Filter}}{Filtro de Partículas}
\newdualacronym{prisma}{PRISMA}{\textit{Preferred Reporting Items for Systematic Reviews and Meta-Analyses}}{}
\newdualacronym{kf}{KF}{\textit{Kalman Filter}}{Filtro de Kalman}

% --- FIN: DEFINICIÓN DE ACRÓNIMOS DUALES ---

% Redefinir formato de primera vez para acrónimos
\renewcommand*{\genacrfullformat}[2]{\glsentryshort{#1}, \glsentrylong{#1}}

% Configuración de encabezados y pies con scrlayer-scrpage
\clearpairofpagestyles
\ohead{\pagemark}
\ihead{\headmark}
\automark[section]{chapter}

\usepackage{tikz}
\usetikzlibrary{shapes.geometric, positioning, intersections, shapes.misc, positioning, arrows.meta, shadows}
\usepackage{amsfonts}
\usepackage{amssymb}
\usepackage[spanish]{babel}

\usepackage{graphicx}
\usepackage{subfigure}
\usepackage{listings}
\usepackage{appendix}
%Márgenes
\usepackage[left=3cm,top=3cm,right=3cm,bottom=3cm]{geometry}

%
\usepackage{multicol}

%para generar índice con hipervínculos
\usepackage{hyperref}

%parametros del documento (sus propiedades)
\hypersetup{
    pdftitle={Trabajo fin de grado},
    pdfsubject={TFG - 2023},
    pdfauthor={Nombre del alumno},
    pdfkeywords={palabraclave1} {palabraclave2} {palabraclave3},
    colorlinks,
    citecolor=black,
    filecolor=black,
    linkcolor=black,
    urlcolor=black,
}

%algunos sÌmbolos matemáticos y paquetes para usar subimágenes
\usepackage{amsmath} 
%Código
\usepackage{listings}
%Rename Listlings
\renewcommand{\lstlistingname}{Código}% Listing -> Algorithm
\renewcommand{\lstlistlistingname}{Lista de \lstlistingname s}
% List of Listings -> Lista de Código

\definecolor{headerblue}{RGB}{220,230,241}
\definecolor{rowgray}{RGB}{245,247,250}

%empieza el documento
\begin{document}  

%elementos antes del trabajo en sÌ se meten dentro de frontmatter
\frontmatter


%cada incluye referencia a un archivo de tipo .tex
% --- Página de Título ---
\begin{titlepage}
    \centering
  
    \vspace{1cm}
    {\Large \textbf{ANÁLISIS DEL EFECTO DE LA OBSTRUCCIÓN CORPORAL EN UN SISTEMA DE POSICIONAMIENTO EN INTERIORES BASADO EN ULTRA-WIDEBAND}\par}
    \vspace{1cm}
    % Centrar imagen y escalar         
    \includegraphics[width=0.3\textwidth]{imagenes/images.jpeg}\par % Ajusta el ancho según sea necesario
    \vspace{1.3cm}
    Anteproyecto Tesis de Maestría\par
    \vspace{0.5cm}
    {\Large \textbf{Danny Daniel Díaz López}}\par
    \vspace{0.8cm}
    Director\hspace{1cm}: Dr. Víctor Manuel Quintero Flórez\par
    Codirectora\hspace{0.5cm}: Claudia Milena Hernández Bonilla\par
    \vspace{1 cm}


{\fontsize{26}{28}\selectfont Universidad del Cauca} \\
    Facultad de Ingeniería Electrónica y Telecomunicaciones \\
    Departamento de Telecomunicaciones \\
    Grupo de Radio e Inalámbricas -- GRIAL \\ % Usar -- para guión largo
    Sistemas de Comunicaciones Móviles e Inalámbricos \\
    Grupo de Nuevas Tecnologías en Telecomunicaciones -- GNTT \\
    %\vspace{1cm}
    Popayán, 2025.
   % \vspace{2cm}


\end{titlepage}




%\include{capitulos/aceptacion}
%
\thispagestyle{empty}
\textcolor[rgb]{1.00,1.00,1.00}{palabra} % Pinta "palabra" de blanco

\newpage

\begin{center}
% Está puesto en portada.tex
\thispagestyle{empty}
%forma de introducir imágenes. el \\[0.5 cm] de final de línea introduce un salto de ese tamaño.
%width=1\textwidth indica el tamaño de la imágen (valores entre 0-1). 
%%\includegraphics[width=1\textwidth]{figuras/cabecera.png}  \\[0.5 cm]

\LARGE UNIVERSIDAD DEL CAUCA \\ [1 cm]

\LARGE FACULTAD DE INGENIERÍA ELECTRÓNICA Y TELECOMUNICACIONES \\ [1 cm]

\LARGE ff\\ [1 cm]

\Large \textbf{TRABAJO FIN DE GRADO}\\[4 cm]

\Huge \textsc{Pfff


%flushleft alinea a la izquierda el texto


%%\begin{flushleft} \Large
\large \emph{Director:}fffn\\
%%\emph{Departamento:} Electrónica, Instrumentación y Control
%%\end{flushleft}

%\end{multicols} 

%rellena de blanco el resto de la página para escribir abajo del todo
\vfill

% Bottom of the page
{\large Popayán, Junio, 2023}

%SE ponen al final firmas.tex
\end{center}


\cleardoublepage 


\newpage
\thispagestyle{empty}
\mbox{}
\newpage
%Licencia opcional
\include{capitulos/licencia}

%\include{capitulos/evaluacion}

%\chapter{Agradecimientos}

\textbf{Jfffo}

dddd

\newpage

\textbf{Jddddez}
ddddddd



% ===========================
% RESUMEN Y ABSTRACT
% ===========================
\chapter*{RESUMEN}
\addcontentsline{toc}{chapter}{RESUMEN}

La tecnología de Banda Ultra Ancha (UWB) se ha consolidado como una de las soluciones más prometedoras para Sistemas de Posicionamiento en Interiores (IPS), gracias a su capacidad de alcanzar exactitudes subdecimétricas. Sin embargo, uno de los desafíos más críticos que enfrenta esta tecnología en aplicaciones de seguimiento de personas es el fenómeno de obstrucción corporal (\textit{Body Shadowing}, BS), donde el cuerpo humano se interpone entre el dispositivo móvil portado y los nodos fijos de referencia, degradando significativamente la exactitud de las mediciones de distancia y, consecuentemente, la estimación de posición.

Este trabajo de maestría presenta una investigación experimental sistemática del efecto de la obstrucción corporal sobre el desempeño de sistemas IPS basados en UWB operando en la banda de 6.5 GHz. A diferencia de estudios previos que se han concentrado principalmente en frecuencias más bajas (3-5 GHz), esta investigación caracteriza el fenómeno en una banda de frecuencia que ha recibido menor atención en la literatura científica, pero que ofrece un compromiso potencialmente ventajoso entre resolución temporal, características de propagación y disponibilidad de ancho de banda regulatorio.

La metodología experimental empleada se fundamenta en el Modelo en V, asegurando una correspondencia rigurosa entre las fases de diseño e implementación del sistema y las fases de verificación y validación. Se desplegó un sistema experimental compuesto por cuatro nodos ancla UWB fijos y un nodo móvil portado por participantes humanos en siete ubicaciones corporales diferentes: cabeza, cadera, mano, muñeca, pecho, rodilla y tobillo. Para cada ubicación, se recolectaron mediciones de Tiempo de Vuelo (ToF) en múltiples posiciones espaciales y orientaciones relativas del cuerpo, cubriendo sistemáticamente condiciones de Línea de Vista (LOS) y Sin Línea de Vista (NLOS).

Los resultados experimentales revelan que la ubicación corporal del dispositivo móvil tiene un impacto crítico sobre la magnitud del error introducido por la BS. La ubicación en la cabeza presentó el desempeño más estable con un error medio de 10.43 cm en condiciones LOS interiores y 18.66 cm en NLOS, mientras que ubicaciones como la cadera y el pecho experimentaron degradaciones severas en NLOS, alcanzando errores promedio de hasta 97.76 cm y 83.98 cm respectivamente. En contraste, ubicaciones en extremidades como la muñeca y la mano mostraron una mejor resiliencia a la obstrucción corporal, con errores en NLOS interiores de 22.00 cm y 26.01 cm respectivamente.

El análisis estadístico demostró que las distribuciones de error en condiciones de obstrucción se apartan significativamente de la gaussianidad, presentando asimetría positiva y colas pesadas que se modelan mejor mediante distribuciones log-normales o mezclas de gaussianas. Esta no-gaussianidad tiene implicaciones directas para el diseño de algoritmos de localización, sugiriendo que enfoques robustos o adaptativos son necesarios para alcanzar desempeño óptimo en presencia de BS.

La correlación entre el error de medición y las características antropométricas de los participantes (estatura, peso, índice de masa corporal) fue significativa, lo cual sugiere que modelos personalizados por usuario podrían mejorar el desempeño.

El desempeño del sistema completo de posicionamiento 2D, evaluado mediante trilateración con las mediciones de distancia hacia los cuatro nodos ancla, alcanzó una exactitud con errores promedio entre 0.60 m y 0.95 m dependiendo de la ubicación corporal del dispositivo. La implementación de un Filtro de Kalman adaptativo logró reducir significativamente la variabilidad y los errores máximos (hasta en un 47\% en el caso del pecho), demostrando la efectividad de técnicas de procesamiento de señal para mitigar parcialmente el efecto de la BS.

Las contribuciones principales de esta investigación incluyen: (1) la primera caracterización experimental sistemática del efecto de BS en sistemas UWB operando en 6.5 GHz, llenando una brecha en el conocimiento científico; (2) el desarrollo de modelos estadísticos de error para diferentes condiciones de propagación y ubicaciones corporales; (3) un protocolo experimental reproducible que puede ser adoptado por la comunidad científica para estudios comparativos; y (4) recomendaciones prácticas para el diseño e implementación de sistemas IPS comerciales en presencia de obstrucción corporal.

Los resultados confirman que la tecnología UWB en la banda de 6.5 GHz es viable para aplicaciones de seguimiento de personas en interiores, alcanzando exactitudes que, si bien se degradan en presencia de obstrucción corporal respecto a condiciones ideales de LOS, resultan suficientes para una amplia gama de aplicaciones prácticas en seguridad industrial, logística, atención médica y deportes. Las direcciones de trabajo futuro identificadas incluyen la extensión a escenarios más diversos, el desarrollo de algoritmos de mitigación más sofisticados, y la fusión con otras modalidades sensoriales para mejorar la robustez del sistema.

\textbf{Palabras clave:} Banda Ultra Ancha (UWB), Sistemas de Posicionamiento en Interiores (IPS), Obstrucción Corporal (\textit{Body Shadowing}), Tiempo de Vuelo (ToF), Medición de Distancia en Dos Vías (TWR), Exactitud de Localización, Propagación NLOS, Frecuencia 6.5 GHz.


\vspace{1cm}

\chapter*{ABSTRACT}
\addcontentsline{toc}{chapter}{ABSTRACT}

Ultra-Wideband (UWB) technology has emerged as one of the most promising solutions for Indoor Positioning Systems (IPS), owing to its capability to achieve sub-decimeter accuracy. However, one of the most critical challenges this technology faces in people-tracking applications is the Body Shadowing (BS) phenomenon, where the human body interposes between the carried mobile device and the fixed reference nodes, significantly degrading distance measurement accuracy and, consequently, position estimation.

This master's thesis presents a systematic experimental investigation of the body shadowing effect on the performance of UWB-based IPS operating in the 6.5 GHz band. Unlike previous studies that have primarily focused on lower frequencies (3-5 GHz), this research characterizes the phenomenon in a frequency band that has received less attention in the scientific literature, but which offers a potentially advantageous tradeoff between temporal resolution, propagation characteristics, and regulatory bandwidth availability.

The experimental methodology employed is grounded in the V-Model, ensuring rigorous correspondence between system design and implementation phases and the verification and validation phases. An experimental system consisting of four fixed UWB anchor nodes and one mobile node carried by human participants was deployed in seven different body locations: head, hip, hand, wrist, chest, knee, and ankle. For each location, Time-of-Flight (ToF) measurements were collected at multiple spatial positions and relative body orientations, systematically covering Line-of-Sight (LOS) and Non-Line-of-Sight (NLOS) conditions.

The experimental results reveal that the body location of the mobile device has a critical impact on the magnitude of the error introduced by BS. The head location presented the most stable performance with a mean error of 10.43 cm in indoor LOS conditions and 18.66 cm in NLOS, while locations such as the hip and chest experienced severe degradations in NLOS, reaching average errors of up to 97.76 cm and 83.98 cm respectively. In contrast, locations on extremities like the wrist and hand showed better resilience to body obstruction, with errors in indoor NLOS of 22.00 cm and 26.01 cm respectively.

Statistical analysis demonstrated that error distributions under shadowing conditions significantly depart from Gaussianity, presenting positive skewness and heavy tails that are better modeled by log-normal distributions or Gaussian mixtures. This non-Gaussianity has direct implications for localization algorithm design, suggesting that robust or adaptive approaches are necessary to achieve optimal performance in the presence of BS.

The correlation between measurement error and participants' anthropometric characteristics (height, weight, body mass index) was significant, indicating that user-personalized models could improve performance.

The performance of the complete 2D positioning system, evaluated through trilateration with distance measurements to the four anchor nodes, achieved an accuracy with mean errors ranging from 0.60 m to 0.95 m depending on the body location of the device. The implementation of an adaptive Kalman Filter significantly reduced variability and maximum errors (by up to 47\% in the case of the chest), demonstrating the effectiveness of signal processing techniques to partially mitigate the BS effect.

The main contributions of this research include: (1) the first systematic experimental characterization of the BS effect in UWB systems operating at 6.5 GHz, filling a gap in scientific knowledge; (2) the development of statistical error models for different propagation conditions and body locations; (3) a reproducible experimental protocol that can be adopted by the scientific community for comparative studies; and (4) practical recommendations for the design and implementation of commercial IPS in the presence of body shadowing.

The results confirm that UWB technology in the 6.5 GHz band is viable for indoor people-tracking applications, achieving accuracies that, while degraded in the presence of body shadowing compared to ideal LOS conditions, are sufficient for a wide range of practical applications in industrial safety, logistics, healthcare, and sports. Identified future work directions include extension to more diverse scenarios, development of more sophisticated mitigation algorithms, and fusion with other sensory modalities to improve system robustness.

\textbf{Keywords:} Ultra-Wideband (UWB), Indoor Positioning Systems (IPS), Body Shadowing, Time-of-Flight (ToF), Two-Way Ranging (TWR), Localization Accuracy, NLOS Propagation, 6.5 GHz Frequency.


% --- Tabla de contenido centrada, en mayúsculas y añadida a preliminares ---
\begingroup
  % Texto del título
  \renewcommand{\contentsname}{\centering \MakeUppercase{Tabla de contenido}}

  % Formato del título centrado (solo para este bloque)
  \setkomafont{chapter}{\normalfont\huge\bfseries\centering}

  % Imprimir el índice
  
  \tableofcontents
  \addcontentsline{toc}{chapter}{LISTA DE FIGURAS}
\endgroup

% ===== LISTA DE FIGURAS =====
\begingroup
  % Título en mayúsculas
  \renewcommand{\listfigurename}{\centering \MakeUppercase{ Lista de figuras}}

  % Formato del título centrado
  \setkomafont{chapter}{\normalfont\huge\bfseries\centering}

  % Imprimir la lista
  \listoffigures
\endgroup

% ===== LISTA DE TABLAS =====
\begingroup
  % Título en mayúsculas
  \renewcommand{\listtablename}{\centering \MakeUppercase{Lista de tablas}}

  % Formato del título centrado
  \setkomafont{chapter}{\normalfont\huge\bfseries\centering}

  % Imprimir la lista
  \listoftables
  \addcontentsline{toc}{chapter}{LISTA DE TABLAS}
\endgroup

\glsaddall 
\printglossary[type=\acronymtype, style=spanishlist, title={\centering LISTA DE ACRÓNIMOS}]
\addcontentsline{toc}{chapter}{LISTA DE ACRÓNIMOS}

%Empieza la parte descriptiva del trabajo
\mainmatter

% ===========================
% ESTRUCTURA DE TESIS DE MAESTRÍA
% ===========================

% Capítulo 1: Introducción



% --- Contenido Principal ---
\chapter{INTRODUCCIÓN}
\label{ch:introduccion}

En las últimas dos décadas, hemos sido testigos de una transformación radical en la forma en que interactuamos con la tecnología y con nuestro entorno. La tecnología de Banda Ultra Ancha (\gls{uwb}) ha surgido como una pieza clave en este panorama evolutivo, especialmente en el ámbito de los Sistemas de Posicionamiento en Interiores (\gls{ips}). Lo que hace particularmente atractiva a esta tecnología es su capacidad para ofrecer una exactitud de localización a nivel subdecimétrico, algo que parecía inalcanzable hace apenas unos años. Sin embargo, como suele ocurrir con las innovaciones tecnológicas, pronto nos encontramos con desafíos que limitan su aplicación en el mundo real.

Uno de estos desafíos, quizás el más intrigante y complejo, es el fenómeno conocido como obstrucción corporal o \textit{Body Shadowing} (\gls{bs}). Imagine por un momento que usted porta un pequeño dispositivo \gls{uwb} en su muñeca o en el bolsillo de su camisa. Cada vez que su cuerpo se interpone entre ese dispositivo y los puntos de referencia fijos en el ambiente, las señales de radiofrecuencia deben atravesar o rodear su cuerpo. El tejido humano, rico en agua y con propiedades electromagnéticas complejas, se convierte en un obstáculo significativo que atenúa, refracta y difracta estas ondas. El resultado no es trivial: errores en la estimación de distancia que pueden superar los 4 metros en las peores condiciones, convirtiendo un sistema que prometía precisión centimétrica en algo poco confiable para aplicaciones críticas.

Esta situación cobra especial relevancia cuando pensamos en las aplicaciones prácticas. Consideremos un trabajador en una planta industrial que porta un dispositivo de seguridad para prevenir colisiones con maquinaria pesada, o un cirujano que necesita rastrear instrumental quirúrgico con precisión milimétrica durante una operación, o incluso un atleta de alto rendimiento cuyos movimientos se analizan para optimizar su técnica. En todos estos escenarios, la obstrucción corporal no es un detalle técnico menor que pueda ignorarse, es un factor determinante que define si el sistema cumplirá o no su propósito fundamental.

La comunidad científica ha reconocido esta problemática y ha comenzado a explorar diferentes estrategias para abordarla. Algunos investigadores han propuesto modelos estadísticos que caracterizan cómo varía el error según la orientación del cuerpo respecto a los nodos de referencia. Otros han desarrollado algoritmos de localización más sofisticados que combinan las mediciones de distancia con información de Unidades de Medida Inercial (\gls{imu}), permitiendo estimar la orientación del portador y ajustar las mediciones en consecuencia. Estos enfoques híbridos han demostrado resultados prometedores, logrando reducciones significativas en el error de posicionamiento cuando se comparan con métodos tradicionales. No obstante, la mayoría de estos estudios se han concentrado en frecuencias relativamente bajas, típicamente entre 3 y 5 GHz.

Aquí surge una pregunta natural: ¿qué ocurre cuando operamos en frecuencias más altas, específicamente en la banda de 6.5 GHz? Esta es una región del espectro que, sorprendentemente, ha recibido menos atención en la literatura científica, a pesar de ofrecer características potencialmente ventajosas. Por un lado, frecuencias más altas implican longitudes de onda más cortas, lo que se traduce en mejor resolución temporal y, teóricamente, mayor precisión en las mediciones de tiempo. Por otro lado, también enfrentamos mayor atenuación en tejidos biológicos y mayor sensibilidad a las obstrucciones. Es este delicado equilibrio el que motiva nuestra investigación.

El presente trabajo de maestría nace precisamente de la necesidad de comprender a profundidad cómo se manifiesta el efecto de la obstrucción corporal en sistemas \gls{uwb} operando en 6.5 GHz. No nos conformamos con repetir lo que ya sabemos de otras frecuencias, sino que buscamos caracterizar sistemática y rigurosamente este fenómeno en condiciones que reflejen aplicaciones reales. Para lograrlo, hemos diseñado un protocolo experimental exhaustivo que evalúa múltiples ubicaciones de portación del dispositivo móvil (pecho, espalda, cadera, muñeca, tobillo), diferentes orientaciones relativas del cuerpo respecto a los nodos de referencia, y participantes con características antropométricas diversas. 

Nuestra investigación no se limita a documentar el problema, sino que busca proporcionar herramientas prácticas para mitigarlo. A través de un análisis estadístico detallado, pretendemos desarrollar modelos que capturen la naturaleza estocástica del error introducido por la obstrucción corporal, modelos que puedan integrarse en algoritmos de localización más robustos. Además, evaluamos técnicas de filtrado y fusión sensorial que permitan compensar, al menos parcialmente, las limitaciones impuestas por la interacción entre las señales electromagnéticas y el cuerpo humano.

Los resultados de este trabajo aspiran a tener un impacto que trascienda lo puramente académico. Si logramos caracterizar adecuadamente el fenómeno de obstrucción corporal en 6.5 GHz, estaremos proporcionando a los ingenieros y desarrolladores de sistemas comerciales una base sólida para tomar decisiones informadas sobre dónde colocar los nodos de referencia, cómo configurar los parámetros del sistema, y qué algoritmos emplear según los requerimientos específicos de cada aplicación. En última instancia, buscamos que la tecnología \gls{uwb} cumpla su promesa de ofrecer localización precisa y confiable en entornos de interiores, incluso cuando los usuarios humanos inevitablemente obstruyen las señales con sus propios cuerpos.

En los capítulos subsecuentes de esta tesis, el lector encontrará no solo datos y gráficas, sino también la historia de cómo fuimos desentrañando, medición tras medición, los secretos de un fenómeno físico que, aunque invisible a nuestros ojos, determina si millones de dispositivos conectados podrán cumplir su función en el mundo real. Es una historia de precisión y error, de señales que viajan a la velocidad de la luz pero se retrasan por meros centímetros de tejido humano, y de cómo la ciencia nos permite comprender y, eventualmente, dominar estos desafíos técnicos para construir tecnologías más útiles y confiables.



% Capítulo 2: Planteamiento del Problema
\section{PLANTEAMIENTO DEL PROBLEMA}
\label{ch:planteamiento}

Las Tecnologías de la Información y la Comunicación (TIC) han tenido un gran desarrollo durante las últimas décadas. Gracias a esto apareció el concepto de Internet de las Cosas (\gls{iot}), un ecosistema de dispositivos interconectados capaz de recolectar e intercambiar información, así como de ejecutar diversos tipos de tareas de forma automatizada, haciéndose presente en múltiples escenarios, tanto domésticos como industriales y científicos \cite{ref1, ref2}. Entre los dispositivos que forman parte del IoT, los dispositivos inalámbricos vestibles (\textit{wearables}) desempeñan un papel clave en aplicaciones relacionadas con el cuidado personal, la asistencia médica, la telemedicina y la seguridad industrial, entre otros. Uno de los retos más relevantes relacionados con el uso de dispositivos IoT \textit{wearables} se encuentra asociado a los Servicios Basados en Localización (\gls{lbs}).

En escenarios de exteriores, la estimación de la localización y/o posición de un objetivo puede ser llevada a cabo usando el Sistema de Posicionamiento Global (\gls{gps}), el cual es uno de los Sistemas Globales de Navegación por Satélite (\gls{gnss}) conformado por una red o constelación de satélites que provee servicios de posicionamiento, navegación y señales de tiempo de referencia a nivel regional y global; sin embargo, en escenarios de interiores, este sistema pierde efectividad debido a que generalmente no existe Línea de Vista (\gls{los}) con los satélites, por obstrucciones tales como los techos y las paredes de las edificaciones, lo que implica la atenuación excesiva de las señales provenientes de los satélites al penetrar los edificios, y por lo tanto señales con niveles por debajo del valor de sensibilidad de los receptores. Por esta razón, se han desarrollado los Sistemas de Posicionamiento Local (\gls{lps}) y más específicamente, los Sistemas de Localización en Tiempo Real  (\gls{rtls}) y dentro de estos, los IPS\footnote{Para el presente trabajo, los términos y conceptos relacionados con RTLS e IPS se tratan de forma indistinta. 
%
Se debe tener en cuenta que la localización proporciona información de la ubicación de un dispositivo o nodo móvil de usuario con respecto a dispositivos fijos o nodos ancla y que son utilizados en aplicaciones tales como la guía de navegación en un museo y la localización de pacientes o especialistas en un hospital, entre otros.}, los cuales permiten realizar una estimación de la localización de un objetivo en escenarios de interiores, haciendo uso de tecnologías de Radiofrecuencia (\gls{rf}) tales como: UWB, Identificación por Radiofrecuencia (\gls{rfid}), Fidelidad Inalámbrica (\gls{wifi}), Zigbee y Bluetooth de Baja Energía (\gls{ble}) \cite{ref6, ref4}, entre otras. Además de sistemas de RF, un IPS puede utilizar tecnologías basadas en ultrasonido y luz, y su funcionamiento se basa en las mismas métricas de señal utilizadas por los sistemas de RF, tales como, la Intensidad de la Señal Recibida (\gls{rss}), el Ángulo de Llegada (\gls{aoa}), el Tiempo de Llegada (\gls{toa}), el Tiempo de Vuelo (\gls{tof}), la Medición de Distancia en Dos Vías (\gls{twr}), y la Fase de Llegada (\gls{poa}). 
%
En los escenarios de interiores con condiciones de canal NLOS, hay interacciones de las señales de RF con obstáculos, tales como mobiliario, paredes, pisos, techos y personas. Lo anterior genera fenómenos de reflexión y difracción de las señales de RF, representados con atenuaciones y cambios en la dirección de las señales que buscan ir de los dispositivos transmisores a los dispositivos receptores, lo cual implica retardos (cambios de fase) de la señal, y lo cual se conoce como multitrayecto, y afecta el desempeño de sistemas de comunicación inalámbricos. En el caso del cuerpo humano, las ondas de RF se reflejan y difractan de diferentes formas dependiendo de la ubicación del dispositivo de comunicación en el cuerpo humano, de los tejidos con los cuales interactúa la señal, lo cual de manera general se considera como BS. Evaluar y analizar el efecto de la BS en un IPS es un problema de significativa complejidad porque cada tejido humano tiene diferentes características eléctricas, i.e., permitividad, permeabilidad y conductividad. Las señales electromagnéticas en condición de NLOS y que atraviesan el cuerpo humano, además de las señales multitrayecto por reflexión y difracción en el cuerpo humano y en el escenario de despliegue del IPS, hacen que el receptor perciba ecos de la misma señal con diferentes atenuaciones y retardos (o cambios de fase), lo cual afecta la detección de señales por parte del receptor, afectando la exactitud y la precisión\footnote{La exactitud mide cuánto se aproximan los resultado al valor verdadero o conocido, y la precisión mide cuánto se aproximan los resultados entre sí \cite{ref5}. Aunque se manejen muchas veces de manera indistinta, cada una tiene su significado.} del IPS.

Para estimar la localización en escenarios de interiores se utilizan diferentes métricas de señal y técnicas asociadas con las que se busca minimizar los errores derivados principalmente de la condición de NLOS y los efectos del multitrayecto. A continuación, se presentan las tres técnicas más utilizadas para localización en escenarios de interiores:

La técnica de \textit{fingerprinting} se basa en la construcción de una base de datos, también conocida como “mapa de huellas”, que almacena información sobre el nivel de potencia de las señales de RF recibidas por un dispositivo o nodo móvil de acuerdo con su posición, y que fueron transmitidas desde dispositivos fijos o nodos ancla en ubicaciones específicas. Esta información se utiliza posteriormente en comparación con las medidas de señales que registra el dispositivo móvil que se desea localizar. Generalmente, la métrica de señal utilizada por esta técnica es la RSS. La técnica de \textit{fingerprinting} se puede implementar con bajo costo y baja complejidad, pero generalmente tiene baja exactitud y precisión \cite{ref6, ref7}.

La técnica de trilateración/multilateración se basa en las medidas de la distancia entre el nodo móvil y los nodos ancla, para lo cual utiliza las métricas de señal ToA, ToF, PoA y RSS para estimar la distancia entre ellos. A partir de las medidas de distancia estimadas, se establece la intersección de figuras geométricas formadas entre el nodo móvil y los nodos ancla, i.e., circunferencias, para estimar la posición del nodo móvil (objetivo) mediante un procedimiento matemático. Una ventaja de la trilateración es que tiene un alto desempeño en términos de exactitud y precisión de la estimación de localización del objetivo; sin embargo, tiene altos costos de implementación y requiere la condición LOS en el canal \cite{ref6}.

La técnica de triangulación, en lugar de medir las distancias entre el nodo móvil y los nodos fijos, mide los ángulos de llegada de la señal recibida por el nodo móvil desde nodos fijos o desde nodos fijos hacia un nodo móvil. Por lo tanto, utiliza la métrica de señal AoA. Al medir el AoA con respecto a dos nodos, se calcula la distancia utilizando cálculos trigonométricos. La ventaja de la triangulación es que funciona sin necesidad de realizar una sincronización entre los nodos; aun así, requiere un arreglo de antenas y condiciones de LOS, por lo que implica un mayor costo y es deficiente en escenarios densos o cuando hay efecto de BS \cite{ref6}.

Las señales de UWB debido al uso de pulsos de muy corta duración, i.e., 2ns, tienen un ancho de banda del orden de 500 MHz, y la distribución de su potencia sobre ese gran ancho de banda le permite tener una baja densidad espectral de potencia y de esa manera no interfiere o se ve interferida de manera significativa por señales de otras tecnologías con menor ancho de banda y mayor densidad espectral de potencia (diversidad en frecuencia). La duración de los pulsos y el tiempo entre pulsos permiten evitar traslapes o colisiones con otros pulsos UWB debido al multitrayecto, i.e., minimiza la Interferencia Intersimbólica (\gls{isi}) \cite{ref8}.

La tecnología UWB para el presente trabajo de grado de maestría opera en la frecuencia de 6.5 GHz, convirtiéndose en una alternativa relevante para la implementación de un IPS, y ofrece un conjunto de métricas de señal que pueden ser utilizadas y que gracias a las características de las señales UWB permiten mitigar los efectos adversos causados por la NLOS y el multitrayecto \cite{ref8, ref9, ref10, ref11, ref12}. 

En UWB se mide el tiempo que le toma a una señal radio viajar de un nodo o dispositivo móvil (TAG) a un nodo o dispositivo fijo (ANCHOR). Este tiempo se conoce como ToF. La medida se realiza en un mismo nodo (fijo o móvil), i.e., evaluando el tiempo de ida y regreso de la señal, el cual toma como referencia su base de tiempo y por lo tanto no requiere sincronismo, lo cual es una gran ventaja.
%
La Figura \ref{fig:twr} ilustra la forma en que se estima el ToF y cómo este permite calcular la distancia entre un nodo móvil o etiqueta y un nodo fijo o ancla utilizando la técnica de Medición de Distancia en Dos Vías (\gls{twr}).
%
El nodo móvil o etiqueta (TAG) inicia el TWR enviando un mensaje de consulta con la dirección conocida de un nodo fijo o ancla (ANCHOR). El nodo fijo o ancla (ANCHOR) registra y procesa el mensaje de consulta y envía una respuesta. Cuando el nodo móvil o etiqueta (TAG) recibe la respuesta, calcula el tiempo entre la ida y el regreso del mensaje, i.e., T$_{\text{round}}$, y estima el tiempo que tardó el nodo fijo o ancla (ANCHOR) en procesar y responder el mensaje de consulta, i.e., T$_{\text{reply}}$. Haciendo uso de estos tiempos se calcula el ToF, calculando la diferencia entre el T$_{\text{round}}$ y T$_{\text{reply}}$ y dividiendo entre 2; luego la distancia se calcula multiplicando el ToF por la velocidad de la luz, tal como se presenta a continuación: 
%

\begin{gather}
    d = \frac{1}{2} \left( t_{\mathrm{round}} - t_{\mathrm{reply}} \right) \times c
    = \frac{1}{2} \,\mathrm{ToF} \times c
\end{gather}




\noindent
donde, $t_{\mathrm{round}}$ es el tiempo total desde que el transmisor envía el mensaje hasta que recibe la respuesta (ida y vuelta); $t_{\mathrm{reply}}$ es el tiempo que el receptor tarda en procesar y responder el mensaje; y $c$ es la  velocidad de la luz.

%
% colocar la ecuacion y mejorar la descripcion con https://www.sewio.net/uwb-technology/two-way-ranging/
%
\begin{figure}[ht]
    \centering
    \includegraphics[width=0.8\textwidth]{imagenes/twr.pdf}
    \caption{Funcionamiento de TWR}
    \label{fig:twr}
\end{figure}
%
El nodo móvil o etiqueta (TAG) puede transmitir la distancia calculada al nodo fijo o ancla (ANCHOR) en un mensaje final, de ser necesario \cite{ref13}.

Uno de los mayores desafíos para lograr un IPS exacto, preciso y confiable es modelar y analizar el efecto de la BS sobre las señales. El cuerpo humano es en sí mismo un medio de transmisión de ondas electromagnéticas, en el cual se pueden presentar fenómenos de reflexión, difracción y absorción que pueden propiciar cambios en la señal, lo que incrementa la incertidumbre en el IPS, afectando su desempeño en términos de exactitud y precisión. Al estudiar el efecto estadístico sobre el error de posicionamiento o localización que genera la BS, y más específicamente, considerando la ubicación de un dispositivo \textit{wearable} en el cuerpo humano, es posible estimar cómo se afecta el desempeño del sistema.

El trabajo de maestría considerará un IPS compuesto por cuatro nodos fijos o anclas (ANCHORS) y un nodo móvil o etiqueta (TAG), el cual será ubicado en distintas partes del cuerpo humano, con la finalidad de analizar el desempeño de un IPS basado en UWB con BS. Con este propósito se plantea la siguiente pregunta de investigación.

¿Cómo analizar el efecto de la obstrucción corporal sobre un sistema de posicionamiento en interiores, considerando diferentes posiciones de dispositivos UWB en el cuerpo humano, cuando este sistema opera en la banda de 6.5 GHz y el dispositivo móvil se ubica en diferentes partes del cuerpo?


\subsection{HIPÓTESIS}
La caracterización del error de estimación de distancia por BS en función de la posición del nodo móvil en el cuerpo humano para un IPS basado en UWB en  6.5 GHz, y su integración con un filtro de estimación como el Filtro de Kalman (\gls{kf}), permitirá una reducción significativa y cuantificable del error de posicionamiento, logrando una mejora en la exactitud en escenarios con NLOS severo, en comparación con un IPS que no implementa el filtro de estimación. 


% Capítulo 3: Objetivos
\section{OBJETIVOS}

Este capítulo presenta los objetivos que guiarán el desarrollo del presente trabajo de maestría. Estos objetivos definen los alcances y las metas a alcanzar en relación con el análisis del efecto de la BS en un IPS basado en UWB. %Se dividen en un objetivo general y varios objetivos específicos.

\subsection{OBJETIVO GENERAL}

Analizar el efecto de la obstrucción corporal sobre el desempeño de un sistema de posicionamiento en un escenario de interiores basado en UWB, cuando este sistema opera en la banda de 6.5 GHz y el dispositivo móvil se ubica en diferentes partes del cuerpo.

\subsection{OBJETIVOS ESPECÍFICOS}
\begin{itemize}
    \item Evaluar el efecto de la obstrucción corporal sobre las medidas de señal de un enlace de comunicación entre dos dispositivos UWB y la estimación de distancias, cuando uno de los dispositivos se ubica en diferentes partes del cuerpo humano.
    \item Analizar el efecto estadístico de la obstrucción corporal sobre el enlace de comunicación basado en UWB.
    \item Evaluar el desempeño de un sistema de posicionamiento, i.e., 4 dispositivos fijos y un dispositivo móvil, basado en UWB y en un escenario de interiores, cuando el dispositivo móvil se ubica en diferentes partes del cuerpo humano.
\end{itemize}





% Capítulo 4: Marco Teórico
\chapter{MARCO TEÓRICO}
\label{ch:marco_teorico}

Este capítulo presenta los fundamentos teóricos necesarios para comprender el desarrollo de esta investigación. Se abordan los conceptos relacionados con sistemas de posicionamiento en interiores, la tecnología de Banda Ultra Ancha y el fenómeno de obstrucción corporal que afecta la propagación de señales de radiofrecuencia.

\section{SISTEMAS DE POSICIONAMIENTO EN INTERIORES}
\label{sec:sistemas_posicionamiento}

La localización en interiores representa un desafío técnico significativamente distinto al posicionamiento en exteriores. Mientras que en espacios abiertos el Sistema de Posicionamiento Global (\gls{gps}) proporciona cobertura satelital directa, los escenarios de interiores presentan limitaciones inherentes que degradan o impiden por completo la recepción de señales GNSS. Las paredes, techos y estructuras metálicas atenúan las señales satelitales hasta niveles por debajo del umbral de sensibilidad de los receptores convencionales, haciendo necesario el desarrollo de tecnologías alternativas específicamente diseñadas para estos entornos.

Los Sistemas de Posicionamiento en Interiores (\gls{ips}) surgieron como respuesta a esta problemática, aprovechando diversas tecnologías de radiofrecuencia que operan en rangos de potencia y frecuencia adecuados para espacios cerrados. Entre las tecnologías más utilizadas se encuentran Wi-Fi, Bluetooth de Baja Energía (\gls{ble}), Zigbee, RFID y, más recientemente, Banda Ultra Ancha (\gls{uwb}). Cada una de estas tecnologías presenta características particulares en términos de exactitud, alcance, consumo energético y costo de implementación.

\subsection{Métricas de Señal para Localización}

La estimación de posición en un IPS se fundamenta en la medición de características específicas de las señales de radiofrecuencia que viajan entre dispositivos fijos (anclas o \textit{anchors}) y dispositivos móviles (etiquetas o \textit{tags}). Las principales métricas utilizadas son:

\subsubsection{Intensidad de Señal Recibida (RSS)}

La \gls{rss} mide la potencia de la señal recibida en el receptor. Aunque es la métrica más simple de implementar, su relación con la distancia está fuertemente afectada por el multitrayecto, las reflexiones y las obstrucciones presentes en entornos de interiores. Su variabilidad temporal y espacial limita su exactitud, ubicándola típicamente en el rango de 2 a 5 metros de error.

\subsubsection{Tiempo de Llegada (ToA) y Tiempo de Vuelo (ToF)}

El \gls{toa} mide el instante en que una señal llega al receptor, mientras que el \gls{tof} cuantifica el tiempo que tarda una señal en viajar desde el transmisor hasta el receptor. Estas métricas requieren sincronización precisa entre los dispositivos, lo cual representa un desafío técnico considerable. Sin embargo, cuando se implementan correctamente, permiten alcanzar exactitudes en el orden de decímetros o incluso centímetros.

\subsubsection{Ángulo de Llegada (AoA)}

El \gls{aoa} determina la dirección desde la cual llega la señal al receptor mediante el uso de arreglos de antenas. Esta técnica permite triangular la posición del dispositivo móvil sin necesidad de medir distancias directamente. No obstante, requiere hardware especializado y es particularmente sensible a las reflexiones del entorno.

\subsubsection{Fase de Llegada (PoA)}

La \gls{poa} analiza el cambio de fase de la señal portadora para estimar la distancia recorrida. Aunque puede proporcionar alta resolución, presenta ambigüedad cuando la distancia excede la longitud de onda de la señal portadora, requiriendo técnicas adicionales para resolver dichas ambigüedades.

\subsection{Técnicas de Localización}

Las métricas de señal descritas anteriormente se utilizan en conjunto con diferentes técnicas algorítmicas para estimar la posición del dispositivo móvil:

\subsubsection{Trilateración y Multilateración}

Estas técnicas calculan la posición del objetivo mediante la intersección de círculos (en 2D) o esferas (en 3D) centrados en los nodos ancla, cuyos radios corresponden a las distancias estimadas. La trilateración requiere al menos tres anclas en 2D o cuatro en 3D. Cuando las mediciones contienen ruido, se emplean métodos de optimización como mínimos cuadrados para encontrar la mejor estimación de posición.

\subsubsection{Triangulación}

Basándose en mediciones de AoA desde múltiples anclas, la triangulación determina la posición del objetivo mediante la intersección de líneas de orientación. A diferencia de la trilateración, no requiere sincronización temporal, pero es más susceptible a errores angulares que se magnifican con la distancia.

\subsubsection{Fingerprinting}

Esta técnica opera en dos fases: una fase offline donde se construye una base de datos (mapa de huellas) que asocia ubicaciones conocidas con patrones de señal característicos, y una fase online donde se compara la señal medida con el mapa para estimar la posición. Aunque es robusta frente al multitrayecto, requiere un proceso de calibración exhaustivo y no se adapta bien a cambios en el entorno.


\section{TECNOLOGÍA DE BANDA ULTRA ANCHA (UWB)}
\label{sec:tecnologia_uwb}

La tecnología de Banda Ultra Ancha representa un avance significativo en las comunicaciones inalámbricas de corto alcance y los sistemas de localización de alta precisión. A diferencia de las tecnologías de banda estrecha tradicionales, UWB transmite información mediante pulsos de radiofrecuencia de duración extremadamente corta, típicamente del orden de 2 nanosegundos, lo que resulta en un espectro de frecuencias extremadamente amplio, superior a 500 MHz.

\subsection{Principios de Operación}

Las señales UWB ocupan un ancho de banda considerable, distribuyendo su energía a través de un amplio rango de frecuencias. Esta característica fundamental confiere a UWB varias ventajas distintivas:

\textbf{Baja densidad espectral de potencia:} Al distribuir la energía de transmisión sobre un ancho de banda extenso, la densidad de potencia por unidad de frecuencia es muy baja, minimizando la interferencia con otros sistemas de radiofrecuencia que operan en rangos de frecuencia superpuestos.

\textbf{Resistencia al multitrayecto:} Los pulsos de corta duración permiten resolver temporalmente múltiples versiones de la señal que llegan al receptor por diferentes trayectos. Mientras que en sistemas de banda estrecha estas réplicas se superponen causando interferencia destructiva, en UWB pueden separarse e identificarse individualmente, reduciendo significativamente la Interferencia Intersimbólica (\gls{isi}).

\textbf{Penetración de materiales:} Las componentes de baja frecuencia del espectro UWB facilitan la penetración a través de paredes y obstáculos, aunque con atenuación variable según las propiedades dieléctricas del material.

\textbf{Alta resolución temporal:} La brevedad de los pulsos UWB se traduce en una resolución temporal excepcional, permitiendo mediciones de ToF con exactitudes en el rango de centímetros cuando se emplea procesamiento de señal adecuado.

\subsection{Regulación y Bandas de Frecuencia}

La regulación del espectro UWB varía según la jurisdicción. En Estados Unidos, la Comisión Federal de Comunicaciones (FCC) permite operación UWB entre 3.1 y 10.6 GHz con restricciones de densidad espectral de potencia. En Europa, el Instituto Europeo de Normas de Telecomunicaciones (ETSI) establece regulaciones similares pero con restricciones más estrictas en ciertas bandas para proteger servicios existentes.

Para aplicaciones de localización en interiores, las bandas más comúnmente utilizadas son:
\begin{itemize}
    \item \textbf{Banda baja (3.1 - 4.8 GHz):} Ofrece mejor penetración de obstáculos y mayor alcance.
    \item \textbf{Banda media (6.0 - 7.0 GHz):} Proporciona un equilibrio entre alcance y resolución espacial.
    \item \textbf{Banda alta (7.25 - 10.6 GHz):} Permite la mayor exactitud de localización pero con alcance reducido.
\end{itemize}

El presente trabajo se enfoca en la banda de 6.5 GHz, seleccionada por representar un compromiso óptimo entre exactitud de localización y características de propagación en escenarios de interiores con presencia de obstrucción corporal.

\subsection{Medición de Distancia en Dos Vías (TWR)}

La técnica de \gls{twr} constituye el método más robusto para estimación de distancias en sistemas UWB, eliminando la necesidad de sincronización estricta entre dispositivos. El proceso opera mediante el siguiente protocolo:

\begin{enumerate}
    \item El nodo móvil (TAG) inicia el proceso transmitiendo un mensaje de consulta (\textit{poll}) dirigido a un nodo ancla (ANCHOR) específico.
    \item El nodo ancla recibe el mensaje, lo procesa durante un tiempo conocido $t_{\text{reply}}$, y transmite una respuesta.
    \item El nodo móvil recibe la respuesta y calcula el tiempo total de ida y vuelta $t_{\text{round}}$.
    \item El ToF se obtiene como: $\text{ToF} = \frac{1}{2}(t_{\text{round}} - t_{\text{reply}})$
    \item La distancia se calcula multiplicando el ToF por la velocidad de la luz: $d = \text{ToF} \times c$
\end{enumerate}

Esta técnica elimina los errores de sincronización de reloj entre dispositivos, ya que ambas mediciones temporales se realizan con el mismo oscilador local. Sin embargo, la exactitud depende críticamente de la precisión con que se mida $t_{\text{reply}}$ y de la estabilidad de los osciladores durante el intervalo de medición.

Existen variantes más sofisticadas como el TWR Simétrico de Doble Cara (\gls{ds-twr}), que realiza un intercambio adicional de mensajes para compensar imperfecciones en los osciladores de los dispositivos, mejorando aún más la exactitud de la medición.


\section{PROPAGACIÓN DE SEÑALES EN INTERIORES}
\label{sec:propagacion_interiores}

La propagación de señales de radiofrecuencia en entornos de interiores difiere sustancialmente de la propagación en espacio libre. Los escenarios de interiores presentan múltiples obstáculos que interactúan con las ondas electromagnéticas, generando fenómenos complejos que afectan tanto la amplitud como la fase de las señales recibidas.

\subsection{Condiciones de Propagación}

\subsubsection{Línea de Vista (LOS)}

Una condición \gls{los} existe cuando hay un trayecto directo y despejado entre el transmisor y el receptor. En esta situación, la señal experimenta mínima atenuación y su comportamiento se aproxima al modelo de propagación en espacio libre. Para señales UWB en LOS, la exactitud de las mediciones de distancia alcanza típicamente valores entre 5 y 15 centímetros, dependiendo del ancho de banda y la relación señal-ruido.

\subsubsection{Sin Línea de Vista (NLOS)}

La condición \gls{nlos} ocurre cuando uno o más obstáculos bloquean el trayecto directo entre transmisor y receptor. La señal debe entonces propagarse mediante reflexión, difracción o difusión alrededor de los obstáculos. Estos trayectos indirectos son más largos que el trayecto directo, introduciendo retardos adicionales que, si no se detectan y corrigen, se traducen directamente en errores de estimación de distancia. En condiciones NLOS severas, el error puede superar los 4 metros.

\subsubsection{Cuasi Línea de Vista (QLOS)}

La condición \gls{qlos} representa un estado intermedio donde el trayecto directo está parcialmente obstruido o donde la señal experimenta difracción significativa en los bordes de los obstáculos. Esta situación es particularmente relevante en el contexto de la obstrucción corporal, cuando el cuerpo humano está orientado de tal manera que la señal puede difractarse alrededor del torso sin experimentar obstrucción completa.

\subsection{Fenómenos de Propagación}

\subsubsection{Multitrayecto}

El multitrayecto se produce cuando réplicas de la señal transmitida alcanzan al receptor por diferentes trayectos debido a reflexiones en paredes, techos, pisos y objetos. Cada réplica llega con diferente amplitud, fase y retardo temporal. En sistemas de banda estrecha, estas réplicas se superponen coherentemente, causando desvanecimiento selectivo en frecuencia. En UWB, la alta resolución temporal permite distinguir las réplicas individuales, aunque siguen afectando los algoritmos de detección del primer arribo de señal.

\subsubsection{Reflexión}

Ocurre cuando una onda electromagnética incide sobre una superficie conductora o con propiedades dieléctricas diferentes al medio de propagación. El coeficiente de reflexión depende de la frecuencia, el ángulo de incidencia y las propiedades del material. En interiores, las superficies metálicas, ventanas y paredes de concreto actúan como reflectores significativos.

\subsubsection{Difracción}

La difracción permite que las ondas electromagnéticas se propaguen alrededor de obstáculos, particularmente en sus bordes. Este fenómeno es más pronunciado cuando las dimensiones del obstáculo son comparables a la longitud de onda de la señal. Para UWB en 6.5 GHz (longitud de onda $\approx$ 4.6 cm), los bordes del cuerpo humano, muebles y estructuras arquitectónicas actúan como fuentes de difracción.

\subsubsection{Dispersión}

Cuando las ondas encuentran superficies rugosas o conjuntos de objetos pequeños, la energía se dispersa en múltiples direcciones. Este fenómeno es relevante en escenarios con mobiliario complejo, estanterías con objetos diversos y elementos decorativos.


\section{OBSTRUCCIÓN CORPORAL (BODY SHADOWING)}
\label{sec:obstruccion_corporal}

La obstrucción corporal, conocida en la literatura técnica como \textit{Body Shadowing} (\gls{bs}), representa uno de los fenómenos más críticos que afectan el desempeño de los sistemas de localización basados en tecnologías inalámbricas, particularmente en aplicaciones donde el dispositivo móvil se porta sobre o cerca del cuerpo humano. Este fenómeno se manifiesta cuando el cuerpo humano se interpone entre el transmisor y el receptor, actuando como un obstáculo que atenúa, refracta y difracta las ondas electromagnéticas.

\subsection{Interacción de las Ondas Electromagnéticas con el Cuerpo Humano}

El cuerpo humano es un medio heterogéneo compuesto por tejidos con propiedades electromagnéticas diversas. Los parámetros fundamentales que caracterizan esta interacción son:

\subsubsection{Permitividad Relativa ($\varepsilon_r$)}

Cuantifica la capacidad del tejido para almacenar energía eléctrica cuando es sometido a un campo eléctrico. Los tejidos con alto contenido de agua, como los músculos y órganos internos, presentan valores de permitividad relativa elevados (típicamente entre 40 y 60 a frecuencias de microondas), mientras que tejidos con bajo contenido de agua como el hueso y la grasa exhiben valores menores (5 a 15).

\subsubsection{Conductividad Eléctrica ($\sigma$)}

Representa la capacidad del tejido para conducir corriente eléctrica. Está directamente relacionada con el contenido de iones y agua en el tejido. La conductividad varía significativamente entre diferentes tipos de tejido y aumenta con la frecuencia debido a mecanismos de relajación dieléctrica.

\subsubsection{Permeabilidad Magnética ($\mu_r$)}

Para tejidos biológicos, la permeabilidad magnética relativa es aproximadamente igual a 1, ya que los tejidos humanos son materiales no magnéticos. Por lo tanto, este parámetro tiene un impacto mínimo en la propagación de señales UWB a través del cuerpo.

La combinación de alta permitividad y conductividad no despreciable resulta en una absorción significativa de la energía electromagnética y en una velocidad de propagación reducida dentro del tejido. La profundidad de penetración (profundidad pelicular) disminuye con el aumento de la frecuencia, haciendo que las señales de frecuencias más altas (como 6.5 GHz) experimenten mayor atenuación al atravesar el cuerpo.

\subsection{Mecanismos de Degradación de Señal}

\subsubsection{Atenuación Directa}

Cuando una señal UWB atraviesa tejido biológico, su amplitud se reduce exponencialmente con la distancia recorrida dentro del tejido. Esta atenuación puede alcanzar valores de 20 a 40 dB cuando la señal atraviesa el torso humano, dependiendo de la frecuencia, la polarización y el grosor de tejido atravesado.

\subsubsection{Difracción Corporal}

Cuando el cuerpo obstruye completamente el trayecto directo, las señales pueden difractarse alrededor del cuerpo, siguiendo trayectos más largos. El retardo adicional introducido se traduce en un error positivo en la estimación de distancia (sobreestimación). Este efecto es particularmente pronunciado en condiciones de NLOS completo.

\subsubsection{Multitrayecto Inducido por el Cuerpo}

El cuerpo humano actúa como un reflector irregular, generando múltiples trayectos de señal que pueden interferir con el trayecto directo o difractado. La naturaleza dieléctrica compleja del cuerpo introduce variaciones de fase que complican la detección del primer arribo de señal.

\subsection{Factores que Influyen en la Magnitud del Efecto}

\subsubsection{Orientación Relativa}

El Ángulo de Orientación Relativo del Cuerpo (\gls{rha}) define la geometría entre el dispositivo UWB portado, el cuerpo humano y el nodo ancla. Cuando el RHA es aproximadamente 0°, el dispositivo se orienta directamente hacia el ancla (condición LOS). A 180°, el cuerpo obstruye completamente el trayecto (NLOS severo). Ángulos intermedios (alrededor de 90° y 270°) corresponden a condiciones QLOS.

\subsubsection{Ubicación del Dispositivo en el Cuerpo}

La posición donde se porta el dispositivo UWB tiene un impacto crítico. Dispositivos colocados en la muñeca, cadera o tobillo experimentan diferentes patrones de obstrucción según los movimientos naturales de la persona. Dispositivos en el pecho o espalda presentan obstrucción más estable pero potencialmente más severa.

\subsubsection{Características Antropométricas}

La estatura, peso, complexión y composición corporal del portador influyen en la magnitud del efecto. Individuos con mayor masa corporal generalmente causan mayor atenuación y retardo de señal.

\subsubsection{Frecuencia de Operación}

Frecuencias más altas experimentan mayor atenuación por unidad de longitud de tejido, pero también presentan longitudes de onda menores que pueden facilitar la difracción alrededor del cuerpo. La banda de 6.5 GHz representa un compromiso entre estos efectos contrapuestos.

\subsection{Impacto en Sistemas de Posicionamiento UWB}

El efecto de BS introduce dos tipos principales de error en las mediciones de ToF:

\textbf{Error de sesgo positivo:} La obstrucción corporal aumenta sistemáticamente el ToF medido con respecto al valor real, debido a que las señales siguen trayectos más largos (difracción) o experimentan retardos de grupo al atravesar tejido biológico. Este error puede variar de 0.1 a 1.5 metros dependiendo de la severidad de la obstrucción.

\textbf{Aumento de la variabilidad:} La naturaleza dinámica del cuerpo humano (movimientos respiratorios, desplazamientos, cambios posturales) introduce variabilidad temporal en las mediciones. La desviación estándar del error puede aumentar de 5 cm en LOS a 40 cm en NLOS severo.

La caracterización estadística de estos errores y el desarrollo de técnicas de mitigación constituyen objetivos centrales de esta investigación, particularmente en la banda de 6.5 GHz donde la evidencia experimental es aún limitada.


\section{FILTRADO Y ESTIMACIÓN DE ESTADO}
\label{sec:filtrado_kalman}

Para mitigar los efectos de la obstrucción corporal y mejorar la exactitud de localización, se emplean técnicas de procesamiento de señal y estimación de estado. El Filtro de Kalman (\gls{kf}) representa una de las herramientas más poderosas y ampliamente utilizadas para este propósito.

\subsection{Fundamentos del Filtro de Kalman}

El Filtro de Kalman es un algoritmo recursivo que estima el estado de un sistema dinámico a partir de mediciones ruidosas. Opera en dos etapas alternadas:

\textbf{Predicción:} Utiliza el modelo dinámico del sistema para predecir el estado futuro y su covarianza asociada.

\textbf{Actualización:} Incorpora una nueva medición para corregir la predicción, ponderando la información según las incertidumbres relativas del modelo y de la medición.

\subsection{Aplicación a Sistemas de Posicionamiento}

En el contexto de IPS, el vector de estado típicamente incluye la posición y velocidad del objetivo móvil. El modelo de movimiento puede ser de velocidad constante, aceleración constante, o más sofisticado según la aplicación. Las mediciones corresponden a las distancias estimadas desde el nodo móvil hacia los nodos ancla.

El KF es particularmente efectivo para suavizar las fluctuaciones de corto plazo en las mediciones de distancia causadas por el multitrayecto y la variabilidad de la obstrucción corporal, mientras preserva la capacidad de seguir cambios reales en la trayectoria del objetivo.

\subsection{Limitaciones y Extensiones}

El KF estándar asume linealidad en el modelo de sistema y gaussianidad en los ruidos de proceso y medición. En sistemas de posicionamiento, la relación entre el estado (posición) y las mediciones (distancias) es inherentemente no lineal, requiriendo el uso de extensiones como el Filtro de Kalman Extendido (EKF) o el Filtro de Kalman Unscented (UKF).

Además, el ruido introducido por la obstrucción corporal no es estrictamente gaussiano, presentando colas pesadas y multimodalidad en condiciones de NLOS severo. Enfoques adaptativos que ajustan dinámicamente las matrices de covarianza del filtro según la condición de propagación detectada pueden mejorar significativamente el desempeño.


\section{SÍNTESIS DEL CAPÍTULO}

Este capítulo ha establecido las bases teóricas necesarias para comprender el fenómeno de la obstrucción corporal en sistemas de posicionamiento UWB. Se han revisado los principios de operación de los IPS, las características distintivas de la tecnología UWB, los mecanismos de propagación en interiores y la naturaleza de la interacción entre las ondas electromagnéticas y el cuerpo humano.

La evidencia presentada revela que la obstrucción corporal constituye un desafío técnico significativo que puede degradar la exactitud de localización de valores subdecimétricos a errores métricos. La banda de 6.5 GHz, aunque prometedora por sus características de compromiso entre penetración y resolución, ha sido insuficientemente estudiada en presencia de obstrucción corporal, justificando la necesidad de la investigación experimental que se desarrollará en los capítulos subsecuentes.


% Capítulo 5: Revisión Sistemática de Literatura
\chapter{REVISIÓN SISTEMÁTICA DE LITERATURA}
\label{ch:revision} % Etiqueta para referenciar en TOC

Este capítulo presenta de manera general una revisión sistemática de la literatura centrada en el fenómeno de la BS y su impacto en la exactitud de los IPS basados en UWB. Dado que la obstrucción causada por el cuerpo humano es uno de los principales factores que degradan el desempeño de estos sistemas, especialmente en aplicaciones de seguimiento de personas, revisar el estado del arte es fundamental para proponer soluciones innovadoras y efectivas.

\section{METODOLOGÍA DE REVISIÓN SISTEMÁTICA DE LITERATURA}
\label{sec:metodologia_revision}

Para asegurar una revisión de la literatura relevante, se llevó a cabo una búsqueda en varias bases de datos académicas, con el fin de identificar los estudios más relevantes sobre el efecto de la BS en los IPS basados en UWB. Las bases de datos seleccionadas fueron: Scopus y Web of Science, ya que estas fuentes proporcionan la mayor cantidad de publicaciones en el campo de la ingeniería y las telecomunicaciones.


\textbf{Palabras clave utilizadas:}

Las palabras clave que guiaron la búsqueda fueron:
\begin{itemize}
    \item \textit{UWB localization.}
    \item \textit{Body Shadowing.}
    \item \textit{Ultra-Wideband Signal Propagation.}
    \item \textit{Indoor Positioning Systems.}
\end{itemize}
Estas palabras clave se combinaron y utilizaron para buscar tanto en los títulos como en los resúmenes de los artículos. Posteriormente, se empleó una estrategia de búsqueda cruzada con las referencias de los artículos encontrados inicialmente para ampliar la cantidad de publicaciones relevantes.

\section{PROCESO DE FILTRADO DE ARTÍCULOS}
\label{sec:filtrado}

Se establecieron criterios de selección específicos para filtrar e identificar los artículos más relevantes en función de los objetivos de la investigación. En una primera etapa, se encontraron 57 artículos. Tras aplicar los criterios de selección y descartar los artículos repetidos, quedaron 20 artículos que abordan específicamente los efectos de la BS sobre la exactitud y exactitud de los IPS basados en UWB, tal como se muestra en la Figura \ref{fig:articulos}.

Los criterios de selección utilizados en la revisión de la literatura científica fueron:
\begin{itemize}
    \item Estudios que analicen IPS basados en UWB como tecnología base.
    \item Estudios que evalúen los efectos de BS en la propagación de señales UWB
    \item Trabajos realizados en escenarios como edificios universitarios, casas, u oficinas.
    \item Estudios basados en experimentos o simulaciones que involucren interacción entre las señales de dispositivos UWB y el cuerpo humano.
    \item Documentos que proporcionen información sobre métodos, resultados y análisis técnico de los sistemas de posicionamiento y la BS.
\end{itemize}
Se definieron también los siguientes criterios de exclusión:
\begin{itemize}
    \item Estudios que no se centren en UWB como tecnología principal; por ejemplo, Bluetooth, WiFi o RFID.
    \item Trabajos en escenarios de exteriores o en escenarios no aplicables al estudio de escenarios de interiores tales como escenarios industriales.
    \item Redes de comunicación enfocadas en aplicaciones específicas diferentes a la localización.
\end{itemize}
%
\begin{figure}[ht]
    \centering
    \includegraphics[width=1\textwidth]{imagenes/diagrama_burbujas_espanol.pdf}
    \caption{Selección y Distribución de Artículos}
    \label{fig:articulos}
\end{figure}

Tras el proceso de filtrado, los 20 artículos seleccionados se organizaron en tres categorías principales según su enfoque metodológico, lo que permite un análisis estructurado del estado del arte. La Figura \ref{fig:articulos}  presenta las categorías mencionadas. La primera categoría corresponde a los estudios basados en simulación, que modelan el impacto de la BS. La segunda agrupa los estudios experimentales, enfocados en la validación empírica en escenarios reales. Finalmente, la tercera categoría reúne los estudios de mitigación de error, que proponen algoritmos para corregir las desviaciones causadas por la BS. 


\subsection{Estudios Basados en Simulación}

Se encontraron cuatro artículos relevantes sobre estudios que emplean simulación para estudiar el efecto de la BS sobre la estimación de ToF en sistemas basados en UWB en escenarios de interiores. Estos trabajos se destacan por modelar los errores que se generan en la estimación de distancia debido a la condición de canal NLOS causada por la presencia del cuerpo humano. Estos artículos analizan las variaciones en ToF introducidas por los fenómenos de reflexión y difracción  en condiciones de BS. %Lo anterior permite analizar el efecto de la BS sobre la exactitud de los IPS.

\subsubsection{Referencias de literatura considerada:}
%
\begin{enumerate}
    \item \textbf{FDTD and Empirical Exploration of Human Body and UWB Radiation Interaction on \gls{tof} Ranging} \cite{Otim2019}: Este estudio utiliza simulaciones basadas en el metodo de Diferencias Finitas en el Dominio del Tiempo (\gls{fdtd}) y medidas experimentales extensivas para explorar cómo la señal UWB en 3990 MHz interactúa con el cuerpo humano en escenarios de interiores y de exteriores, y para diferentes condiciones de canal, i.e., LOS, Cuasi Línea de Vista (\gls{qlos}) y NLOS.
    
    Los autores investigan cómo la BS afecta la estimación de distancia basada en el ToF entre un nodo móvil y un nodo fijo y cómo las condiciones del cuerpo modifican el nivel del campo eléctrico recibido. Aunque no se mide potencia directamente, el análisis relaciona las variaciones del campo eléctrico con variaciones de  ganancia en el receptor.

    La Figura \ref{fig:ganancia_wave} visualiza el fenómeno físico de la BS mediante una simulación que genera un mapa de calor de la intensidad de la señal UWB, donde los colores cálidos (amarillo, ~0 dB) indican una señal fuerte y los fríos (azul oscuro, $<$ -40 dB) representan una atenuación severa. Para entender la figura, se debe notar que la señal se propaga de izquierda a derecha.
    
    \begin{itemize}
        \item En el panel (a), se ilustra una condición de LOS, donde el dispositivo receptor se encuentra frente al cuerpo. Se observa que la señal incidente llega con alta intensidad (color amarillo), sin ser atenuada significativamente.
        \item En el panel (b), se representa una condición de NLOS. Aquí, el cuerpo está girado y bloquea completamente la trayectoria directa de la señal. El dispositivo receptor queda inmerso en la "sombra" de atenuación (color azul oscuro), donde la intensidad de la señal se reduce en más de 40 dB.
    \end{itemize}

    
%%
    \begin{figure}[ht]
    \centering   \includegraphics[width=0.9\textwidth]{imagenes/datos_LOS_NLOS_QLOS.png}
    \caption{Visualización del Fenómeno de Obstrucción Corporal mediante Simulación FDTD en Condiciones LOS (a) y NLOS (b)}
    \label{fig:ganancia_wave}
    \end{figure}

    \item \textbf{Human Body Shadowing Effect on UWB-Based Ranging System for Pedestrian Tracking} \cite{ref16}: Los autores desarrollaron un modelo basado en simulación que predice cómo el error de ToA afecta la estimación de distancia, considerando la interacción entre las señales UWB y el cuerpo humano. El modelo distingue entre la ubicación del nodo móvil en el cuerpo y su orientación con respecto a un nodo ancla fijo.
    Los autores realizaron simulaciones utilizando un modelo geométrico simplificado de un cilindro vertical para representar el torso humano. En su esquema, un nodo ancla se encuentra en el techo y un nodo móvil sobre el cuerpo. Para analizar el efecto de la obstrucción o Ángulo de Orientación Relativo (\gls{rha}), definen el ángulo de orientación corporal, $\theta$, como el ángulo entre la dirección frontal del torso y la línea de visión directa hacia el nodo ancla.

    \begin{figure}[ht]
        \centering   \includegraphics[width=1\textwidth]{imagenes/RHA_ref17.pdf}
        \caption{Posición Relativa entre el Nodo Móvil y el Nodo Fijo (a) Ejemplo 1, (b) Ejemplo 2.}
        \label{fig:posicion_relativa_nodo_fijo_movil}
    \end{figure}

    El estudio clasifica los escenarios de propagación en tres categorías según el RHA o $\theta$, este ángulo mide la orientación del nodo móvil con respecto al nodo fijo: la condición LOS ocurre cuando el RHA es cercano a $0\degree$, lo que significa que la dirección de la caminata se alinea con este eje de referencia (linea punteada) y el cuerpo del peatón queda detrás del dispositivo. En contraste, la condición de NLOS se produce cuando el RHA se aproxima a $180\degree$, indicando que el peatón se mueve en dirección opuesta al ancla, interponiendo su torso como un obstáculo directo. Las zonas de \gls{qlos} representan los estados de transición críticos, que se manifiestan con un RHA cercano a $90\degree$ y $270\degree$, donde la trayectoria de la señal no está ni despejada ni completamente bloqueada, sino que se difracta por los contornos laterales del cuerpo, estas posiciones de la dirección del movimiento de pueden ver en la figura \ref{fig:posicion_relativa_nodo_fijo_movil}. Los experimentos se llevaron a cabo en un escenario de interiores con dimensiones de 13 m $\times$ 83 m $\times$ 2.5 m, evaluando distancias entre transmisor y receptor de 1 a 6 m con un muestreo a 3.5 Hz. Se utilizaron nodos UWB portátiles acoplados al cuerpo de los participantes, quienes presentaban diferencias en altura y peso, i.e., B1: 1.73 m, 77 kg; B2: 1.66 m, 50 kg.
     
    Los resultados del estudio indican que en condiciones de LOS, el error de estimación de distancia presenta un comportamiento estable y de baja dispersión, con una media que varía entre 0.12 m y 0.15 m y una desviación estándar entre 0.09 m y 0.15 m. Este comportamiento se asocia con una distribución aproximadamente gaussiana. En contraste, bajo condiciones de NLOS, este error de estimación de distancia se incrementa significativamente, alcanzando valores medios de hasta 0.5 m y desviaciones estándar de hasta 0.7 m, lo que evidencia una marcada degradación en el desempeño del sistema.

    Un aspecto fundamental del artículo es que no se limita a este modelo binario (LOS/NLOS), sino que analiza los estados de QLOS. Para ello, los autores introducen la métrica del grado de obstrucción, que va de 0 (visión directa) a 1 (obstrucción total). Los resultados muestran cómo el error aumenta progresivamente: con una obstrucción lateral de 0.5, el error medio es de 0.17 m; y con una obstrucción casi completa de 0.8, el error medio sube a 0.36 m, demostrando una relación directa entre el grado de bloqueo y la imprecisión.

    Los autores concluyen que para la condición LOS, i.e., RHA $\approx$ 0, el error es mínimo, con una media de 5.5~cm y una desviación estándar de 3.6~cm. En la zona de transición con, i.e., QLOS y RHA de $\approx 92\degree$ la difracción de la señal por el costado del cuerpo eleva el error a una media de 11.8~cm con una desviación estándar de 4.5~cm. En condición NLOS, i.e., RHA = $180\degree$, la obstrucción total del torso provoca que el error alcance su punto máximo, con una media de 38.9~cm y una desviación estándar de 13.9~cm.

    \item \textbf{Non-Line-of-Sight Identification based on Unsupervised Machine Learning in Ultra Wideband Systems} \cite{Fan2019}: Aunque este estudio se clasifica en simulación, también es relevante en la mitigación del error NLOS mediante técnicas de Aprendizaje Automático (ML, \textit{Machine Learning}) no supervisado, las cuales se utilizan para identificar condiciones de propagación LOS y NLOS en sistemas de localización basados en UWB. 
    
    Para este análisis se realizaron simulaciones en MATLAB utilizando un modelo de canal UWB en un escenario de interiores, generando 1000 formas de onda, de las cuales 500 representaban condiciones LOS y 500 NLOS. A partir de estas señales, se extrajeron tres características clave: número de caminos significativos (NP), retardo en exceso medio ($\tau_{\textrm{MED}}$) y dispersión de retardo RMS ($\tau_{\textrm{RMS}}$). Estas características fueron utilizadas para alimentar un algoritmo de ML  no supervisado basado en Maximización de la Expectativa (EM, \textit{Expectation Maximization}) para Modelos de Mezcla Gaussiana (GMM, \textit{Gaussian Mixture Model}) \cite{ref20}. Un GMM es un modelo probabilístico que asume que los datos pueden representarse como una combinación de varias distribuciones gaussianas (o normales). Cada una de estas distribuciones gaussianas representa un componente de la mezcla y puede describir un subconjunto específico de los datos, es decir, que un GMM asume que cada grupo de datos sigue una distribución normal. 
    %
    %En el caso de este artículo, cada grupo de datos corresponde a las condiciones de canal LOS y NLOS. 
    %
    
    Si bien el estudio no se enfoca exclusivamente en la BS, el enfoque permite clasificar señales LOS y NLOS sin etiquetado manual, un avance importante en la automatización del proceso de identificación de las condiciones de canal. 
    
    Los autores presentan que la clasificación de señales LOS y NLOS tiene una tasa de clasificación correcta del 86.5\%; una tasa de falsos negativos del 12.7\% donde se identifican condiciones LOS cuando en realidad son NLOS; y de falsos positivos del 0.8\% donde se identifican condiciones NLOS cuando en realidad son LOS.
    
    \item \textbf{Modeling, Validation and Performance Evaluation of Body Shadowing Effect in Ultra-Wideband Networks} \cite{Ruonan2019}: Los autores de este estudio modelaron la BS en redes UWB, presentando que este fenómeno introduce pérdidas significativas de señal que dependen de la orientación del cuerpo humano respecto al transmisor y receptor, es decir, de la dirección en la que está orientado el cuerpo en relación con la trayectoria de la señal. Además, los autores proponen un modelo matemático que mejora la exactitud de los resultados de simulaciones y predicciones del comportamiento del canal en presencia de BS.

    Los autores modelaron y simularon los efectos de la BS sobre los sistemas de comunicación UWB en escenarios de interiores, lo cual es clave para mejorar la exactitud en aplicaciones de localización y seguimiento en tiempo real. Para este análisis, los autores emplearon simulaciones basadas en IEEE 802.15.3a que es una propuesta de extensión al estándar IEEE 802.15.3 y que fue cancelada en 2006, pero aun así se reconoce su relevancia como un estándar de facto para simulaciones UWB debido a su capacidad para representar escenarios de interiores bajo distintas condiciones de canal, i.e., LOS y NLOS.

    Aunque el trabajo es de simulación, se incluye una fase de validación que hace uso de medidas  experimentales previamente disponibles.
    
    Las simulaciones consideran un transmisor y un receptor separados por una distancia de 3 m, situados a 1.2 m de altura. El cuerpo humano se modela como una elipse vertical con un ancho de 0.5 m, y se representa como una barrera que bloquea completamente la propagación de la señal UWB en la dirección frontal, es decir, cuando el cuerpo humano se interpone en línea directa entre el transmisor y el receptor. En el estudio, los ángulos se determinan a partir de un modelo geométrico que define la orientación del cuerpo humano en relación con la línea recta entre el transmisor y el receptor. No se utilizaron nodos físicos; en cambio, se calcularon los ángulos de incidencia de la señal sobre el cuerpo para asignar una pérdida de señal proporcional. Esta pérdida es máxima, 20 dB, cuando la señal incide de forma frontal, i.e., 0$^\circ$, y disminuye progresivamente hacia los lados, hasta anularse en direcciones laterales, i.e., $\pm90^\circ$. Esta función de atenuación se implementa como un filtro direccional sobre las trayectorias del canal.
    
    En las simulaciones, el modelo de canal base utilizado es el IEEE 802.15.3a para condiciones de canal de interiores NLOS.  

    Los resultados muestran que, en presencia de BS, la media de la potencia recibida disminuye entre 5 y 15 dB, dependiendo del ángulo de incidencia. Por ejemplo, cuando la dirección de propagación es frontal, i.e.,  0$^\circ$, se observa una atenuación media de 13.2 dB, mientras que para ángulos cercanos a $\pm90^\circ$, la atenuación es casi nula. El impacto sobre el error de estimación de distancia se analiza mediante simulaciones de localización, donde se observó un aumento promedio del error del 35\% en presencia de BS, con errores que pasaron de 0.18 m sin BS a 0.24 m con BS en escenarios NLOS.

    Los autores también evaluaron la Tasa de Paquetes Pérdidos (PER, \textit{Packet Error Rate}) como función de la posición relativa del cuerpo. En el peor caso, i.e., bloqueo completo de la LOS, el PER alcanzó hasta un 70\%, mientras que en ausencia de obstrucción fue inferior al 10\%. Estos resultados validan el modelo direccional propuesto, que incorpora la orientación corporal como un parámetro clave para simular con exactitud los efectos de la BS en sistemas UWB.
\end{enumerate}

\textbf{Contribución}\\
La contribución conjunta de los estudios de simulación es fundamental para modelar el efecto de la BS a través de distintas métricas para las condiciones LOS, NLOS y QLOS. Estos trabajos abarcan un espectro de enfoques que van desde modelos de FDTD, que correlacionan la atenuación del campo eléctrico con el MAE, hasta modelos geométricos-probabilísticos más simples que definen RHA para cada condición y predicen la distribución estadística del error. Además, otros enfoques utilizan la simulación para entrenar algoritmos de ML capaces de clasificar automáticamente el estado del canal basándose en características de la señal, o proponen modelos de atenuación paramétricos para evaluar el impacto en métricas a nivel de sistema, como la Tasa de Paquetes Pérdidos (PER).


\subsection{Estudios Experimentales}

\subsubsection{Referencias de literatura considerada:}

\begin{enumerate}
    %\setcounter{enumi}{3} % Continúa la numeración de la lista anterior
    \item \textbf{Impact of Body Wearable Sensor Positions on UWB Ranging} \cite{ref14}: Este estudio analiza cómo la posición de los nodos ubicados en diferentes partes del cuerpo, específicamente en frente, mano, pecho, muñeca, brazo, muslo y tobillo, afecta la exactitud de la estimación de distancia utilizando tecnología UWB. 
    
    Los autores determinaron que la posición en la frente proporcionaba los mejores resultados en términos de exactitud de localización. Este estudio proporciona información acerca de la ubicación del receptor UWB en el cuerpo humano para aplicaciones de seguimiento de peatones. Los investigadores utilizaron el kit de desarrollo TREK1000 fabricado por Decawave, el cual opera a una velocidad de transmisión de datos de 110 kbps en el canal 2 a una frecuencia de 3990 MHz.

    Los resultados mostraron que la ubicación del nodo móvil en el cuerpo afecta significativamente la exactitud del ToF, lo que repercute en la estimación de distancia. En LOS, el nodo móvil en la frente presentó los mejores resultados con un error medio de 20 cm, seguido de la mano, muñeca, tobillo, brazo, muslo y pecho. En NLOS, los errores aumentaron considerablemente, alcanzando valores de hasta 2.2 m en el pecho debido al multitrayecto. Se observó que en NLOS, la propagación de ondas UWB está dominada por fenómenos de difracción y propagación por ondas rasantes, siendo estas últimas un tipo de propagación de ondas electromagnéticas en la que la señal viaja muy cerca o alrededor de una superficie.

    Para modelar el error de estimación de distancia, se ajustaron distribuciones estadísticas a la distribución de los resultados experimentales. En LOS y QLOS, los errores siguieron una distribución de probabilidad gaussiana, con una media de 13 cm y una desviación estándar de 8 cm para la frente. En NLOS, los errores fueron modelados con una distribución de probabilidad gamma, con valores máximos de 2.08 m, 0.91 m y 0.47 m en el pecho, la muñeca y la mano, respectivamente. Se observó que la relación entre el RHA y el error de distancia no es lineal en algunas ubicaciones, especialmente en el pecho, donde la correlación entre el error medio de distancia y el RHA fue de -0.69, lo que indica una variabilidad moderada.

    \item \textbf{Gaussian mixture model, IMU, UWB, human body shadowing, indoor localization, particle filter} \cite{Tanghe2023}: Este estudio analiza los valores de ToF de una señal UWB por efecto de la BS y cómo esta afecta la estimación de distancias entre nodos y como consecuencia, su efecto sobre la estimación de la localización del nodo móvil haciendo uso de la técnica de trilateración. 
    
    El IPS basado en UWB está conformado por tres nodos ancla y un nodo móvil operando a una frecuencia de 3990 MHz, el cual fue diseñado para el seguimiento de peatones. El estudio se basa en pruebas experimentales en tres escenarios distintos: una oficina de 3.95 m $\times$ 6.15 m, un laboratorio de 10.9 m $\times$ 21.0 m y una cancha de baloncesto al aire libre. Se utilizaron cuatro nodos EVB1000 de Decawave, configurados como nodos ancla (ANC) y nodos móviles (TAG), operando en el canal 2, i.e., 3993 MHz, con una velocidad de transmisión de datos de 110 kbps y una frecuencia de toma de medidas igual a 3.57 Hz. Las medidas se registraron durante 174 minutos, lo que representa más de 37000 medidas de distancia.

    Los resultados muestran que la BS introduce errores sistemáticos en la estimación de ToF, lo que afecta negativamente la exactitud del sistema de localización. Los resultados mostraron que el error de distancia en LOS sigue una distribución de probabilidad gaussiana, con un error medio de 8 cm y una desviación estándar de 6 cm. En NLOS, el error se ajusta mejor a una distribución de probabilidad gamma, alcanzando valores máximos de 1.6 m cuando el cuerpo bloquea completamente la señal.

    \item \textbf{Experimental Evaluation Scheme of UWB Radio Propagation Channel with Human Body} \cite{Pradabphon2019}: Este artículo evalúa experimentalmente cómo el cuerpo humano afecta el canal de propagación de señales UWB en el rango de frecuencia de 3 GHz a 11 GHz. 
    
    Las pruebas experimentales se realizaron en dos rangos de frecuencia específicos: de 3 GHz a 7 GHz y de 7 GHz a 11 GHz. Estas mediciones fueron realizadas con un Analizador Vectorial de Redes (VNA, \textit{Vector Network Analyzer}) para evaluar la pérdida de propagación y la distorsión de la señal UWB por BS, en la banda definida por la Comisión Federal de Comunicaciones (FCC, \textit{Federal Communications Commission}) para sistemas UWB. %Los autores analizaron con medidas en laboratorio el impacto de la BS sobre las pérdidas de propagación y la distorsión sobre la señal. 
    
    En LOS, el coeficiente de correlación entre la señal medida y la señal esperada (teórica) disminuyó de 73.96\% a 33.77\% al aumentar la distancia de 2 m a 10 m, respectivamente. En NLOS, el coeficiente de correlación fue aún menor, reduciéndose de 21.4\% a 9.65\% en las mismas condiciones de separación. La pérdida de propagación en la salida del filtro acoplado aumentó de 57.86 dB a 73.38 dB en LOS, mientras que en NLOS, la pérdida de propagación aumentó de 72.75 dB a 80.32 dB para las distancias dadas. Al utilizar un filtro acoplado óptimo, la pérdida de señal se redujo ligeramente, alcanzando valores de 55.24 dB a 63.95 dB en LOS y de 62.36 dB a 66.03 dB en NLOS.

    \item \textbf{Effects of the Body Wearable Sensor Position on the UWB Localization Accuracy} \cite{ref4}: Este artículo analiza la influencia de la posición de un nodo móvil ubicado en el cuerpo humano sobre la exactitud de la localización de un IPS basado en UWB. 
    
    El nodo móvil se ubicó en la frente, mano, pecho, muñeca, brazo, muslo y tobillo, estimando el error de localización en distintas condiciones de canal: LOS y NLOS. Los autores caracterizaron el RHA del cuerpo respecto a los nodos ancla, lo cual permitió definir la condición de canal: LOS o NLOS. 
    
    Los experimentos se realizaron en el Laboratorio Luis Mercader de la Universidad Pública de Navarra, en un área de 78 m$^2$ (13 m $\times$ 6 m $\times$ 4 m), utilizando los kits UWB TREK1000 de Decawave, configurados a una velocidad de transmisión de datos de 110 kbps y a una frecuencia central de 3990 MHz (canal 2). 
    
    En el escenario de pruebas se dispuso de cuatro nodos fijos dentro del laboratorio, mientras que un participante masculino portó el nodo móvil en diferentes partes del cuerpo y recorrió una trayectoria predefinida con 26 puntos de referencia. Las medidas se tomaron en discreto, i.e., con paradas en cada punto de referencia, y continuo, i.e., movimiento sin interrupciones entre las tomas de medidas.

    El error de localización estimado varió significativamente según la ubicación del nodo móvil en el cuerpo. En LOS, la ubicación en la frente presentó el mejor desempeño, con un error promedio de 20 cm y una desviación estándar de 11 cm, mientras que el pecho presentó el peor desempeño, con errores de hasta 4.5 m en NLOS. Los autores identificaron que el mecanismo de propagación dominante en la frente es la difracción, basándose en la baja variabilidad de los errores de posicionamiento observados en dicha ubicación del dispositivo móvil, incluso en condiciones NLOS. Esta afirmación se apoya en estudios previos como \cite{ref11}, donde se establece que la forma curva de la cabeza favorece la propagación de ondas por difracción más que por transmisión directa o reflexión, lo que facilita la propagación de la señal incluso en NLOS. En contraste, los autores identificaron que el pecho genera la mayor cantidad de multitrayecto, basándose en los altos errores de posicionamiento observados que fueron de hasta 4.5 m en NLOS y en la variabilidad de las mediciones. Estos resultados sugieren una fuerte presencia de multitrayecto, probablemente debida a la geometría frontal del torso, que facilita la reflexión de señales hacia múltiples rutas. Aunque el artículo no presenta directamente la Respuesta al Impulso del Canal (\gls{cir}), la interpretación se fundamenta en los errores registrados y en conocimientos previos sobre propagación en presencia de superficies planas y reflejantes del cuerpo humano. 
    
    Los autores utilizaron el Filtro de Kalman Extendido (\gls{ekf}) para estimar la localización del nodo móvil. Los autores compararon los errores de posicionamiento obtenidos en las distintas ubicaciones del nodo móvil en el cuerpo humano con respecto al error de localización obtenido cuando el nodo móvil se montó en un trípode estático. Los resultados mostraron que la frente proporcionó el mejor desempeño del IPS, con un error medio de 0.20 m y un error de 0.35 m en el percentil 90 (P90). En contraste, el pecho presentó el mayor error, con un error medio de 2.46 m y de 4.04 m en el percentil 90 (P90). Las posiciones del nodo móvil en la mano, tobillo, muñeca y muslo mostraron errores intermedios, con valores de 0.62 m, 0.97 m, 1.14 m y 1.46 m en el percentil 90 (P90), respectivamente. 
    
    Los autores concluyeron que la exactitud de localización en sistemas UWB depende críticamente de la posición del nodo móvil en el cuerpo. La frente es la mejor ubicación en términos de desempeño, con respecto al error de localización del nodo móvil en un trípode, i.e., 0.12 m de error de distancia. El pecho es la peor ubicación, debido a la propagación NLOS y la generación de multitrayecto. Los autores recomiendan explícitamente considerar el efecto de la BS y la posición del nodo móvil sobre el cuerpo humano en el diseño de sistemas de localización, especialmente para aplicaciones de seguimiento de peatones en escenarios de interiores. Además, proponen y validan un enfoque de estimación basado en un EKF, lo cual demuestra una estrategia práctica para mejorar la exactitud de localización. 

    \item \textbf{The Effect of Human-Body Shadowing on Indoor UWB TOA-Based Ranging Systems} \cite{ref18}: Este estudio analizó el impacto de la BS en la exactitud de los sistemas de estimación de distancia basados en UWB utilizando TOA. 
    
    Se realizaron dos campañas de medición en frecuencias desde 3 hasta 5.5 GHz, una en una cámara anecoica y otra en un laboratorio. Los autores estimaron CIR, obtenida a partir de la Transformada Inversa Rápida de Fourier (\gls{ifft}) de los resultados en frecuencia. Los autores implementaron un algoritmo de detección del primer pico de señal basado en un umbral para la estimación del TOA, utilizando umbrales de $-70$, $-80$ y $-85$ dB, observando que un umbral menor permite identificar trayectorias difractadas de la señal antes que la componente directa atenuada. Los autores adicionalmente estimaron la atenuación de la señal causada por la BS, así como los errores en la estimación de distancia. 

    Las mediciones se realizaron con un analizador de redes vectoriales Rohde \& Schwarz ZVB-8, con una resolución temporal de 0.4 ns, i.e., equivalente a una resolución espacial de 12 cm. Se emplearon dos antenas Time Domain Broadspec UWB con una ganancia nominal de 3 dBi y un patrón de radiación omnidireccional. El sujeto de prueba fue posicionado entre el transmisor y el receptor en distintas orientaciones. Se realizaron pruebas con distancias entre 4 m y 7 m, variando la orientación del cuerpo en posiciones paralelas y perpendiculares a la línea de propagación.

    Los resultados mostraron que la atenuación de la señal directa causada por la BS depende tanto de la posición como de la distancia entre antenas. En la cámara anecoica, se observaron diferencias en la atenuación de la señal directa causadas por la ubicación del cuerpo humano respecto a los nodos. Específicamente, cuando el cuerpo estaba cerca del receptor, se estimó una pérdida promedio de 12 dB, y de 9 dB cuando estaba cerca del transmisor, lo cual se obtuvo con una separación entre antenas de 8 m. La diferencia de la atenuación percibida puede atribuirse a la geometría del cuerpo, que ofrece diferentes formas de obstrucción según su orientación y cercanía a cada nodo. Cabe aclarar que en esta configuración experimental ninguno de los nodos se mueve y que el cuerpo se desplaza entre posiciones definidas a lo largo de la trayectoria entre transmisor y receptor. En el laboratorio, la atenuación de la señal directa fue, en promedio, mayor que en la cámara anecoica, debido a la presencia de múltiples superficies reflectantes que provocan condiciones de canal donde el efecto de multitrayecto es más complejo. En este escenario, se observaron valores de atenuación de hasta 15 dB cuando el cuerpo se encontraba bloqueando la línea directa entre los nodos. Además, los resultados mostraron que la posición central del cuerpo entre las antenas generaba la menor atenuación cuando el cuerpo estaba orientado en paralelo a la dirección de propagación. Esta disminución se atribuye a que, en esa orientación y ubicación, la proyección del cuerpo sobre la línea directa es mínima, lo que reduce el efecto de la obstrucción.

    Los autores evaluaron el error de distancia mediante el cálculo del Error Cuadrático Medio (\gls{mse}), considerando dos configuraciones: una con 4 m y otra con 7 m de separación entre antenas. Para la configuración de 4 m, el error RMS fue de 26 cm cuando el cuerpo humano se alineó con la trayectoria de la señal en LOS, y aumentó hasta 46 cm bajo condiciones de fuerte atenuación en condición de canal NLOS. Esta última fue inducida al fijar el umbral de detección (\gls{pt}) en $-70$ dB, lo que limitó la capacidad del sistema para detectar señales difractadas. Cuando el PT se redujo a $-80$ dB y $-85$ dB, el error disminuyó  a 14 cm, ya que fueron detectadas señales indirectas alrededor del cuerpo. En la configuración de 7 m, se obtuvieron errores RMS de 21 cm en LOS y hasta 38 cm cuando hubo BS.

    El análisis mostró tres regiones de comportamiento en función del umbral de detección. En la Región I (PT $< -90$ dB), el error aumentó debido a detecciones falsas causadas por ruido. En la Región II ($-85$ dB $< $ PT $< -70$ dB), los errores fueron mínimos, ya que los caminos difractados alrededor del cuerpo dominaban la estimación del TOA. En la Región III (PT $> -70$ dB), el error aumentó porque el TOA se estimaba sobre la trayectoria indirecta más fuerte en lugar del camino directo atenuado.

    Los autores concluyen que, aunque la obstrucción causada por BS atenúa significativamente la señal LOS, esta condición no genera errores adicionales en la estimación de distancia siempre que el TOA sea detectado correctamente a partir de trayectorias difractadas. El estudio demuestra que una adecuada selección del umbral de detección permite minimizar el error de estimación incluso en presencia de BS. No obstante, en escenarios con baja Relación Señal a Ruido (SNR, \textit{Signal-to-Noise Ratio}), la pérdida del componente directo de la señal causada por la BS puede deteriorar la exactitud del sistema. Finalmente, los autores recomiendan que futuras investigaciones consideren configuraciones dinámicas, en las cuales el movimiento del usuario pueda modificar las condiciones de canal para la propagación de la señal en tiempo real.

    \item \textbf{The Effects of Human Body Shadowing in RF-based Indoor Localization} \cite{Schimtt2019}: Este artículo aborda el impacto de la BS en la exactitud de los IPS basados en RF. 
    
    Se llevaron a cabo experimentos utilizando mediciones de ToF a una frecuencia de operación de 2.4 GHz, considerando diferentes posiciones de un dispositivo móvil en el cuerpo y distintas técnicas de localización. Se analizaron los efectos de la BS en escenarios reales de seguimiento sobre la estimación de distancia (error de distancia) como sobre la localización (error de localización). Los experimentos se realizaron en dos escenarios distintos. En el primer escenario, se llevaron a cabo mediciones en un entorno controlado con un único nodo móvil y un nodo ancla, ubicados a distancias de 10 m, 15 m y 20 m. Se evaluaron tres posturas corporales: el nodo móvil colocado al lado derecho de la cadera, al lado izquierdo y en la espalda, con orientaciones de 0$^\circ$, 45$^\circ$ y 90$^\circ$ respecto al nodo ancla. En el segundo escenario, se evaluó un caso práctico de localización inspirado en situaciones de emergencia, como rescates en edificios. Para ello, se desplegaron 14 nodos fijos UWB en un edificio de oficinas, a lo largo de un recorrido total de 60 m. En este segundo escenario, se probaron dos configuraciones: una configuración ideal con los 14 nodos ancla (tanto interiores como exteriores) activos. Otra configuración degradada con solo 12 nodos ancla activos, específicamente los ubicados en el exterior del edificio. Esto se hizo para simular un escenario de emergencia realista donde se pierde la comunicación con los nodos desplegados dentro del área de rescate.

    Los resultados mostraron que la BS genera un error de estimación de distancia de entre 4 m y 14 m, afectando directamente la exactitud de posicionamiento. En el escenario controlado, el Error  Absoluto Medio (\gls{mae}) en la estimación de distancia fue de 4.03 m para 0$^\circ$, 7.93 m para 45$^\circ$ y 14.04 m para 90$^\circ$, lo que demuestra que el error aumenta considerablemente con el ángulo de orientación o RHA del nodo móvil respecto al nodo ancla. En distancias mayores, el efecto fue aun más pronunciado. En 20 m, el MAE alcanzó 4.10 m a 0$^\circ$, 13.56 m a 45$^\circ$ y 8.8 m a 90$^\circ$, lo que sugiere que, en ciertos casos, el efecto del multitrayecto puede compensar la atenuación por la BS.
%
    En el segundo escenario, el error de localización dependió del algoritmo utilizado. Se compararon cuatro algoritmos: Centroid, NLLS (\textit{Non-Linear Least Squares}), MD-Min-Max y Geo-n. Con los 14 nodos ancla activos, los mejores resultados se obtuvieron con Geo-n (MAE de 1.22 m) y MD-Min-Max (MAE de 1.80 m), mientras que NLLS presentó el error más alto con 4.64 m. Cuando se utilizaron 12 nodos ancla en exteriores, la exactitud se redujo significativamente. En este caso, el MAE aumentó a 2.44 m con Geo-n, 3.70 m con NLLS y hasta 6.58 m con MD-Min-Max. Para las distintas ubicaciones del nodo móvil, se encontró que la mejor ubicación fue en la cadera derecha, con un MAE de 1.25 m utilizando Geo-n, mientras que la peor fue en la espalda con un MAE de 2.91 m.

    El análisis también reveló que la cantidad de nodos ancla visibles afecta directamente la exactitud de la localización. En la configuración con todos los nodos ancla activos (tanto en escenarios de interiores como de exteriores), el promedio de nodos visibles por ciclo de medición fue de 6.46. Mientras que al excluir los nodos ancla en interiores, es decir, aquellos ubicados dentro del edificio y más cercanos al usuario, este número se redujo a 4.61, lo que generó un incremento del error de localización del 28.8\%. En la configuración más adversa, con el nodo móvil en el lado izquierdo de la cadera y únicamente nodos ancla exteriores activos, el número de anclas visibles cayó a 2.56, con un aumento del error de localización del 37.8\% respecto a la configuración de referencia.

    Los autores concluyen que el efecto de BS degrada significativamente la exactitud del IPS basado en ToF. La mejor exactitud se obtuvo utilizando el algoritmo Geo-n con un MAE de 1.22 m en condiciones ideales y 2.44 m en el peor caso. La BS incrementó el error en más de 3 m en algunos casos, dependiendo de la orientación y la cantidad de nodos ancla visibles. Los autores recomiendan que los IPS implementen modelos de corrección adaptativos que consideren la posición del nodo móvil y el número de nodos ancla visibles, con el fin de mitigar los efectos de la BS, la pérdida de visibilidad de anclas y la orientación desfavorable del nodo móvil en escenarios reales.

    \item \textbf{Comparing Ubisense, Bespoon and Decawave UWB Location Systems: Indoor Performance Analysis} \cite{Falcone2012}: Este trabajo compara el desempeño de tres IPS distintos basados en tecnología UWB, desarrollados por los fabricantes Ubisense, Bespoon y Decawave. 
    
    Los autores realizaron un análisis experimental en un escenario industrial con condiciones de LOS y NLOS, evaluando la exactitud en la medición de distancias, el desempeño en la estimación de AoA y el desempeño en la localización tridimensional. Además, se empleó un Filtro de Partículas (\gls{pf}) para mitigar la incertidumbre en las medidas y mejorar la exactitud en la localización de los sistemas.

    El experimento se llevó a cabo en un escenario de 24 m $\times$ 14 m (336 m$^2$) dentro del Centro de Automatización y Robótica de España. Se instalaron seis nodos ancla a alturas de 2.32 m, 2.48 m y 2.60 m para el caso de Ubisense, Bespoon y Decawave, respectivamente. Se designaron 70 puntos de prueba en el área de evaluación, con tres repeticiones de mediciones por sistema. El nodo móvil se situó sobre un banco de pruebas a una altura de 0.76 m , 0.72 m y 0.85 m para Ubisense, Bespoon y  Decawave, respectivamente, con tiempos de medición de 30 segundos por punto y tres repeticiones por sistema, lo que equivale a un total de 2100 segundos de observación experimental por sistema. En la estimación de distancias, los resultados mostraron que Decawave presentó la mejor exactitud, con un error medio de 0.12 m y una desviación estándar de 0.232 m. Bespoon tuvo un desempeño ligeramente inferior, con un error medio de 0.43 m y una desviación estándar de 0.634 m. Ubisense, que emplea Diferencia de Tiempo de Llegada (TDoA, \textit{Time Difference of Arrival}), mostró el peor desempeño, con errores superiores a 10 m en múltiples casos y una desviación estándar de 10.6 m. Se observó que tanto Decawave como Bespoon presentan distribuciones de error simétricas con sesgo positivo debido a NLOS, mientras que Ubisense mostró errores más dispersos e inconsistentes.

    En la evaluación de exactitud de localización se utilizaron los datos de estimación de distancia para calcular la posición del nodo móvil mediante trilateración y PF. Decawave obtuvo la mejor exactitud, con un error de localización medio de 0.49 m, mediana de 0.39 m, RMS de 0.59 m y en el percentil 90 de 1.09 m. Bespoon presentó un error de localización medio de 0.71 m, mediana de 0.58 m y RMS de 0.86 m. Ubisense mostró el peor desempeño en la estimación de posición, con un error de localización medio de 1.93 m, RMS de 2.66 m y un percentil 90 de 5.17 m. Sin embargo, cuando se incorporó información de AoA, la exactitud de Ubisense mejoró significativamente, reduciendo el error de localización medio a 1.10 m y RMS a 1.82 m. El análisis de exactitud haciendo uso de información de ángulo reveló que Ubisense es el único sistema que proporciona estimaciones de AoA utilizando un arreglo de antenas en cada nodo. Se estimaron ángulos de azimut y elevación, obteniendo una exactitud de $\pm5.4^\circ$ en azimut y $\pm10.6^\circ$ en elevación, con una tasa de medición válida del 75.4\%, es decir, solo tres de cada cuatro estimaciones angulares fueron aceptadas como confiables. A pesar de la capacidad de estimación de ángulos, la variabilidad de las mediciones sugiere que el sistema requiere un procesamiento adicional para mejorar su confiabilidad.

    El estudio también incluyó una evaluación del impacto del NLOS y la propagación multitrayecto. Se observó que Decawave y Bespoon presentaron distribuciones de error con asimetrías significativas y presencia de valores atípicos, alcanzando errores máximos de hasta 1.2 m y 2 m, respectivamente, como resultado de obstrucciones físicas y trayectorias reflejadas. Por su parte, Ubisense mostró un comportamiento más errático con errores extremos que superaron los 30 m y una tasa de mediciones inválidas del 70\%. Para mitigar estos efectos, se utilizó un modelo de error basado en distribuciones gaussiana y gamma, incorporando el PF para mejorar la exactitud de la localización. En términos de desempeño en escenarios reales, se observó que Decawave y Bespoon son adecuados para IPS, mientras que Ubisense presentó mayores desafíos debido a su alta variabilidad en las estimaciones de distancia. Aunque la integración de AoA en Ubisense mejoró su desempeño, la cantidad de valores atípicos y la necesidad de infraestructura adicional para mejorar la sincronización limitaron su aplicabilidad en comparación con las soluciones basadas en ToF.

    Los autores concluyeron que los dispositivos Decawave ofrecieron el mejor desempeño en la estimación de distancias y la localización en escenarios de interiores, seguidos de Bespoon, que proporcionó una alternativa viable con un desempeño ligeramente inferior. Ubisense, a pesar de mejorar su desempeño al incluir información de AoA, mostró la mayor variabilidad en las estimaciones y la peor exactitud de localización, lo que sugiere que su implementación requiere procesamiento adicional mediante técnicas de filtrado estadístico o fusión de datos, con el fin de compensar su alta variabilidad y mejorar la confiabilidad de las estimaciones en escenarios industriales.

    \item \textbf{The Effects of the Human Body on UWB Signal Propagation in an Indoor Environment} \cite{ref15}: Este trabajo analiza el impacto del cuerpo humano en la propagación de señales UWB en escenarios de interiores. 
    
    Los autores realizaron mediciones experimentales en una cámara anecoica, un auditorio y una oficina, con el propósito de comparar el efecto de la BS en condiciones de baja y alta dispersión de señales. Se evaluaron tres aspectos clave: la atenuación de la señal, las variaciones en el patrón de radiación y la degradación de la calidad del enlace, con el fin de caracterizar la influencia del cuerpo humano en sistemas de comunicación UWB. 
    
    Las mediciones se realizaron con una antena PulseOn Diamond Dipole en el laboratorio de antenas de la Academia de la Fuerza Aérea de los Estados Unidos. En la cámara anecoica, la antena fue montada en un pedestal giratorio e iluminada con una antena Marconi 1804 log-periódica. Se llevaron a cabo mediciones a frecuencias de 2, 3 y 5 GHz con incrementos angulares de 5$^\circ$, cubriendo el patrón de radiación, es decir, la distribución espacial de la intensidad de la señal emitida por la antena en distintas direcciones. Se observó una asimetría en dicho patrón, con una desviación de $\pm2$ dB atribuida a la interacción con el cableado del sistema. Adicionalmente, se realizaron mediciones en el dominio del tiempo, alimentando la antena con un pulso de duración igual a 50 ps, y se analizó la CIR mediante la Transformada Rápida de Fourier (FFT, \textit{Fast Fourier Transform}). En el análisis espectral de esta respuesta se identificó una resonancia principal en 1.7 GHz, entendida como la frecuencia en la que la antena presenta su mayor eficiencia de radiación, es decir, donde transfiere la mayor parte de la energía emitida o recibida, revelando su comportamiento resonante natural ante pulsos ultra anchos en frecuencia. Aunque el sistema fue alimentado con un pulso extremadamente corto de 50 ps, la duración del impulso observado fue de aproximadamente 600 ps, según se reporta directamente en el artículo. Esta diferencia se debe a que la CIR incluye efectos del sistema como resonancia, dispersión y acoplamiento, que extienden la señal en el tiempo. Por tanto, la duración del impulso observado no refleja únicamente la señal de entrada, sino la respuesta completa del sistema bajo prueba.

    En la segunda fase del estudio, se realizaron mediciones en un auditorio, con el transmisor y el receptor separados 5 m y la antena receptora montada en la cintura del usuario. Para estas pruebas se utilizó un sistema con referencia Time Modulated Ultra-Wideband (TM-UWB) de la empresa PulseOn, que permitió obtener mediciones con resolución temporal de 1.1 ns, lo que posibilita una caracterización exacta de las trayectorias múltiples sin requerir un sistema basado en ToF. Los resultados mostraron que, en un escenario de baja dispersión multitrayecto, i.e., con escasa reflexión y baja densidad de trayectorias indirectas, la BS  generó una atenuación pronunciada al bloquear la señal LOS, lo que se manifestó como un nulo profundo de 23.6 dB en la región angular comprendida entre los 180$^\circ$ y los 210$^\circ$. Este valor corresponde a una reducción de 23.6 dB en la potencia de la señal recibida debido a la BS y no a una variación del patrón de radiación. Esta atenuación fue registrada directamente en el receptor, como una caída abrupta en la intensidad de la señal medida. Además, se observó que la magnitud del nulo dependía significativamente de la orientación relativa del cuerpo respecto a la antena receptora, evidenciando que el desempeño del sistema está estrechamente relacionado con la geometría y el posicionamiento del usuario. Aunque este sistema no emplea ToF ni estima AoA directamente, su alta resolución temporal permite analizar cómo varía la recepción en función de la dirección y la obstrucción, aportando información útil para evaluar condiciones de visibilidad entre nodos.

    En el tercer experimento realizado en un escenario de oficina con alto multitrayecto, i.e., alta dispersión de la señal, la separación entre transmisor y receptor se redujo a 3.14 m debido a las restricciones del espacio. Se observó que el impacto de la BS en la propagación de la señal se redujo significativamente en comparación con los resultados de las pruebas en el auditorio. La máxima atenuación de la señal fue de 6.8 dB, en contraste con los 23.6 dB en el auditorio. Esto sugiere que en entornos densos con multitrayecto, las reflexiones o ecos de la señal pueden compensar parcialmente la atenuación causada por la BS, mejorando la cobertura del sistema UWB.

    Los resultados indican que la BS puede generar atenuaciones de hasta 23.6 dB en escenarios de interiores con baja dispersión multitrayecto, como un auditorio, lo que afecta negativamente la calidad del enlace en condiciones NLOS. Sin embargo, en escenarios con alta densidad de objetos reflectantes, como oficinas, la atenuación fue considerablemente menor: 6.8 dB, debido a la contribución de trayectorias indirectas generadas por la reflexión de la señal. Esta diferencia se explica porque, en escenarios con alta dispersión, los ecos provenientes de múltiples trayectos pueden compensar parcialmente la atenuación de la señal directa. Los autores concluyen que los sistemas UWB pueden beneficiarse en este tipo de escenarios de alta dispersión por multitrayecto, lo que los hace adecuados para aplicaciones de comunicación en escenarios de interiores complejos.

    Los autores recomiendan que futuras investigaciones evalúen modelos de corrección adaptativa para compensar la atenuación sobre la señal causada por la BS en escenarios de baja dispersión. También sugieren analizar el impacto en diferentes configuraciones de antena y la posibilidad de implementar técnicas de diversidad espacial para mejorar la confiabilidad del enlace en presencia de BS. Cabe señalar que, según el estudio, el patrón de radiación de las antenas no se ve afectado directamente por la BS. La degradación del enlace ocurre por obstrucción de la señal directa, no por una modificación del patrón de radiación.

    \item \textbf{Effect of Human Body Shadowing on UWB Radio Channel} \cite{SasakiEffectHumanBody}: Este trabajo analiza el efecto de la BS en la propagación de señales UWB en Redes de Área Corporal (\gls{ban}). 
    
    Los autores realizaron mediciones experimentales en una cámara anecoica para evaluar la atenuación de la señal cuando un nodo móvil está ubicado en diferentes bolsillos del vestido, i.e., bolsillo derecho de la camisa, el bolsillo frontal del pantalón y el bolsillo trasero del pantalón, variando los ángulos de orientación del cuerpo con respecto al receptor. Los autores analizaron la atenuación de la señal en función del RHA, así como el impacto de la polarización cruzada, es decir, la pérdida de señal causada por la desalineación entre la polarización de las antenas en transmisión y recepción. Los autores compararon los resultados en diferentes posiciones del nodo móvil en el cuerpo y con distintos sujetos de prueba.

     Los autores utilizaron un VNA para obtener la CIR, con una antena de transmisión omnidireccional (monopolo) y una antena receptora tipo bocina situada a 3 m de distancia, a una altura de 0.8 m. Las pruebas se hicieron a una frecuencia central de 8 GHz y una potencia de transmisión de 0 dBm. La discriminación de polarización cruzada del transmisor fue de -17.3 dB a 8 GHz, según los datos de calibración presentados en el estudio. Este valor indica la diferencia de nivel entre la señal transmitida en la polarización principal y la que se recibe en polarización cruzada, reflejando así una pérdida por desalineación de polarización entre las antenas. Esta pérdida fue considerada en el análisis de propagación para evaluar su impacto en la atenuación de la señal bajo diferentes orientaciones del cuerpo.

    La atenuación de la señal debida a BS depende de la ubicación del nodo móvil sobre el cuerpo humano y su orientación con respecto a los nodos fijos. Para el caso del bolsillo de la camisa, se observaron mínimos de señal en ángulos cercanos a 150$^\circ$, mientras que para el bolsillo frontal del pantalón la atenuación máxima se produjo a 45$^\circ$. En el bolsillo trasero del pantalón, los niveles de señal fueron más variables, con fluctuaciones entre 10 y 16 dB dependiendo de la orientación del cuerpo. En algunos casos, la señal recibida en presencia de BS fue aproximadamente 3 dB mayor que en condiciones de espacio libre, debido a reflexiones favorables en la superficie del cuerpo humano, que redirigieron parte de la señal hacia la antena receptora.

    Se evaluó el efecto de la frecuencia en la propagación, es decir, cómo se afecta la transmisión y recepción de señales, comparando mediciones a 4 GHz, 8 GHz y 12 GHz, con un ancho de banda de 0.5 GHz. Se encontró que a 4 GHz, la señal varía de manera más uniforme; esto implica que la señal es más estable, mientras que a 8 GHz y 12 GHz, la atenuación se vuelve más dependiente del RHA. Con el incremento de la frecuencia, i.e., disminución de la longitud de onda, se fortalecen los fenómenos de difracción y dispersión de la señal.

    Los autores también compararon mediciones utilizando diferentes anchos de banda: 0.5 GHz y 3 GHz, ambas con una frecuencia central de 8 GHz. Se observó que con un ancho de banda de 0.5 GHz, los mínimos de señal se presentaron en 50$^\circ$ y 280$^\circ$, mientras que con 3 GHz la atenuación significativa ocurrió únicamente en 50$^\circ$. Esto indica que un mayor ancho de banda proporciona una mayor robustez frente a la BS, debido a la ganancia por diversidad de frecuencia, ya que la señal se distribuye en un rango más amplio en frecuencia. Además, el efecto del multitrayecto es menos severo con anchos de banda mayores, lo que favorece una propagación más estable y menos sensible a los desvanecimientos causados por la orientación del cuerpo.

    Para evaluar la variabilidad entre sujetos, los autores realizaron pruebas con 20 personas: 10 hombres y 10 mujeres. Se calculó la desviación estándar y la mediana de la potencia recibida, encontrando variaciones de 10 a 16 dB dependiendo de la orientación del sujeto. Sin embargo, no se identificaron diferencias estadísticamente significativas entre géneros.

    Los autores concluyeron que la BS afecta significativamente la propagación de señales UWB en BAN, generando atenuaciones que dependen tanto de la ubicación del nodo móvil sobre el cuerpo como de su orientación respecto al receptor. Se observó que las frecuencias más altas presentan una mayor atenuación angular, y que el efecto de multitrayecto, junto con fenómenos de difracción alrededor del cuerpo humano, puede permitir que parte de la señal rodee el cuerpo y contribuya al nivel de potencia de la señal recibida. Además, reflexiones en la superficie corporal pueden incrementar puntualmente la potencia recibida. Por ello, los autores recomiendan considerar cuidadosamente la posición de los nodos UWB en BAN y seleccionar apropiadamente tanto la frecuencia de operación como el ancho de banda, para mejorar la calidad del enlace frente a los efectos de la BS.
\end{enumerate}

\textbf{Contribución:}

Los estudios experimentales permiten una validación empírica de los modelos teóricos y ofrecen datos importantes sobre cómo la posición del nodo en el cuerpo humano afecta la exactitud de los IPS basados en UWB. Estos resultados no solo son esenciales para comprender el impacto de la interacción entre el cuerpo y la propagación UWB, sino que también proporcionan una base sólida para mejorar la exactitud de los IPS en aplicaciones reales, tales como el seguimiento de peatones.

\subsection{Estudios Basados en la Mitigación del Error}

En esta categoría se encuentran estudios que se enfocan en identificar y mitigar los errores introducidos por la BS en IPS basados en UWB. En estos estudios se hace procesamiento de señal para extraer y analizar las características de las señales UWB, las cuales alimentan algoritmos de ML y filtrado que buscan identificar y mitigar el impacto de las condiciones NLOS causadas por la BS sobre el desempeño de los sistemas de comunicación y localización.

\subsubsection{Referencias de la literatura considerada}


\begin{enumerate}
    %\setcounter{enumi}{6} % Continúa la numeración de la lista anterior
    \item \textbf{Feature Selection for Real-Time NLOS Identification and Mitigation for Body-Mounted UWB Transceivers} \cite{Ferreira2021}: Este artículo propone un enfoque basado en ML para identificar y mitigar en tiempo real las condiciones NLOS causadas por la BS en transceptores UWB operando a 3990 MHz, puestos en distintas posiciones del cuerpo. 
    
    Los autores extraen características como el rango ToA, la energía total, el número de trayectorias detectadas y parámetros estadísticos como la desviación estándar del retardo y la varianza de amplitud. Estas variables alimentan algoritmos de clasificación como Random Forest, Gradient Boosting, Máquinas de Vectores de Soporte (\gls{svm}) y K-Vecinos más Cercanos (\gls{knn}), los cuales permiten distinguir entre condiciones LOS y NLOS. Esta clasificación posibilita la mitigación del error de distancia en tiempo real, logrando una mejora significativa en la exactitud, especialmente en posiciones como el pecho y el muslo, donde la BS genera mayor afectación.
    
    \item \textbf{NLOS Identification and Mitigation for UWB Localization Systems} \cite{Guvenc}: Este estudio se centra en la identificación de condiciones NLOS en IPS basados en UWB, utilizando un rango de frecuencias no continuo entre 3 y 11 GHz.
    
    El análisis se enfoca en cómo las señales de UWB interactúan con el cuerpo humano y cómo dicha interacción afecta la estimación del ToF, lo que puede introducir errores en la localización. Para mitigar estos errores, los autores emplean un algoritmo de Mínimos Cuadrados Ponderados (WLS, \textit{Weighted least squares}), que asigna distintos pesos a cada observación según su confiabilidad. Este método ajusta la localización minimizando los errores en función de la calidad de los datos. Los resultados experimentales muestran que identificar correctamente las condiciones NLOS mejora sustancialmente la exactitud del sistema, especialmente en entornos con alta dispersión, como oficinas. El artículo sugiere que el algoritmo WLS puede ser especialmente útil en escenarios reales, donde la calidad de las señales puede variar considerablemente debido a la BS.
    
    \item \textbf{Non-Line-of-Sight Identification based on Unsupervised Machine Learning in Ultra Wideband Systems} \cite{Fan2019}: Aunque este estudio se clasifica también como simulación, es relevante en la mitigación del error NLOS mediante técnicas de ML no supervisado. El GMM permite clasificar automáticamente señales en condiciones LOS y NLOS, ayudando a mitigar el impacto de la BS en la exactitud de los sistemas de posicionamiento.

    \item \textbf{Effects of the Body Wearable Sensor Position on the UWB Localization Accuracy} \cite{ref4}: Este estudio no solo analiza cómo la ubicación de un nodo móvil en el cuerpo afecta la exactitud, sino que también implementa y valida una técnica de mitigación para mejorar la estimación de la localización. Los autores realizaron experimentos en un laboratorio, colocando un nodo UWB en diferentes partes del cuerpo (frente, pecho, muñeca, etc.) y registrando el error en condiciones LOS y NLOS. Para mitigar la incertidumbre y estimar la posición final, los autores emplearon un EKF. Los resultados, ya procesados por el EKF, mostraron que la exactitud y precisión dependen críticamente de la ubicación del nodo en el cuerpo humano: la frente fue la mejor posición, con un error medio de 0.20 m y un error en el percentil 90 (P90) de 0.35 m. En contraste, el pecho fue la peor ubicación, con un error medio de 2.46 m y de 4.04 m en el P90, incluso después del filtrado. La contribución de mitigación del error de este trabajo es la validación de un enfoque práctico basado en el EKF, demostrando que es una estrategia viable para mejorar la exactitud, evaluando su efectividad en función de la posición del dispositivo móvil  en el cuerpo.

    \item \textbf{The Effect of Human-Body Shadowing on Indoor UWB TOA-Based Ranging Systems} \cite{ref18}: Este trabajo investiga el impacto de la BS en sistemas de rango basados en ToA y, fundamentalmente, propone y valida una técnica de mitigación basada en el ajuste de parámetros del receptor. Los autores evaluaron la Raíz Cuadrática Media (\gls{rms}) en configuraciones de 4 m y 7 m de separación. Su contribución a la mitigación reside en cómo manejaron las condiciones NLOS: demostraron que un umbral de PT poco sensible, i.e., -70 dB, limita la capacidad del sistema para detectar las señales débiles difractadas alrededor del cuerpo, lo que resulta en un gran error i.e., hasta 46 cm. La técnica de mitigación propuesta consiste en reducir este umbral para hacerlo más sensible, i.e., -80 dB y -85 dB. Al hacerlo, el sistema fue capaz de detectar correctamente la primera señal difractada, reduciendo el error de localización drásticamente a 14 cm. Por lo tanto, el estudio concluye que una adecuada selección del umbral de detección es una poderosa y eficiente técnica de mitigación para minimizar el error en presencia de BS.

    \item \textbf{The Effects of Human Body Shadowing in RF-based Indoor Localization} \cite{Schimtt2019}: 
    Este artículo aborda el impacto de la BS en la localización en interiores y su principal contribución a la mitigación es la evaluación comparativa de diferentes algoritmos de localización para determinar cuál es más robusto frente a los errores inducidos por la BS. En un escenario práctico simulando un rescate en un edificio de oficinas con 14 nodos UWB, los autores compararon el rendimiento de cuatro algoritmos de mitigación: Centroid, mínimos cuadrados no lineales (\gls{nlls}), método geométrico multimencional min-max y método geométrico optimizado. Los resultados demostraron que la elección del algoritmo es crítica. Con todos los nodos activos, Geo-n fue el más efectivo, logrando un MAE de solo 1.22 m, mientras que NLLS presentó un error de 4.64 m. En el peor de los casos i.e., con solo 12 nodos exteriores activos, el método geométrico optimizado mantuvo su superioridad i.e., MAE de 2.44 m, mientras que el método multidimensional falló drásticamente i.e., MAE de 6.58 m. El estudio concluye que la selección de un algoritmo de localización robusto como el Geo-n es una estrategia de mitigación fundamental y recomienda la implementación de modelos de corrección adaptativos.

    \item \textbf{Comparing Ubisense, Bespoon and Decawave UWB Location Systems: Indoor Performance Analysis} \cite{Falcone2012}: Este trabajo compara el desempeño de tres sistemas comerciales de localización UWB en un entorno industrial realista, y su principal aporte a la mitigación es la implementación y validación de un PF para mejorar la exactitud en condiciones reales de LOS y NLOS. Los autores utilizaron los datos de distancia de cada sistema para alimentar el PF, el cual fue diseñado para manejar la incertidumbre y las distribuciones de error no gaussianas típicas del multitrayecto y la obstrucción. Para mejorar aún más la mitigación, el filtro incorporó un modelo de error estadístico basado en distribuciones gaussiana y gamma. Los resultados demostraron la efectividad de este enfoque: el sistema de Decawave, combinado con el PF, logró la mejor exactitud de localización, con un error medio de 0.49 m y una mediana de 0.39 m. La contribución de mitigación es, por tanto, la demostración de que una técnica de filtrado estadístico avanzado es esencial para compensar la alta variabilidad y mejorar la fiabilidad de los sistemas UWB comerciales en escenarios industriales complejos.
\end{enumerate}

\begin{figure}[ht]
    \centering
    \begin{tikzpicture}[
        % Estilo para los círculos de los conjuntos
        set/.style={circle, minimum size=5.5cm, fill opacity=0.3},
        % Estilo para las etiquetas de los conjuntos
        label/.style={font=\large\bfseries},
        % Estilo para las citas
        citation/.style={font=\bfseries}
    ]
        % --- Dibuja los círculos (conjuntos) ---
        % Círculo para "Estudios Experimentales" (izquierda)
        \node[set, fill=blue] (E) at (-2,0) {};
        
        % Círculo para "Estudios de Mitigación" (derecha)
        \node[set, fill=red] (M) at (2,0) {};

        % --- Coloca los títulos de los conjuntos ---
        \node[label] at ([yshift=0.7cm]E.north) {Experimentales};
        \node[label] at ([yshift=0.7cm]M.north) {Mitigación};
        
        % --- Coloca las citas en sus respectivas zonas ---

        % Zona 1: Solo Experimentales
        % Estas citas se colocan en la parte del círculo azul que no se solapa.
        \node[citation] at (-3.2, 1.5) {\cite{ref14}};
        \node[citation] at (-2.5, 0.7) {\cite{Tanghe2023}};
        \node[citation] at (-3.2, -0.1) {\cite{ref15}};
        \node[citation] at (-2.5, -0.9) {\cite{Pradabphon2019}};
        \node[citation] at (-3.2, -1.7) {\cite{SasakiEffectHumanBody}};


        % Zona 2: Intersección (Estudios Híbridos)
        % Estas citas se colocan en el área de solapamiento.
        \node[citation] at (-0.1, 1.2) {\cite{ref4}};
        \node[citation] at (-0.1, 0.4) {\cite{ref18}};
        \node[citation] at (-0.1, -0.4) {\cite{Schimtt2019}};
        \node[citation] at (-0.1, -1.2) {\cite{Falcone2012}};
        
        % Zona 3: Solo Mitigación
        % Estas citas se colocan en la parte del círculo rojo que no se solapa.
        \node[citation] at (3, 1.2) {\cite{Ferreira2021}};
        \node[citation] at (2.5, 0) {\cite{Guvenc}};
        \node[citation] at (3, -1.2) {\cite{Fan2019}};

    \end{tikzpicture}
    \caption{Clasificación de los Artículos Analizados en Estudios Experimentales, de Mitigación y Estudios Híbridos que Combinan Ambas Metodologías.}
    \label{fig:venn_clasificacion}
\end{figure}

\textbf{Contribución}\\
La contribución fundamental de los estudios de mitigación es el desarrollo y la validación de un diverso ecosistema de técnicas capaces de contrarrestar el error inducido por la BS. Una característica destacada de este grupo de trabajos, como se observa en la figura \ref{fig:venn_clasificacion} es que no son puramente teóricos; la gran mayoría valida sus propuestas a través de implementaciones experimentales, demostrando su aplicabilidad en el mundo real. Estas estrategias de mitigación se pueden agrupar en tres enfoques principales: 1) Clasificación de Canal basada en Machine Learning, donde se emplean algoritmos como SVM, k-NN \cite{ref7} o GMM \cite{ref20} para identificar condiciones LOS/NLOS en tiempo real; 2) Filtrado Estadístico y Estimación de Estados, que utiliza técnicas como Mínimos Cuadrados Ponderados \cite{ref11}, el EKF \cite{ref4} o PF Bayesianos \cite{Falcone2012} para corregir y suavizar las mediciones; y 3) Optimización a Nivel de Sistema, que se enfoca en el ajuste de parámetros del receptor como el PT \cite{ref18} o en la selección del algoritmo de localización más robusto, como Geo-n \cite{ref19}. En conjunto, estos estudios establecen que la mitigación del error por BS es un problema multifacético que puede ser abordado desde el procesamiento de la señal hasta la fusión de datos, proporcionando una caja de herramientas validada experimentalmente para el desarrollo de IPS más precisos y fiables.



\section{ANÁLISIS DE LA REVISIÓN SISTEMÁTICA DE LITERATURA}
\label{sec:analisis_estado_arte}

\subsection{Metodología de la  Revisión}

Para garantizar un proceso de revisión de literatura transparente, riguroso y reproducible, se adoptó la metodología PRISMA (\textit{Preferred Reporting Items for Systematic Reviews and Meta-Analyses}) \cite{PRISMA2020}. PRISMA proporciona una guía basada en evidencia para estructurar la revisión en cuatro fases: Identificación, Cribado (\textit{Screening}), Elegibilidad e Inclusión. Este enfoque documenta el flujo de información desde la búsqueda exhaustiva hasta la selección final de los estudios que conforman el núcleo de este análisis.

\begin{figure}[ht]
    \centering
    \begin{tikzpicture}[
        node distance=1.5cm,
        box/.style={
            rectangle, 
            draw, 
            thick, 
            fill={rgb,255:red,51;green,153;blue,255},
            text=white,
            text width=10cm, 
            minimum height=1.5cm, 
            align=center,
            drop shadow
        },
        arrow/.style={
            -{Stealth[length=3mm]},
            thick
        }
    ]
        % --- Nodos del diagrama ---
        \node[box] (identification) {
            \textbf{Identificación: Registros identificados mediante búsqueda en bases de datos} \\
            (Scopus, WoS) \\
            \textbf{(n = 208)}
        };

        \node[box, below=of identification] (screening) {
            \textbf{Cribado (Screening): Registros únicos tras eliminar duplicados} \\
            \textbf{(n = 204)}
        };
        
        \node[box, below=of screening] (eligibility) {
            \textbf{Elegibilidad: Artículos evaluados a texto completo} \\
            \textbf{(n = 84)}
        };
        
        \node[box, below=of eligibility] (included) {
            \textbf{Inclusión: Estudios incluidos en la síntesis cualitativa} \\
            \textbf{(n = 20)}
        };

        % --- Flechas y texto de exclusión ---
        \draw[arrow] (identification) -- (screening);
        \node[right=0.5cm of screening, text width=4cm, align=left] (dup_removed) {
            Registros duplicados eliminados \\
            \textbf{(n = 4)}
        };
        \draw[arrow] (screening.east) -> (dup_removed.west);
        
        \draw[arrow] (screening) -- (eligibility);
        \node[right=0.5cm of eligibility, text width=4cm, align=left] (screen_excluded) {
            Registros excluidos por título y resumen (no relevantes) \\
            \textbf{(n = 120)}
        };
        \draw[arrow] (eligibility.east) ->(screen_excluded.west);
        
        \draw[arrow] (eligibility) -- (included);
        \node[right=0.5cm of included, text width=4cm, align=left] (fulltext_excluded) {
            Artículos excluidos tras revisión de texto completo \\
            \textbf{(n = 64)} \\
        };
        \draw[arrow] (included.east) -> (fulltext_excluded.west);

    \end{tikzpicture}
    \caption{Diagrama de Flujo PRISMA del Proceso de Revisión Sistemática de Literatura.}
    \label{fig:prisma_flow}
\end{figure}

\subsubsection{Fase de Identificación}
El objetivo de esta fase fue encontrar la mayor cantidad posible de literatura relevante. Se realizaron búsquedas sistemáticas en cuatro bases de datos académicas clave para el campo de la ingeniería y las telecomunicaciones: Scopus y Web of Science (WoS). Se utilizó una combinación de las siguientes palabras clave: UWB localization, Body Shadowing, Ultra-Wideband Signal Propagation e Indoor Positioning Systems. Las bases de datos arrojaron 107 registros y WoS, con 101 registros. Contando todas las fuentes, la búsqueda inicial identificó un total de 208 registros. En esta fase se filtraron los 208 registros. Primero, utilizando software de gestión de referencias Mendeley, se identificaron y eliminaron 4 registros duplicados, quedando así 204 artículos para la fase de cribado o screening. 

\subsubsection{Fase de Cribado (Screening)}
 Los 204 registros únicos restantes fueron sometidos a un cribado basado en título y resumen para excluir aquellos que eran claramente irrelevantes. Se excluyeron 120 artículos en esta etapa por no estar centrados en UWB como tecnología principal, por tratar con aplicaciones distintas a la localización o por enfocarse en escenarios no aplicables.

\subsubsection{Fase de Elegibilidad}
Los 84 artículos que superaron la fase de cribado fueron sometidos a una revisión de texto completo para determinar su elegibilidad final. En esta etapa, se evaluaron en profundidad la metodología, los resultados y el análisis de cada estudio. Se excluyeron 27 artículos adicionales por razones como:
\begin{itemize}
    \item Falta de datos cuantitativos sobre el efecto de la BS.
    \item Metodología experimental no reproducible o insuficientemente descrita.
    \item Enfoque principal en diseño de antenas en lugar de en sistemas de posicionamiento.
    \item Ser resúmenes de conferencias sin un artículo completo disponible.
\end{itemize}

\subsubsection{Fase de Inclusión}
El proceso culminó con la selección de 20 artículos que cumplieron con todos los criterios de inclusión. Estos 20 estudios constituyen la base de la síntesis cualitativa y el análisis del estado del arte presentados en este capítulo. El flujo completo de este proceso se visualiza en el diagrama PRISMA de la Figura \ref{fig:prisma_flow}.
\label{subsec:contexto_analisis}



\subsection{Distribución de Estudios por Tipo de Investigación}
\label{subsec:distribucion_tipos}
%
\begin{figure}[ht]
    \centering
    \includegraphics[width=1.2\textwidth]{imagenes/clasificacion_tipo_investigacion.pdf}
    \caption{Distribución de 20 Artículos por Tipo de Investigación}
    \label{fig:distribucion_tipos}
\end{figure}
%
La Figura \ref{fig:distribucion_tipos} muestra que el 45\% de los estudios son experimentales (9 artículos), seguidos por 35\% de estudios de mitigación (7 artículos) y 20\% de simulación (4 artículos). Esta distribución revela una tendencia marcada hacia la validación práctica de los efectos de la BS, lo cual es fundamental para el desarrollo de sistemas comerciales viables. Los estudios experimentales predominantes incluyen trabajos como \cite{ref4} y \cite{ref14}, que evalúan directamente el impacto de diferentes posiciones del nodo móvil en el cuerpo humano en escenarios controlados.

La proporción relativamente baja de estudios de simulación (20\%) presenta tanto ventajas como desafíos. Por un lado, indica que la comunidad científica prioriza la validación empírica sobre modelos teóricos. Por otro lado, la escasez de simulaciones sofisticadas podría limitar la exploración de escenarios complejos o frecuencias no disponibles comercialmente, como es precisamente el caso de la banda de 6.5 GHz que constituye el foco de esta investigación.

Es importante destacar que esta distribución también refleja la madurez tecnológica del campo. En las etapas iniciales de una tecnología, los estudios de simulación suelen dominar debido a la falta de hardware disponible. Sin embargo, la predominancia actual de estudios experimentales sugiere que la tecnología UWB ha alcanzado un nivel de madurez que permite validación directa, aunque, paradójicamente, esta madurez se concentra principalmente en las bandas de frecuencia inferiores.

\subsection{Frecuencias de Operación Utilizadas en los Estudios}
\label{subsec:frecuencias_operacion}

\begin{figure}[ht]
    \centering
    \includegraphics[width=0.8\textwidth]{imagenes/frecuencias_por_anio.pdf}
    \caption{Distribución de Estudios por Rango de Frecuencia}
    \label{fig:frecuencias}
\end{figure}

La Figura \ref{fig:frecuencias} presenta una concentración abrumadora de investigaciones en las bandas inferiores del espectro UWB. Específicamente, el 70\% de los estudios, i.e., 14 artículos, operan en el rango de 3-5 GHz, mientras que solo el 5\% i.e., 1 artículo, explora frecuencias entre 6 y 7 GHz. Esta distribución no es casual, sino que responde a factores históricos y tecnológicos: la banda de 3.1-5 GHz fue la primera autorizada por la FCC para aplicaciones UWB comerciales, y consecuentemente, la mayoría del hardware disponible opera en este rango.

Solo el 5\% de los estudios analizados operan en frecuencias superiores a 6 GHz. La banda de 6-6.5 GHz permanece prácticamente inexplorada en el contexto de la BS, lo que justifica plenamente el objetivo de esta investigación de analizar el desempeño de sistemas de posicionamiento en 6.5 GHz. Esta frecuencia ofrece ventajas potenciales en términos de resolución espacial mejorada y menor interferencia con otros sistemas de comunicaciones.

La concentración de sistemas en frecuencias bajas tiene implicaciones significativas para el desarrollo futuro de la tecnología. Las frecuencias más altas ofrecen ventajas teóricas importantes, tales como:
%
\begin{itemize}
    \item \textbf{Mayor resolución espacial:} La principal ventaja de UWB es su alta resolución espacial, i.e., la distancia mínima que el sistema puede resolver es del orden de los centímetros, que está determinada por su alta resolución temporal, i.e., pulsos muy angostos del orden de los nanosegundos y que implican un ancho de banda del orden de los 500 MHz. Esta capacidad permite al receptor UWB distinguir con gran exactitud la señal que llega por el primer trayecto o camino directo de los ecos de la misma señal que llegan nanosegundos después. Los sistemas que operan en la banda de 6.5 GHz pueden alcanzar una resolución espacial del orden de centímetros, lo que es fundamental para la localización de alta exactitud y precisión.    
    
    \item \textbf{Menor tamaño de antena:} Las dimensiones de las antenas son proporcionales a la longitud de onda. A 6.5 GHz, las antenas pueden ser aproximadamente 35\% más pequeñas que a 4 GHz, facilitando la integración en dispositivos portátiles.
    
    \item \textbf{Diferentes características de propagación:} Las frecuencias más altas alteran fundamentalmente la interacción de la señal con el entorno. Por un lado, se ha observado que frecuencias mayores, como 8 y 12 GHz, experimentan una atenuación más pronunciada y dependiente del RHA al interactuar con el cuerpo humano en comparación con frecuencias más bajas como 4 GHz \cite{ref19}. Por otro lado, estas mismas frecuencias exhiben patrones de difracción más complejos. Como demuestran estudios en entornos de alta dispersión (como oficinas), estos fenómenos de multitrayecto y difracción pueden ser explotados, ya que los ecos y las señales que rodean el cuerpo pueden compensar la pérdida de la señal directa, mejorando la robustez y la cobertura general del sistema \cite{ref15, ref18}.
\end{itemize}

\subsection{Posiciones del Cuerpo Evaluadas}
\label{subsec:posiciones_cuerpo}
%
\begin{figure}[ht]
    \centering  \includegraphics[width=0.8\textwidth]{imagenes/posiciones_cuerpo.png}
    \caption{Distribución de Estudios por Posición del Nodo Móvil en el Cuerpo Humano}
    \label{fig:posiciones}
\end{figure}
%
La Figura \ref{fig:posiciones} presenta un diagrama radar que ilustra la frecuencia con la que diferentes posiciones corporales han sido evaluadas en la literatura. El pecho y la muñeca emergen como las ubicaciones más estudiadas i.e., 18\% cada una, seguidas por la frente i.e., 15\%. Esta preferencia se debe a consideraciones prácticas: el pecho representa el peor escenario en términos de BS, con errores de localización reportados de hasta 4.5 $m$ en condiciones NLOS \cite{ref4}, mientras que la muñeca simula dispositivos \textit{wearables} comunes como relojes inteligentes.

El análisis detallado de las posiciones revela patrones interesantes:
%
\begin{itemize}
    \item \textbf{Pecho (18\%):} Representa el caso más desafiante debido a la máxima BS. Los estudios como \cite{ref4} reportan errores de hasta 4.5 m en condiciones NLOS cuando el nodo se ubica en esta posición.
    
    \item \textbf{Muñeca (18\%):} Popular debido a la proliferación de dispositivos \textit{wearables}. Sin embargo, esta posición presenta alta variabilidad debido al movimiento natural del brazo durante la marcha.
    
    \item \textbf{Frente (15\%):} Consistentemente muestra el mejor desempeño, con errores típicos de 20 cm en LOS según \cite{ref14}. La forma curva de la cabeza favorece la difracción de señales UWB.
    
    \item \textbf{Posiciones subrepresentadas:} El brazo (8\%), muslo (7\%) y espalda (2\%) han recibido atención marginal, a pesar de su relevancia en aplicaciones específicas.
\end{itemize}

Esta distribución heterogénea sugiere la necesidad imperante de estudios exhaustivos e integrales que evalúen sistemáticamente múltiples posiciones del nodo móvil en el cuerpo humano bajo las mismas condiciones experimentales, permitiendo comparaciones directas y la identificación de patrones.

\subsection{Técnicas de Mitigación Empleadas}
\label{subsec:tecnicas_mitigacion}
%
\begin{figure}[ht]
    \centering
    \includegraphics[width=1.2\textwidth]{imagenes/tecnicas_mitigacion.pdf}
    \caption{Técnicas de Mitigación de Error NLOS}
    \label{fig:mitigacion}
\end{figure}
%
La Figura \ref{fig:mitigacion} ilustra mediante un gráfico polar la distribución de técnicas empleadas para mitigar los efectos de la BS. El análisis revela tendencias significativas en los enfoques adoptados:

\subsubsection{Aprendizaje Automático (43\%)}

Las técnicas de ML dominan el panorama con 6 estudios, reflejando la tendencia actual hacia soluciones basadas en datos. Estas técnicas incluyen:

\begin{itemize}
    \item \textbf{Clasificadores supervisados:} Random Forest, SVM y Gradient Boosting, como en \cite{ref7}, que logran tasas de clasificación LOS/NLOS superiores al 85\%.
    
    \item \textbf{Aprendizaje no supervisado:} GMM con algoritmo EM, como en \cite{ref20}, que permite identificación de NLOS sin etiquetado manual previo.
    
    \item \textbf{Redes neuronales:} Aunque menos comunes, algunos estudios exploran arquitecturas profundas para capturar relaciones no lineales complejas.
\end{itemize}

\subsubsection{Filtros Estadísticos (43\%)}

Representan soluciones más tradicionales pero efectivas:\\

\textbf{A. Filtros de ponderación y partículas}
\begin{itemize}
    \item \textbf{Mínimos Cuadrados Ponderados (WLS):} Como en \cite{ref11}, asignan pesos según la confiabilidad de cada medición.
    
    \item \textbf{Filtros de partículas (29\%)} Utilizados en \cite{Falcone2012} para manejar distribuciones no gaussianas en NLOS.
    
    \item \textbf{Estimadores robustos:} son técnicas estadísticas avanzadas que se utilizan para mejorar la exactitud de los sistemas de posicionamiento UWB. Su función principal es identificar y minimizar el impacto de mediciones extremadamente erróneas (outliers).
\end{itemize}


\textbf{B. Filtro de Kalman (14\%)}\\

Sorprendentemente subutilizado considerando sus ventajas:

\begin{itemize}
    \item Eficiencia computacional superior a la de técnicas de ML.
    \item Capacidad de fusión de datos multinodo.
    \item Predicción de estados futuros útil para seguimiento  continuo.
    \item Implementación madura y bien documentada.
\end{itemize}

Esta subutilización presenta una oportunidad significativa, especialmente para sistemas embebidos con recursos limitados operando a 6.5 GHz.

\subsection{Evolución Temporal de las Investigaciones (2019-2024)}
\label{subsec:evolucion_temporal}
%
\begin{figure}[ht]
    \centering
    \includegraphics[width=1.1\textwidth]{imagenes/evolucion_temporal.pdf}
    \caption{Evolución Temporal de Investigaciones sobre Obstrucción Corporal en UWB}
    \label{fig:temporal}
\end{figure}
%
La evolución temporal presentada en la Figura \ref{fig:temporal} revela patrones interesantes que merecen un análisis detallado:

\subsubsection{Fase de Crecimiento (2019-2022)}

El período 2019-2022 muestra un crecimiento exponencial, con factores catalizadores clave:

\begin{itemize}
    \item \textbf{2019:} Apple introduce el chip U1 en iPhone 11, democratizando el acceso a tecnología UWB.
    \item \textbf{2020:} La pandemia de COVID-19 impulsa la demanda de soluciones de distanciamiento social basadas en localización exacta.
    \item \textbf{2021:} Estandarización IEEE 802.15.4z mejora la seguridad y exactitud en la estimación de distancia de  UWB.
    \item \textbf{2022:} Pico de investigación con 6 estudios, incluyendo primeras exploraciones en frecuencias superiores.
\end{itemize}

\subsubsection{Fase de Consolidación (2023-2024)}

La aparente disminución no indica pérdida de interés, sino una transición hacia:

\begin{itemize}
    \item Investigación más especializada y profunda.
    \item Desarrollo de productos comerciales basados en hallazgos previos.
    \item Exploración de nuevas bandas de frecuencia, como 6.5 GHz.
    \item Integración con otras tecnologías, tales como 5G y WiFi 6E.
\end{itemize}



%%________

\section{ANÁLISIS Y DISCUSIÓN DE LOS HALLAZGOS DE LA REVISIÓN SISTEMÁTICA DE LITERATURA}
\label{sec:analisis_discusion}

\subsection{Síntesis de Hallazgos Principales}
\label{subsec:sintesis_hallazgos}

El análisis de la revisión sistemática de literatura sobre BS en IPS basados en UWB revela un campo de investigación con desarrollo asimétrico. Mientras que las frecuencias entre 3 y 5 GHz han sido exhaustivamente caracterizadas, existe un vacío significativo en el conocimiento sobre el comportamiento de las señales UWB en frecuencias superiores a 6 GHz. Esta disparidad no es meramente académica; tiene implicaciones profundas para el desarrollo futuro de sistemas de localización más exactos, precisos y robustos.

\subsection{Impacto Cuantificado de la Obstrucción Corporal}
\label{subsec:impacto_bs_cuantificado}

Los estudios de simulación han establecido que la BS introduce atenuaciones que varían entre 12 y 20 dB en condiciones de NLOS, con errores de posicionamiento que pueden alcanzar los 60 cm en los peores escenarios. El trabajo en \cite{ref17} utilizando FDTD reveló un fenómeno particularmente intrigante: en condiciones NLOS severas, se observan aparentes "ganancias" de señal de hasta 18 dB debido a la interferencia constructiva entre componentes multitrayecto, principalmente causadas por difracción alrededor del cuerpo humano. Este hallazgo desafía las concepciones simplistas sobre la propagación de señales y subraya la complejidad del fenómeno.

La validación experimental de estos modelos ha confirmado que la posición del nodo en el cuerpo humano es crítica para el desempeño de un sistema de localización. La frente emerge consistentemente como la ubicación óptima, con errores típicos de 20 cm en condiciones LOS que se degradan solo hasta 35 cm en NLOS, representando un incremento del 75\%. En contraste, nodos ubicados en el pecho pueden experimentar degradaciones de hasta 900\%, con errores que saltan de 45 cm en LOS a 450 cm en NLOS. Esta variabilidad extrema entre ubicaciones corporales sugiere que no existe una solución única para todas las aplicaciones, sino que cada caso de uso requiere una evaluación cuidadosa de la ubicación óptima del nodo.

\subsection{Evolución de las Técnicas de Mitigación}
\label{subsec:evolucion_mitigacion}

El panorama de las técnicas de mitigación está dominado por el ML, empleado en el 43\% de los estudios analizados. Algoritmos como Random Forest y SVM han demostrado capacidades impresionantes, alcanzando exactitudes de clasificación LOS/NLOS superiores al 90\%. Sin embargo, esta mejora conlleva costos computacionales significativos que pueden ser prohibitivos para sistemas embebidos con recursos limitados. Además, estos modelos entrenados en frecuencias específicas no son fácilmente transferibles a nuevas bandas de operación, lo que representa un desafío particular para la exploración de la banda de 6.5 GHz.

Los filtros estadísticos tradicionales, aunque menos glamorosos, ofrecen un balance pragmático entre efectividad y eficiencia computacional. El método de Mínimos Cuadrados Ponderados (WLS) ha demostrado reducciones del error RMSE de hasta 45\% con complejidad computacional lineal, siendo atractivo para implementaciones en tiempo real. Sorprendentemente, el EKF, a pesar de su probada eficacia en sistemas de navegación durante décadas, representa solo el 14\% de las implementaciones en la literatura revisada. Esta subutilización presenta una oportunidad significativa, especialmente considerando que el KF ofrece capacidades nativas de predicción de estados y fusión multinodo que son ideales para el seguimiento continuo de personas en movimiento.

\subsection{La Promesa Inexplorada de 6.5 GHz}
\label{subsec:promesa_65ghz}

La migración hacia frecuencias superiores no es simplemente una progresión incremental; representa un cambio cualitativo en las características de propagación que podría revolucionar la mitigación de la BS. A 6.5 GHz, la longitud de onda se reduce a 4.6 cm comparada con 7.5 cm a 4 GHz, lo que implica una mejora del 38.3\% en la resolución espacial teórica. Esta mejora en resolución permite una discriminación más exacta entre componentes de señal directa y multitrayecto, facilitando la identificación del primer pico de señal incluso en condiciones desafiantes.

Paradójicamente, la mayor atenuación corporal esperada a 6.5 GHz podría ser ventajosa para la detección de la condición NLOS. Con una atenuación teórica 30\% mayor que a 4 GHz, las señales que atraviesan el cuerpo serán más claramente distinguibles de aquellas que lo rodean por difracción, simplificando potencialmente los algoritmos de clasificación LOS/NLOS. Además, el coeficiente de difracción mejorado en un 8.2\% sugiere que las señales podrán ``envolver'' el cuerpo más efectivamente, reduciendo las zonas muertas y mejorando la cobertura general del sistema.

\subsection{Limitaciones Sistemáticas en la Literatura Actual}
\label{subsec:limitaciones_sistematicas}

La consideración hacia frecuencias bajas en la literatura no es accidental sino estructural, derivada de la disponibilidad histórica de hardware y la inercia regulatoria. Esta consideración ha creado un círculo vicioso donde la falta de datos experimentales en frecuencias superiores desalienta el desarrollo de hardware, que a su vez limita la investigación experimental. El 95\% de los estudios concentrados en la banda 3-5 GHz han creado una base de conocimiento robusta pero estrecha de conocimiento que no puede extrapolarse confiablemente a frecuencias superiores debido a las diferencias fundamentales en los mecanismos de propagación.

La fragmentación metodológica representa otra limitación crítica. Los estudios típicamente evalúan solo 2-3 posiciones del nodo móvil en el cuerpo humano, como se muestra en la sección \ref{subsec:posiciones_cuerpo}. La mayoría de los estudios se limitan a estudiar típicamente una combinación de pecho, muñeca y frente utilizando protocolos experimentales divergentes, lo que hace imposibles las comparaciones directas. Esta fragmentación se extiende a los escenarios de prueba, donde el 70\% de los experimentos se realizan en laboratorios controlados que no capturan la complejidad de entornos reales con múltiples personas, mobiliario metálico y fuentes de interferencia electromagnética.

\subsection{Oportunidades de Innovación y Contribución}
\label{subsec:oportunidades_contribucion}

Esta investigación se posiciona estratégicamente para abordar estas limitaciones mediante un enfoque metodológico. La evaluación sistemática de seis posiciones del nodo móvil en el cuerpo humano bajo condiciones experimentales idénticas en un aula universitaria real proporcionará el primer conjunto de datos verdaderamente comparable para análisis cruzado. Más allá de llenar brechas de datos, este trabajo establecerá un protocolo experimental reproducible que podrá servir como estándar de facto para futuras investigaciones.

La caracterización estadística del error de posicionamiento por BS en un IPS basado en UWB a 6.5 GHz  representará una contribución fundamental al campo. Esta caracterización no solo incluirá las distribuciones de probabilidad para errores de posicionamiento en condiciones LOS y NLOS. La validación experimental en un escenario real con alta densidad de ocupación y complejidad electromagnética asegurará que los resultados sean directamente aplicables al desarrollo de productos comerciales.

\subsection{Implicaciones para el Futuro del Campo de Investigación}
\label{subsec:implicaciones_futuro}

El impacto anticipado de esta investigación se extiende más allá de las contribuciones técnicas inmediatas. En el corto plazo, los resultados obtenidos permitirán a la academia e industria tomar decisiones informadas sobre aplicaciones e inversiones en tecnología de 6.5 GHz. La caracterización del comportamiento del sistema de posicionamiento favorecerá el desarrollo de chips, tarjetas y reducirá el tiempo de llegada al mercado de productos de próxima generación. En el mediano plazo, se espera que estos hallazgos influyan en la evolución de estándares como IEEE 802.15.4ab, proporcionando resultados empíricos base para decisiones de diseño de protocolos y aplicaciones.

A largo plazo, esta investigación podría catalizar un cambio de paradigma del concepto de los IPS. La demostración de ventajas significativas a 6.5 GHz podría impulsar la exploración de frecuencias aún más altas, incluyendo las bandas milimétricas donde la exactitud subcentimétrica podría ser alcanzable. Además, la metodología desarrollada para caracterización sistemática de la BS podría aplicarse a otras tecnologías de localización emergentes, desde WiFi 6E hasta las futuras redes 6G.

\subsection{Conclusión}
\label{subsec:conclusion_analisis}

El estado actual de la literatura revela un campo maduro en ciertos aspectos, pero con oportunidades significativas sin explotar. La concentración de investigación en frecuencias bajas ha creado una base sólida de conocimiento que ahora debe expandirse hacia nuevas bandas de frecuencia. La banda de 6.5 GHz representa no solo una frontera técnica sino una oportunidad para repensar fundamentalmente cómo se aborda el desafío de la BS en los IPS basados en UWB. 

Esta investigación se posiciona en la intersección crítica entre la necesidad académica de expandir el conocimiento fundamental y la demanda industrial de soluciones prácticas para sistemas de próxima generación. Al abordar sistemáticamente las limitaciones identificadas en la literatura actual y explorar las oportunidades únicas de la banda de 6.5 GHz, este trabajo no solo llenará brechas existentes sino que establecerá nuevas direcciones para la investigación futura en IPS basados en UWB.

%%________

% Capítulo 6: Aportes Investigativos

\chapter{METODOLOGÍA Y ACTIVIDADES}
\label{ch:metodologia}

\section{METODOLOGÍA}
\label{sec:metodologia_propuesta}
Para el desarrollo del presente trabajo de grado, se adoptará la metodología del Modelo en V, un enfoque riguroso derivado del modelo en cascada que enfatiza la relación entre cada fase de desarrollo y su correspondiente fase de pruebas \cite{forsberg1991relationship}. La elección de este modelo responde a la necesidad de una validación continua en un proyecto de carácter técnico y experimental. A diferencia de un modelo puramente secuencial, el Modelo en V establece un paralelismo entre la construcción y la validación, lo que garantiza que los objetivos definidos en las primeras etapas sean verificados sistemáticamente en las últimas.
Este modelo es especialmente adecuado para este proyecto, ya que asegura que el diseño del sistema de posicionamiento, la configuración del hardware y la implementación del software sean validados mediante pruebas específicas, minimizando errores y garantizando la fiabilidad de los resultados. El proceso se visualiza como una "V", donde el brazo izquierdo representa la descomposición del problema y el diseño e implementación, y el brazo derecho representa la integración y validación de los componentes del sistema.
El flujo de trabajo se estructurará siguiendo las fases del Modelo en V, como se muestra en la Figura  \ref{fig:modelV}, adaptadas a los objetivos de esta investigación. 
%
\begin{figure}[ht]
    \centering
    \includegraphics[width=0.8\textwidth]{imagenes/ModelV.pdf}
    \caption{Modelo en V}
    \label{fig:modelV}
\end{figure}

\subsection*{Brazo Izquierdo: Definición del proyecto}

En esta trayectoria descendente, se definen y diseñan los componentes del proyecto.
\begin{enumerate}
\item \textbf{Análisis de Requerimientos:} Se establecen los objetivos fundamentales de la investigación: analizar cómo el efecto de la BS impacta las mediciones de ToF en un sistema UWB, cuantificar el error en la estimación de distancia y utilizar el KF para mitigar el efecto de la BS sobre la exactitud del sistema de posicionamiento. El resultado de esta fase es la especificación detallada de los objetivos, los requerimientos y los criterios de aceptación de los resultados.

\item \textbf{Diseño del Sistema:} Se diseña la arquitectura general del experimento. Esto incluye la definición del escenario de prueba y del modelo general del sistema, la selección y justificación del hardware y software, y la especificación de los parámetros de operación del sistema UWB (frecuencia, ancho de banda).

\end{enumerate}

\subsection*{Brazo Derecho: Pruebas y Resultados}
En esta trayectoria ascendente, cada entregable de la fase de diseño es probado y validado.
\begin{enumerate}

\item \textbf{Pruebas Unitarias:} Corresponden a la fase de implementación. Se verifica que cada componente de software y hardware funcione correctamente de forma aislada.

\item \textbf{Integración:} Se integran todos los componentes de software y hardware en un único sistema que es capaz de recolectar los datos en cada posición del TAG en el cuerpo humano y en cada posición (x,y) dentro del escenario. 

\item \textbf{Pruebas de Desempeño y Validación de Resultados:} Es la fase cumbre y corresponde a la recolección y análisis de los datos. Se ejecuta el experimento completo en el escenario definido. Se  recolectan y se analizan los datos. Se verifica si el sistema desarrollado permite responder a las preguntas de investigación, requerimientos planteados y se relacionan los resultados con los de otros trabajos.

\end{enumerate}

Este enfoque metodológico garantiza que cada etapa de diseño sea validada por una fase de prueba correspondiente, lo que asegura la trazabilidad, minimiza la propagación de errores y confiere una alta fiabilidad y reproducibilidad a los resultados experimentales, aspectos cruciales en una investigación científica rigurosa.



\subsection{Diseño Experimental y Protocolo de Recolección de Datos}

\subsubsection{Población y Muestra}
Para garantizar la generalización de los resultados, el estudio contará con un mínimo de 3 participantes con diversidad de complexión física (estatura y peso). Cada participante realizará el recorrido predefinido con el nodo móvil en la peor posición en NLOS en interiores arrojada por el plan de pruebas de validación.

\subsubsection{Entorno de Prueba Detallado}
Las pruebas se realizarán en un salón de clases de la universidad del Cauca. Se incluirá un mapa del entorno, señalando la posición de los nodos ancla (hardware y altura), el mobiliario principal y las fuentes potenciales de interferencia. Se realizará una calibración inicial del sistema sin obstrucción para establecer una línea base.

\subsubsection{Técnicas de Aumentación de Datos}
Dada la dificultad de reclutar un número masivo de participantes, se explorarán técnicas de aumentación de datos, como la introducción de pequeños desplazamientos aleatorios en las trayectorias o la simulación de diferentes complexiones corporales, para enriquecer el conjunto de datos y mejorar la robustez de los modelos estadísticos \textcolor{red}{Esto se haría basado en simulaciones monte carlo}.

\section{ACTIVIDADES}
\label{sec:actividades}

Las actividades del proyecto se estructuran en fases que se corresponden con la progresión a través del Modelo en V, asegurando que cada etapa de diseño sea seguida por una etapa de validación.

\subsection{Fase 1: Análisis de Requerimientos}
Esta fase inicial, ubicada en la cima del brazo izquierdo del Modelo en V, se centra en la revisión conceptual y la definición del alcance de la investigación.
\begin{itemize}
    \item \textbf{Actividad 1.1}: Revisión de información sobre la tecnología UWB y la métrica ToF, además, su uso en posicionamiento.
    \item \textbf{Actividad 1.2}: Revisión de información sobre la técnica de multilateración y su aplicación.
    \item \textbf{Actividad 1.3}: Revisión de información sobre la BS y su efecto en posicionamiento.
    \item \textbf{Actividad 1.4}: Revisión de filtros de estimación y mitigación, como el KF.
\end{itemize}

\subsection{Fase 2: Diseño del Sistema}
Descendiendo por el brazo izquierdo, esta fase se enfoca en el diseño arquitectónico del sistema y la planificación detallada de los experimentos.
\begin{itemize}
    \item \textbf{Actividad 2.1}: Selección de escenarios para pruebas.
    \item \textbf{Actividad 2.2}: Diseño del sistema prototipo de posicionamiento UWB.
    \item \textbf{Actividad 2.3}: Diseño de la metodología experimental definiendo posiciones del TAG, una parte del cuerpo y la ruta a seguir por los sujetos bajo pruebas, considerando una muestra diversa en características antropométricas (i.e, altura, complexión) para garantizar la generalización de los resultados.
\end{itemize}

\subsection{Fase 3: Implementación}
Esta fase se sitúa en el vértice de la V y consiste en la materialización del diseño.
\begin{itemize}
    \item \textbf{Actividad 3.1}: Selección de software y hardware para la implementación del sistema UWB.
    \item \textbf{Actividad 3.2}: Definición del plan de pruebas unitarias o de validación. 
    \item \textbf{Actividad 3.3}: Ejecución del plan de pruebas unitario o de validación.
\end{itemize}

\subsection{Fase 4: Pruebas de Desempeño}
Iniciando el ascenso por el brazo derecho del modelo, esta fase valida que los componentes del sistema interactúan correctamente.
\begin{itemize}
    \item \textbf{Actividad 4.1}: Definición del plan de pruebas de desempeño.
    \item \textbf{Actividad 4.2}: Ejecución del plan de pruebas de desempeño.
\end{itemize}

\subsection{Fase 5: Análisis de Resultados}
En la cima del brazo derecho, se analizan los resultados obtenidos sobre el efecto de la BS en interiores.
%
\begin{itemize}
    \item \textbf{Actividad 5.1}: Evaluación de la exactitud y precisión del sistema de posicionamiento considerando la BS.
    \item \textbf{Actividad 5.2}: Análisis final del efecto de la BS en el sistema de posicionamiento.
\end{itemize}

\subsection{Fase 6: Entrega}
Esta fase concluye el proyecto con la documentación y presentación de los resultados.
\begin{itemize}
    \item \textbf{Actividad 6.1}: Elaboración del documento final y del artículo del trabajo de maestría.
    \item \textbf{Actividad 6.2}: Entrega final del documento final de maestría y sustentación.
\end{itemize}

\section{ESTADO ACTUAL DEL TRABAJO Y AVANCES}

A la fecha de presentación de este documento, el trabajo de maestría  ha avanzado más allá de la fase de diseño y se encuentra en una etapa avanzada de ejecución: fase 3, implementación; con los siguientes logros clave completados:

\begin{itemize}
    \item \textbf{Revisión Sistemática de Literatura y Marco Teórico:} Se ha finalizado una revisión sistemática de la literatura, permitiendo contextualizar los resultados experimentales e identificar las contribuciones específicas de este trabajo, particularmente en la inexplorada banda de 6.5 GHz.

    \item \textbf{Definición Metodológica} Se ha estructurado y documentado parte de la metodología del proyecto para la recolección de los datos experimentales.

    \item \textbf{Finalización de Campaña de Medidas de Validación Punto a Punto:} Se han completado las pruebas en escenarios de interiores y exteriores utilizando un nodo móvil (TAG) y un nodo fijo (ANCLA). Las mediciones se realizaron de manera lineal, en disposición punto a punto, determinando la distancia entre la TAG y el ANCLA, ubicando la TAG en diferentes partes del cuerpo humano, i.e., frente, pecho, espalda, entre otras, y situándola a distancias previamente definidas.

    \item \textbf{Publicación de Resultados Iniciales en Conferencia:} Parte de los resultados preliminares fueron presentados en la conferencia IEEE C3, realizada en la Universidad del Magdalena. En este evento se expuso cómo la ubicación del TAG en diferentes partes del cuerpo influye significativamente en la estimación de su posición a lo largo de una trayectoria recta, tanto en condiciones LOS como NLOS.
    
\end{itemize}

El trabajo restante se centrará en la creación del plan y ejecución de pruebas de desempeño y  análisis de los resultados, en el escenario 2-dimensional en interiores, la evaluación de la BS y la aplicación del algoritmo de mitigación, y la redacción de las conclusiones, de acuerdo con el cronograma.

\newpage

\section{CRONOGRAMA}
\label{sec:cronograma}
\renewcommand{\arraystretch}{1.2}

% Definición de colores personalizados
\definecolor{azulEncabezado}{HTML}{003366}
\definecolor{textoEncabezado}{HTML}{FFFFFF}
\definecolor{grisFase}{HTML}{E6E6E6}
\definecolor{textoFase}{HTML}{333333}
\definecolor{grisAlterno}{HTML}{F9F9F9}
\definecolor{barraDuracion}{HTML}{1E90FF}
\definecolor{bordeTabla}{HTML}{CCCCCC}

\begin{longtable}{|>{\raggedright\arraybackslash}p{4cm}|*{9}{>{\centering\arraybackslash}p{0.9cm}|}}
\caption{Cronograma de Actividades \label{tab:cronograma}} \\
\rowcolor{azulEncabezado}
\textcolor{textoEncabezado}{\textbf{Actividad}} & 
\textcolor{textoEncabezado}{\textbf{M1}} & 
\textcolor{textoEncabezado}{\textbf{M2}} & 
\textcolor{textoEncabezado}{\textbf{M3}} & 
\textcolor{textoEncabezado}{\textbf{M4}} & 
\textcolor{textoEncabezado}{\textbf{M5}} & 
\textcolor{textoEncabezado}{\textbf{M6}} & 
\textcolor{textoEncabezado}{\textbf{M7}} & 
\textcolor{textoEncabezado}{\textbf{M8}} & 
\textcolor{textoEncabezado}{\textbf{M9}} \\
\hline
\endfirsthead

\rowcolor{azulEncabezado}
\textcolor{textoEncabezado}{\textbf{Actividad}} & 
\textcolor{textoEncabezado}{\textbf{M1}} & 
\textcolor{textoEncabezado}{\textbf{M2}} & 
\textcolor{textoEncabezado}{\textbf{M3}} & 
\textcolor{textoEncabezado}{\textbf{M4}} & 
\textcolor{textoEncabezado}{\textbf{M5}} & 
\textcolor{textoEncabezado}{\textbf{M6}} & 
\textcolor{textoEncabezado}{\textbf{M7}} & 
\textcolor{textoEncabezado}{\textbf{M8}} & 
\textcolor{textoEncabezado}{\textbf{M9}} \\
\hline
\endhead

% Fase 1
\rowcolor{grisFase}
\multicolumn{10}{|l|}{\textcolor{textoFase}{\textbf{Fase 1: Análisis de requerimientos}}} \\
\hline
\rowcolor{white}
Actividad 1.1 & \cellcolor{barraDuracion} & \cellcolor{barraDuracion} &  & & & & & & \\
\rowcolor{grisAlterno}
Actividad 1.2 & \cellcolor{barraDuracion} & \cellcolor{barraDuracion} & & & & & & & \\
\rowcolor{white}
Actividad 1.3 & & \cellcolor{barraDuracion} & & & & & & & \\
\rowcolor{grisAlterno}
Actividad 1.4 & & \cellcolor{barraDuracion} &  & & & & & & \\
\hline

% Fase 2
\rowcolor{grisFase}
\multicolumn{10}{|l|}{\textcolor{textoFase}{\textbf{Fase 2: Diseño}}} \\
\hline
\rowcolor{white}
Actividad 2.1 & & \cellcolor{barraDuracion} & \cellcolor{barraDuracion} & & & & & & \\
\rowcolor{grisAlterno}
Actividad 2.2 & & & \cellcolor{barraDuracion} & \cellcolor{barraDuracion} & & & & & \\
\rowcolor{white}
Actividad 2.3 & & & & \cellcolor{barraDuracion} & \cellcolor{barraDuracion} & & & & \\
\hline

% Fase 3
\rowcolor{grisFase}
\multicolumn{10}{|l|}{\textcolor{textoFase}{\textbf{Fase 3: Implementación}}} \\
\hline
\rowcolor{grisAlterno}
Actividad 3.1 & & & & & \cellcolor{barraDuracion} & \cellcolor{barraDuracion} & & & \\
\rowcolor{white}
Actividad 3.2 & & & & & & \cellcolor{barraDuracion} & & & \\
\rowcolor{grisAlterno}
Actividad 3.3 & & & & & & \cellcolor{barraDuracion} & & & \\
\hline

% Fase 4
\rowcolor{grisFase}
\multicolumn{10}{|l|}{\textcolor{textoFase}{\textbf{Pruebas de desempeño}}} \\
\hline
\rowcolor{white}
Actividad 4.1 & & & & & & \cellcolor{barraDuracion} & & & \\
\rowcolor{grisAlterno}
Actividad 4.2 & & & & & & \cellcolor{barraDuracion} & & & \\
\hline

% Fase 5
\rowcolor{grisFase}
\multicolumn{10}{|l|}{\textcolor{textoFase}{\textbf{Análisis de resultados}}} \\
\hline
\rowcolor{grisAlterno}
Actividad 5.1 & & & & & & & \cellcolor{barraDuracion} & & \\
\rowcolor{white}
Actividad 5.2 & & & & & & & \cellcolor{barraDuracion} & & \\
\hline

% Fase 6
\rowcolor{grisFase}
\multicolumn{10}{|l|}{\textcolor{textoFase}{\textbf{Fase 6: Entrega}}} \\
\hline
\rowcolor{grisAlterno}
Actividad 6.1 & & & & & & & & \cellcolor{barraDuracion} & \\
\rowcolor{white}
Actividad 6.2 & & & & & & & & & \cellcolor{barraDuracion} \\
\hline

\end{longtable}


% --- Condiciones de Entrega ---
\chapter{CONDICIONES DE ENTREGA Y PRESUPUESTO}
\label{ch:condiciones}

\section{CONDICIONES DE ENTREGA}

Finalizado el trabajo de grado de maestría se entregará:

\begin{itemize}
    \item Documento final de trabajo de maestría impreso y en formato digital, que incluye su desarrollo y sus resultados.
    \item Archivos con resultados de la implementación de los planes de pruebas de validación y desempeño.
    \item Artículo de investigación en formato IEEE.
\end{itemize}

\section{PRESUPUESTO ESTIMADO Y FUENTES DE FINANCIACIÓN}

A continuación, se presenta una estimación de los costos asociados a los recursos materiales, técnicos y humanos necesarios para el desarrollo del trabajo de maestría basado en el documento \cite{unicauca_guia_anteproyecto}.
%
\begin{table}[htb]
\centering
\caption{Presupuesto Estimado del Proyecto}
\label{tab:presupuesto}
\scriptsize

% Definición de colores personalizados
\definecolor{azulEncabezado}{HTML}{003366}
\definecolor{textoEncabezado}{HTML}{FFFFFF}
\definecolor{grisFase}{HTML}{E6E6E6}
\definecolor{textoFase}{HTML}{333333}
\definecolor{grisAlterno}{HTML}{F9F9F9}
\definecolor{bordeTabla}{HTML}{CCCCCC}

\arrayrulecolor{bordeTabla}
\renewcommand{\arraystretch}{1.3}

\begin{tabular}{|p{3.5cm}|>{\raggedleft\arraybackslash}p{2cm}|>{\raggedleft\arraybackslash}p{2cm}|>{\raggedleft\arraybackslash}p{2cm}|}
\hline
\rowcolor{azulEncabezado}
\textcolor{textoEncabezado}{\textbf{Rubro}} &
\textcolor{textoEncabezado}{\textbf{Estudiante}} &
\textcolor{textoEncabezado}{\textbf{FIET}} &
\textcolor{textoEncabezado}{\textbf{Total (COP)}} \\
\hline

% Recursos Humanos
\rowcolor{grisFase}
\multicolumn{4}{|l|}{\textcolor{textoFase}{\textbf{Recursos Humanos}}} \\
\hline
Estudiante de Maestría & \numprint{\TotalEstudiante} & \numprint{0.00} & \numprint{\TotalEstudiante} \\
Director del Proyecto & \numprint{0.00} & \numprint{\TotalDirector}  & \numprint{\TotalDirector} \\
\rowcolor{grisAlterno}
\multicolumn{3}{|l|}{\textbf{Subtotal Recursos Humanos}} & \numprint{\TotalRH} \\
\hline

% Hardware
\rowcolor{grisFase}
\multicolumn{4}{|l|}{\textcolor{textoFase}{\textbf{Hardware y Equipos}}} \\
\hline
Computador (depreciación 9 meses) & \numprint{\DepreciacionPC} & \numprint{0.00} & \numprint{\DepreciacionPC} \\
Equipos UWB (50 horas) & \numprint{0.00} & \numprint{\DepreciacionUWB} & \numprint{\DepreciacionUWB}  \\
\rowcolor{grisAlterno}
\multicolumn{3}{|l|}{\textbf{Subtotal Hardware}} & \numprint{\TotalHardware} \\
\hline

\rowcolor{grisAlterno}
\hline

% Publicaciones
\rowcolor{grisFase}
\multicolumn{4}{|l|}{\textcolor{textoFase}{\textbf{Publicación y Difusión}}} \\
\hline
APC + Impresión & \numprint{\TotalPublicaciones} & \numprint{0.00}& \numprint{\TotalPublicaciones} \\
\hline

%primer subtotal
\rowcolor{azulEncabezado}
\multicolumn{3}{|l|}{\textcolor{textoEncabezado}{\textbf{Primer Sub Total}}} & \textcolor{textoEncabezado}{\numprint{\SubtotalDirecto}}\\
\hline
% Comunicaciones
\rowcolor{grisFase}
\multicolumn{4}{|l|}{\textcolor{textoFase}{\textbf{Comunicaciones}}} \\
\hline
Comunicaciones (2\% subtotal) & \numprint{0.00} & \numprint{\comunicacionesRubro} & \numprint{\comunicacionesRubro} \\
\hline

% Segundo sobtotal
%primer subtotal
\rowcolor{azulEncabezado}
\multicolumn{3}{|l|}{\textcolor{textoEncabezado}{\textbf{Segundo Sub Total}}} & \textcolor{textoEncabezado}{\numprint{\fpeval{\SubtotalDirecto + \comunicacionesRubro}}}\\
\hline

% AUI
\rowcolor{grisFase}
\multicolumn{4}{|l|}{\textcolor{textoFase}{\textbf{AUI}}} \\
\hline
Infraestructura y administración (20\% subtotal) & \numprint{0.00} & \numprint{\AUI} & \numprint{\AUI} \\
\hline



% TOTAL
\rowcolor{azulEncabezado}
\textcolor{textoEncabezado}{\textbf{TOTAL ESTIMADO}} &
\textcolor{textoEncabezado}{\numprint{\TotalEstudianteRecurso}} &
\textcolor{textoEncabezado}{\numprint{\TotalFIET}} &
\textcolor{textoEncabezado}{\textbf{\numprint{\TotalProyecto}}} \\
\hline

\end{tabular}

\end{table}

% --- Párrafo Descriptivo ---
\subsection{Descripción y Fuentes de Financiación del Presupuesto:}

El presupuesto presentado en la Tabla \ref{tab:presupuesto} detalla los costos estimados para la ejecución completa del proyecto. 

\begin{itemize}
    \item \textbf{Recursos Humanos:} El costo se calcula con base en el valor del punto salarial vigente para empleados públicos a nivel nacional, i.e., \$22,358 COP por punto, con una asignación de 2.5 puntos / hora, según la el documento \cite{unicauca_guia_anteproyecto}, considerando una dedicación de 30 horas semanales por parte del estudiante de maestría a lo largo de los 9 meses de duración del proyecto, para un total de 36 semanas. En cuanto al director y codirector, en conjunto se asignan 4 horas por semana con una asignación de 4 puntos/hora a lo largo de 36 semanas.  % $ 30.183.300.

    \item \textbf{Hardware y Equipos:} Los costos de los componentes electrónicos (módulos UWB) son equipos que son obtenidos de los grupos de investigación de la Universidad de Cauca. Por lo que el costo se calcula por el uso de estos. Los dispositivos se van a utilizar un máximo de 50 horas durante todo el trabajo de maestría. En cuanto a los computadores utilizados para la redacción de los documentos, validación y análisis de los datos se va a utilizar un computador que tiene un costo de \$ 4.250.000 COP, el costo de uso se calcula con base en el documento \cite{unicauca_guia_anteproyecto}.

    \item \textbf{AUI}: 20\% del total del proyecto.

    \item \textbf{Comunicaciones}: 2\% del total del proyecto. 
    
    \item \textbf{Publicación y Difusión:} Se contempla un rubro para cubrir las posibles tarifas de publicación (APC, \textit{Article Processing Charge}) en una conferencia o revista científica de alto impacto. Los costos de impresión corresponden a la entrega del documento final de tesis de maestría.
\end{itemize}



\textbf{Fuentes de Financiación:} El proyecto será financiado a través de una combinación de recursos. Los costos de hardware y publicación serán cubiertos con fondos propios del investigador. Asimismo, se aprovecharán los recursos de infraestructura, equipos y laboratorios proporcionados por los grupos de investigación \textbf{GRIAL} y \textbf{GNTT} de la Facultad de Ingeniería Electrónica y Telecomunicaciones de la Universidad del Cauca.


% Capítulo 7: Diseño Metodológico
\chapter{DISEÑO METODOLÓGICO}
\label{ch:metodologia_experimental}

En este capítulo se describe detalladamente la metodología empleada para llevar a cabo la investigación experimental, estructurada en dos fases secuenciales:

\textbf{Fase 1 - Plan de Pruebas de Validación:} Evaluación exhaustiva de la exactitud en la estimación de distancia UWB analizando el impacto de la ubicación del dispositivo móvil en diferentes partes del cuerpo humano bajo condiciones de propagación LOS y NLOS en dos escenarios representativos (exterior e interior). Esta fase, ya ejecutada y documentada en el artículo de validación, establece la línea base de desempeño del sistema.

\textbf{Fase 2 - Sistema de Posicionamiento 2D:} Implementación y evaluación del sistema completo de localización utilizando cuatro nodos ancla para determinar posiciones 2D mediante trilateración. Esta fase integrará los hallazgos de la Fase 1 para evaluar el error de posicionamiento en el plano.

Este capítulo se enfoca en describir la metodología de la Fase 1, que constituye el plan de pruebas de validación completo del sistema de estimación de distancia a 6.5 GHz.

\section{CONFIGURACIÓN DEL HARDWARE}
\label{sec:configuracion_hardware}

\subsection{Dispositivos UWB}

\subsubsection{Selección de Dispositivos}

Para este estudio se han seleccionado módulos de desarrollo Qorvo DWM1001, que integran un transceptor UWB DW1000, un microcontrolador Nordic Semiconductor nRF52832 y un sensor de movimiento. Los módulos operan con el firmware PANS (Positioning and Networking Stack) de Qorvo, el cual implementa la técnica TWR para estimación de distancia. Estos dispositivos están configurados para operar en la banda de frecuencia central de 6.5 GHz. La elección de estos dispositivos se fundamenta en:

\begin{itemize}
\item Capacidad de operar en la banda de 6.5 GHz con el ancho de banda necesario para lograr resolución temporal adecuada.
\item Soporte nativo para mediciones TWR con registro de timestamps de alta precisión.
\item Interfaz de comunicación accesible que permite la extracción de datos crudos de ToF y otras métricas de señal.
\item Consumo energético compatible con operación portátil prolongada.
\item Disponibilidad de documentación técnica detallada y herramientas de desarrollo.
\end{itemize}

\subsubsection{Parámetros de Configuración}

Los dispositivos UWB se configuran con los siguientes parámetros operacionales:

\begin{itemize}
\item \textbf{Frecuencia central:} 6489.6 MHz (Canal 5)
\item \textbf{Ancho de banda:} 499.2 MHz
\item \textbf{Tasa de transmisión de datos:} 6.8 Mbps
\item \textbf{Frecuencia de repetición de pulsos (PRF):} 64 MHz
\item \textbf{Longitud del preámbulo:} 128 símbolos
\item \textbf{Código del preámbulo (TX/RX):} 10
\item \textbf{Potencia de transmisión:} -17 dBm
\item \textbf{Sensibilidad del receptor:} -93 dBm
\end{itemize}

Estos parámetros se seleccionan para optimizar el compromiso entre exactitud de medición, alcance efectivo y consumo energético, siguiendo las recomendaciones del fabricante y las mejores prácticas identificadas en la revisión de literatura.


\section{DISEÑO DEL ESCENARIO EXPERIMENTAL - FASE 1}
\label{sec:escenario_experimental}

\subsection{Descripción General}

El plan de pruebas de validación (Fase 1) se implementó en dos escenarios principales diseñados para evaluar diferentes condiciones de propagación características de aplicaciones reales de sistemas de posicionamiento indoor y outdoor. La selección de estos escenarios permite caracterizar el comportamiento del sistema UWB tanto en ambientes con mínima reflexión (exterior) como en entornos con alta dispersión multitrayecto (interior tipo corredor).

\subsection{Escenario Exterior}

Se utilizó un campo abierto con dimensiones similares a las de un campo de fútbol. Este escenario fue diseñado para:

\begin{itemize}
\item Minimizar las reflexiones provenientes de estructuras verticales como paredes y edificios.
\item Proporcionar un entorno base para analizar los efectos primarios de propagación.
\item Aislar el impacto de la obstrucción corporal (BS) sobre la señal directa.
\item Considerar fenómenos fundamentales de propagación como las Pérdidas en Espacio Libre (FSPL) y el multitrayecto generado por reflexión en el suelo.
\end{itemize}

\textbf{Características del Escenario Exterior:}
\begin{itemize}
\item \textbf{Tipo:} Campo abierto (dimensiones ~100 m × 60 m)
\item \textbf{Superficie:} Césped natural
\item \textbf{Obstrucciones:} Mínimas (ausencia de paredes y estructuras verticales cercanas)
\item \textbf{Condiciones climáticas:} Mediciones realizadas en días despejados para minimizar variables ambientales
\end{itemize}

\subsection{Escenario Interior}

Se empleó un corredor de edificio universitario para representar condiciones típicas de aplicaciones indoor:

\textbf{Características del Escenario Interior:}
\begin{itemize}
\item \textbf{Dimensiones:} 20 m de largo × 3 m de ancho
\item \textbf{Materiales constructivos:} Paredes de concreto, piso de baldosa cerámica, techo de placa de concreto
\item \textbf{Mobiliario:} Corredor típico con mobiliario mínimo
\item \textbf{Características de propagación:} Introduce efectos de multitrayecto controlados resultantes de reflexiones en paredes, techo y piso
\item \textbf{Fuentes de interferencia:} Redes WiFi institucionales activas en bandas de 2.4 GHz y 5 GHz, equipos electrónicos en oficinas adyacentes
\end{itemize}

Este escenario permite evaluar cómo las reflexiones especulares en superficies confinadas afectan la estimación de distancia, representando condiciones realistas para sistemas de localización indoor.

\subsection{Configuración de Nodos - Fase 1}

En la Fase 1 del plan de pruebas de validación, la configuración experimental consistió en:

\begin{itemize}
\item \textbf{Nodo Fijo de Referencia:} Un módulo DWM1001 montado en un trípode a una altura estándar de 1.5 m sobre el nivel del piso, actuando como ancla de referencia para las mediciones de distancia.
\item \textbf{Nodo Móvil:} Un módulo DWM1001 portado por el sujeto de prueba en diferentes ubicaciones corporales.
\end{itemize}

Esta configuración punto a punto permite evaluar la exactitud fundamental de la estimación de distancia antes de implementar el sistema completo de posicionamiento 2D. El nodo fijo permanece en una ubicación constante durante cada sesión de medición, mientras el sujeto se desplaza a distancias conocidas siguiendo un protocolo sistemático.

\textbf{Nota:} La Fase 2 implementará un sistema con cuatro nodos ancla en configuración geométrica optimizada para trilateración y posicionamiento 2D.


\section{PROTOCOLO DE RECOLECCIÓN DE DATOS}
\label{sec:protocolo_recoleccion}

\subsection{Sujetos de Prueba}

La Fase 1 de este estudio fue conducida como un caso de estudio fundacional con el Sujeto de Prueba 2. Este enfoque metodológico permitió una caracterización detallada de los efectos de la ubicación del dispositivo en el cuerpo sin variabilidad inter-sujeto, estableciendo una línea base de desempeño del sistema y un control riguroso de variables experimentales.

Para la Fase 2, se contó con la participación de los tres sujetos de prueba (Sujeto 1, Sujeto 2 y Sujeto 3, siendo este último el mismo participante de la Fase 1). Este plan de pruebas exhaustivo, que involucró a múltiples sujetos con diversas características antropométricas, ha permitido una evaluación más robusta y generalizable del sistema de posicionamiento. Los sujetos de prueba participaron bajo consentimiento informado, con conocimiento completo del propósito del estudio, los procedimientos a realizar y el uso de los datos recolectados.

A continuación, se presentan las características antropométricas detalladas de los tres sujetos participantes en la Fase 2, así como sus respectivas fotografías:

\begin{figure}[hbt]
    \centering
    \includegraphics[width=0.3\textwidth]{imagenes/sujeto1.jpg}
    \caption{Sujeto de prueba 1}
    \label{fig:sujeto1}
\end{figure}

\begin{table}[hbt]
    \centering
    \caption{Características Antropométricas - Sujeto 1}
    \label{tab:antropometricas_sujeto1}
    \begin{tabular}{lcc}
        \toprule
        \rowcolor{headerblue}
        \textbf{Característica} & \textbf{Valor} & \textbf{Unidad} \\
        \midrule
        Estatura                & 1.75           & m                \\
        Peso                    & 70             & kg               \\
        Índice de Masa Corporal & 22.86          & kg/m²            \\
        Género                  & Masculino      & -                \\
        Edad                    & 25             & años             \\
        \bottomrule
    \end{tabular}
\end{table}

\begin{figure}[hbt]
    \centering
    \includegraphics[width=0.3\textwidth]{imagenes/sujeto2.jpg}
    \caption{Sujeto de prueba 2}
    \label{fig:sujeto2}
\end{figure}

\begin{table}[hbt]
    \centering
    \caption{Características Antropométricas - Sujeto 2}
    \label{tab:antropometricas_sujeto2}
    \begin{tabular}{lcc}
        \toprule
        \rowcolor{headerblue}
        \textbf{Característica} & \textbf{Valor} & \textbf{Unidad} \\
        \midrule
        Estatura                & 1.62           & m                \\
        Peso                    & 60             & kg               \\
        Índice de Masa Corporal & 22.86          & kg/m²            \\
        Género                  & Masculino      & -                \\
        Edad                    & 28             & años             \\
        \bottomrule
    \end{tabular}
\end{table}

\begin{figure}[hbt]
    \centering
    \includegraphics[width=0.3\textwidth]{imagenes/sujeto3.jpg}
    \caption{Sujeto de prueba 3}
    \label{fig:sujeto3}
\end{figure}

\begin{table}[hbt]
    \centering
    \caption{Características Antropométricas - Sujeto 3}
    \label{tab:antropometricas_sujeto3}
    \begin{tabular}{lcc}
        \toprule
        \rowcolor{headerblue}
        \textbf{Característica} & \textbf{Valor} & \textbf{Unidad} \\
        \midrule
        Estatura                & 1.80           & m                \\
        Peso                    & 85             & kg               \\
        Índice de Masa Corporal & 26.23          & kg/m²            \\
        Género                  & Masculino      & -                \\
        Edad                    & 30             & años             \\
        \bottomrule
    \end{tabular}
\end{table}

\subsection{Ubicaciones del Dispositivo Móvil en el Cuerpo}

Se evaluaron siete ubicaciones corporales del dispositivo móvil UWB, seleccionadas estratégicamente por su relevancia en aplicaciones prácticas de seguimiento de personas y dispositivos wearables:

\begin{enumerate}
\item \textbf{Cabeza:} Dispositivo ubicado en la parte superior craneal. Esta posición es relevante para aplicaciones de realidad aumentada y cascos inteligentes.

\item \textbf{Pecho:} Dispositivo colocado en el centro del pecho, aproximadamente a la altura del esternón, orientado hacia adelante. Representa la ubicación típica de badges de identificación y monitores cardíacos.

\item \textbf{Cadera:} Dispositivo posicionado lateralmente a nivel de la cintura/cadera. Común en dispositivos clip-on y sistemas de rastreo industrial.

\item \textbf{Mano:} Dispositivo sostenido directamente en la mano dominante. Simula el escenario de un usuario portando activamente un dispositivo móvil o herramienta.
\item \textbf{Muñeca:} Dispositivo colocado en la muñeca como si fuera un reloj inteligente (smartwatch), representando uno de los wearables más populares.

\item \textbf{Rodilla:} Dispositivo ubicado en la superficie frontal o lateral de la rodilla. Relevante para aplicaciones de análisis biomecánico y monitoreo deportivo.

\item \textbf{Tobillo:} Dispositivo posicionado en la superficie externa del tobillo. Utilizado en sistemas de monitoreo de marcha y rastreo en aplicaciones de seguridad.
\end{enumerate}

Para cada ubicación corporal, el dispositivo fue fijado de manera segura para minimizar el movimiento relativo durante las mediciones. Se documentó fotográficamente cada configuración para asegurar reproducibilidad en futuras réplicas del experimento.

\subsection{Condiciones de Propagación: LOS y NLOS}

Para cada combinación de ubicación corporal del dispositivo y escenario (exterior/interior), se evaluaron dos condiciones de propagación fundamentales definidas por la orientación del cuerpo del sujeto respecto al nodo fijo de referencia:

\textbf{Condición LOS (Line of Sight):} Se aseguró la existencia de una trayectoria visual directa entre el nodo fijo y el dispositivo móvil. El sujeto se orientó de manera que la parte del cuerpo que portaba el dispositivo mantuviera visibilidad directa con el nodo de referencia. Esta condición representa el escenario óptimo donde la señal UWB puede propagarse sin obstrucción corporal significativa.

\textbf{Condición NLOS (Non-Line of Sight):} Se obstruyó deliberadamente la línea de visión directa entre el nodo fijo y el dispositivo móvil mediante el cuerpo humano. Esta condición se logró orientando al sujeto de espaldas al nodo fijo (rotación de 180°), asegurando que el torso y la parte del cuerpo con el dispositivo bloquearan la propagación directa de la señal.

\textbf{Control de Postura:} Para garantizar la consistencia y reproducibilidad en todas las mediciones NLOS, se instruyó al sujeto a mantener una postura corporal estandarizada:
\begin{itemize}
\item Postura erguida y vertical
\item Orientación directamente opuesta al nodo fijo (180°)
\item Brazos descansando naturalmente a los costados del cuerpo
\item Evitar movimientos durante la adquisición de cada conjunto de mediciones
\end{itemize}

Esta dicotomía LOS/NLOS permite caracterizar el efecto máximo de la obstrucción corporal (BS) sobre la exactitud de la estimación de distancia, aislando este factor de otras fuentes de error.

\subsection{Protocolo de Medición - Fase 1}

\subsubsection{Configuración del Grid de Medición}

Las mediciones se realizaron siguiendo un protocolo sistemático de posiciones estáticas donde el sujeto permanece inmóvil durante la adquisición de datos:

\begin{itemize}
\item \textbf{Distancias evaluadas:} 1 m, 2 m, 3 m, ..., 13 m (incrementos de 1 metro)
\item \textbf{Rango total:} 1 m a 13 m (13 puntos de medición por configuración)
\item \textbf{Método de marcación:} Distancias reales marcadas con cinta métrica de alta precisión
\item \textbf{Alineación:} El sujeto se posiciona en línea recta desde el nodo fijo de referencia
\end{itemize}

\subsubsection{Muestras por Punto de Medición}

Para cada combinación de:
\begin{itemize}
\item Ubicación corporal (7 opciones: cabeza, pecho, cadera, mano, muñeca, rodilla, tobillo)
\item Condición de canal (2 opciones: LOS, NLOS)
\item Escenario (2 opciones: exterior, interior)
\item Distancia (13 puntos: 1-13 m)
\end{itemize}

Se registraron \textbf{250 mediciones de distancia estimada} por el sistema UWB. Este tamaño de muestra asegura análisis estadístico robusto y permite caracterizar adecuadamente la variabilidad de las mediciones.\\

\textbf{Total de mediciones - Fase 1:}
$$7 \text{ ubicaciones} \times 2 \text{ condiciones} \times 2 \text{ escenarios} \times 13 \text{ distancias} \times 250 \text{ muestras} = 91,000 \text{ mediciones}$$

\textbf{Nota:} Cabe aclarar que también se tomaron muestras en el dispositivo móvil tanto en interiores como en exteriores en los 13 puntos del experimento. 

\subsection{Condiciones de Control}

Para garantizar la reproducibilidad del experimento y minimizar la variabilidad no controlada:

\begin{itemize}
\item Las mediciones se realizan en horarios donde la actividad en las áreas circundantes es mínima, reduciendo interferencia de personas transitando.
\item Se verifica antes de cada sesión que no haya cambios significativos en el entorno (mobiliario movido, nuevos objetos metálicos, etc.).
\item Se registran las condiciones ambientales: temperatura, humedad relativa, presencia de dispositivos electrónicos activos.
\item Se realiza una medición de calibración en LOS (sin participante presente) al inicio de cada sesión experimental para verificar la estabilidad del sistema.
\end{itemize}

\subsection{Procedimiento Detallado de Medición}

El protocolo experimental implementado en cada sesión de medición se estructuró de manera sistemática para garantizar la integridad y reproducibilidad de los datos. Inicialmente, se procedió con la preparación del sistema, que abarcaba el encendido y la verificación funcional de los módulos DWM1001, así como el montaje del nodo fijo en un trípode a una altura estándar de 1.5 metros. Posteriormente, se realizó la inicialización del firmware PANS junto con la configuración de los parámetros UWB, finalizando esta etapa con la confirmación de la comunicación efectiva entre el nodo fijo y el nodo móvil.

De manera paralela, se llevó a cabo la marcación del grid de medición, utilizando una cinta métrica para delimitar con precisión las distancias desde 1 metro hasta 13 metros en línea recta desde el nodo fijo.. Antes de iniciar las pruebas con el participante, se ejecutó una calibración inicial en línea de vista (LOS) situando el nodo móvil en un trípode sin la presencia del sujeto; esto permitía verificar la exactitud del sistema en ausencia de obstrucciones e identificar posibles sesgos sistemáticos.

Una vez validado el sistema, se procedió a la instrumentación del sujeto, fijando de manera segura el nodo móvil en la ubicación corporal correspondiente y verificando su correcta orientación, proceso que era documentado fotográficamente. La adquisición de datos comenzó en condición LOS y en cada parte del cuerpo, donde el sujeto se posicionaba a 1 metro de distancia con orientación frontal hacia el nodo fijo. En esta posición, permaneciendo inmóvil durante aproximadamente 30 segundos, se registraban automáticamente 250 mediciones de distancia. Este procedimiento se repetía secuencialmente para cada punto marcado hasta alcanzar los 13 metros.

Posteriormente, se realizó la adquisición de datos en condición de no línea de vista (NLOS). Para ello, el sujeto se ubicó nuevamente en la primera marca de distancia y realizó una rotación de 180 grados para quedar de espaldas al nodo fijo, bloqueando la señal con con el nodo en cada una de las 7 posiciones corporales. Se adquirieron entonces otras 250 mediciones automáticas, repitiendo el proceso en cada una de las distancias establecidas.

Al completar las mediciones en una posición específica, se efectuó el cambio de ubicación corporal del nodo móvil a la siguiente zona de interés, repitiendo los ciclos de adquisición LOS y NLOS para las siete ubicaciones definidas. Una vez cubiertas todas las ubicaciones corporales, se replicó el procedimiento completo en el segundo escenario experimental (Exterior).

\section{PROCESAMIENTO Y ANÁLISIS DE DATOS - FASE 1}
\label{sec:procesamiento_datos}

\subsection{Preprocesamiento de Datos Crudos}

Los datos crudos registrados por el sistema UWB incluyen timestamps de alta resolución, mediciones de ToF entre el nodo móvil y el nodo fijo, estimaciones de potencia de señal recibida y otros parámetros de diagnóstico. El preprocesamiento de la Fase 1 comprende:

\begin{itemize}
\item \textbf{Filtrado de valores atípicos:} Detección y remoción de mediciones claramente erróneas (ej. ToF negativo, distancias fuera del rango físico posible de 1-13 m) mediante criterios estadísticos robustos.

\item \textbf{Conversión de ToF a Distancia:} Aplicación de la ecuación $d = \text{ToF} \times c$, donde $c = 299{,}792{,}458$ m/s es la velocidad de la luz en el vacío.

\item \textbf{Corrección de sesgos sistemáticos:} Si se identifican sesgos consistentes en las mediciones de calibración LOS (sin sujeto), se aplican correcciones de offset al nodo de referencia.
\end{itemize}

\subsection{Cálculo de Métricas de Error}

Para cada medición de distancia entre el nodo móvil y el nodo fijo de referencia, se calcula el error de distancia:

\begin{equation}
e_i = d_{\text{medida},i} - d_{\text{real},i}
\end{equation}

donde $d_{\text{medida},i}$ es la distancia estimada por el sistema UWB y $d_{\text{real},i}$ es la distancia física real marcada con cinta métrica (1 m, 2 m, ..., 13 m).

Se definen las siguientes métricas estadísticas:

\begin{itemize}

\item \textbf{Error Absoluto Medio (Mean Absolute Error, MAE):}
\begin{equation}
\text{MAE} = \frac{1}{N} \sum_{j=1}^{N} |e_j|
\end{equation}

\item \textbf{Raíz del Error Cuadrático Medio (Root Mean Square Error, RMSE):}
\begin{equation}
\text{RMSE} = \sqrt{\frac{1}{N} \sum_{j=1}^{N} e_j^2}
\end{equation}


\item \textbf{Percentiles:} Se calculan los percentiles 50 (mediana), 90 y 95 del error absoluto para caracterizar la distribución completa del error.
\end{itemize}

\subsection{Análisis Estadístico - Fase 1}

\subsubsection{Caracterización de Distribuciones}

Para cada condición experimental (combinación de ubicación corporal, condición LOS/NLOS, escenario, distancia), se caracteriza la distribución estadística de los errores observados. Se evalúan distribuciones de probabilidad teóricas:

\begin{itemize}
\item Normal (Gaussiana)
\item Log-normal  
\item Gamma
\end{itemize}

La bondad de ajuste se evalúa mediante pruebas de Kolmogorov-Smirnov y criterios de información (AIC, BIC). Esta caracterización permite identificar si las distribuciones de error son simétricas (gaussianas) o presentan asimetría positiva con colas extendidas hacia errores mayores, lo cual tiene implicaciones para el diseño de algoritmos de mitigación.

\subsubsection{Factores Experimentales Analizados}

Los principales factores experimentales evaluados en la Fase 1 son:

\begin{itemize}
\item \textbf{Ubicación Corporal:} 7 niveles (cabeza, pecho, cadera, mano, muñeca, rodilla, tobillo)
\item \textbf{Condición de Canal:} 2 niveles (LOS, NLOS)
\item \textbf{Escenario:} 2 niveles (exterior, interior)
\item \textbf{Distancia:} Variable continua o discretizada (1-13 m)
\end{itemize}

Se analiza el efecto individual de cada factor sobre el error de distancia, así como las interacciones entre factores (e.g., si el efecto de la condición LOS/NLOS depende de la ubicación corporal).


\section{HERRAMIENTAS DE SOFTWARE}
\label{sec:herramientas_software}

\subsection{Software de Adquisición de Datos}

Para el desarrollo del presente proyecto se empleó un software provisto por la empresa Quorvo, el cual consiste en una librería integral que conforma un ecosistema completo de software. Dicho ecosistema incluye el PANS (Positioning and Navigation Software), encargado de los procesos de posicionamiento y navegación. Esta librería contiene todas las funciones y estructuras necesarias para controlar la comunicación entre los dispositivos denominados anclas (anchors) y el dispositivo etiqueta (tag).

El software está desarrollado en lenguaje C y se utiliza el entorno de desarrollo SEGGER Embedded Studio para la validación del algoritmo, la compilación del código y la carga del binario en el dispositivo. Este entorno emplea J-Link como controlador (driver) para la programación del hardware.

Las funciones de comunicación y adquisición de datos se encuentran predefinidas dentro de la librería. El trabajo realizado en esta tesis de maestría consiste en la modificación de una parte del algoritmo con el objetivo de establecer la conexión de la etiqueta con cuatro anclas y habilitar la transmisión de los datos obtenidos a través de una interfaz serial. Los algoritmos desarrollados y modificados se encuentran disponibles en el siguiente repositorio de GitHub, tanto para el dispositivo anchor como para el tag \cite{DiazAlgoritmosUWB}.


\subsection{Software de Procesamiento y Análisis}

El procesamiento de datos y el análisis estadístico se llevan a cabo utilizando Python 3.x, apoyado en un conjunto de bibliotecas científicas. NumPy se emplea para operaciones numéricas y álgebra lineal, Pandas para la manipulación y análisis de datos estructurados, y SciPy para la aplicación de funciones estadísticas avanzadas, ajuste de distribuciones y análisis ANOVA. La visualización de los resultados se realiza mediante Matplotlib y Seaborn, mientras que scikit-learn se utiliza para la implementación de filtros y algoritmos de localización.

\subsection{Control de Versiones y Reproducibilidad}

Todo el código desarrollado se gestiona mediante Git y se almacena en un repositorio público. Los datos experimentales se organizan en formato personalizado con metadatos descriptivos. Se proporciona un entorno computacional reproducible mediante un archivo requirements.txt que especifica las versiones exactas de todas las dependencias de software que se instalan en un entorno virtual.


\section{CONSIDERACIONES ÉTICAS}
\label{sec:consideraciones_eticas}

El protocolo experimental ha sido aprobado por el Comité de Ética correspondiente y diseñado siguiendo los lineamientos éticos vigentes. Los participantes son informados detalladamente sobre el propósito del estudio y los procedimientos involucrados, firman un consentimiento informado antes de participar y pueden retirarse del estudio en cualquier momento sin consecuencias. Asimismo, sus datos personales (nombre, edad, entre otros) son anonimizados y protegidos conforme a las normativas de protección de datos aplicables, tales como la Ley 1581 de 2012 en Colombia o el GDPR en Europa. Finalmente, se garantiza que los participantes no son sometidos a ningún procedimiento invasivo ni a riesgos para su salud o integridad física.

% Capítulo 8: Resultados Experimentales
\chapter{FASE 1: VALIDACIÓN DE ESTIMACIÓN DE DISTANCIA}
\label{ch:resultados}

Este capítulo presenta los resultados obtenidos durante la Fase 1 de la investigación experimental, correspondiente a la validación del sistema de estimación de distancia UWB. Se exponen de manera sistemática los datos recolectados y procesados según la metodología descrita en el capítulo anterior, organizados por categorías que facilitan su interpretación posterior.

Los resultados presentados en este capítulo se enfocan en caracterizar la exactitud de las mediciones de distancia punto a punto bajo diferentes condiciones de obstrucción corporal y propagación. La Fase 2, correspondiente al sistema completo de posicionamiento 2D con cuatro nodos ancla, será presentada en secciones subsiguientes una vez completada la experimentación.

\section{VERIFICACIÓN Y CALIBRACIÓN DEL SISTEMA}
\label{sec:verificacion_sistema}

\subsection{Calibración de Dispositivos}

Previo a la recolección de datos experimentales, se realizó un procedimiento de calibración de los retardos de antena (\textit{antenna delays}) del transceptor DW1000 presente en los módulos DWM1001. Estos parámetros compensan los retardos intrínsecos del hardware entre el instante temporal registrado por el transceptor y la emisión/recepción física de la señal en la interfaz radioeléctrica.

El DW1000 utiliza dos constantes de retardo que deben calibrarse para garantizar la exactitud de las mediciones de distancia mediante Two-Way Ranging (TWR):

\begin{itemize}
\item \texttt{TX\_ANT\_DLY}: retardo efectivo desde que el DW1000 genera el instante temporal de transmisión hasta que la señal electromagnética es emitida físicamente por la antena. 
\item \texttt{RX\_ANT\_DLY}: retardo desde que la señal llega físicamente a la antena hasta que el DW1000 registra el instante temporal de recepción.
\end{itemize}

Los valores por defecto proporcionados por el fabricante son aproximaciones generales que no consideran las particularidades de cada implementación de hardware (diseño de PCB, tipo de antena, trazas de RF). Por tanto, es necesario calibrarlos empíricamente para cada sistema específico.

\subsubsection{Procedimiento de Calibración}

El método de calibración implementado se basa en la comparación entre distancias conocidas y distancias medidas, ajustando iterativamente los \textit{antenna delays} hasta minimizar el error sistemático. El procedimiento consistió en los siguientes pasos:

\textbf{1. Configuración inicial:} Se configuraron dos módulos DWM1001 en modo TWR unilateral (SS-TWR), con valores iniciales de \texttt{TX\_ANT\_DLY} = 16380 y \texttt{RX\_ANT\_DLY} = 16450 (unidades de tiempo internas del DW1000, equivalentes a $\approx$15.65 ps por unidad).

\textbf{2. Posicionamiento de referencia:} Los módulos fueron ubicados a una distancia fija conocida $D_{\text{real}}$ = 1.00 m, medida con cinta métrica láser de precisión $\pm$1 mm, en ambiente exterior despejado para minimizar efectos de multitrayecto.

\textbf{3. Medición no calibrada:} Se realizaron 250 mediciones continuas de distancia, registrando el valor promedio $D_{\text{medida}}$ y calculando el error sistemático:

\begin{equation}
E = D_{\text{real}} - D_{\text{medida}}
\end{equation}

\textbf{4. Ajuste de retardos:} El error sistemático se traduce a un incremento/decremento en las constantes de retardo mediante el factor de conversión empírico para el DW1000 ($\approx$0.015 m por unidad de delay):

\begin{equation}
\Delta = \frac{E}{0.015}
\end{equation}

El ajuste se distribuyó simétricamente entre ambos retardos para mantener el balance TX/RX:

\begin{align}
\texttt{TX\_ANT\_DLY}_{\text{nuevo}} &= \texttt{TX\_ANT\_DLY}_{\text{anterior}} - \frac{\Delta}{2} \\
\texttt{RX\_ANT\_DLY}_{\text{nuevo}} &= \texttt{RX\_ANT\_DLY}_{\text{anterior}} - \frac{\Delta}{2}
\end{align}

\textbf{5. Iteración:} El proceso se repitió hasta lograr convergencia, definida como error sistemático $|E| < 5$ mm. La calibración convergió en tres iteraciones.

\subsubsection{Resultados de Calibración}

Los valores finales calibrados fueron \texttt{TX\_ANT\_DLY} = 16388 y \texttt{RX\_ANT\_DLY} = 16458. En la validación final a 1.00 m de distancia, el sistema registró MAE = 3.64 cm ($\sigma$ = 3.09 cm) en condiciones exteriores con línea de vista despejada, correspondiente al mejor caso de referencia (dispositivo en pecho, exterior, LOS). Este nivel de exactitud se considera adecuado para las aplicaciones de localización wearable objetivo de este trabajo.

Es importante destacar que la calibración fue realizada para la configuración específica de hardware utilizada (módulos DWM1001 con antenas integradas). Diferentes configuraciones de antena o modificaciones en el diseño de PCB requerirían repetir el procedimiento de calibración para mantener la trazabilidad metrológica del sistema.


\section{CARACTERIZACIÓN ANTROPOMÉTRICA DE LOS PARTICIPANTES}
\label{sec:caracteristicas_participantes}

Tal como se indicó en la fase de validación, este trabajo fue conducido como un caso de estudio fundacional con un único sujeto de prueba. Este enfoque permitió una caracterización detallada de los efectos de la ubicación del dispositivo y las condiciones de canal, minimizando la variabilidad inter-sujeto. Se reconoce que para garantizar la generalización de estos hallazgos para aplicaciones wearables robustas, futuros trabajos deberán enfocarse en un plan de pruebas exhaustivo involucrando múltiples sujetos con diversas condiciones corporales y características antropométricas.

\section{ANÁLISIS DEL ERROR DE DISTANCIA POR UBICACIÓN CORPORAL}
\label{sec:error_por_ubicacion}

En esta sección se presentan los resultados de la Fase 1, correspondientes a las mediciones de distancia entre un nodo móvil (ubicado en diferentes partes del cuerpo) y un nodo fijo de referencia. Estas mediciones constituyen la base fundamental para el sistema de posicionamiento 2D, ya que la exactitud de la localización depende directamente de la calidad de las estimaciones de distancia hacia cada nodo ancla.

Se evaluaron siete ubicaciones corporales del dispositivo móvil: cabeza, cadera, mano, muñeca, pecho, rodilla y tobillo, bajo dos condiciones de propagación (LOS y NLOS) en dos escenarios diferentes (exterior e interior). A continuación se presentan los resultados organizados por ubicación corporal.

\subsection{Dispositivo en el Pecho}

El pecho representa una de las ubicaciones más críticas para la evaluación de la obstrucción corporal, ya que el torso constituye la mayor masa de tejido que puede obstruir la señal directa entre el nodo móvil y el nodo fijo.

\begin{table}[ht]
\centering
\caption{Error de Distancia - Dispositivo en Pecho}
\label{tab:error_pecho}
\begin{tabular}{lcc}
\toprule
\rowcolor{headerblue}
\textbf{Escenario} & \textbf{Condición} & \textbf{MAE (cm)} \\
\midrule
\multirow{2}{*}{Exterior}   & LOS & 4.62 \\
                            & NLOS & 63.69 \\
\midrule
\multirow{2}{*}{Interior}   & LOS & 13.74 \\
                            & NLOS & 83.98 \\
\bottomrule
\end{tabular}
\end{table}

Los resultados muestran que el pecho alcanza valores de MAE de 4.62 cm en condiciones LOS exteriores, siendo esta la mejor configuración registrada en todo el estudio. Sin embargo, experimenta una degradación significativa al pasar a condiciones NLOS, alcanzando 63.69 cm en exterior y 83.98 cm en interior (factor de degradación de 13.8× y 6.1× respectivamente). Esta degradación se debe a que el torso bloquea completamente la línea de vista directa, forzando al sistema a estimar la distancia basándose en señales difractadas alrededor del cuerpo o reflejadas por el entorno.

\subsection{Dispositivo en la Cabeza}

La cabeza representa una ubicación estratégica debido a su geometría curva que favorece la difracción de señales UWB. Los resultados demuestran el comportamiento más consistente y predecible entre todas las ubicaciones evaluadas.

\begin{table}[ht]
\centering
\caption{Error de Distancia - Dispositivo en Cabeza}
\label{tab:error_cabeza}
\begin{tabular}{lcccc}
\rowcolor{headerblue}
\toprule
\textbf{Escenario} & \textbf{Condición} & \textbf{MAE (cm)} & \textbf{Desv. Est. (cm)} & \textbf{Error Máx. (cm)} \\
\midrule
\multirow{2}{*}{Exterior} & LOS  & 4.87  & 3.72  & 15 \\
                          & NLOS & 32.05 & 10.16 & 53 \\
\midrule
\multirow{2}{*}{Interior} & LOS  & 10.43 & 9.21  & 48 \\
                          & NLOS & 18.66 & 11.20 & 64 \\
\bottomrule
\end{tabular}
\end{table}

La cabeza alcanza MAE de 4.87 cm en exterior LOS, comparable al pecho. Presenta degradación moderada en NLOS exterior (factor 6.6×) y excelente desempeño en interior con factor de degradación de solo 1.8× entre LOS y NLOS. La geometría esférica del cráneo permite que las señales se difracten efectivamente alrededor de la obstrucción, manteniendo la exactitud incluso en condiciones NLOS.

\subsection{Dispositivo en la Cadera}

La cadera exhibe el comportamiento más variable del estudio, con excelente desempeño en LOS pero degradación extrema en NLOS.

\begin{table}[ht]
\centering
\caption{Error de Distancia - Dispositivo en Cadera}
\label{tab:error_cadera}
\begin{tabular}{lcccc}
\toprule
\rowcolor{headerblue}
\textbf{Escenario} & \textbf{Condición} & \textbf{MAE (cm)} & \textbf{Desv. Est. (cm)} & \textbf{Error Máx. (cm)} \\
\midrule
\multirow{2}{*}{Exterior}   & LOS & 8.31 & 5.34 & 25 \\
                            & NLOS & 94.78 & 52.46 & 189 \\
\midrule
\multirow{2}{*}{Exterior}   & LOS & 7.19 & 7.86 & 37 \\
                            & NLOS & 97.76 & 71.00 & 290 \\
\bottomrule
\end{tabular}
\end{table}

La cadera presenta los mayores errores máximos registrados: 189 cm (exterior NLOS) y 290 cm (interior NLOS), con factores de degradación de 11.4× y 13.6× respectivamente. Esta ubicación sufre obstrucción severa por la pelvis y el torso, generando caminos de propagación altamente variables con elevada dispersión estadística.

\subsection{Dispositivo en la Rodilla}

La rodilla muestra errores consistentemente elevados en todas las configuraciones, con alta variabilidad especialmente en condiciones NLOS.

\begin{table}[ht]
\centering
\caption{Error de Distancia - Dispositivo en Rodilla}
\label{tab:error_rodilla}
\begin{tabular}{lcccc}
\toprule
\rowcolor{headerblue}
\textbf{Escenario} & \textbf{Condición} & \textbf{MAE (cm)} & \textbf{Desv. Est. (cm)} & \textbf{Error Máx. (cm)} \\
\midrule
\multirow{2}{*}{Exterior}   & LOS & 14.99 & 10.59 & 46 \\
                            & NLOS & 55.54 & 36.11 & 131 \\
\midrule
\multirow{2}{*}{Interior}   & LOS & 12.04 & 8.43 & 35 \\
                            & NLOS & 49.55 & 37.68 & 181 \\
\bottomrule
\end{tabular}
\end{table}

La rodilla presenta MAE superiores a 12 cm incluso en LOS, y alcanza 55.54 cm en exterior NLOS (factor 3.7×) y 49.55 cm en interior NLOS (factor 4.1×). La proximidad al suelo y la obstrucción por las extremidades inferiores generan multitrayecto complejo y atenuación significativa.

\subsection{Dispositivo en el Tobillo}

El tobillo exhibe el comportamiento más atípico, con errores elevados incluso en LOS y un fenómeno contraintuitivo de mejora en NLOS interior.

\begin{table}[ht]
\centering
\caption{Error de Distancia - Dispositivo en Tobillo}
\label{tab:error_tobillo}
\begin{tabular}{lcccc}
\toprule
\rowcolor{headerblue}
\textbf{Escenario} & \textbf{Condición} & \textbf{MAE (cm)} & \textbf{Desv. Est. (cm)} & \textbf{Error Máx. (cm)} \\
\midrule
\multirow{2}{*}{Exterior}   & LOS & 29.87 & 20.86 & 105 \\
                            & NLOS & 41.86 & 19.44 & 85 \\
\midrule
\multirow{2}{*}{Interior}   & LOS & 29.41 & 28.67 & 81 \\
                            & NLOS & 38.67 & 15.89 & 147 \\
\bottomrule
\end{tabular}
\end{table}

El tobillo presenta MAE superiores a 29 cm en todas las configuraciones. En interior NLOS (38.67 cm) supera el desempeño de Exterior NLOS (41.86 cm). Este fenómeno sugiere que las reflexiones del piso y paredes en el corredor compensan parcialmente la atenuación por obstrucción corporal.

\subsection{Dispositivo en Mano y Muñeca}

Mano y muñeca mantienen desempeño moderado y estable en la mayoría de configuraciones.

\begin{table}[ht]
\centering
\caption{Error de Distancia - Dispositivo en Mano y Muñeca}
\label{tab:error_mano_muneca}
\begin{tabular}{lllccc}
\toprule
\rowcolor{headerblue}
\textbf{Ubicación} & \textbf{Escenario} & \textbf{Condición} & \textbf{MAE(cm)} & \textbf{Std(cm)} & \textbf{Error Máx.(cm)} \\
\midrule
\multirow{4}{*}{Mano} & \multirow{2}{*}{Exterior}& LOS & 8.98 & 6.39 & 32 \\
&  & NLOS & 41.60 & 12.14 & 73 \\
\cmidrule{3-6}
& \multirow{2}{*}{Interior} & LOS & 9.47 & 7.74 & 35 \\
&  & NLOS & 26.01 & 23.92 & 156 \\
\midrule
\multirow{4}{*}{Muñeca} & \multirow{2}{*}{Exterior} & LOS & 7.78 & 5.99 & 32 \\
&  & NLOS & 24.07 & 7.71 & 45 \\
\cmidrule{3-6}
& \multirow{2}{*}{Interior} & LOS & 6.42 & 4.09 & 19 \\
&  & NLOS & 22.00 & 13.21 & 81 \\
\bottomrule
\end{tabular}
\end{table}

Mano y muñeca presentan MAE entre 6.42 cm y 9.47 cm en LOS, con degradación moderada en NLOS (factores entre 2.9× y 4.6×). La muñeca muestra mejor consistencia que la mano, con menor variabilidad y errores máximos más bajos. Ambas ubicaciones se benefician de la movilidad natural de las extremidades superiores que reduce la obstrucción persistente.

\subsection{Comparativo General}

La Tabla \ref{tab:resumen_todas_ubicaciones} consolida los resultados de todas las ubicaciones evaluadas.

\begin{table}[ht]
\centering
\caption{Resumen de MAE por Ubicación Corporal y Condición - Fase 1}
\label{tab:resumen_todas_ubicaciones}
\begin{tabular}{lcccc}
\toprule
\multirow{2}{*}{\cellcolor{headerblue}}
& \multicolumn{2}{c}{\cellcolor{headerblue}\textbf{Exterior}}
& \multicolumn{2}{c}{\cellcolor{headerblue}\textbf{Interior}} \\


\cellcolor{headerblue}\textbf{Ubicación}& \cellcolor{headerblue}\textbf{LOS (cm)}
& \cellcolor{headerblue}\textbf{NLOS (cm)}
& \cellcolor{headerblue}\textbf{LOS (cm)}
& \cellcolor{headerblue}\textbf{NLOS (cm)} \\
\midrule
Cabeza   & 4.87 & 32.43 & 8.91 & 18.16 \\
Cadera   & 8.31 & 89.99 & 7.22 & 95.11 \\
Mano     & 9.28 & 42.17 & 9.09 & 26.01 \\
Muñeca   & 7.78 & 24.21 & 6.42 & 22.00 \\
Pecho    & 4.62 & 63.69 & 13.74 & 83.98 \\
Rodilla  & 14.99 & 51.36 & 12.04 & 49.52 \\
Tobillo  & 29.81 & 38.45 & 41.48 & 29.96 \\
\bottomrule
\end{tabular}
\end{table}


\textbf{Hallazgos principales:}

\begin{itemize}
\item \textbf{Cabeza:} Muestra el comportamiento más consistente entre todas las ubicaciones evaluadas. En exteriores presenta degradación moderada entre LOS (4.87 cm) y NLOS (32.43 cm) con factor 6.7×. En interiores mantiene valores relativamente bajos: 8.91 cm (LOS) y 18.16 cm (NLOS) con factor 2.0×. La forma curva de la cabeza favorece la difracción de señales UWB alrededor del cráneo.

\item \textbf{Cadera:} Exhibe el desempeño más variable del estudio. Alcanza valores excelentes en LOS (8.31 cm exterior, 7.22 cm interior), pero experimenta la degradación más severa en NLOS con MAE de 89.99 cm (exterior) y 95.11 cm (interior), representando factores de degradación de 10.8× y 13.2× respectivamente.

\item \textbf{Rodilla:} Presenta errores consistentemente elevados en todas las configuraciones. Los valores en LOS son moderados (14.99 cm exterior, 12.04 cm interior), pero en NLOS alcanza 51.36 cm (exterior) y 49.52 cm (interior), con factores de degradación de 3.4× y 4.1×.

\item \textbf{Tobillo:} Muestra un coportamiento interesante del estudio. Presenta MAE elevados en exteriores NLOS que en el escenario interior NLOS, por lo que este último exhibe mejor desempeño.

\item \textbf{Mano y Muñeca:} Mantienen desempeño moderado y consistente. La mano registra 9.28 cm (exterior LOS) y 9.09 cm (interior LOS), degradándose a 42.17 cm y 26.01 cm en NLOS respectivamente. La muñeca presenta valores similares: 7.78 cm (exterior LOS), 6.42 cm (interior LOS), degradándose a 24.21 cm y 22.00 cm en NLOS. Ambas ubicaciones muestran factores de degradación entre 2.9× y 4.5×.
\end{itemize}


\section{ANÁLISIS ESTADÍSTICO GLOBAL}
\label{sec:distribuciones_error}

\subsection{Métricas Globales del Sistema}

Considerando las 28 configuraciones evaluadas (7 ubicaciones × 2 condiciones × 2 escenarios), el análisis global del sistema de estimación de distancia arroja las siguientes métricas:

\begin{itemize}
\item \textbf{MAE Global:} 0.3944 m
\item \textbf{RMSE Global:} 0.6690 m  
\item \textbf{Percentil 95:} 1.1424 m
\end{itemize}

El percentil 95 indica que el 95\% de todas las mediciones presentan errores inferiores a 1.14 m, proporcionando una medida de la variabilidad extrema del sistema. Se observó una tendencia sistemática a sobreestimar las distancias, fenómeno atribuible a efectos de multitrayecto y tiempos de procesamiento de señal en los dispositivos UWB.

\subsection{Distribución Probabilística de Errores}

La distribución probabilística de los errores evidencia una asimetría positiva, con una cola extendida hacia valores de error elevados, lo que indica un comportamiento claramente no gaussiano en configuraciones específicas, particularmente bajo condiciones severas de no línea de vista (NLOS). Este comportamiento sugiere que los modelos de error tradicionales basados en distribuciones gaussianas resultan inadecuados para caracterizar el desempeño de sistemas UWB en presencia de obstrucción corporal.

Uno de los gráficos que permite observar este fenómeno es el diagrama de cajas y bigotes, mostrado en la Figura~\ref{fig:box_plot_los_nlos}, donde se aprecia de manera clara el contraste en la distribución del error al pasar de condiciones LOS a NLOS.

\begin{figure}[hbt]
    \centering
    \includegraphics[width=0.8\linewidth]{imagenes/boxplot_los_nlos.png}
    \caption{Distribución probabilística de los errores en condiciones LOS y NLOS}
    \label{fig:box_plot_los_nlos}
\end{figure}

Con base en la Figura~\ref{fig:box_plot_los_nlos}, se observa que, cuando se presenta obstrucción corporal, el error puede alcanzar valores de hasta aproximadamente 5 m. Este comportamiento implica que ciertas ubicaciones corporales introducen una distorsión significativa en la señal, atenuándola de tal forma que el sistema pierde confiabilidad y su desempeño se degrada considerablemente.

\begin{figure}[hbt]
    \centering
    \includegraphics[width=0.8\linewidth]{imagenes/exactitud_vs_precision_NLOS.png}
    \caption{Exactitud versus precisión del MAE en configuración NLOS}
    \label{fig:exactitud_vs_precision_nlos}
\end{figure}

Al analizar el gráfico de exactitud versus precisión, presentado en la Figura~\ref{fig:exactitud_vs_precision_nlos}, se observa que las ubicaciones correspondientes a la cadera y el pecho presentan simultáneamente los valores más altos de MAE y de desviación estándar. Este resultado es consistente con la estructura anatómica de dichas partes del cuerpo, la cual dificulta la propagación de la señal y limita severamente su llegada al receptor, generando así un desempeño deficiente del sistema.

Si bien estos resultados proporcionan una primera caracterización del fenómeno, resulta necesario extender el análisis a escenarios en interiores, que constituyen el foco principal del presente trabajo. En este contexto, el siguiente paso consiste en evaluar el comportamiento del MAE bajo configuración NLOS en entornos interiores, tal como se muestra en la Figura~\ref{fig:exactitud_vs_precision_nlos_interiores}.

\begin{figure}[hbt]
    \centering
    \includegraphics[width=0.8\linewidth]{imagenes/exactitud_vs_precision_nlos_pasillo.png}
    \caption{Exactitud versus precisión del MAE en configuración NLOS en interiores}
    \label{fig:exactitud_vs_precision_nlos_interiores}
\end{figure}

La Figura~\ref{fig:exactitud_vs_precision_nlos_interiores} muestra que la mayor parte de la variabilidad observada en condiciones NLOS está dominada por el escenario del Interiores. En este contexto, tanto la cadera como el pecho continúan presentando errores absolutos elevados y altas desviaciones estándar, confirmando su bajo rendimiento en términos de exactitud y precisión.

Un hallazgo relevante es que las ubicaciones correspondientes a la cabeza, la mano y la muñeca presentan consistentemente valores de MAE y desviación estándar cercanos a los umbrales establecidos. Este comportamiento constituye una evidencia del fenómeno conocido como \textit{multitrayecto constructivo}, en el cual ciertas configuraciones geométricas favorecen la superposición coherente de trayectorias reflejadas, mejorando el desempeño del sistema.

Si bien estos resultados representan un primer hallazgo significativo, resulta pertinente profundizar el análisis mediante la aplicación de pruebas estadísticas formales, tales como ANOVA y la prueba post-hoc de Tukey o pruebas no paramétricas. La inclusión de estas pruebas permitiría sustentar de manera más robusta las conclusiones sobre la BS y relacionadas con el fenómeno de multitrayecto constructivo y su impacto en el desempeño del sistema.




\subsection{Análisis Estadístico No Paramétrico}

\subsubsection{Justificación de la Metodología No Paramétrica}

Un análisis preliminar de la distribución del Error Absoluto (AE) de las mediciones reveló una desviación significativa de la normalidad. La prueba de Shapiro-Wilk sobre una muestra representativa de los datos arrojó un p-valor extremadamente bajo ($p \ll 0.001$), lo que obliga a rechazar la hipótesis de normalidad. Adicionalmente, la prueba de Levene para la homogeneidad de varianzas entre las distintas ubicaciones corporales también resultó altamente significativa ($p \ll 0.001$), indicando que las varianzas no son iguales entre los grupos.

Dada la violación de los supuestos de normalidad y homocedasticidad, el uso de pruebas paramétricas como ANOVA o t-test sería metodológicamente incorrecto. Por consiguiente, todo el análisis inferencial se fundamenta en pruebas no paramétricas, específicamente la prueba U de Mann-Whitney, que es robusta ante estas condiciones y permite comparar distribuciones entre grupos independientes y levene para detectar diferencias significativas.

\subsubsection{Análisis de Configuraciones Extremas y Relevancia Práctica}

Para evaluar la magnitud de las diferencias de rendimiento, se compararon las dos mejores configuraciones (menor MAE y dispersión) con las dos peores (mayor MAE y dispersión), tanto en condiciones de Línea de Vista (LOS) como de No Línea de Vista (NLOS). La Tabla~\ref{tab:extremos} resume los resultados.

\begin{table}[h!]
\centering
\caption{Comparación de rendimiento entre las configuraciones extremas. Se reporta el MAE, su Intervalo de Confianza (IC) del 95\% estimado por bootstrap, y el tamaño del efecto (correlación rank-biserial, $r$).}
\label{tab:extremos}
\begin{tabular}{llcccc}
\toprule
\rowcolor{headerblue}
\textbf{Condición} & \textbf{Grupo} & \textbf{MAE [m]} & \textbf{IC 95\% [m]} & \textbf{Tamaño Efecto ($r$)} & \textbf{$p$-valor} \\
\midrule
\addlinespace
LOS & Mejores & 0.047 & [0.047, 0.048] & \multirow{2}{*}{0.705} & \multirow{2}{*}{$\ll 0.001$} \\
 & Peores & 0.282 & [0.276, 0.287] & & \\
\addlinespace
\midrule
\addlinespace
NLOS & Mejores & 0.101 & [0.100, 0.103] & \multirow{2}{*}{0.968} & \multirow{2}{*}{$\ll 0.001$} \\
 & Peores & 0.935 & [0.923, 0.947] & & \\
\bottomrule
\end{tabular}
\end{table}

Los resultados muestran que las diferencias son estadísticamente significativas en ambos casos. Sin embargo, el análisis va más allá del p-valor: el tamaño del efecto es \textbf{grande} en ambas condiciones ($r > 0.5$), y los intervalos de confianza para los grupos de 'Mejores' y 'Peores' no se superponen en absoluto. Esto demuestra de manera concluyente que las diferencias no solo son estadísticamente detectables, sino también de una magnitud muy relevante desde el punto de vista práctico. La condición NLOS exhibe una diferencia de casi un orden de magnitud en el MAE, con un tamaño de efecto cercano a la máxima disimilitud ($r \approx 1$).

\subsubsection{Impacto del Escenario en Condiciones NLOS}

Con el fin de analizar cómo el entorno influye en el desempeño del sistema bajo las condiciones más adversas, se realizó una comparación pareada entre los escenarios \textit{Exteriores} y \textit{Interiores} para cada ubicación corporal, considerando exclusivamente mediciones en condición NLOS. Los resultados de este análisis se presentan en la Tabla~\ref{tab:nlos_escenario}.

\begin{table}[h!]
\centering
\caption{Análisis comparativo del MAE en condiciones NLOS entre los escenarios \textit{Exteriores} y \textit{Interiores} para cada ubicación corporal.}
\label{tab:nlos_escenario}
\begin{tabular}{lccccc}
\toprule
\rowcolor{headerblue}
\textbf{Ubicación} & \textbf{Exteriores[m]} & \textbf{Interiores[m]} & \textbf{$p$-valor} & Efecto ($r$) & \textbf{Magnitud} \\
\midrule
CABEZA  & 0.320 & 0.187 & $\ll 0.001$ & -0.660 & grande  \\
CADERA  & 0.948 & 0.978 & 0.028       &  0.049 & trivial \\
MANO    & 0.416 & 0.260 & $\ll 0.001$ & -0.693 & grande  \\
MUÑECA  & 0.241 & 0.220 & $\ll 0.001$ & -0.273 & pequeño \\
PECHO   & 0.790 & 0.895 & 0.590       & -0.012 & trivial \\
RODILLA & 0.555 & 0.496 & $\ll 0.001$ & -0.392 & medio   \\
TOBILLO & 0.389 & 0.300 & $\ll 0.001$ & -0.488 & medio   \\
\bottomrule
\end{tabular}
\end{table}

\subsubsection{Discusión e Interpretación Física}

Un resultado relevante y, a primera vista, contraintuitivo de la Tabla~\ref{tab:nlos_escenario} es que para varias ubicaciones corporales la mediana del MAE es \emph{menor en el escenario Interiores que en Exteriores}, a pesar de que el Interiores constituye un entorno más confinado. Este comportamiento no representa una inconsistencia estadística, sino una manifestación clara de la interacción entre obstrucción corporal y propagación por multitrayecto.

En ubicaciones del torso como \textbf{CADERA} y \textbf{PECHO}, el error es elevado en ambos escenarios y las diferencias observadas son estadísticamente triviales. Esto indica que, en estas posiciones, la \textbf{obstrucción corporal severa} domina completamente el proceso de propagación, atenuando la señal de forma tal que el entorno circundante tiene un impacto marginal. En otras palabras, una vez que la línea de vista está fuertemente bloqueada por el cuerpo, ni un entorno abierto ni uno confinado logra mitigar significativamente el error.

En contraste, para ubicaciones en extremidades como \textbf{CABEZA}, \textbf{MANO}, \textbf{RODILLA} y \textbf{TOBILLO}, se observa un comportamiento sistemático donde el error es mayor en \textit{Exteriores} que en \textit{Interiores}, con tamaños de efecto que van de moderados a grandes. Este resultado se explica por el fenómeno de \textbf{multitrayecto constructivo}. En un Interiores, las reflexiones en paredes, suelo y techo generan múltiples trayectorias indirectas que pueden compensar parcialmente la obstrucción de la señal directa, reduciendo el error mediano. En espacios abiertos, estas superficies reflectantes están ausentes o demasiado alejadas, por lo que, ante una obstrucción corporal, el sistema carece de trayectorias alternativas robustas, incrementando el error.

El caso de la ubicación \textbf{MANO} es particularmente ilustrativo: presenta un tamaño del efecto grande ($r = -0.693$), lo que indica que el entorno tiene una influencia práctica muy relevante. Esto sugiere que la mano, al ser una extremidad altamente móvil y con geometría variable, es especialmente sensible a la disponibilidad de trayectorias reflejadas.

En síntesis, el hecho de que el escenario \textit{Interiores} presente MAE menor que \textit{Exteriores} para varias ubicaciones no constituye una anomalía, sino una evidencia clara de que, en condiciones NLOS, un entorno confinado puede favorecer el rendimiento mediante multitrayecto constructivo. Estos resultados refuerzan la necesidad de analizar conjuntamente la ubicación corporal y el entorno, y tienen implicaciones directas para el diseño y despliegue de sistemas UWB, particularmente en aplicaciones donde la NLOS es inevitable.

\section{Análisis de la Distribución de Probabilidad del AE}
\label{sec:analisis_distribucion_mae}

Con el objetivo de caracterizar formalmente el comportamiento del error de medición y validar la hipótesis de no normalidad sugerida en las secciones anteriores, se realizó un análisis de ajuste de distribuciones probabilísticas sobre el conjunto total de datos experimentales ($N=101552$ muestras). Se evaluaron seis distribuciones teóricas candidatas: Normal, Log-normal, Gamma, Weibull, Exponencial y Uniforme. El ajuste se determinó mediante el Criterio de Información de Akaike (AIC) y el Criterio de Información Bayesiano (BIC), donde valores menores indican un mejor compromiso entre la fidelidad del ajuste y la complejidad del modelo.

\subsection{Ajuste Global de Datos}

El análisis del conjunto global de datos confirma de manera concluyente que la distribución Normal es inadecuada para modelar el error de distancia en sistemas UWB bajo condiciones mixtas (LOS/NLOS). El análisis del ajuste revela que la distribución Normal presenta valores de AIC significativamente superiores (peor ajuste) en comparación con distribuciones asimétricas como Gamma o Log-normal.

La Figura~\ref{fig:ajuste_todos} presenta el histograma del error absoluto para la totalidad de las mediciones, superpuesto con las curvas de densidad de probabilidad ajustadas. Se evidencia claramente la asimetría positiva de los datos experimentales, característica que es capturada eficazmente por la distribución Gamma (mejor ajuste global), mientras que la curva Normal subestima la densidad en la región cercana a cero y no logra modelar adecuadamente la cola pesada de la distribución.

\begin{table}[ht]
\centering
\caption{Ajuste para el conjunto total de datos ($N=101,552$). Se ordenan las distribuciones por el criterio AIC de menor a mayor (mejor ajuste).}
\label{tab:ajuste_todos}
\begin{tabular}{lcccc}
\toprule
\rowcolor{headerblue}
\textbf{Distribución} & \textbf{AIC} & \textbf{BIC} & \textbf{KS Stat} & \textbf{P-value} \\
\midrule
Gamma       & -125,845.02 & -125,816.43 & 0.2261 & 0.00 \\
Log-normal  & -83,510.28  & -83,481.70  & 0.0378 & 0.00 \\
Exponencial & -79,051.37  & -79,032.31  & 0.1008 & 0.00 \\
Normal      & 55,016.18   & 55,035.24   & 0.2160 & 0.00 \\
Uniforme    & 339,868.33  & 339,887.39  & 0.8005 & 0.00 \\
Weibull     & 2,289,166.27& 2,289,194.85& 0.0561 & 0.00 \\
\bottomrule
\end{tabular}
\end{table}

\begin{figure}[hbt]
    \centering
    \includegraphics[width=0.9\linewidth]{imagenes/ajuste_hist_Todos_los_Datos.png}
    \caption{Histograma y ajuste de distribuciones para el conjunto total de datos.}
    \label{fig:ajuste_todos}
\end{figure}

\subsection{Análisis Comparativo LOS vs. NLOS}

Al segmentar los datos por condición de propagación, se revelan diferencias fundamentales en la estructura probabilística del error.

\textbf{Condición de Línea de Vista (LOS):} En condiciones LOS ($N=56,571$), la distribución Gamma se mantiene como el modelo óptimo (ver Figura~\ref{fig:ajuste_los}). El error se concentra fuertemente en valores bajos, con una dispersión controlada. Aunque la asimetría persiste, es menos pronunciada que en el caso global.

\begin{figure}[hbt]
    \centering
    \includegraphics[width=0.9\linewidth]{imagenes/ajuste_hist_Datos_LOS.png}
    \caption{Ajuste de distribuciones para mediciones en condición LOS.}
    \label{fig:ajuste_los}
\end{figure}

La Tabla~\ref{tab:ajuste_los} detalla las métricas de ajuste para el escenario LOS, donde la distribución Gamma minimiza tanto el AIC como el BIC.

\begin{table}[ht]
\centering
\caption{Ajuste para condiciones de Línea de Vista (LOS).}
\label{tab:ajuste_los}
\begin{tabular}{lcccc}
\toprule
\rowcolor{headerblue}
\textbf{Distribución} & \textbf{AIC} & \textbf{BIC} & \textbf{KS Stat} & \textbf{P-value} \\
\midrule
Gamma       & -132,051.35 & -132,024.52 & 0.0883 & 0.00 \\
Log-normal  & -124,128.73 & -124,101.90 & 0.0551 & 0.00 \\
Exponencial & -123,109.07 & -123,091.19 & 0.0746 & 0.00 \\
Normal      & -61,789.28  & -61,771.40  & 0.2069 & 0.00 \\
\bottomrule
\end{tabular}
\end{table}

\textbf{Condición de No Línea de Vista (NLOS):} En condiciones NLOS ($N=44,981$), la distribución del error sufre una transformación drástica. La Figura~\ref{fig:ajuste_nlos} muestra una dispersión mucho mayor y una cola pesada significativa hacia la derecha. El análisis de AIC muestra que la Log-normal supera ligeramente a la Gamma como mejor ajuste. Esto indica efectos multiplicativos por obstrucción y multitrayecto, típicos de procesos log-normales

\begin{figure}[hbt]
    \centering
    \includegraphics[width=0.9\linewidth]{imagenes/ajuste_hist_Datos_NLOS.png}
    \caption{Ajuste de distribuciones para mediciones en condición NLOS.}
    \label{fig:ajuste_nlos}
\end{figure}

Como se observa en la Tabla~\ref{tab:ajuste_nlos}, en condiciones NLOS la distribución Log-normal presenta el mejor ajuste (menor AIC), desplazando a la Gamma.

\begin{table}[ht]
\centering
\caption{Ajuste para condiciones de No Línea de Vista (NLOS).}
\label{tab:ajuste_nlos}
\begin{tabular}{lcccc}
\toprule
\rowcolor{headerblue}
\textbf{Distribución} & \textbf{AIC} & \textbf{BIC} & \textbf{KS Stat} & \textbf{P-value} \\
\midrule
Log-normal  & 7,145.36    & 7,171.50    & 0.0522 & 0.00 \\
Gamma       & 7,762.93    & 7,789.07    & 0.0663 & 0.00 \\
Exponencial & 9,075.48    & 9,092.91    & 0.0717 & 0.00 \\
Normal      & 44,631.39   & 44,648.82   & 0.1829 & 0.00 \\
\bottomrule
\end{tabular}
\end{table}

Para visualizar mejor la desviación respecto a la normalidad en las colas de la distribución (donde ocurren los errores grandes, críticos para la confiabilidad del sistema), se presenta el gráfico Q-Q (Quantile-Quantile) para los datos NLOS en la Figura~\ref{fig:qq_nlos}. La divergencia de los puntos experimentales (azul) respecto a la línea roja de referencia (distribución normal teórica) en los extremos confirma que los errores grandes son mucho más frecuentes de lo que predeciría un modelo gaussiano.

\begin{figure}[hbt]
    \centering
    \includegraphics[width=0.9\linewidth]{imagenes/ajuste_qq_Datos_NLOS.png}
    \caption{Gráfico Q-Q comparando los datos NLOS con una distribución Normal teórica.}
    \label{fig:qq_nlos}
\end{figure}

\subsection{Parametrización de los Modelos Ajustados}
\label{sec:parametrizacion_modelos}

Con base en el análisis de bondad de ajuste, se determinaron los parámetros óptimos para los modelos probabilísticos que describen el error de medición. A continuación se presentan las ecuaciones teóricas de las distribuciones seleccionadas y la interpretación física de sus parámetros, conforme al marco teórico utilizado para la caracterización.

\subsubsection{Distribuciones de Probabilidad Teóricas}

\textbf{1. Distribución Gamma} (Frecuentemente óptima para LOS y errores asimétricos):
\begin{equation}
f(x) = \frac{(x-\gamma)^{\alpha-1} e^{-(x-\gamma)/\theta}}{\theta^\alpha \Gamma(\alpha)}
\end{equation}
Donde:
\begin{itemize}
    \item $x$: Variable aleatoria (Error de posicionamiento en metros).
    \item $\alpha$ (o $k$): Parámetro de forma. Define la curtosis o qué tan "picuda" es la curva.
    \item $\theta$ (o \textit{scale}): Parámetro de escala. Define la dispersión hacia la derecha.
    \item $\Gamma(\cdot)$: Función Gamma.
    \item $\gamma$ (o \textit{loc}): Desplazamiento del origen.
\end{itemize}

\textbf{2. Distribución Weibull} (Común en análisis de confiabilidad):
\begin{equation}
f(x) = \frac{k}{\lambda} \left(\frac{x-\gamma}{\lambda}\right)^{k-1} \exp\left(-\left(\frac{x-\gamma}{\lambda}\right)^k\right)
\end{equation}
Donde,
\begin{itemize}
    \item $k$ (o \textit{shape}): Parámetro de forma. Si $k < 1$, la tasa de probabilidad decrece.
    \item $\lambda$ (o \textit{scale}): Parámetro de escala.
    \item $\gamma$ (o \textit{loc}): Desplazamiento del origen.
\end{itemize}

\textbf{3. Distribución Log-normal} (Común en NLOS y procesos multiplicativos):
\begin{equation}
f(x) = \frac{1}{(x-\gamma) \sigma \sqrt{2\pi}} \exp\left(-\frac{(\ln(x-\gamma) - \mu)^2}{2\sigma^2}\right)
\end{equation}
Donde:
\begin{itemize}
    \item $\mu$: Media del logaritmo de los datos.
    \item $\sigma$ (o \textit{shape}): Desviación estándar del logaritmo.
    \item $\gamma$ (o \textit{loc}): Desplazamiento.
\end{itemize}

\subsubsection{Metodología de Obtención de Parámetros}

El paso de los datos experimentales a los parámetros del modelo se realizó mediante el \textbf{Método de Máxima Verosimilitud (MLE)}. A diferencia del método de momentos (que iguala medias y varianzas), el MLE busca iterativamente los valores de los parámetros que maximizan la probabilidad matemática de haber observado exactamente la muestra de datos recolectada ($x_1, x_2, ..., x_n$).

No obstante, existe una relación conceptual entre la estadística descriptiva y los parámetros obtenidos:
\begin{itemize}
    \item \textbf{Caso Gamma:} $\alpha \approx E[x]^2 / Var[x]$ y $\theta \approx Var[x] / E[x]$. Si la varianza es alta respecto a la media, $\alpha$ tiende a ser bajo.
    \item \textbf{Caso Log-normal:} Los parámetros surgen de transformar los datos $y = \ln(x-\gamma)$, donde $\mu = \bar{y}$ y $\sigma = S_y$.
    \item \textbf{Caso Weibull:} $\lambda$ representa un valor característico cercano a la media, mientras que $k$ controla la pendiente de la distribución.
\end{itemize}

La Tabla~\ref{tab:modelos_parametricos} resume los parámetros estimados para las condiciones más relevantes, junto con una validación cruzada comparando el MAE teórico del modelo con el experimental.

\begin{table}[ht]
\centering
\scriptsize
\caption{Parámetros de los modelos de error ajustados para escenarios globales y específicos.}
\label{tab:modelos_parametricos}
\begin{tabular}{lllcc}
\toprule
\rowcolor{headerblue}
\textbf{Escenario} & \textbf{Modelo} & \textbf{Parámetros Estimados} & \textbf{MAE Exp.[m]} & \textbf{MAE Teor.[m]} \\
\midrule
\textbf{Global LOS} & Gamma & $\alpha=0.9407$, $\theta=0.1292$ & 0.1239 & 0.1215 \\
\addlinespace
\textbf{Global NLOS} & Log-normal & $\sigma=0.8283$, $\mu=-1.1511$ & 0.4069 & 0.4115 \\
\addlinespace
\textbf{Cadera LOS} & Weibull & $k=1.5768$, $\lambda=0.0901$ & 0.0771 & 0.0769 \\
\addlinespace
\textbf{Cadera NLOS} & Gamma & $\alpha=3.4311$, $\theta=0.2464$ & 0.9723 & 0.9723 \\
\addlinespace
\textbf{Pecho LOS} & Weibull & $k=0.8679$, $\lambda=0.1265$ & 0.1117 & 0.1359 \\
\addlinespace
\textbf{Pecho NLOS} & Gamma & $\alpha=2.5971$, $\theta=0.3542$ & 0.8783 & 0.8783 \\
\bottomrule
\end{tabular}
\end{table}

A continuación se detallan las ecuaciones de densidad de probabilidad parametrizadas para los casos globales, las cuales constituyen insumo directo para simulaciones y diseño de filtros de navegación.

\subsubsection{Modelo Global para Condiciones de Línea de Vista (LOS)}

El error en condiciones LOS se modela óptimamente mediante una distribución Gamma con parámetros de forma $\alpha \approx 0.94$ y escala $\theta \approx 0.13$. La función de densidad de probabilidad (PDF) está dada por:

\begin{equation}
f_{LOS}(x) = \frac{x^{-0.0593} e^{-x/0.1292}}{0.1292^{0.9407} \Gamma(0.9407)}
\end{equation}

El modelo predice un valor esperado del error de 0.1215 m, lo cual difiere del valor experimental (0.1239 m) en apenas un 1.95\%, validando la alta fidelidad del modelo para representar el comportamiento nominal del sistema.

\subsubsection{Modelo Global para Condiciones de No Línea de Vista (NLOS)}

En condiciones NLOS, el error sigue una distribución Log-normal, caracterizada por parámetros $\sigma \approx 0.83$ y $\mu \approx -1.15$. La PDF se expresa como:

\begin{equation}
f_{NLOS}(x) = \frac{1}{(x + 0.0342) \cdot 0.8283 \sqrt{2\pi}} \exp\left(-\frac{(\ln(x + 0.0342) + 1.1511)^2}{1.3722}\right)
\end{equation}

Es importante notar que este modelo incluye un parámetro de desplazamiento (\textit{loc}) de -0.0342 m para optimizar el ajuste. El modelo predice un MAE teórico de 0.4115 m frente a un experimental de 0.4069 m (diferencia de 1.14\%), demostrando su capacidad para capturar la naturaleza de cola pesada de los errores por obstrucción.

\subsubsection{Modelos Específicos por Ubicación}

El análisis desagregado para las ubicaciones más críticas (Cadera y Pecho) revela patrones específicos que difieren del modelo global, lo cual justifica el uso de modelos adaptativos si se conoce la ubicación del dispositivo.

\paragraph{Ubicación: Cadera}
En condiciones LOS, la distribución Weibull ofrece el mejor ajuste, con un parámetro de forma $k \approx 1.58$. En contraste, para NLOS, el error sigue una distribución Gamma desplazada significativamente ($\gamma \approx 0.12$ m) y con un parámetro de forma alto ($\alpha \approx 3.43$), lo que refleja una distribución más simétrica y alejada del cero que el caso global.

\begin{itemize}
    \item \textbf{Cadera LOS (Weibull):}
    \begin{equation}
    f(x) = \frac{1.5768}{0.0901} \left(\frac{x + 0.0040}{0.0901}\right)^{0.5768} \exp\left(-\left(\frac{x + 0.0040}{0.0901}\right)^{1.5768}\right)
    \end{equation}
    
    \item \textbf{Cadera NLOS (Gamma):}
    \begin{equation}
    f(x) = \frac{(x - 0.1268)^{2.4311} e^{-(x - 0.1268)/0.2464}}{0.2464^{3.4311} \Gamma(3.4311)}
    \end{equation}
\end{itemize}

\paragraph{Ubicación: Pecho}
Similar a la cadera, el pecho en LOS se modela mejor con una Weibull, pero con $k < 1$ ($k \approx 0.87$), lo que sugiere una alta probabilidad de errores pequeños pero con una cola pesada de errores ocasionales grandes incluso con línea de vista. En NLOS, la distribución Gamma nuevamente es dominante.

\begin{itemize}
    \item \textbf{Pecho LOS (Weibull):}
    \begin{equation}
    f(x) = \frac{0.8679}{0.1265} \left(\frac{x}{0.1265}\right)^{-0.1321} \exp\left(-\left(\frac{x}{0.1265}\right)^{0.8679}\right)
    \end{equation}
    
    \item \textbf{Pecho NLOS (Gamma):}
    \begin{equation}
    f(x) = \frac{(x + 0.0416)^{1.5971} e^{-(x + 0.0416)/0.3542}}{0.3542^{2.5971} \Gamma(2.5971)}
    \end{equation}
\end{itemize}








% Capítulo 9: Análisis y Discusión
\chapter{ANÁLISIS Y DISCUSIÓN}
\label{ch:analisis}

En este capítulo se interpretan y discuten los resultados experimentales presentados en el capítulo anterior. Se contrastan los hallazgos con el estado del arte revisado en el capítulo de revisión sistemática de literatura, se explican los fenómenos físicos subyacentes que dan cuenta de los patrones observados, y se analizan las implicaciones prácticas de los resultados para el diseño e implementación de sistemas de posicionamiento UWB en escenarios reales.

\section{INTERPRETACIÓN DE LOS PATRONES DE ERROR OBSERVADOS}
\label{sec:interpretacion_patrones}

\subsection{Efecto de la Ubicación Corporal del Dispositivo}

Los resultados experimentales revelan que la ubicación donde se porta el dispositivo UWB tiene un impacto crítico sobre el error de estimación de distancia. [COMPLETAR: Basándose en los resultados obtenidos, explicar por qué ciertas ubicaciones exhiben mayor/menor error. Por ejemplo:]

La ubicación en el pecho presentó un comportamiento [COMPLETAR: describir patrón observado] debido a que cuando el participante se orienta de frente a un nodo ancla (orientación $\approx$ 0°), existe línea de vista directa entre el dispositivo y el ancla. En contraste, cuando el participante se orienta en dirección opuesta (orientación $\approx$ 180°), el torso completo actúa como obstáculo, forzando a las señales UWB a difractarse alrededor del cuerpo o atravesar tejido biológico con alta constante dieléctrica y conductividad.

La ubicación en la muñeca mostró [COMPLETAR: describir patrón] lo cual se explica porque el brazo presenta menor masa corporal que el torso, y además experimenta mayor movilidad natural. Los movimientos involuntarios del brazo durante las mediciones, aun intentando mantener postura estática, introducen variabilidad adicional que se refleja en mayor desviación estándar del error.

\subsection{Dependencia del Ángulo de Orientación del Cuerpo}

El análisis del error en función del ángulo de orientación del cuerpo muestra una transición clara entre dos regímenes de propagación:

\textbf{Régimen LOS (orientación $\approx$ 0° y 360°):} El error se mantiene [COMPLETAR: bajo/moderado], con una media de [X.X] cm y desviación estándar de [X.X] cm, valores comparables a los obtenidos en la calibración sin obstrucción. Esto confirma que cuando el dispositivo está orientado directamente hacia el nodo ancla, el cuerpo no interfiere significativamente con la propagación de la señal.

\textbf{Régimen NLOS (orientación $\approx$ 90°, 180° y 270°):} La obstrucción aumenta significativamente, con errores que alcanzan [X.X] cm en promedio y picos de hasta [X.X] metros. En estas configuraciones, las señales directas son severamente atenuadas (estimadas en 20-40 dB según la literatura), y los trayectos dominantes corresponden a difracción alrededor del cuerpo o reflexiones en paredes y objetos del entorno. El retardo introducido por estos trayectos más largos se traduce directamente en sobreestimación de la distancia.

\subsection{Comparación con Frecuencias Inferiores}

La banda de 6.5 GHz presenta características de propagación particulares en comparación con las bandas más comúnmente estudiadas (3-5 GHz). [COMPLETAR: Si se puede contrastar con literatura:]

A frecuencias más altas, la atenuación por unidad de longitud en tejido biológico aumenta, lo cual podría sugerir mayor degradación en condiciones de BS. Sin embargo, la longitud de onda más corta también facilita la difracción alrededor de obstáculos con dimensiones corporales. Los resultados obtenidos [COMPLETAR: sugieren que en la banda de 6.5 GHz, el error en NLOS es comparable/ligeramente superior/inferior a los valores reportados en X GHz por Autor et al.].

Adicionalmente, la mayor resolución temporal inherente al mayor ancho de banda disponible en frecuencias altas permite una mejor separación de trayectos múltiples, lo cual podría compensar parcialmente la mayor atenuación. Los algoritmos de detección del primer arribo de señal se benefician de pulsos más estrechos para identificar con mayor exactitud el ToF del trayecto directo o el primer trayecto significativo.


\section{ANÁLISIS DE LA VARIABILIDAD ESTADÍSTICA}
\label{sec:analisis_variabilidad}

\subsection{Naturaleza de las Distribuciones de Error}

El ajuste de distribuciones teóricas reveló que en condiciones LOS, el error se aproxima a una distribución normal, consistente con ruido de medición gaussiano dominado por efectos de cuantización temporal y ruido térmico en los receptores.

En condiciones NLOS, las distribuciones exhiben [COMPLETAR: asimetría positiva/colas pesadas/multimodalidad], lo cual se modeló adecuadamente mediante distribuciones [COMPLETAR: log-normal/gamma/mixtura de gaussianas]. Esta no-gaussianidad refleja la naturaleza determinista del fenómeno de obstrucción corporal: cuando el cuerpo bloquea el trayecto directo, el error no es una fluctuación aleatoria simétrica, sino un sesgo sistemático positivo debido a trayectos de propagación más largos.

La presencia de multimodalidad en algunas condiciones sugiere la coexistencia de múltiples mecanismos de propagación (difracción, reflexión especular, dispersión) que resultan en "familias" de errores con magnitudes características distintas.

\subsection{Implicaciones para Algoritmos de Localización}

La no-gaussianidad del error en NLOS tiene implicaciones importantes para el diseño de algoritmos de localización y filtrado:

Los algoritmos tradicionales de trilateración por mínimos cuadrados asumen ruido gaussiano en las mediciones de distancia. Cuando este supuesto se viola severamente, como en presencia de BS, el estimador de mínimos cuadrados puede producir estimaciones de posición significativamente sesgadas.

Enfoques robustos como mínimos cuadrados ponderados iterativamente (IRLS) o M-estimadores pueden mejorar el desempeño al reducir el peso de mediciones con errores atípicos. Alternativamente, si se puede identificar qué enlaces están en NLOS (mediante técnicas de clasificación basadas en características de la señal), se pueden aplicar modelos de error específicos para cada condición de propagación.

El Filtro de Kalman estándar también asume gaussianidad. Las extensiones como el Filtro de Kalman Robusto o Filtros de Partículas pueden manejar mejor distribuciones de error no gaussianas y multimodales. [COMPLETAR si se implementó: Los resultados de aplicar el KF mostraron que...]


\section{CORRELACIÓN CON CARACTERÍSTICAS ANTROPOMÉTRICAS}
\label{sec:discusion_antropometria}

El análisis de correlación entre el error y las características antropométricas de los participantes reveló [COMPLETAR: basándose en los resultados obtenidos].

Una correlación [positiva/negativa/inexistente] entre el error NLOS y el peso o IMC de los participantes [COMPLETAR: sugiere que individuos con mayor masa corporal causan mayor/menor obstrucción, o que el efecto es independiente de estas variables dentro del rango estudiado].

La estatura mostró [COMPLETAR: interpretación]. Esto puede explicarse porque una mayor altura implica que el dispositivo portado en el torso se encuentra a mayor distancia vertical de los nodos ancla ubicados [ESPECIFICAR altura de montaje], alterando la geometría de propagación.

Es importante notar que con [N] participantes, el poder estadístico para detectar correlaciones débiles es limitado. Estudios futuros con mayor número de participantes permitirían caracterizar con mayor precisión la influencia de la variabilidad interindividual.


\section{DESEMPEÑO DEL SISTEMA DE POSICIONAMIENTO COMPLETO}
\label{sec:discusion_posicionamiento}

\subsection{Exactitud de Localización Alcanzada}

El análisis del error de posición 2D (que integra las mediciones de distancia hacia los cuatro nodos ancla) muestra que el sistema alcanza una exactitud [COMPLETAR: métrica específica, ej. "mediana de X cm y percentil 95 de Y cm"] cuando el dispositivo se porta en [ubicación óptima].

Esta exactitud es [COMPLETAR: comparable/superior/inferior] a los sistemas comerciales de localización UWB reportados en la literatura, que típicamente alcanzan exactitudes de [COMPLETAR valores de referencia] en condiciones de operación reales con BS presente.

\subsection{Dependencia de la Geometría del Despliegue}

La configuración geométrica de los nodos ancla influye en la exactitud alcanzable mediante el factor de Dilución de Precisión Geométrica (GDOP). [COMPLETAR: Analizar si ciertas posiciones en el escenario presentan sistemáticamente mayor error debido a geometría desfavorable].

Las zonas del escenario donde el nodo móvil está aproximadamente equidistante de los cuatro nodos ancla presentan [COMPLETAR: mejor/peor] exactitud debido a [explicación geométrica].

\subsection{Propagación del Error de Distancia al Error de Posición}

Un aspecto relevante es cómo el error en las mediciones individuales de distancia se propaga al error final de posición estimada. En trilateración, errores de magnitud similar en las cuatro distancias medidas pueden producir errores de posición [COMPLETAR: amplificados/reducidos] dependiendo de si los errores están correlacionados o son independientes.

Los resultados muestran que [COMPLETAR: interpretación específica basada en los datos obtenidos].


\section{COMPARACIÓN CON EL ESTADO DEL ARTE}
\label{sec:comparacion_estado_arte}

\subsection{Contraste con Estudios de Simulación}

Los resultados experimentales pueden contrastarse con los modelos de simulación revisados en el capítulo de estado del arte. [COMPLETAR: Referirse a estudios específicos mencionados en Cap4]

El estudio de [Autor, Año] predijo mediante simulaciones FDTD que el error en NLOS en [frecuencia] sería de [valor]. Nuestros resultados experimentales en 6.5 GHz muestran [comparación]. Las diferencias pueden atribuirse a [factores: simplificación del modelo corporal, propiedades dieléctricas asumidas vs. reales, efectos del entorno no modelados].

\subsection{Concordancia con Estudios Experimentales}

[COMPLETAR: Comparar cuantitativamente con los resultados de estudios experimentales citados en Cap4]

El trabajo de [Autor, Año] reportó un error medio de [X] cm en condición NLOS con el dispositivo en [ubicación] operando en [frecuencia]. Nuestro estudio en 6.5 GHz obtuvo [Y] cm para configuración comparable. La [concordancia/discrepancia] puede explicarse por [diferencias en hardware, escenarios, metodología].

\subsection{Aporte Diferencial de este Estudio}

Este trabajo aporta evidencia experimental específica para la banda de 6.5 GHz, que ha sido menos estudiada que bandas inferiores. Los resultados sugieren que [COMPLETAR: conclusión sobre ventajas/desventajas de operar en esta frecuencia].

Adicionalmente, la caracterización sistemática de múltiples ubicaciones corporales bajo un protocolo experimental uniforme permite comparaciones directas que no estaban disponibles en la literatura previa, donde diferentes estudios evaluaban ubicaciones distintas bajo condiciones experimentales heterogéneas.


\section{IMPLICACIONES PARA APLICACIONES PRÁCTICAS}
\label{sec:implicaciones_practicas}

\subsection{Recomendaciones de Diseño}

Basándose en los resultados obtenidos, se pueden derivar recomendaciones prácticas para el diseño de sistemas IPS basados en UWB cuando se anticipa obstrucción corporal:

\begin{enumerate}
\item \textbf{Ubicación óptima del dispositivo:} Los resultados indican que portar el dispositivo en [COMPLETAR: ubicación con mejor desempeño] minimiza el error medio y la variabilidad. Para aplicaciones donde la exactitud es crítica, se recomienda esta ubicación.

\item \textbf{Número y distribución de nodos ancla:} Configuraciones con [COMPLETAR: más nodos ancla / distribución específica] pueden mejorar la probabilidad de que al menos [N] enlaces operen en condiciones LOS, permitiendo descartar mediciones en NLOS severo.

\item \textbf{Estrategias de mitigación:} [COMPLETAR: basándose en si se probó alguna técnica: El uso de Filtro de Kalman redujo el error de posición en X\%, o: se recomienda implementar algoritmos de detección NLOS para descartar mediciones con error > umbral].

\item \textbf{Expectativas realistas de desempeño:} En aplicaciones con movilidad libre del usuario, donde la orientación relativa varía continuamente, debe anticiparse una exactitud de localización del orden de [COMPLETAR] cm (percentil 95), significativamente mayor que la exactitud nominal en LOS.
\end{enumerate}

\subsection{Aplicabilidad a Casos de Uso Específicos}

\textbf{Seguimiento de personal en entornos industriales:} [COMPLETAR: Analizar si la exactitud alcanzada es suficiente para aplicaciones como prevención de colisiones, control de acceso a zonas restringidas, optimización de flujos de trabajo]

\textbf{Localización en hospitales:} [COMPLETAR: Analizar aplicabilidad para seguimiento de pacientes, personal médico, activos móviles]

\textbf{Deportes y fitness:} [COMPLETAR: Analizar si la exactitud es adecuada para análisis de movimiento, tracking de atletas]


\section{LIMITACIONES DEL ESTUDIO}
\label{sec:limitaciones}

Es importante reconocer las limitaciones inherentes a este estudio experimental:

\begin{enumerate}
\item \textbf{Tamaño muestral limitado:} Con [N] participantes, la generalización de las conclusiones sobre la influencia de características antropométricas debe tomarse con cautela. Estudios futuros con mayor número de participantes permitirían análisis estadísticos más robustos.

\item \textbf{Escenario experimental único:} Los experimentos se realizaron en [descripción breve del escenario]. Las características específicas de este entorno (materiales constructivos, dimensiones, mobiliario) pueden influir en los resultados. La replicación en múltiples escenarios con geometrías y características de propagación diversas fortalecería la validez externa de las conclusiones.

\item \textbf{Mediciones estáticas predominantes:} [Si aplica: Aunque se incluyeron algunas mediciones dinámicas, la mayoría de los datos corresponden a condiciones estáticas. En aplicaciones reales donde el usuario camina continuamente, efectos adicionales como aceleración de movimiento y variaciones posturales podrían influir en el desempeño.]

\item \textbf{Configuración de hardware específica:} Los resultados se obtuvieron con [modelo específico de dispositivos UWB]. Diferentes implementaciones de tecnología UWB pueden presentar variaciones en algoritmos de procesamiento de señal, potencia de transmisión efectiva, y sensibilidad del receptor que afectarían los resultados cuantitativos.

\item \textbf{Condiciones controladas:} Los experimentos se realizaron con actividad humana externa minimizada. En escenarios operacionales reales con mayor densidad de personas en movimiento, el efecto de BS múltiple (varias personas obstruyendo simultáneamente diferentes enlaces) y la interferencia inter-usuario podrían degradar adicional mente el desempeño.
\end{enumerate}


\section{SÍNTESIS DEL CAPÍTULO}

Este capítulo ha interpretado los resultados experimentales presentados anteriormente, explicando los mecanismos físicos subyacentes al fenómeno de obstrucción corporal y sus manifestaciones en las mediciones de ToF y estimaciones de posición.

Se ha establecido que la banda de 6.5 GHz [COMPLETAR: presenta un compromiso favorable/desfavorable/comparable a bandas inferiores] para aplicaciones de posicionamiento en presencia de BS, con exactitudes que [COMPLETAR: resumen cuantitativo].

La comparación con el estado del arte confirma que [COMPLETAR: validación de resultados vs. literatura previa] y contribuye con caracterización sistemática en una banda de frecuencia previamente poco explorada.

Las implicaciones prácticas de estos hallazgos incluyen [COMPLETAR: resumen de recomendaciones principales] que pueden informar el diseño de sistemas IPS UWB para aplicaciones reales donde la obstrucción corporal es inevitable.


% Capítulo 10: Conclusiones y Recomendaciones
\chapter{CONCLUSIONES Y RECOMENDACIONES}
\label{ch:conclusiones}

Este capítulo final presenta de manera integrada las conclusiones derivadas de la investigación realizada, evalúa el grado de cumplimiento de los objetivos planteados, sintetiza las principales contribuciones del trabajo y plantea líneas de investigación futura orientadas a profundizar y extender los resultados obtenidos.

El objetivo general de este trabajo de maestría consistió en analizar el efecto de la obstrucción corporal sobre el desempeño de un sistema de posicionamiento en interiores basado en UWB, operando en la banda de 6.5 GHz, cuando el dispositivo móvil se ubica en diferentes partes del cuerpo humano. Dicho objetivo fue cumplido satisfactoriamente mediante el diseño e implementación de un sistema experimental completo que permitió caracterizar de forma sistemática el fenómeno de la obstrucción corporal bajo múltiples configuraciones. El conjunto de datos recolectados y su posterior análisis proporcionan una comprensión cuantitativa del impacto de la body shadowing sobre la exactitud de las mediciones de distancia y sobre el error final de localización del nodo móvil, con especial énfasis en la banda de 6.5 GHz, la cual ha sido escasamente abordada en la literatura previa.

Con cada una de las fases del proyecto, en primer lugar se evaluó el efecto de la obstrucción corporal sobre las medidas de señal y la estimación de distancias en un enlace de comunicación UWB, considerando distintas ubicaciones del dispositivo sobre el cuerpo humano. Para ello se analizaron sistemáticamente cinco posiciones corporales —pecho, espalda, cadera, muñeca y tobillo—, recolectando mediciones de tiempo de vuelo (ToF) bajo múltiples orientaciones relativas del cuerpo y diferentes posiciones espaciales. Los resultados obtenidos permitieron cuantificar cómo cada ubicación presenta patrones característicos de error dependientes del ángulo de orientación corporal, con variaciones que abarcan desde [COMPLETAR: rango de errores observados], evidenciando la fuerte influencia de la geometría cuerpo–dispositivo sobre el desempeño del enlace.

En segundo lugar, se abordó el análisis estadístico del efecto de la obstrucción corporal sobre el enlace de comunicación basado en UWB. Este análisis incluyó la caracterización de distribuciones de error, la aplicación de análisis de varianza multifactorial y el ajuste de modelos probabilísticos. Se determinó que, bajo condiciones de no línea de vista (NLOS) inducidas por el cuerpo humano, las distribuciones de error se apartan significativamente del comportamiento gaussiano, presentando [COMPLETAR: características observadas, como asimetría, colas pesadas, etc.]. Estos hallazgos tienen implicaciones directas para el diseño y la selección de algoritmos de localización robustos, especialmente aquellos que asumen modelos estadísticos simplificados del error.

Finalmente, se evaluó el desempeño del sistema completo de posicionamiento en interiores, conformado por cuatro dispositivos fijos y un dispositivo móvil, considerando nuevamente las diferentes ubicaciones corporales del nodo móvil. A partir de las mediciones obtenidas se calcularon estimaciones de posición bidimensional mediante técnicas de trilateración, y se analizaron diversas métricas de desempeño, entre ellas el error medio, la mediana del error, el RMSE, el CEP y el percentil 95. Los resultados muestran que el sistema alcanza exactitudes del orden de [COMPLETAR: valores obtenidos], las cuales dependen de manera significativa tanto de la ubicación del dispositivo sobre el cuerpo como de las condiciones de propagación predominantes, confirmando la relevancia de considerar explícitamente la obstrucción corporal en el diseño y evaluación de sistemas UWB de posicionamiento en interiores.


\section{CONCLUSIONES PRINCIPALES}
\label{sec:conclusiones_principales}

Basándose en los resultados experimentales y el análisis realizado, se derivan las siguientes conclusiones:

\begin{enumerate}
\item \textbf{La ubicación corporal del dispositivo es un factor crítico para el desempeño del sistema.} Los resultados demuestran que no todas las ubicaciones de portación son equivalentes. [COMPLETAR: Especificar cuál fue mejor/peor y por qué]. Esta conclusión tiene implicaciones directas para el diseño de sistemas comerciales de seguimiento de personas: la elección de dónde portar el dispositivo debe considerar el compromiso entre comodidad del usuario y exactitud de localización requerida por la aplicación.

\item \textbf{El efecto de la obstrucción corporal presenta una fuerte dependencia con la orientación relativa.} La transición entre condiciones LOS y NLOS se manifiesta claramente en la variación del error con la orientación del cuerpo. Los errores en NLOS pueden ser [COMPLETAR: X veces] superiores a los errores en LOS, confirmando que la BS constituye uno de los limitantes principales de la exactitud en aplicaciones de seguimiento de personas.

\item \textbf{La banda de 6.5 GHz presenta características favorables/desfavorables para mitigación de BS.} [COMPLETAR: Basándose en comparaciones con literatura en otras frecuencias]. La mayor atenuación en tejido biológico se compensa parcialmente por [COMPLETAR: mejor resolución temporal, características de difracción, etc.], resultando en un desempeño [comparable/superior/inferior] a frecuencias más bajas.

\item \textbf{Las distribuciones de error en NLOS son no-gaussianas.} Esta no-gaussianidad tiene implicaciones para el diseño de algoritmos de localización. Los enfoques tradicionales que asumen ruido gaussiano producen estimaciones subóptimas. Se recomienda el uso de técnicas robustas [COMPLETAR: si se implementó alguna: como el KF adaptativo empleado en este estudio, que logró reducir el error en X\%].

\item \textbf{La variabilidad interindividual es [significativa/moderada/limitada].} [COMPLETAR: Basándose en el análisis de correlación con antropometría]. Esto sugiere que [si es significativa: sistemas operacionales podrían beneficiarse de calibración personalizada por usuario / si es limitada: los modelos de error desarrollados son generalizables a una población diversa].

\item \textbf{La exactitud de localización alcanzable en condiciones reales con BS es del orden de [X] cm.} Esta exactitud es [COMPLETAR: suficiente/insuficiente] para aplicaciones como [ejemplos de aplicaciones]. Para aplicaciones que requieren exactitud [mejor/peor] será necesario [estrategias de mitigación adicionales / la tecnología UWB en la banda estudiada es adecuada directamente].
\end{enumerate}


\section{CONTRIBUCIONES DE LA INVESTIGACIÓN}
\label{sec:contribuciones}

Esta investigación ha generado las siguientes contribuciones al campo de los sistemas de posicionamiento en interiores:

\subsection{Contribuciones Científicas}

\begin{itemize}
\item \textbf{Caracterización experimental en 6.5 GHz:} Se proporciona la primera caracterización sistemática (según revisión de literatura realizada) del efecto de BS en sistemas UWB operando en 6.5 GHz, llenando una brecha en el conocimiento científico que se concentraba en frecuencias inferiores.

\item \textbf{Modelos estadísticos de error:} Se han desarrollado modelos de distribución de probabilidad del error de distancia para diferentes condiciones de propagación y ubicaciones corporales, que pueden ser utilizados para simulaciones realistas de desempeño de IPS.

\item \textbf{Protocolo experimental reproducible:} Se ha diseñado y documentado un protocolo experimental sistemático que puede ser adoptado por otros investigadores para estudios comparativos, facilitando la reproducibilidad y la comparación de resultados entre estudios.
\end{itemize}

\subsection{Contribuciones Prácticas}

\begin{itemize}
\item \textbf{Guías de diseño para sistemas comerciales:} Las recomendaciones derivadas de este estudio pueden informar el diseño de sistemas IPS UWB comerciales, optimizando la ubicación de nodos ancla, la selección de frecuencia de operación, y la ubicación recomendada del dispositivo móvil según los requerimientos de exactitud de la aplicación.

\item \textbf{Benchmarks de desempeño:} Los resultados cuantitativos obtenidos establecen valores de referencia (benchmarks) contra los cuales se pueden comparar desarrollos tecnológicos futuros o implementaciones alternativas.

\item \textbf{Validación de viabilidad tecnológica:} Los resultados confirman que [COMPLETAR: la tecnología UWB en 6.5 GHz es viable para aplicaciones de seguimiento de personas con exactitudes del orden de X cm, o: se requieren técnicas de mitigación adicionales para alcanzar exactitudes subdecimétricas en presencia de BS].
\end{itemize}


\section{TRABAJOS FUTUROS}
\label{sec:trabajos_futuros}

Los hallazgos de esta investigación abren múltiples direcciones para trabajos futuros que pueden profundizar o extender el conocimiento generado:

\subsection{Extensiones Experimentales}

\begin{enumerate}
\item \textbf{Evaluación en múltiples escenarios:} Replicar el estudio en entornos con características arquitectónicas y de propagación diversas (espacios abiertos tipo almacén, oficinas con particiones, áreas con alta densidad de mobiliario metálico) para evaluar la generalización de los resultados.

\item \textbf{Aumento del tamaño muestral:} Incrementar el número de participantes para reforzar el análisis estadístico de la influencia de características antropométricas y explorar efectos de segunda orden (edad, género, composición corporal).

\item \textbf{Mediciones dinámicas exhaustivas:} Extender la caracterización a escenarios dinámicos donde el participante realiza actividades naturales (caminar en diferentes velocidades, subir escaleras, movimientos complejos), capturando la variabilidad adicional introducida por el movimiento.

\item \textbf{Evaluación de BS múltiple:} Analizar el efecto de múltiples personas presentes simultáneamente en el escenario, donde un nodo móvil puede experimentar obstrucción no solo por el cuerpo que lo porta sino también por otras personas cercanas.
\end{enumerate}

\subsection{Desarrollo de Técnicas de Mitigación}

\begin{enumerate}
\item \textbf{Algoritmos de detección NLOS:} Desarrollar y validar algoritmos de aprendizaje automático que clasifiquen cada medición de ToF como LOS o NLOS basándose en características de la señal (potencia recibida, relación entre primer trayecto y trayectos subsecuentes, etc.), permitiendo descartar o ponderar diferencialmente mediciones según su condición de propagación.

\item \textbf{Filtros adaptativos:} Implementar y comparar diferentes esquemas de filtrado adaptativo (EKF, UKF, filtros de partículas) que ajusten dinámicamente sus parámetros según la condición de propagación detectada.

\item \textbf{Fusión con sensores inerciales:} Integrar datos de IMU (acelerómetros, giroscopios, magnetómetros) para estimar la orientación del cuerpo en tiempo real y aplicar modelos de corrección de error dependientes de la orientación.

\item \textbf{Corrección basada en modelos de propagación:} Desarrollar modelos fenomenológicos o basados en datos (machine learning) que predigan el sesgo de error esperado dada la configuración geométrica y la orientación estimada, aplicando correcciones antes del algoritmo de trilateración.
\end{enumerate}

\subsection{Estudios Comparativos}

\begin{enumerate}
\item \textbf{Comparación multi-frecuencia:} Realizar estudios experimentales controlados donde se comparen directamente múltiples bandas de frecuencia UWB (3.5, 4.5, 6.5, 8 GHz) bajo condiciones idénticas para cuantificar el compromiso entre atenuación, difracción y resolución temporal.

\item \textbf{Comparación multi-tecnología:} Contrastar el desempeño de UWB con tecnologías alternativas (WiFi RTT, Bluetooth AoA, sistemas híbridos) en presencia de BS para identificar las fortalezas y debilidades relativas de cada tecnología.
\end{enumerate}

\subsection{Modelado y Simulación}

\begin{enumerate}
\item \textbf{Modelos electromagnéticos refinados:} Desarrollar simulaciones FDTD de alta fidelidad que incorporen modelos antropomórficos detallados del cuerpo humano con propiedades dieléctricas diferenciadas por tejido, validándolos contra los datos experimentales de este estudio.

\item \textbf{Simuladores de nivel de sistema:} Crear simuladores que integren los modelos estadísticos de error derivados de este estudio, permitiendo evaluar rápidamente diferentes arquitecturas de sistemas IPS sin necesidad de despliegues experimentales costosos.
\end{enumerate}

\subsection{Aplicaciones Especializadas}

\begin{enumerate}
\item \textbf{Sistemas para grupos específicos de usuarios:} Adaptar y optimizar los sistemas IPS para aplicaciones específicas como seguimiento de pacientes geriátricos en hospitales, tracking de bomberos en emergencias (considerando equipo de protección adicional que afecta la propagación), o atletas de alto rendimiento con requerimientos de exactitud centimétrica.

\item \textbf{Integración con arquitecturas IoT:} Desarrollar arquitecturas completas que integren los IPS UWB con infraestructura IoT, plataformas de analítica de datos y sistemas de toma de decisiones para aplicaciones industriales 4.0.
\end{enumerate}




%genera doble hoja en blanco
\cleardoublepage

% ===== BIBLIOGRAFÍA =====
\begingroup
  % Título en mayúsculas
  \renewcommand{\refname}{\MakeUppercase{Bibliografía}} % Para scrartcl o artículos
  \renewcommand{\bibname}{\MakeUppercase{Bibliografía}} % Para scrbook o scrreprt

  % Formato del título centrado
  \setkomafont{chapter}{\normalfont\huge\bfseries\centering}

  % Imprimir la bibliografía
  %\printbibliography % Si usas biblatex
  \bibliography{bibliografia/bibliografia} % Si usas BibTeX tradicional
  \bibliographystyle{IEEEtran}

  % Añadir entrada al índice general
  \addcontentsline{toc}{chapter}{BIBLIOGRAFÍA}
\endgroup

\chapter*{ }

\begin{center}
    %\Large \textbf{UNIVERSIDAD DEL CAUCA}\\[0.3cm]
    %\textbf{FACULTAD DE INGENIERÍA ELECTRÓNICA Y TELECOMUNICACIONES}\\[0.5cm]
    \textbf{ACTA DE ACUERDO SOBRE LA PROPIEDAD INTELECTUAL DE LA TESIS DE MAESTRÍA} % Y DOCTORADO
\end{center}

\vspace{0.5cm}

En atención al acuerdo del Honorable Consejo Superior de la Universidad del Cauca, número 008 del 23 de Febrero de 1999, donde se estipula todo lo concerniente a la producción intelectual en la institución, los abajo firmantes, reunidos el día      \hspace{1cm} del mes de \hspace{2cm} de 202 \hspace{1cm}  en el salón del Consejo de Facultad, acordamos las siguientes condiciones para el desarrollo y posible usufructo del siguiente proyecto. \\

\noindent \textbf{Materia del acuerdo:} Tesis de Maestría  para optar al título de Magíster en Electrónica y Telecomunicaciones. \\
Área: Electrónica y Telecomunicaciones. \\

\noindent \textbf{Título de la Tesis:} ANÁLISIS DEL EFECTO DE LA OBSTRUCCIÓN
CORPORAL EN UN SISTEMA DE
POSICIONAMIENTO EN INTERIORES BASADO
EN ULTRA-WIDEBAND \\

\noindent \textbf{Objetivo de la Tesis:} Analizar el efecto de la obstrucción corporal sobre el desempeño de un sistema de posicionamiento en un escenario de interiores basado en UWB, cuando este sistema opera en la banda de 6.5 GHz y el dispositivo móvil se ubica en diferentes partes del cuerpo. \\

\noindent \textbf{Duración de la Tesis:}  9 Meses\\

\noindent \textbf{Cronograma de actividades:} El cronograma establece un plan de trabajo organizado en seis fases. En los dos primeros meses se va a realizar el análisis de requerimientos, lo que implica identificar, definir y validar las necesidades del sistema. Posteriormente, entre los meses dos y cinco, se va a llevar a cabo el diseño, donde se van a definir la arquitectura, los componentes y las especificaciones técnicas. A partir del mes cinco se va a implementar el sistema, proceso que se extenderá hasta el mes seis, para luego dar paso a las pruebas de desempeño, con el fin de verificar su funcionalidad y confiabilidad. En el mes siete se va a efectuar el análisis de resultados, evaluando el cumplimiento de los objetivos planteados. Finalmente, entre los meses ocho y nueve se va a realizar la entrega, garantizando la correcta transferencia y cierre del proyecto.\\

\noindent \textbf{Término de vinculación de cada partícipe:} 9 Meses\\

\noindent \textbf{Organismo financiador:} Grupos de investigación GRIAL y GNTT, Departamento de Telecomunicaciones y Universidad de Cauca.\\

\noindent \textbf{Naturaleza y Cuantía de sus Aportes}: 
\begin{itemize}
    \item Universidad del Cauca: apoyo institucional, apoyo en infraestructura, apoyo académico (\num[round-mode=places,round-precision=2]{\fpeval{100*\TotalFIET / \TotalProyecto}} \% del proyecto).
    \item Grupos GRIAL y GNTT: apoyo en la facilitación de hardware (\num[round-mode=places,round-precision=2]{\fpeval{100*\DepreciacionUWB / \TotalProyecto}} \% del proyecto). 
\end{itemize}

\vspace{0.5cm}

Los participantes de la Tesis, el señor estudiante de maestría, Danny Daniel Diaz Lopez, identificado con la cédula de ciudadanía número 76.331.174, a quien en adelante se le llamará ``estudiante'', y los ingenieros Víctor Manuel Quintero Florez y Claudia Milena Hernandez Bonilla en calidad de Directores del trabajo de maestría, identificados con la cédula de ciudadanía 76.323.426 y 25.291.154, a quienes en adelante se le llamará ``docentes'', y la Universidad del Cauca, representada por el Decano de la FIET, manifiestan que: \\

\begin{enumerate}
    \item La idea original del proyecto es de los docentes  quienes la propusieron y presentaron a los grupos de investigación: GRIAL y GNTT, quienes la aceptaron como tema para el proyecto de grado en referencia.
    \item La idea mencionada fue acogida por el estudiante como proyecto para obtener el grado de Magíster en Electrónica y Telecomunicaciones, quien la desarrollará bajo la dirección de los docentes.
    \item Los derechos intelectuales y morales corresponden al docente y a los estudiantes.
    \item Los derechos patrimoniales corresponden al docente, a los estudiantes y a la Universidad del Cauca por partes iguales y continuarán vigentes, aún después de la desvinculación de alguna de las partes de la Universidad.
    \item Los participantes se comprometen a cumplir con todas las condiciones de tiempo, recursos, infraestructura, dirección, asesoría, establecidas en el anteproyecto, a estudiar, analizar, documentar y hacer acta de cambios aprobados por el Consejo de Facultad, durante el desarrollo del proyecto, los cuales entran a formar parte de las condiciones generales.
    \item El estudiante se compromete a restituir en efectivo y de manera inmediata a la Universidad los aportes recibidos y los pagos hechos por la Institución a terceros por servicios o equipos, si el comité de Postgrados, previo concepto del Comité de Maestría respectivo, declara suspendido el proyecto por incumplimiento del cronograma o de las demás obligaciones contraídas por los estudiantes; y en cualquier caso de suspensión, la obligación de devolver en el estado en que les fueron proporcionados y de manera inmediata, los equipos de laboratorio, de cómputo y demás bienes suministrados por la Universidad para la realización del proyecto.
    \item Los docentes y el  estudiante se comprometen a dar crédito a la Universidad y a hacer mención del Fondo de Fomento de Investigación en caso de existir, en los informes de avance y de resultados, y en el registro de éstos, cuando ha habido financiación de la Universidad o del Fondo.
    \item Cuando por razones de incumplimiento, legalmente comprobadas, de las condiciones de desarrollo planteadas en el anteproyecto y sus modificaciones, el participante deba ser excluido del proyecto, los derechos aquí establecidos concluyen para él. Además, se tendrán en cuenta los principios establecidos en el reglamento del programa y el acuerdo 035 de 1992 vigente de la Universidad del Cauca en lo concerniente a la cancelación y la pérdida del derecho a continuar estudios.
    \item El documento del anteproyecto y las actas de modificaciones si las hubiere, forman parte integral de la presente acta.
    \item Los aspectos no contemplados en la presente acta serán definidos en los términos del acuerdo 008 del 23 de febrero de 1999 expedido por el Consejo Superior de la Universidad del Cauca, del cual los participantes del acuerdo aseguran tener pleno conocimiento.
\end{enumerate}


\noindent \textbf{Firmas:} \\[0.8cm]

\begin{tabular}{@{}l l@{}}
Director:        & \rule{7cm}{0.4pt} \\[0.4cm]
Directora:        & \rule{7cm}{0.4pt} \\[0.4cm]
Estudiante:      & \rule{7cm}{0.4pt} \\[0.4cm]
Decano Facultad: & \rule{7cm}{0.4pt} \\
\end{tabular}

%fin del documento
\end{document}