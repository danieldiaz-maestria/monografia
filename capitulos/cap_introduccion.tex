


% --- Contenido Principal ---
\chapter{INTRODUCCIÓN}
\label{ch:introduccion}

En las últimas dos décadas, hemos sido testigos de una transformación radical en la forma en que interactuamos con la tecnología y con nuestro entorno. La tecnología de Banda Ultra Ancha (\gls{uwb}) ha surgido como una pieza clave en este panorama evolutivo, especialmente en el ámbito de los Sistemas de Posicionamiento en Interiores (\gls{ips}). Lo que hace particularmente atractiva a esta tecnología es su capacidad para ofrecer una exactitud de localización a nivel subdecimétrico, algo que parecía inalcanzable hace apenas unos años. Sin embargo, como suele ocurrir con las innovaciones tecnológicas, pronto nos encontramos con desafíos que limitan su aplicación en el mundo real.

Uno de estos desafíos, quizás el más intrigante y complejo, es el fenómeno conocido como obstrucción corporal o \textit{Body Shadowing} (\gls{bs}). Imagine por un momento que usted porta un pequeño dispositivo \gls{uwb} en su muñeca o en el bolsillo de su camisa. Cada vez que su cuerpo se interpone entre ese dispositivo y los puntos de referencia fijos en el ambiente, las señales de radiofrecuencia deben atravesar o rodear su cuerpo. El tejido humano, rico en agua y con propiedades electromagnéticas complejas, se convierte en un obstáculo significativo que atenúa, refracta y difracta estas ondas. El resultado no es trivial: errores en la estimación de distancia que pueden superar los 4 metros en las peores condiciones, convirtiendo un sistema que prometía precisión centimétrica en algo poco confiable para aplicaciones críticas.

Esta situación cobra especial relevancia cuando pensamos en las aplicaciones prácticas. Consideremos un trabajador en una planta industrial que porta un dispositivo de seguridad para prevenir colisiones con maquinaria pesada, o un cirujano que necesita rastrear instrumental quirúrgico con precisión milimétrica durante una operación, o incluso un atleta de alto rendimiento cuyos movimientos se analizan para optimizar su técnica. En todos estos escenarios, la obstrucción corporal no es un detalle técnico menor que pueda ignorarse, es un factor determinante que define si el sistema cumplirá o no su propósito fundamental.

La comunidad científica ha reconocido esta problemática y ha comenzado a explorar diferentes estrategias para abordarla. Algunos investigadores han propuesto modelos estadísticos que caracterizan cómo varía el error según la orientación del cuerpo respecto a los nodos de referencia. Otros han desarrollado algoritmos de localización más sofisticados que combinan las mediciones de distancia con información de Unidades de Medida Inercial (\gls{imu}), permitiendo estimar la orientación del portador y ajustar las mediciones en consecuencia. Estos enfoques híbridos han demostrado resultados prometedores, logrando reducciones significativas en el error de posicionamiento cuando se comparan con métodos tradicionales. No obstante, la mayoría de estos estudios se han concentrado en frecuencias relativamente bajas, típicamente entre 3 y 5 GHz.

Aquí surge una pregunta natural: ¿qué ocurre cuando operamos en frecuencias más altas, específicamente en la banda de 6.5 GHz? Esta es una región del espectro que, sorprendentemente, ha recibido menos atención en la literatura científica, a pesar de ofrecer características potencialmente ventajosas. Por un lado, frecuencias más altas implican longitudes de onda más cortas, lo que se traduce en mejor resolución temporal y, teóricamente, mayor precisión en las mediciones de tiempo. Por otro lado, también enfrentamos mayor atenuación en tejidos biológicos y mayor sensibilidad a las obstrucciones. Es este delicado equilibrio el que motiva nuestra investigación.

El presente trabajo de maestría nace precisamente de la necesidad de comprender a profundidad cómo se manifiesta el efecto de la obstrucción corporal en sistemas \gls{uwb} operando en 6.5 GHz. No nos conformamos con repetir lo que ya sabemos de otras frecuencias, sino que buscamos caracterizar sistemática y rigurosamente este fenómeno en condiciones que reflejen aplicaciones reales. Para lograrlo, hemos diseñado un protocolo experimental exhaustivo que evalúa múltiples ubicaciones de portación del dispositivo móvil (pecho, espalda, cadera, muñeca, tobillo), diferentes orientaciones relativas del cuerpo respecto a los nodos de referencia, y participantes con características antropométricas diversas. 

Nuestra investigación no se limita a documentar el problema, sino que busca proporcionar herramientas prácticas para mitigarlo. A través de un análisis estadístico detallado, pretendemos desarrollar modelos que capturen la naturaleza estocástica del error introducido por la obstrucción corporal, modelos que puedan integrarse en algoritmos de localización más robustos. Además, evaluamos técnicas de filtrado y fusión sensorial que permitan compensar, al menos parcialmente, las limitaciones impuestas por la interacción entre las señales electromagnéticas y el cuerpo humano.

Los resultados de este trabajo aspiran a tener un impacto que trascienda lo puramente académico. Si logramos caracterizar adecuadamente el fenómeno de obstrucción corporal en 6.5 GHz, estaremos proporcionando a los ingenieros y desarrolladores de sistemas comerciales una base sólida para tomar decisiones informadas sobre dónde colocar los nodos de referencia, cómo configurar los parámetros del sistema, y qué algoritmos emplear según los requerimientos específicos de cada aplicación. En última instancia, buscamos que la tecnología \gls{uwb} cumpla su promesa de ofrecer localización precisa y confiable en entornos de interiores, incluso cuando los usuarios humanos inevitablemente obstruyen las señales con sus propios cuerpos.

En los capítulos subsecuentes de esta tesis, el lector encontrará no solo datos y gráficas, sino también la historia de cómo fuimos desentrañando, medición tras medición, los secretos de un fenómeno físico que, aunque invisible a nuestros ojos, determina si millones de dispositivos conectados podrán cumplir su función en el mundo real. Es una historia de precisión y error, de señales que viajan a la velocidad de la luz pero se retrasan por meros centímetros de tejido humano, y de cómo la ciencia nos permite comprender y, eventualmente, dominar estos desafíos técnicos para construir tecnologías más útiles y confiables.

