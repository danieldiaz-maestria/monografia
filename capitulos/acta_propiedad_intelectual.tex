\chapter*{ }

\begin{center}
    %\Large \textbf{UNIVERSIDAD DEL CAUCA}\\[0.3cm]
    %\textbf{FACULTAD DE INGENIERÍA ELECTRÓNICA Y TELECOMUNICACIONES}\\[0.5cm]
    \textbf{ACTA DE ACUERDO SOBRE LA PROPIEDAD INTELECTUAL DE LA TESIS DE MAESTRÍA} % Y DOCTORADO
\end{center}

\vspace{0.5cm}

En atención al acuerdo del Honorable Consejo Superior de la Universidad del Cauca, número 008 del 23 de Febrero de 1999, donde se estipula todo lo concerniente a la producción intelectual en la institución, los abajo firmantes, reunidos el día      \hspace{1cm} del mes de \hspace{2cm} de 202 \hspace{1cm}  en el salón del Consejo de Facultad, acordamos las siguientes condiciones para el desarrollo y posible usufructo del siguiente proyecto. \\

\noindent \textbf{Materia del acuerdo:} Tesis de Maestría  para optar al título de Magíster en Electrónica y Telecomunicaciones. \\
Área: Electrónica y Telecomunicaciones. \\

\noindent \textbf{Título de la Tesis:} ANÁLISIS DEL EFECTO DE LA OBSTRUCCIÓN
CORPORAL EN UN SISTEMA DE
POSICIONAMIENTO EN INTERIORES BASADO
EN ULTRA-WIDEBAND \\

\noindent \textbf{Objetivo de la Tesis:} Analizar el efecto de la obstrucción corporal sobre el desempeño de un sistema de posicionamiento en un escenario de interiores basado en UWB, cuando este sistema opera en la banda de 6.5 GHz y el dispositivo móvil se ubica en diferentes partes del cuerpo. \\

\noindent \textbf{Duración de la Tesis:}  9 Meses\\

\noindent \textbf{Cronograma de actividades:} El cronograma establece un plan de trabajo organizado en seis fases. En los dos primeros meses se va a realizar el análisis de requerimientos, lo que implica identificar, definir y validar las necesidades del sistema. Posteriormente, entre los meses dos y cinco, se va a llevar a cabo el diseño, donde se van a definir la arquitectura, los componentes y las especificaciones técnicas. A partir del mes cinco se va a implementar el sistema, proceso que se extenderá hasta el mes seis, para luego dar paso a las pruebas de desempeño, con el fin de verificar su funcionalidad y confiabilidad. En el mes siete se va a efectuar el análisis de resultados, evaluando el cumplimiento de los objetivos planteados. Finalmente, entre los meses ocho y nueve se va a realizar la entrega, garantizando la correcta transferencia y cierre del proyecto.\\

\noindent \textbf{Término de vinculación de cada partícipe:} 9 Meses\\

\noindent \textbf{Organismo financiador:} Grupos de investigación GRIAL y GNTT, Departamento de Telecomunicaciones y Universidad de Cauca.\\

\noindent \textbf{Naturaleza y Cuantía de sus Aportes}: 
\begin{itemize}
    \item Universidad del Cauca: apoyo institucional, apoyo en infraestructura, apoyo académico (\num[round-mode=places,round-precision=2]{\fpeval{100*\TotalFIET / \TotalProyecto}} \% del proyecto).
    \item Grupos GRIAL y GNTT: apoyo en la facilitación de hardware (\num[round-mode=places,round-precision=2]{\fpeval{100*\DepreciacionUWB / \TotalProyecto}} \% del proyecto). 
\end{itemize}

\vspace{0.5cm}

Los participantes de la Tesis, el señor estudiante de maestría, Danny Daniel Diaz Lopez, identificado con la cédula de ciudadanía número 76.331.174, a quien en adelante se le llamará ``estudiante'', y los ingenieros Víctor Manuel Quintero Florez y Claudia Milena Hernandez Bonilla en calidad de Directores del trabajo de maestría, identificados con la cédula de ciudadanía 76.323.426 y 25.291.154, a quienes en adelante se le llamará ``docentes'', y la Universidad del Cauca, representada por el Decano de la FIET, manifiestan que: \\

\begin{enumerate}
    \item La idea original del proyecto es de los docentes  quienes la propusieron y presentaron a los grupos de investigación: GRIAL y GNTT, quienes la aceptaron como tema para el proyecto de grado en referencia.
    \item La idea mencionada fue acogida por el estudiante como proyecto para obtener el grado de Magíster en Electrónica y Telecomunicaciones, quien la desarrollará bajo la dirección de los docentes.
    \item Los derechos intelectuales y morales corresponden al docente y a los estudiantes.
    \item Los derechos patrimoniales corresponden al docente, a los estudiantes y a la Universidad del Cauca por partes iguales y continuarán vigentes, aún después de la desvinculación de alguna de las partes de la Universidad.
    \item Los participantes se comprometen a cumplir con todas las condiciones de tiempo, recursos, infraestructura, dirección, asesoría, establecidas en el anteproyecto, a estudiar, analizar, documentar y hacer acta de cambios aprobados por el Consejo de Facultad, durante el desarrollo del proyecto, los cuales entran a formar parte de las condiciones generales.
    \item El estudiante se compromete a restituir en efectivo y de manera inmediata a la Universidad los aportes recibidos y los pagos hechos por la Institución a terceros por servicios o equipos, si el comité de Postgrados, previo concepto del Comité de Maestría respectivo, declara suspendido el proyecto por incumplimiento del cronograma o de las demás obligaciones contraídas por los estudiantes; y en cualquier caso de suspensión, la obligación de devolver en el estado en que les fueron proporcionados y de manera inmediata, los equipos de laboratorio, de cómputo y demás bienes suministrados por la Universidad para la realización del proyecto.
    \item Los docentes y el  estudiante se comprometen a dar crédito a la Universidad y a hacer mención del Fondo de Fomento de Investigación en caso de existir, en los informes de avance y de resultados, y en el registro de éstos, cuando ha habido financiación de la Universidad o del Fondo.
    \item Cuando por razones de incumplimiento, legalmente comprobadas, de las condiciones de desarrollo planteadas en el anteproyecto y sus modificaciones, el participante deba ser excluido del proyecto, los derechos aquí establecidos concluyen para él. Además, se tendrán en cuenta los principios establecidos en el reglamento del programa y el acuerdo 035 de 1992 vigente de la Universidad del Cauca en lo concerniente a la cancelación y la pérdida del derecho a continuar estudios.
    \item El documento del anteproyecto y las actas de modificaciones si las hubiere, forman parte integral de la presente acta.
    \item Los aspectos no contemplados en la presente acta serán definidos en los términos del acuerdo 008 del 23 de febrero de 1999 expedido por el Consejo Superior de la Universidad del Cauca, del cual los participantes del acuerdo aseguran tener pleno conocimiento.
\end{enumerate}


\noindent \textbf{Firmas:} \\[0.8cm]

\begin{tabular}{@{}l l@{}}
Director:        & \rule{7cm}{0.4pt} \\[0.4cm]
Directora:        & \rule{7cm}{0.4pt} \\[0.4cm]
Estudiante:      & \rule{7cm}{0.4pt} \\[0.4cm]
Decano Facultad: & \rule{7cm}{0.4pt} \\
\end{tabular}