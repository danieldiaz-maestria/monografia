\section{PLANTEAMIENTO DEL PROBLEMA}
\label{ch:planteamiento}

Las Tecnologías de la Información y la Comunicación (TIC) han tenido un gran desarrollo durante las últimas décadas. Gracias a esto apareció el concepto de Internet de las Cosas (\gls{iot}), un ecosistema de dispositivos interconectados capaz de recolectar e intercambiar información, así como de ejecutar diversos tipos de tareas de forma automatizada, haciéndose presente en múltiples escenarios, tanto domésticos como industriales y científicos \cite{ref1, ref2}. Entre los dispositivos que forman parte del IoT, los dispositivos inalámbricos vestibles (\textit{wearables}) desempeñan un papel clave en aplicaciones relacionadas con el cuidado personal, la asistencia médica, la telemedicina y la seguridad industrial, entre otros. Uno de los retos más relevantes relacionados con el uso de dispositivos IoT \textit{wearables} se encuentra asociado a los Servicios Basados en Localización (\gls{lbs}).

En escenarios de exteriores, la estimación de la localización y/o posición de un objetivo puede ser llevada a cabo usando el Sistema de Posicionamiento Global (\gls{gps}), el cual es uno de los Sistemas Globales de Navegación por Satélite (\gls{gnss}) conformado por una red o constelación de satélites que provee servicios de posicionamiento, navegación y señales de tiempo de referencia a nivel regional y global; sin embargo, en escenarios de interiores, este sistema pierde efectividad debido a que generalmente no existe Línea de Vista (\gls{los}) con los satélites, por obstrucciones tales como los techos y las paredes de las edificaciones, lo que implica la atenuación excesiva de las señales provenientes de los satélites al penetrar los edificios, y por lo tanto señales con niveles por debajo del valor de sensibilidad de los receptores. Por esta razón, se han desarrollado los Sistemas de Posicionamiento Local (\gls{lps}) y más específicamente, los Sistemas de Localización en Tiempo Real  (\gls{rtls}) y dentro de estos, los IPS\footnote{Para el presente trabajo, los términos y conceptos relacionados con RTLS e IPS se tratan de forma indistinta. 
%
Se debe tener en cuenta que la localización proporciona información de la ubicación de un dispositivo o nodo móvil de usuario con respecto a dispositivos fijos o nodos ancla y que son utilizados en aplicaciones tales como la guía de navegación en un museo y la localización de pacientes o especialistas en un hospital, entre otros.}, los cuales permiten realizar una estimación de la localización de un objetivo en escenarios de interiores, haciendo uso de tecnologías de Radiofrecuencia (\gls{rf}) tales como: UWB, Identificación por Radiofrecuencia (\gls{rfid}), Fidelidad Inalámbrica (\gls{wifi}), Zigbee y Bluetooth de Baja Energía (\gls{ble}) \cite{ref6, ref4}, entre otras. Además de sistemas de RF, un IPS puede utilizar tecnologías basadas en ultrasonido y luz, y su funcionamiento se basa en las mismas métricas de señal utilizadas por los sistemas de RF, tales como, la Intensidad de la Señal Recibida (\gls{rss}), el Ángulo de Llegada (\gls{aoa}), el Tiempo de Llegada (\gls{toa}), el Tiempo de Vuelo (\gls{tof}), la Medición de Distancia en Dos Vías (\gls{twr}), y la Fase de Llegada (\gls{poa}). 
%
En los escenarios de interiores con condiciones de canal NLOS, hay interacciones de las señales de RF con obstáculos, tales como mobiliario, paredes, pisos, techos y personas. Lo anterior genera fenómenos de reflexión y difracción de las señales de RF, representados con atenuaciones y cambios en la dirección de las señales que buscan ir de los dispositivos transmisores a los dispositivos receptores, lo cual implica retardos (cambios de fase) de la señal, y lo cual se conoce como multitrayecto, y afecta el desempeño de sistemas de comunicación inalámbricos. En el caso del cuerpo humano, las ondas de RF se reflejan y difractan de diferentes formas dependiendo de la ubicación del dispositivo de comunicación en el cuerpo humano, de los tejidos con los cuales interactúa la señal, lo cual de manera general se considera como BS. Evaluar y analizar el efecto de la BS en un IPS es un problema de significativa complejidad porque cada tejido humano tiene diferentes características eléctricas, i.e., permitividad, permeabilidad y conductividad. Las señales electromagnéticas en condición de NLOS y que atraviesan el cuerpo humano, además de las señales multitrayecto por reflexión y difracción en el cuerpo humano y en el escenario de despliegue del IPS, hacen que el receptor perciba ecos de la misma señal con diferentes atenuaciones y retardos (o cambios de fase), lo cual afecta la detección de señales por parte del receptor, afectando la exactitud y la precisión\footnote{La exactitud mide cuánto se aproximan los resultado al valor verdadero o conocido, y la precisión mide cuánto se aproximan los resultados entre sí \cite{ref5}. Aunque se manejen muchas veces de manera indistinta, cada una tiene su significado.} del IPS.

Para estimar la localización en escenarios de interiores se utilizan diferentes métricas de señal y técnicas asociadas con las que se busca minimizar los errores derivados principalmente de la condición de NLOS y los efectos del multitrayecto. A continuación, se presentan las tres técnicas más utilizadas para localización en escenarios de interiores:

La técnica de \textit{fingerprinting} se basa en la construcción de una base de datos, también conocida como “mapa de huellas”, que almacena información sobre el nivel de potencia de las señales de RF recibidas por un dispositivo o nodo móvil de acuerdo con su posición, y que fueron transmitidas desde dispositivos fijos o nodos ancla en ubicaciones específicas. Esta información se utiliza posteriormente en comparación con las medidas de señales que registra el dispositivo móvil que se desea localizar. Generalmente, la métrica de señal utilizada por esta técnica es la RSS. La técnica de \textit{fingerprinting} se puede implementar con bajo costo y baja complejidad, pero generalmente tiene baja exactitud y precisión \cite{ref6, ref7}.

La técnica de trilateración/multilateración se basa en las medidas de la distancia entre el nodo móvil y los nodos ancla, para lo cual utiliza las métricas de señal ToA, ToF, PoA y RSS para estimar la distancia entre ellos. A partir de las medidas de distancia estimadas, se establece la intersección de figuras geométricas formadas entre el nodo móvil y los nodos ancla, i.e., circunferencias, para estimar la posición del nodo móvil (objetivo) mediante un procedimiento matemático. Una ventaja de la trilateración es que tiene un alto desempeño en términos de exactitud y precisión de la estimación de localización del objetivo; sin embargo, tiene altos costos de implementación y requiere la condición LOS en el canal \cite{ref6}.

La técnica de triangulación, en lugar de medir las distancias entre el nodo móvil y los nodos fijos, mide los ángulos de llegada de la señal recibida por el nodo móvil desde nodos fijos o desde nodos fijos hacia un nodo móvil. Por lo tanto, utiliza la métrica de señal AoA. Al medir el AoA con respecto a dos nodos, se calcula la distancia utilizando cálculos trigonométricos. La ventaja de la triangulación es que funciona sin necesidad de realizar una sincronización entre los nodos; aun así, requiere un arreglo de antenas y condiciones de LOS, por lo que implica un mayor costo y es deficiente en escenarios densos o cuando hay efecto de BS \cite{ref6}.

Las señales de UWB debido al uso de pulsos de muy corta duración, i.e., 2ns, tienen un ancho de banda del orden de 500 MHz, y la distribución de su potencia sobre ese gran ancho de banda le permite tener una baja densidad espectral de potencia y de esa manera no interfiere o se ve interferida de manera significativa por señales de otras tecnologías con menor ancho de banda y mayor densidad espectral de potencia (diversidad en frecuencia). La duración de los pulsos y el tiempo entre pulsos permiten evitar traslapes o colisiones con otros pulsos UWB debido al multitrayecto, i.e., minimiza la Interferencia Intersimbólica (\gls{isi}) \cite{ref8}.

La tecnología UWB para el presente trabajo de grado de maestría opera en la frecuencia de 6.5 GHz, convirtiéndose en una alternativa relevante para la implementación de un IPS, y ofrece un conjunto de métricas de señal que pueden ser utilizadas y que gracias a las características de las señales UWB permiten mitigar los efectos adversos causados por la NLOS y el multitrayecto \cite{ref8, ref9, ref10, ref11, ref12}. 

En UWB se mide el tiempo que le toma a una señal radio viajar de un nodo o dispositivo móvil (TAG) a un nodo o dispositivo fijo (ANCHOR). Este tiempo se conoce como ToF. La medida se realiza en un mismo nodo (fijo o móvil), i.e., evaluando el tiempo de ida y regreso de la señal, el cual toma como referencia su base de tiempo y por lo tanto no requiere sincronismo, lo cual es una gran ventaja.
%
La Figura \ref{fig:twr} ilustra la forma en que se estima el ToF y cómo este permite calcular la distancia entre un nodo móvil o etiqueta y un nodo fijo o ancla utilizando la técnica de Medición de Distancia en Dos Vías (\gls{twr}).
%
El nodo móvil o etiqueta (TAG) inicia el TWR enviando un mensaje de consulta con la dirección conocida de un nodo fijo o ancla (ANCHOR). El nodo fijo o ancla (ANCHOR) registra y procesa el mensaje de consulta y envía una respuesta. Cuando el nodo móvil o etiqueta (TAG) recibe la respuesta, calcula el tiempo entre la ida y el regreso del mensaje, i.e., T$_{\text{round}}$, y estima el tiempo que tardó el nodo fijo o ancla (ANCHOR) en procesar y responder el mensaje de consulta, i.e., T$_{\text{reply}}$. Haciendo uso de estos tiempos se calcula el ToF, calculando la diferencia entre el T$_{\text{round}}$ y T$_{\text{reply}}$ y dividiendo entre 2; luego la distancia se calcula multiplicando el ToF por la velocidad de la luz, tal como se presenta a continuación: 
%

\begin{gather}
    d = \frac{1}{2} \left( t_{\mathrm{round}} - t_{\mathrm{reply}} \right) \times c
    = \frac{1}{2} \,\mathrm{ToF} \times c
\end{gather}




\noindent
donde, $t_{\mathrm{round}}$ es el tiempo total desde que el transmisor envía el mensaje hasta que recibe la respuesta (ida y vuelta); $t_{\mathrm{reply}}$ es el tiempo que el receptor tarda en procesar y responder el mensaje; y $c$ es la  velocidad de la luz.

%
% colocar la ecuacion y mejorar la descripcion con https://www.sewio.net/uwb-technology/two-way-ranging/
%
\begin{figure}[ht]
    \centering
    \includegraphics[width=0.8\textwidth]{imagenes/twr.pdf}
    \caption{Funcionamiento de TWR}
    \label{fig:twr}
\end{figure}
%
El nodo móvil o etiqueta (TAG) puede transmitir la distancia calculada al nodo fijo o ancla (ANCHOR) en un mensaje final, de ser necesario \cite{ref13}.

Uno de los mayores desafíos para lograr un IPS exacto, preciso y confiable es modelar y analizar el efecto de la BS sobre las señales. El cuerpo humano es en sí mismo un medio de transmisión de ondas electromagnéticas, en el cual se pueden presentar fenómenos de reflexión, difracción y absorción que pueden propiciar cambios en la señal, lo que incrementa la incertidumbre en el IPS, afectando su desempeño en términos de exactitud y precisión. Al estudiar el efecto estadístico sobre el error de posicionamiento o localización que genera la BS, y más específicamente, considerando la ubicación de un dispositivo \textit{wearable} en el cuerpo humano, es posible estimar cómo se afecta el desempeño del sistema.

El trabajo de maestría considerará un IPS compuesto por cuatro nodos fijos o anclas (ANCHORS) y un nodo móvil o etiqueta (TAG), el cual será ubicado en distintas partes del cuerpo humano, con la finalidad de analizar el desempeño de un IPS basado en UWB con BS. Con este propósito se plantea la siguiente pregunta de investigación.

¿Cómo analizar el efecto de la obstrucción corporal sobre un sistema de posicionamiento en interiores, considerando diferentes posiciones de dispositivos UWB en el cuerpo humano, cuando este sistema opera en la banda de 6.5 GHz y el dispositivo móvil se ubica en diferentes partes del cuerpo?


\subsection{HIPÓTESIS}
La caracterización del error de estimación de distancia por BS en función de la posición del nodo móvil en el cuerpo humano para un IPS basado en UWB en  6.5 GHz, y su integración con un filtro de estimación como el Filtro de Kalman (\gls{kf}), permitirá una reducción significativa y cuantificable del error de posicionamiento, logrando una mejora en la exactitud en escenarios con NLOS severo, en comparación con un IPS que no implementa el filtro de estimación. 
