\chapter{FASE 2: SISTEMA DE POSICIONAMIENTO 2D COMPLETO}
\label{sec:desempeno_posicionamiento}
En esta fase se evalúa el desempeño del sistema de posicionamiento en un entorno 2D completo, utilizando las técnicas y algoritmos desarrollados en las fases anteriores. Se realizan experimentos en un salón cerrado con dimensiones conocidas, donde se colocan anclas en posiciones fijas y se mide la precisión del sistema al estimar la posición de una etiqueta móvil a lo largo de 18 posiciones distribuidas en el área del salón.


\section{ESCENARIO EXPERIMENTAL}
\label{sec:escenario_experimental_2d}

\subsection{Configuración de anclas}

La información de las anclas se refiere a cada una de las paredes del salón. Cada par de medidas indica: la primera medida respecto de la sección más larga del salón y la segunda respecto de la sección más corta.

La primera ancla se coloca en la puerta y, a partir de ella, en sentido horario se colocan las restantes en cada esquina. Se toma como sistema de referencia un eje cartesiano con origen $(0,0)$ ubicado en la esquina próxima al punto junto al Ancla 2. El eje $x$ recorre el largo del salón (0 a 10.4 m) y el eje $y$ recorre el ancho (0 a 7.4 m). Las medidas de las anclas se expresan en centímetros en el croquis original; se convierten a metros para obtener coordenadas.

Dados los pares (distancia respecto de la sección más larga, distancia respecto de la sección más corta) y siguiendo la disposición en sentido horario, las coordenadas relativas a $(0,0)$ son:
\begin{itemize}
    \item Ancla 1 (37,29) → $A=(0.37\,\mathrm{m},\,0.29\,\mathrm{m})$ en el croquis; como está en la esquina opuesta en $x$: $x=10.4-0.29=10.11\,$m, $y=0.37\,$m. Coordenadas: $(10.11,\;0.37)$ m.
    \item Ancla 2 (38,28) → cercana al origen: $x=0.28\,$m, $y=0.38\,$m. Coordenadas: $(0.28,\;0.38)$ m.
    \item Ancla 3 (68,28) → esquina opuesta en $y$: $x=0.28\,$m, $y=7.4-0.68=6.72\,$m. Coordenadas: $(0.28,\;6.72)$ m.
    \item Ancla 4 (43,31) → esquina lejana en $x$ y $y$: $x=10.4-0.31=10.09\,$m, $y=7.4-0.43=6.97\,$m. Coordenadas: $(10.09,\;6.97)$ m.
\end{itemize}

Todas las anclas están colocadas a una altura de 1.52 m sobre el suelo.

El salón experimental presenta las siguientes dimensiones:
\begin{itemize}
    \item Ancho: 7.4 m
    \item Largo: 10.4 m
\end{itemize}

Las marcas de la etiqueta (posiciones del nodo móvil/etiqueta) se disponen tal como se muestra en la figura correspondiente del croquis experimental.

Las filas de marcas se colocan a las siguientes distancias desde el lado más largo del salón donde está ubicado el punto de referencia (origen $(0,0)$):

\begin{itemize}
    \item Primera fila: 0.70 m
    \item Segunda fila: 3.00 m
    \item Tercera fila: 5.40 m
\end{itemize}

La separación entre marcas a lo largo de cada fila es de 1.5 m y las 18 posiciones de medición se distribuyen en una cuadrícula de 3 filas × 6 columnas. Las coordenadas $(x, y)$ de cada marca se presentan en la figura \ref{fig:croquis_experimental_2d}.

\begin{figure}[hbt]
    \centering
    \includegraphics[width=0.8\textwidth]{imagenes/RepresentacionPosiciones.jpeg}
    \caption{Croquis del escenario experimental 2D con las posiciones de las anclas y las marcas de la etiqueta y la orientación del sistema de referencia.}
    \label{fig:croquis_experimental_2d}
\end{figure}


\section{RESULTADOS GRÁFICOS}
\label{sec:resultados_graficos_2d}

\subsection{Resultados del filtro de Kalman}

Las Figuras \ref{fig:kalman_tag} a \ref{fig:kalman_pecho_s3} muestran las trayectorias estimadas aplicando el filtro de Kalman para diferentes configuraciones de medición. En estas gráficas se comparan las posiciones reales (referencia) con las estimaciones obtenidas mediante multilateración (sin filtrado) y las estimaciones refinadas con el filtro de Kalman.

\begin{figure}[hbt]
    \centering
    \includegraphics[width=0.8\textwidth]{imagenes/kalman_tag.pdf}
    \caption{Trayectoria estimada para el tag: comparación entre posiciones reales, estimación por multilateración y estimación con filtro de Kalman.}
    \label{fig:kalman_tag}
\end{figure}

\begin{figure}[hbt]
    \centering
    \includegraphics[width=0.8\textwidth]{imagenes/kalman_cadera_sujeto1.pdf}
    \caption{Trayectoria estimada para cadera del sujeto 1: comparación entre posiciones reales, estimación por multilateración y estimación con filtro de Kalman.}
    \label{fig:kalman_cadera_s1}
\end{figure}

\begin{figure}[hbt]
    \centering
    \includegraphics[width=0.8\textwidth]{imagenes/kalman_cadera_sujeto2.pdf}
    \caption{Trayectoria estimada para cadera del sujeto 2: comparación entre posiciones reales, estimación por multilateración y estimación con filtro de Kalman.}
    \label{fig:kalman_cadera_s2}
\end{figure}

\begin{figure}[hbt]
    \centering
    \includegraphics[width=0.8\textwidth]{imagenes/kalman_cadera_sujeto3.pdf}
    \caption{Trayectoria estimada para cadera del sujeto 3: comparación entre posiciones reales, estimación por multilateración y estimación con filtro de Kalman.}
    \label{fig:kalman_cadera_s3}
\end{figure}

\begin{figure}[hbt]
    \centering
    \includegraphics[width=0.8\textwidth]{imagenes/kalman_pecho_sujeto1.pdf}
    \caption{Trayectoria estimada para pecho del sujeto 1: comparación entre posiciones reales, estimación por multilateración y estimación con filtro de Kalman.}
    \label{fig:kalman_pecho_s1}
\end{figure}

\begin{figure}[hbt]
    \centering
    \includegraphics[width=0.8\textwidth]{imagenes/kalman_pecho_sujeto2.pdf}
    \caption{Trayectoria estimada para pecho del sujeto 2: comparación entre posiciones reales, estimación por multilateración y estimación con filtro de Kalman.}
    \label{fig:kalman_pecho_s2}
\end{figure}

\begin{figure}[hbt]
    \centering
    \includegraphics[width=0.8\textwidth]{imagenes/kalman_pecho_sujeto3.pdf}
    \caption{Trayectoria estimada para pecho del sujeto 3: comparación entre posiciones reales, estimación por multilateración y estimación con filtro de Kalman.}
    \label{fig:kalman_pecho_s3}
\end{figure}


\section{ANÁLISIS ESTADÍSTICO}
\label{sec:analisis_estadistico_2d}

A continuación se presentan los resultados estadísticos del error de estimación para cada configuración evaluada, comparando el desempeño del filtro de Kalman con la multilateración sin filtrado.

\subsection{Estadísticas del tag}

Para el caso del tag, que representa la medición de referencia sin cuerpo humano, se obtuvieron los siguientes resultados:

\begin{table}[hbt]
\centering
\caption{Estadísticas de error para el tag}
\label{tab:estadisticas_tag}
\begin{tabular}{lcc}
\hline
\rowcolor{headerblue}
\textbf{Métrica} & \textbf{Filtro de Kalman (m)} & \textbf{Multilateración (m)} \\
\hline
Error mínimo & 0.0000 & 0.0386 \\
Error máximo & 1.8900 & 0.8652 \\
Error promedio & 0.7985 & 0.4802 \\
Desviación std & 0.4355 & 0.2178 \\
Error mediana & 0.8657 & 0.5341 \\
\hline
\end{tabular}
\end{table}

En este caso, la multilateración presenta un mejor desempeño con un error promedio de 0.4802 m, siendo 0.3183 m menor que el filtro de Kalman. Esto sugiere que para el tag sin interferencia del cuerpo humano, el algoritmo de multilateración directo proporciona estimaciones más precisas.

\subsection{Estadísticas para cadera}

Los resultados para las mediciones con el dispositivo ubicado en la cadera de los tres sujetos se presentan en las Tablas \ref{tab:estadisticas_cadera_s1}, \ref{tab:estadisticas_cadera_s2} y \ref{tab:estadisticas_cadera_s3}.

\begin{table}[hbt]
\centering
\caption{Estadísticas de error para cadera del sujeto 1}
\label{tab:estadisticas_cadera_s1}
\begin{tabular}{lcc}
\hline
\rowcolor{headerblue}
\textbf{Métrica} & \textbf{Filtro de Kalman (m)} & \textbf{Multilateración (m)} \\
\hline
Error mínimo & 0.0000 & 0.1970 \\
Error máximo & 1.0622 & 1.0994 \\
Error promedio & 0.6422 & 0.6049 \\
Desviación std & 0.2813 & 0.2936 \\
Error mediana & 0.7006 & 0.6124 \\
\hline
\end{tabular}
\end{table}

Para el sujeto 1, la multilateración muestra un mejor desempeño con una diferencia de 0.0373 m en el error promedio. Ambos métodos presentan errores máximos similares (alrededor de 1.1 m).

\begin{table}[hbt]
\centering
\caption{Estadísticas de error para cadera del sujeto 2}
\label{tab:estadisticas_cadera_s2}
\begin{tabular}{lcc}
\hline
\rowcolor{headerblue}
\textbf{Métrica} & \textbf{Filtro de Kalman (m)} & \textbf{Multilateración (m)} \\
\hline
Error mínimo & 0.0000 & 0.0841 \\
Error máximo & 2.3179 & 1.4634 \\
Error promedio & 0.9525 & 0.5521 \\
Desviación std & 0.5055 & 0.3487 \\
Error mediana & 0.8871 & 0.4754 \\
\hline
\end{tabular}
\end{table}

Para el sujeto 2, la multilateración muestra un desempeño notablemente superior con un error promedio de 0.5521 m, siendo 0.4004 m menor que el filtro de Kalman. Sin embargo, el filtro de Kalman logra reducir el error máximo de 2.32 m a 1.46 m.

\begin{table}[hbt]
\centering
\caption{Estadísticas de error para cadera del sujeto 3}
\label{tab:estadisticas_cadera_s3}
\begin{tabular}{lcc}
\hline
\rowcolor{headerblue}
\textbf{Métrica} & \textbf{Filtro de Kalman (m)} & \textbf{Multilateración (m)} \\
\hline
Error mínimo & 0.0000 & 0.0792 \\
Error máximo & 1.6716 & 2.9926 \\
Error promedio & 0.7664 & 0.8081 \\
Desviación std & 0.3928 & 0.6984 \\
Error mediana & 0.8135 & 0.7099 \\
\hline
\end{tabular}
\end{table}

En el caso del sujeto 3, el filtro de Kalman presenta un mejor desempeño con un error promedio de 0.7664 m, siendo 0.0417 m menor que la multilateración. Además, el filtro reduce significativamente el error máximo de 2.99 m a 1.67 m y la desviación estándar.

\subsection{Estadísticas para pecho}

Los resultados para las mediciones con el dispositivo ubicado en el pecho de los tres sujetos se presentan en las Tablas \ref{tab:estadisticas_pecho_s1}, \ref{tab:estadisticas_pecho_s2} y \ref{tab:estadisticas_pecho_s3}.

\begin{table}[hbt]
\centering
\caption{Estadísticas de error para pecho del sujeto 1}
\label{tab:estadisticas_pecho_s1}
\begin{tabular}{lcc}
\hline
\rowcolor{headerblue}
\textbf{Métrica} & \textbf{Filtro de Kalman (m)} & \textbf{Multilateración (m)} \\
\hline
Error mínimo & 0.0000 & 0.2390 \\
Error máximo & 1.3731 & 2.6677 \\
Error promedio & 0.7454 & 0.8471 \\
Desviación std & 0.3386 & 0.5835 \\
Error mediana & 0.7545 & 0.7389 \\
\hline
\end{tabular}
\end{table}

En el caso del pecho del sujeto 1, el filtro de Kalman muestra un mejor desempeño con un error promedio de 0.7454 m, siendo 0.1017 m menor que la multilateración. Además, el filtro de Kalman reduce significativamente el error máximo (1.3731 m vs 2.6677 m).

\begin{table}[hbt]
\centering
\caption{Estadísticas de error para pecho del sujeto 2}
\label{tab:estadisticas_pecho_s2}
\begin{tabular}{lcc}
\hline
\rowcolor{headerblue}
\textbf{Métrica} & \textbf{Filtro de Kalman (m)} & \textbf{Multilateración (m)} \\
\hline
Error mínimo & 0.0000 & 0.0644 \\
Error máximo & 1.6964 & 3.2228 \\
Error promedio & 0.9108 & 0.7761 \\
Desviación std & 0.4176 & 0.7136 \\
Error mediana & 0.9639 & 0.6766 \\
\hline
\end{tabular}
\end{table}

Para el pecho del sujeto 2, la multilateración presenta un error promedio ligeramente menor (0.7761 m vs 0.9108 m), pero el filtro de Kalman es especialmente efectivo en reducir el error máximo de 3.22 m a 1.70 m y la desviación estándar de 0.71 m a 0.42 m, lo que indica mayor estabilidad en las estimaciones.

\begin{table}[hbt]
\centering
\caption{Estadísticas de error para pecho del sujeto 3}
\label{tab:estadisticas_pecho_s3}
\begin{tabular}{lcc}
\hline
\rowcolor{headerblue}
\textbf{Métrica} & \textbf{Filtro de Kalman (m)} & \textbf{Multilateración (m)} \\
\hline
Error mínimo & 0.0000 & 0.0863 \\
Error máximo & 1.4035 & 2.6466 \\
Error promedio & 0.7027 & 0.7175 \\
Desviación std & 0.3238 & 0.5985 \\
Error mediana & 0.6438 & 0.5506 \\
\hline
\end{tabular}
\end{table}

En el pecho del sujeto 3, el filtro de Kalman logra un error promedio marginalmente mejor (0.7027 m vs 0.7175 m). El beneficio más significativo del filtro se observa en la reducción del error máximo (de 2.65 m a 1.40 m) y la desviación estándar (de 0.60 m a 0.32 m), lo que demuestra su capacidad para estabilizar las estimaciones en presencia de condiciones NLOS.


\section{ANÁLISIS DEL TIEMPO DE VUELO (ToF)}
\label{sec:analisis_tiempo_vuelo_2d}

El análisis del Tiempo de Vuelo (ToF) permite caracterizar directamente la calidad de las mediciones de distancia en el sistema de posicionamiento 2D. En esta sección se presentan los resultados del análisis comparativo entre condiciones LOS y NLOS para cada configuración evaluada.

\subsection{Estadísticas Globales de ToF}

El análisis consolidado de todas las mediciones (252 mediciones LOS y 252 NLOS) revela patrones importantes en el comportamiento del sistema:

\begin{table}[hbt]
\centering
\caption{Estadísticas globales de ToF: Comparación LOS vs NLOS}
\label{tab:tof_global}
\begin{tabular}{lccc}
\hline
\rowcolor{headerblue}
\textbf{Estadística} & \textbf{LOS (ns)} & \textbf{NLOS (ns)} & \textbf{$\Delta$ (ns)} \\
\hline
ToF Promedio & 24.58 & 21.85 & -2.73 \\
ToF Mediana & 24.72 & 22.45 & -2.27 \\
ToF Desv. Std & 7.33 & 8.18 & 0.86 \\
ToF Rango & 28.52 & 42.86 & 14.35 \\
Desv. Std Promedio & 0.18 & 0.63 & 0.45 \\
Coef. Variación (\%) & 29.63 & 36.44 & 6.81 \\
\hline
\end{tabular}
\end{table}

El hallazgo más crítico es el incremento de 251.64\% en la desviación estándar promedio al pasar de condiciones LOS a NLOS, señalando mayor incertidumbre en las mediciones. El rango de valores ToF en NLOS (42.86 ns) es 1.50× mayor que en LOS, indicando mayor dispersión temporal de las mediciones. Este aumento en variabilidad es directamente responsable de la degradación en exactitud de posicionamiento observada en la sección anterior.

\subsection{Resultados por Configuración}

\subsubsection{Tag (Referencia)}

Para el tag sin obstrucción corporal, se obtuvieron los siguientes resultados:

\begin{table}[hbt]
\centering
\caption{Estadísticas de ToF para el tag}
\label{tab:tof_tag}
\begin{tabular}{lccc}
\hline
\rowcolor{headerblue}
\textbf{Métrica} & \textbf{LOS (ns)} & \textbf{NLOS (ns)} & \textbf{Diferencia (ns)} \\
\hline
ToF Desv. Estándar & 7.50 & 8.07 & 0.57 \\
ToF Rango & 27.97 & 28.44 & 0.47 \\
Desv. Std Promedio & 0.10 & 0.09 & -0.02 \\
Coef. Variación (\%) & 29.39 & 37.70 & 8.31 \\
\hline
\end{tabular}
\end{table}

El tag presenta variabilidad mínima y consistente entre LOS y NLOS, confirmando que sin obstrucción corporal, el sistema mantiene excelente estabilidad temporal.

\subsubsection{Cadera}

Los resultados para cadera muestran incremento moderado en variabilidad:

\begin{table}[hbt]
\centering
\caption{Estadísticas de ToF para cadera (promedio de 3 sujetos)}
\label{tab:tof_cadera}
\begin{tabular}{lccc}
\hline
\rowcolor{headerblue}
\textbf{Métrica} & \textbf{LOS (ns)} & \textbf{NLOS (ns)} & \textbf{Diferencia (ns)} \\
\hline
ToF Desv. Std - S1 & 7.34 & 7.62 & 0.28 \\
ToF Desv. Std - S2 & 7.22 & 7.47 & 0.25 \\
ToF Desv. Std - S3 & 7.31 & 8.22 & 0.91 \\
Desv. Std Prom - S1 & 0.21 & 0.61 & 0.40 \\
Desv. Std Prom - S2 & 0.18 & 0.59 & 0.40 \\
Desv. Std Prom - S3 & 0.19 & 0.97 & 0.78 \\
\hline
\end{tabular}
\end{table}

La cadera presenta incremento en desviación estándar promedio de 0.40-0.78 ns entre sujetos, con el sujeto 3 mostrando mayor degradación (0.97 ns en NLOS).

\subsubsection{Pecho}

El pecho exhibe la mayor variabilidad en condiciones NLOS:

\begin{table}[hbt]
\centering
\caption{Estadísticas de ToF para pecho (3 sujetos)}
\label{tab:tof_pecho}
\begin{tabular}{lccc}
\hline
\rowcolor{headerblue}
\textbf{Métrica} & \textbf{LOS (ns)} & \textbf{NLOS (ns)} & \textbf{Diferencia (ns)} \\
\hline
ToF Desv. Std - S1 & 7.40 & 8.36 & 0.95 \\
ToF Desv. Std - S2 & 7.22 & 9.08 & 1.86 \\
ToF Desv. Std - S3 & 7.28 & 8.20 & 0.92 \\
ToF Rango - S2 & 27.37 & 42.86 & 15.49 \\
Desv. Std Prom - S1 & 0.20 & 0.90 & 0.70 \\
Desv. Std Prom - S2 & 0.18 & 0.56 & 0.38 \\
Desv. Std Prom - S3 & 0.18 & 0.69 & 0.52 \\
\hline
\end{tabular}
\end{table}

El pecho del sujeto 2 presenta el máximo rango ToF registrado (42.86 ns) y desviación estándar de 9.08 ns en NLOS, correlacionando directamente con los errores de posicionamiento más altos observados en este sujeto.

\subsection{Visualización de ToF por Ancla}

La Figura \ref{fig:tof_pecho_s1} muestra la distribución temporal de ToF para cada una de las cuatro anclas en el caso del pecho del sujeto 1, permitiendo identificar patrones de variabilidad específicos por ancla.

\begin{figure}[hbt]
    \centering
    \begin{minipage}[b]{0.48\textwidth}
        \centering
        \includegraphics[width=\textwidth]{imagenes/pecho_sujeto1_ancla1_tof_sin_outliers.png}
        \\ (a) Ancla 1
    \end{minipage}
    \hfill
    \begin{minipage}[b]{0.48\textwidth}
        \centering
        \includegraphics[width=\textwidth]{imagenes/pecho_sujeto1_ancla2_tof_sin_outliers.png}
        \\ (b) Ancla 2
    \end{minipage}
    
    \vspace{0.3cm}
    
    \begin{minipage}[b]{0.48\textwidth}
        \centering
        \includegraphics[width=\textwidth]{imagenes/pecho_sujeto1_ancla3_tof_sin_outliers.png}
        \\ (c) Ancla 3
    \end{minipage}
    \hfill
    \begin{minipage}[b]{0.48\textwidth}
        \centering
        \includegraphics[width=\textwidth]{imagenes/pecho_sujeto1_ancla4_tof_sin_outliers.png}
        \\ (d) Ancla 4
    \end{minipage}
    
    \caption{Distribución de ToF para cada ancla - Pecho Sujeto 1. Se observa la diferenciación clara entre condiciones LOS y NLOS, con mayor dispersión temporal en NLOS.}
    \label{fig:tof_pecho_s1}
\end{figure}

Las gráficas revelan que la Ancla 2 presenta la mayor variabilidad en condiciones NLOS (desviación estándar de 8.52 ns, rango 33.70 ns), mientras que las Anclas 3 y 4 mantienen mejor estabilidad. Esta variabilidad diferencial por ancla sugiere que factores geométricos y de obstrucción corporal específicos afectan cada enlace de manera distinta.

\subsection{Implicaciones para Detección NLOS}

El análisis de ToF permite establecer que la variabilidad (desviación estándar) constituye el indicador más confiable para distinguir condiciones LOS/NLOS:

\begin{itemize}
\item \textbf{Umbral sugerido:} Mediciones con desviación estándar > 0.49 ns son probablemente NLOS
\item \textbf{Incremento de variabilidad:} 251.64\% en promedio al pasar de LOS a NLOS
\item \textbf{Factor de peso:} Se recomienda aplicar un factor de peso a las mediciones NLOS en los algoritmos de posicionamiento para mitigar su impacto negativo. Este factor se obtiene mediante la relación entre las desviaciones estándar promedio de ambas condiciones:

\begin{equation}
    \text{Factor de peso} = \frac{\text{Desviación Estándar Promedio (LOS)}}{\text{Desviación Estándar Promedio (NLOS)}} = \frac{0.18}{0.63} \approx 0.28
\end{equation}

Donde:
\begin{itemize}
    \item \textbf{Desviación Estándar Promedio (LOS):} es el promedio de las desviaciones estándar de todas las mediciones realizadas bajo condiciones de Línea de Vista (LOS).
    \item \textbf{Desviación Estándar Promedio (NLOS):} es el promedio de las desviaciones estándar de todas las mediciones realizadas bajo condiciones de No Línea de Vista (NLOS).
\end{itemize}
\end{itemize}
Estos hallazgos respaldan la implementación de algoritmos de detección NLOS basados en análisis de variabilidad temporal de ToF, los cuales podrían mejorar significativamente la robustez del sistema de posicionamiento al permitir descartar o ponderar adecuadamente mediciones degradadas por obstrucción corporal.
