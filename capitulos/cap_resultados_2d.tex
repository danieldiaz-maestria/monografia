\chapter{FASE 2: SISTEMA DE POSICIONAMIENTO 2D COMPLETO}
\label{sec:desempeno_posicionamiento}

La Fase 2 de esta investigación corresponde a la implementación y evaluación del sistema completo de posicionamiento 2D utilizando cuatro nodos ancla en configuración geométrica. Esta fase integrará las mediciones de distancia hacia múltiples anclas para estimar posiciones 2D del nodo móvil mediante algoritmos de trilateración.

Los resultados de la Fase 1 presentados en el anterior capítulo establecen la línea base de desempeño del sistema de estimación de distancia bajo diferentes condiciones de obstrucción corporal. La Fase 2 utilizará estos hallazgos para:

\begin{itemize}
\item Implementar el sistema completo de localización con cuatro nodos ancla en el escenario experimental.
\item Evaluar el error de posicionamiento 2D (distancia euclidiana entre posición estimada y real).
\item Analizar métricas como el error euclidiano, exactitud al 95\%, y RMSE de posición.
\item Evaluar la robustez de un filtro de Kalman extendido para mitigar el error de posicionamiento.
\item Generar mapas de calor de error de posición en el área del escenario experimental.
\end{itemize}

Los resultados de la Fase 2 serán presentados en sección subsiguiente una vez completada la experimentación correspondiente.

[COMPLETAR: Incluir figuras mostrando mapas de calor del error de posición en el área del escenario experimental para cada ubicación corporal]

\begin{figure}[ht]
    \centering
    % \includegraphics[width=1.0\textwidth]{imagenes/mapa_error_pecho.pdf}
    \caption{Mapa de Calor del Error de Posición - Dispositivo en Pecho}
    \label{fig:mapa_error_pecho}
\end{figure}


\section{ANÁLISIS DE CASOS EXTREMOS}
\label{sec:casos_extremos}

\subsection{Condiciones de Peor Caso}

Se identificaron las combinaciones de ubicación corporal, orientación del cuerpo y posición en el escenario que resultaron en los mayores errores de estimación de distancia y posición.

[COMPLETAR: Describir los escenarios de peor caso observados, con datos cuantitativos]

\subsection{Condiciones de Mejor Caso}

[COMPLETAR: Similar al anterior, para condiciones óptimas]


\section{EFECTOS DE SEGUNDA ORDEN}
\label{sec:efectos_segunda_orden}

\subsection{Influencia de la Distancia}

[COMPLETAR: Analizar si el error de BS depende de la distancia entre el nodo móvil y los nodos ancla]

\subsection{Efecto del Multitrayecto del Entorno}

[COMPLETAR: Si es posible distinguir, analizar la contribución del multitrayecto del entorno vs. el efecto directo de la obstrucción corporal]


\section{SÍNTESIS DE RESULTADOS}

Este capítulo ha presentado los resultados experimentales obtenidos al evaluar sistemáticamente el efecto de la obstrucción corporal en un sistema de posicionamiento UWB operando en la banda de 6.5 GHz. Los datos recolectados abarcan múltiples ubicaciones de portación del dispositivo, orientaciones relativas del cuerpo, y participantes con características antropométricas diversas.

Los resultados evidencian que [COMPLETAR: resumen cualitativo de los hallazgos principales sin interpretación profunda, ej. "el error de distancia varía significativamente según la ubicación corporal del dispositivo y la orientación relativa, con valores que van desde X cm en condiciones óptimas hasta X metros en condiciones de obstrucción severa"].

El siguiente capítulo profundizará en la interpretación de estos resultados, comparándolos con el estado del arte y discutiendo las implicaciones prácticas y teóricas de los hallazgos.
