\chapter{FASE 2: SISTEMA DE POSICIONAMIENTO 2D COMPLETO}
\label{sec:desempeno_posicionamiento}
En esta fase se evalúa el desempeño del sistema de posicionamiento en un entorno 2D completo, utilizando las técnicas y algoritmos desarrollados en las fases anteriores. Se realizan experimentos en un salón cerrado con dimensiones conocidas, donde se colocan anclas en posiciones fijas y se mide la precisión del sistema al estimar la posición de una etiqueta móvil a lo largo de 18 posiciones distribuidas en el área del salón.


\section{ESCENARIO EXPERIMENTAL}
\label{sec:escenario_experimental_2d}

\subsection{Configuración de anclas}

La información de las anclas se refiere a cada una de las paredes del salón. Cada par de medidas indica: la primera medida respecto de la sección más larga del salón y la segunda respecto de la sección más corta.

La primera ancla se coloca en la puerta y, a partir de ella, en sentido horario se colocan las restantes en cada esquina. Se toma como sistema de referencia un eje cartesiano con origen $(0,0)$ ubicado en la esquina próxima al punto junto al Ancla 2. El eje $x$ recorre el largo del salón (0 a 10.4 m) y el eje $y$ recorre el ancho (0 a 7.4 m). Las medidas de las anclas se expresan en centímetros en el croquis original; se convierten a metros para obtener coordenadas.

Dados los pares (distancia respecto de la sección más larga, distancia respecto de la sección más corta) y siguiendo la disposición en sentido horario, las coordenadas relativas a $(0,0)$ son:
\begin{itemize}
    \item Ancla 1 (37,29) → $A=(0.37\,\mathrm{m},\,0.29\,\mathrm{m})$ en el croquis; como está en la esquina opuesta en $x$: $x=10.4-0.29=10.11\,$m, $y=0.37\,$m. Coordenadas: $(10.11,\;0.37)$ m.
    \item Ancla 2 (38,28) → cercana al origen: $x=0.28\,$m, $y=0.38\,$m. Coordenadas: $(0.28,\;0.38)$ m.
    \item Ancla 3 (68,28) → esquina opuesta en $y$: $x=0.28\,$m, $y=7.4-0.68=6.72\,$m. Coordenadas: $(0.28,\;6.72)$ m.
    \item Ancla 4 (43,31) → esquina lejana en $x$ y $y$: $x=10.4-0.31=10.09\,$m, $y=7.4-0.43=6.97\,$m. Coordenadas: $(10.09,\;6.97)$ m.
\end{itemize}

Todas las anclas están colocadas a una altura de 1.52 m sobre el suelo.

El salón experimental presenta las siguientes dimensiones:
\begin{itemize}
    \item Ancho: 7.4 m
    \item Largo: 10.4 m
\end{itemize}

Las marcas de la etiqueta (posiciones del nodo móvil/etiqueta) se disponen tal como se muestra en la figura correspondiente del croquis experimental.

Las filas de marcas se colocan a las siguientes distancias desde el lado más largo del salón donde está ubicado el punto de referencia (origen $(0,0)$):

\begin{itemize}
    \item Primera fila: 0.70 m
    \item Segunda fila: 3.00 m
    \item Tercera fila: 5.40 m
\end{itemize}

La separación entre marcas a lo largo de cada fila es de 1.5 m.

Las 18 posiciones de medición se distribuyen en una cuadrícula de 3 filas × 6 columnas. Las coordenadas $(x, y)$ de cada marca se presentan en la siguiente \textcolor{red}{figura}

