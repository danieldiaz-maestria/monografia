\chapter{APORTES INVESTIGATIVOS}


La presente investigación busca generar aportes significativos en el campo de los IPS basados en tecnología UWB. Responde a la necesidad crítica de mejorar la exactitud y confiabilidad de la localización en escenarios NLOS, particularmente cuando la obstrucción es causada por el cuerpo humano (BS), un factor que degrada severamente el desempeño de estos sistemas. Los aportes son de especial relevancia al explorar la banda de frecuencia de 6.5 GHz, un espectro poco estudiado que promete ventajas sustanciales para la mitigación de dicho fenómeno.


Se desarrollará una caracterización estadística fundamental del error de posicionamiento inducido por la obstrucción corporal en la banda de 6.5 GHz. Este aporte es novedoso, ya que la literatura actual se concentra mayoritariamente en frecuencias inferiores (3-5 GHz). El estudio generará modelos de distribución de probabilidad del error para distintas ubicaciones del dispositivo en el cuerpo (frente, pecho, espalda, etc.), estableciendo una base teórica sobre cómo la interacción de las señales UWB de mayor frecuencia con el cuerpo humano afecta la estimación de la distancia. Esta caracterización permitirá comprender si la mayor atenuación y el coeficiente de difracción mejorado en 6.5 GHz pueden, paradójicamente, simplificar la detección de condiciones NLOS.


Se diseñará y validará un protocolo experimental sistemático y reproducible para la evaluación del efecto de la obstrucción corporal. A diferencia de la fragmentación metodológica existente, donde se comparan pocas posiciones en condiciones dispares, este trabajo evaluará múltiples ubicaciones del nodo móvil bajo un conjunto idéntico de condiciones experimentales en un entorno real. Este enfoque permitirá, por primera vez, un análisis cruzado verdaderamente comparable de los datos. Dicho protocolo podrá ser adoptado en futuras investigaciones para estandarizar las pruebas de desempeño de sistemas IPS, no solo en UWB sino también en tecnologías emergentes como Wi-Fi 6E y futuras redes 6G.


Los resultados de la investigación ofrecerán una guía práctica para el diseño y la implementación de sistemas UWB comerciales más precisos y robustos. La caracterización del error en 6.5 GHz permitirá a los desarrolladores de hardware y software (chips, módulos de localización) tomar decisiones informadas para optimizar sus productos. Además, la integración y validación de un filtro de Kalman como técnica de mitigación proporcionará una solución de baja complejidad computacional, ideal para dispositivos embebidos y aplicaciones en tiempo real, como el seguimiento de personal en entornos industriales, la logística, la seguridad en fábricas y la asistencia médica.