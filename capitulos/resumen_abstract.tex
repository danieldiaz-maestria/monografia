\chapter*{RESUMEN}
\addcontentsline{toc}{chapter}{RESUMEN}

La tecnología de Banda Ultra Ancha (UWB) se ha consolidado como una de las soluciones más prometedoras para Sistemas de Posicionamiento en Interiores (IPS), gracias a su capacidad de alcanzar exactitudes subdecimétricas. Sin embargo, uno de los desafíos más críticos que enfrenta esta tecnología en aplicaciones de seguimiento de personas es el fenómeno de obstrucción corporal (\textit{Body Shadowing}, BS), donde el cuerpo humano se interpone entre el dispositivo móvil portado y los nodos fijos de referencia, degradando significativamente la exactitud de las mediciones de distancia y, consecuentemente, la estimación de posición.

Este trabajo de maestría presenta una investigación experimental sistemática del efecto de la obstrucción corporal sobre el desempeño de sistemas IPS basados en UWB operando en la banda de 6.5 GHz. A diferencia de estudios previos que se han concentrado principalmente en frecuencias más bajas (3-5 GHz), esta investigación caracteriza el fenómeno en una banda de frecuencia que ha recibido menor atención en la literatura científica, pero que ofrece un compromiso potencialmente ventajoso entre resolución temporal, características de propagación y disponibilidad de ancho de banda regulatorio.

La metodología experimental empleada se fundamenta en el Modelo en V, asegurando una correspondencia rigurosa entre las fases de diseño e implementación del sistema y las fases de verificación y validación. Se desplegó un sistema experimental compuesto por cuatro nodos ancla UWB fijos y un nodo móvil portado por participantes humanos en cinco ubicaciones corporales diferentes: pecho, espalda, cadera, muñeca y tobillo. Para cada ubicación, se recolectaron mediciones de Tiempo de Vuelo (ToF) en múltiples posiciones espaciales y orientaciones relativas del cuerpo, cubriendo sistemáticamente condiciones de Línea de Vista (LOS), Cuasi Línea de Vista (QLOS) y Sin Línea de Vista (NLOS).

Los resultados experimentales revelan que la ubicación corporal del dispositivo móvil tiene un impacto crítico sobre la magnitud del error introducido por la BS. [COMPLETAR CON RESULTADOS ESPECÍFICOS: Por ejemplo, "La ubicación en el pecho presentó un error medio de X cm en condiciones LOS que aumentaba hasta Y cm en NLOS severo, mientras que la ubicación en la muñeca exhibió mayor variabilidad con errores de hasta Z cm debido a la movilidad natural del brazo."]

El análisis estadístico demostró que las distribuciones de error en condiciones de obstrucción se apartan significativamente de la gaussianidad, presentando asimetría positiva y colas pesadas que se modelan mejor mediante distribuciones log-normales o mezclas de gaussianas. Esta no-gaussianidad tiene implicaciones directas para el diseño de algoritmos de localización, sugiriendo que enfoques robustos o adaptativos son necesarios para alcanzar desempeño óptimo en presencia de BS.

La correlación entre el error de medición y las características antropométricas de los participantes (estatura, peso, índice de masa corporal) fue [COMPLETAR: significativa/moderada/limitada], lo cual [sugiere que modelos personalizados por usuario podrían mejorar el desempeño / indica que los modelos de error son generalizables a poblaciones diversas].

El desempeño del sistema completo de posicionamiento 2D, evaluado mediante trilateración con las mediciones de distancia hacia los cuatro nodos ancla, alcanzó una exactitud de [COMPLETAR: valores de error medio, mediana, RMSE, percentil 95] dependiendo de la ubicación corporal del dispositivo. [COMPLETAR: Si se implementó alguna técnica de mitigación: "La implementación de un Filtro de Kalman adaptativo logró reducir el error de posición en X\%, demostrando la efectividad de técnicas de procesamiento de señal para mitigar parcialmente el efecto de la BS."]

Las contribuciones principales de esta investigación incluyen: (1) la primera caracterización experimental sistemática del efecto de BS en sistemas UWB operando en 6.5 GHz, llenando una brecha en el conocimiento científico; (2) el desarrollo de modelos estadísticos de error para diferentes condiciones de propagación y ubicaciones corporales; (3) un protocolo experimental reproducible que puede ser adoptado por la comunidad científica para estudios comparativos; y (4) recomendaciones prácticas para el diseño e implementación de sistemas IPS comerciales en presencia de obstrucción corporal.

Los resultados confirman que la tecnología UWB en la banda de 6.5 GHz es viable para aplicaciones de seguimiento de personas en interiores, alcanzando exactitudes que, si bien se degradan en presencia de obstrucción corporal respecto a condiciones ideales de LOS, resultan suficientes para una amplia gama de aplicaciones prácticas en seguridad industrial, logística, atención médica y deportes. Las direcciones de trabajo futuro identificadas incluyen la extensión a escenarios más diversos, el desarrollo de algoritmos de mitigación más sofisticados, y la fusión con otras modalidades sensoriales para mejorar la robustez del sistema.

\textbf{Palabras clave:} Banda Ultra Ancha (UWB), Sistemas de Posicionamiento en Interiores (IPS), Obstrucción Corporal (\textit{Body Shadowing}), Tiempo de Vuelo (ToF), Medición de Distancia en Dos Vías (TWR), Exactitud de Localización, Propagación NLOS, Frecuencia 6.5 GHz.


\vspace{1cm}

\chapter*{ABSTRACT}
\addcontentsline{toc}{chapter}{ABSTRACT}

Ultra-Wideband (UWB) technology has emerged as one of the most promising solutions for Indoor Positioning Systems (IPS), owing to its capability to achieve sub-decimeter accuracy. However, one of the most critical challenges this technology faces in people-tracking applications is the Body Shadowing (BS) phenomenon, where the human body interposes between the carried mobile device and the fixed reference nodes, significantly degrading distance measurement accuracy and, consequently, position estimation.

This master's thesis presents a systematic experimental investigation of the body shadowing effect on the performance of UWB-based IPS operating in the 6.5 GHz band. Unlike previous studies that have primarily focused on lower frequencies (3-5 GHz), this research characterizes the phenomenon in a frequency band that has received less attention in the scientific literature, but which offers a potentially advantageous tradeoff between temporal resolution, propagation characteristics, and regulatory bandwidth availability.

The experimental methodology employed is grounded in the V-Model, ensuring rigorous correspondence between system design and implementation phases and the verification and validation phases. An experimental system consisting of four fixed UWB anchor nodes and one mobile node carried by human participants was deployed in five different body locations: chest, back, hip, wrist, and ankle. For each location, Time-of-Flight (ToF) measurements were collected at multiple spatial positions and relative body orientations, systematically covering Line-of-Sight (LOS), Quasi-Line-of-Sight (QLOS), and Non-Line-of-Sight (NLOS) conditions.

The experimental results reveal that the body location of the mobile device has a critical impact on the magnitude of the error introduced by BS. [TO COMPLETE WITH SPECIFIC RESULTS: For example, "The chest location presented a mean error of X cm in LOS conditions that increased to Y cm in severe NLOS, while the wrist location exhibited greater variability with errors up to Z cm due to the natural mobility of the arm."]

Statistical analysis demonstrated that error distributions under shadowing conditions significantly depart from Gaussianity, presenting positive skewness and heavy tails that are better modeled by log-normal distributions or Gaussian mixtures. This non-Gaussianity has direct implications for localization algorithm design, suggesting that robust or adaptive approaches are necessary to achieve optimal performance in the presence of BS.

The correlation between measurement error and participants' anthropometric characteristics (height, weight, body mass index) was [TO COMPLETE: significant/moderate/limited], which [suggests that user-personalized models could improve performance / indicates that error models are generalizable to diverse populations].

The performance of the complete 2D positioning system, evaluated through trilateration with distance measurements to the four anchor nodes, achieved an accuracy of [TO COMPLETE: mean error, median, RMSE, 95th percentile values] depending on the body location of the device. [TO COMPLETE if any mitigation technique was implemented: "The implementation of an adaptive Kalman Filter reduced position error by X\%, demonstrating the effectiveness of signal processing techniques to partially mitigate the BS effect."]

The main contributions of this research include: (1) the first systematic experimental characterization of the BS effect in UWB systems operating at 6.5 GHz, filling a gap in scientific knowledge; (2) the development of statistical error models for different propagation conditions and body locations; (3) a reproducible experimental protocol that can be adopted by the scientific community for comparative studies; and (4) practical recommendations for the design and implementation of commercial IPS in the presence of body shadowing.

The results confirm that UWB technology in the 6.5 GHz band is viable for indoor people-tracking applications, achieving accuracies that, while degraded in the presence of body shadowing compared to ideal LOS conditions, are sufficient for a wide range of practical applications in industrial safety, logistics, healthcare, and sports. Identified future work directions include extension to more diverse scenarios, development of more sophisticated mitigation algorithms, and fusion with other sensory modalities to improve system robustness.

\textbf{Keywords:} Ultra-Wideband (UWB), Indoor Positioning Systems (IPS), Body Shadowing, Time-of-Flight (ToF), Two-Way Ranging (TWR), Localization Accuracy, NLOS Propagation, 6.5 GHz Frequency.
