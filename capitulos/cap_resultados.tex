\chapter{RESULTADOS EXPERIMENTALES}
\label{ch:resultados}

Este capítulo presenta los resultados obtenidos durante la fase experimental de la investigación. Se exponen de manera sistemática los datos recolectados y procesados según la metodología descrita en el capítulo anterior, organizados por categorías que facilitan su interpretación posterior. Los resultados se presentan mediante tablas, gráficas y análisis estadísticos descriptivos, evitando en esta sección la interpretación profunda que se realizará en el capítulo siguiente.

\section{VERIFICACIÓN Y CALIBRACIÓN DEL SISTEMA}
\label{sec:verificacion_sistema}

\subsection{Pruebas de Línea de Vista (LOS)}

Previo a la recolección de datos con participantes, se realizaron mediciones de calibración en condiciones de Línea de Vista sin obstrucción corporal. Estas mediciones establecen la línea base de desempeño del sistema UWB en el escenario experimental.

[COMPLETAR CON DATOS REALES:
\begin{table}[ht]
\centering
\caption{Resultados de Calibración en Condición LOS}
\label{tab:calibracion_los}
\begin{tabular}{lcccc}
\toprule
\textbf{Métrica} & \textbf{Ancla 1} & \textbf{Ancla 2} & \textbf{Ancla 3} & \textbf{Ancla 4} \\
\midrule
Error Medio (cm) & X.X & X.X & X.X & X.X \\
Desv. Estándar (cm) & X.X & X.X & X.X & X.X \\
RMSE (cm) & X.X & X.X & X.X & X.X \\
MAE (cm) & X.X & X.X & X.X & X.X \\
\bottomrule
\end{tabular}
\end{table}
]

Los resultados de calibración indican que el sistema UWB operando en la banda de 6.5 GHz alcanza una exactitud [COMPLETAR: "subdecimétrica/centimétrica"] en ausencia de obstrucciones, consistente con las especificaciones del fabricante y los valores reportados en la literatura para esta tecnología.

\subsection{Estabilidad Temporal del Sistema}

[COMPLETAR: Se evaluó la estabilidad temporal del sistema realizando mediciones repetidas en una posición fija durante X horas. La deriva temporal observada fue de X cm/hora, considerada aceptable para la duración de las sesiones experimentales.]


\section{CARACTERIZACIÓN ANTROPOMÉTRICA DE LOS PARTICIPANTES}
\label{sec:caracteristicas_participantes}

El estudio incluyó la participación de [COMPLETAR número] voluntarios con las siguientes características antropométricas:

[COMPLETAR CON DATOS REALES:
\begin{table}[ht]
\centering
\caption{Características Antropométricas de los Participantes}
\label{tab:participantes}
\begin{tabular}{lccccc}
\toprule
\textbf{Participante} & \textbf{Género} & \textbf{Edad} & \textbf{Estatura (cm)} & \textbf{Peso (kg)} & \textbf{IMC} \\
\midrule
P1 & M/F & XX & XXX & XX & XX.X \\
P2 & M/F & XX & XXX & XX & XX.X \\
P3 & M/F & XX & XXX & XX & XX.X \\
... & ... & ... & ... & ... & ... \\
\bottomrule
\end{tabular}
\end{table}
]

Esta diversidad en características físicas permite evaluar si el efecto de la obstrucción corporal depende significativamente de la complexión del portador del dispositivo.


\section{ANÁLISIS DEL ERROR DE DISTANCIA POR UBICACIÓN CORPORAL}
\label{sec:error_por_ubicacion}

\subsection{Dispositivo en el Pecho (Frente)}

[COMPLETAR: Presentar resultados para el dispositivo ubicado en el pecho]

\begin{figure}[ht]
    \centering
    % \includegraphics[width=0.8\textwidth]{imagenes/error_pecho_vs_rha.pdf}
    \caption{Error de Distancia vs. Ángulo de Orientación Relativo (RHA) - Dispositivo en Pecho}
    \label{fig:error_pecho_rha}
\end{figure}

[COMPLETAR:
\begin{table}[ht]
\centering
\caption{Estadísticas de Error de Distancia - Dispositivo en Pecho}
\label{tab:error_pecho}
\begin{tabular}{lcccc}
\toprule
\textbf{Condición RHA} & \textbf{ME (cm)} & \textbf{Desv. Est. (cm)} & \textbf{RMSE (cm)} & \textbf{Percentil 95 (cm)} \\
\midrule
LOS (0°-45°) & X.X & X.X & X.X & X.X \\
QLOS (45°-135°) & X.X & X.X & X.X & X.X \\
NLOS (135°-225°) & X.X & X.X & X.X & X.X \\
QLOS (225°-315°) & X.X & X.X & X.X & X.X \\
LOS (315°-360°) & X.X & X.X & X.X & X.X \\
\bottomrule
\end{tabular}
\end{table}
]

\subsection{Dispositivo en la Espalda}

[COMPLETAR: Similar estructura que la sección anterior, con gráficas y tablas correspondientes]

\subsection{Dispositivo en la Cadera}

[COMPLETAR]

\subsection{Dispositivo en la Muñeca}

[COMPLETAR]

\subsection{Dispositivo en el Tobillo}

[COMPLETAR]

\subsection{Comparación Entre Ubicaciones Corporales}

[COMPLETAR: Gráfica comparativa mostrando el error medio para cada ubicación corporal en función del RHA]

\begin{figure}[ht]
    \centering
    % \includegraphics[width=1.0\textwidth]{imagenes/comparacion_ubicaciones.pdf}
    \caption{Comparación del Error de Distancia Entre Diferentes Ubicaciones Corporales}
    \label{fig:comparacion_ubicaciones}
\end{figure}


\section{ANÁLISIS ESTADÍSTICO DE LAS DISTRIBUCIONES DE ERROR}
\label{sec:distribuciones_error}

\subsection{Ajuste de Distribuciones Teóricas}

Para cada combinación de ubicación corporal y condición de propagación (LOS/QLOS/NLOS), se ajustaron distribuciones de probabilidad teóricas a los errores observados.

[COMPLETAR:
\begin{table}[ht]
\centering
\caption{Mejores Ajustes de Distribución para Cada Condición}
\label{tab:distribuciones}
\begin{tabular}{lcc}
\toprule
\textbf{Condición} & \textbf{Distribución} & \textbf{Parámetros} \\
\midrule
Pecho - LOS & Normal & $\mu = X.X$ cm, $\sigma = X.X$ cm \\
Pecho - NLOS & Log-normal & $\mu = X.X$, $\sigma = X.X$ \\
Espalda - LOS & Normal & $\mu = X.X$ cm, $\sigma = X.X$ cm \\
... & ... & ... \\
\bottomrule
\end{tabular}
\end{table}
]

\subsection{Análisis de Varianza (ANOVA)}

Se realizó un análisis de varianza multifactorial para determinar qué variables experimentales tienen efecto estadísticamente significativo sobre el error de distancia.

[COMPLETAR:
\begin{table}[ht]
\centering
\caption{Resultados del ANOVA Multifactorial}
\label{tab:anova}
\begin{tabular}{lccccc}
\toprule
\textbf{Factor} & \textbf{Suma Cuad.} & \textbf{GL} & \textbf{Media Cuad.} & \textbf{F} & \textbf{p-valor} \\
\midrule
Ubicación corporal & XXX.X & X & XXX.X & XX.X & < 0.001 \\
Ángulo RHA & XXX.X & X & XXX.X & XX.X & < 0.001 \\
Distancia & XXX.X & X & XXX.X & XX.X & < 0.001 \\
Ubicación × RHA & XXX.X & X & XXX.X & XX.X & < 0.001 \\
Residuos & XXX.X & XXX & XX.X & & \\
\bottomrule
\end{tabular}
\end{table}
]

Los resultados del ANOVA indican que [COMPLETAR: todos los factores principales tienen efecto significativo (p < 0.001), así como la interacción entre ubicación corporal y ángulo RHA, sugiriendo que el patrón de variación del error con la orientación depende de dónde se porte el dispositivo].


\section{CORRELACIÓN CON VARIABLES ANTROPOMÉTRICAS}
\label{sec:correlacion_antropometria}

Se analizó la correlación entre el error de distancia y las características antropométricas de los participantes.

[COMPLETAR:
\begin{table}[ht]
\centering
\caption{Coeficientes de Correlación de Pearson}
\label{tab:correlacion}
\begin{tabular}{lcccc}
\toprule
\textbf{Variable} & \textbf{LOS} & \textbf{QLOS} & \textbf{NLOS} & \textbf{p-valor} \\
\midrule
Estatura & 0.XX & 0.XX & 0.XX & X.XXX \\
Peso & 0.XX & 0.XX & 0.XX & X.XXX \\
IMC & 0.XX & 0.XX & 0.XX & X.XXX \\
\bottomrule
\end{tabular}
\end{table}
]


\section{DESEMPEÑO DEL SISTEMA DE POSICIONAMIENTO COMPLETO}
\label{sec:desempeno_posicionamiento}

A partir de las mediciones de distancia hacia los cuatro nodos ancla, se calcularon las posiciones 2D del nodo móvil mediante trilateración por mínimos cuadrados.

\subsection{Error de Posición 2D}

[COMPLETAR:
\begin{table}[ht]
\centering
\caption{Error de Posición 2D por Ubicación Corporal}
\label{tab:error_posicion}
\begin{tabular}{lccccc}
\toprule
\textbf{Ubicación} & \textbf{ME (cm)} & \textbf{Mediana (cm)} & \textbf{RMSE (cm)} & \textbf{CEP (cm)} & \textbf{95\% (cm)} \\
\midrule
Pecho & XX.X & XX.X & XX.X & XX.X & XX.X \\
Espalda & XX.X & XX.X & XX.X & XX.X & XX.X \\
Cadera & XX.X & XX.X & XX.X & XX.X & XX.X \\
Muñeca & XX.X & XX.X & XX.X & XX.X & XX.X \\
Tobillo & XX.X & XX.X & XX.X & XX.X & XX.X \\
\bottomrule
\end{tabular}
\end{table}
]

\subsection{Mapas de Calor de Error de Posición}

[COMPLETAR: Incluir figuras mostrando mapas de calor del error de posición en el área del escenario experimental para cada ubicación corporal]

\begin{figure}[ht]
    \centering
    % \includegraphics[width=1.0\textwidth]{imagenes/mapa_error_pecho.pdf}
    \caption{Mapa de Calor del Error de Posición - Dispositivo en Pecho}
    \label{fig:mapa_error_pecho}
\end{figure}


\section{ANÁLISIS DE CASOS EXTREMOS}
\label{sec:casos_extremos}

\subsection{Condiciones de Peor Caso}

Se identificaron las combinaciones de ubicación corporal, orientación RHA y posición en el escenario que resultaron en los mayores errores de estimación de distancia y posición.

[COMPLETAR: Describir los escenarios de peor caso observados, con datos cuantitativos]

\subsection{Condiciones de Mejor Caso}

[COMPLETAR: Similar al anterior, para condiciones óptimas]


\section{EFECTOS DE SEGUNDA ORDEN}
\label{sec:efectos_segunda_orden}

\subsection{Influencia de la Distancia}

[COMPLETAR: Analizar si el error de BS depende de la distancia entre el nodo móvil y los nodos ancla]

\subsection{Efecto del Multitrayecto del Entorno}

[COMPLETAR: Si es posible distinguir, analizar la contribución del multitrayecto del entorno vs. el efecto directo de la obstrucción corporal]


\section{SÍNTESIS DE RESULTADOS}

Este capítulo ha presentado los resultados experimentales obtenidos al evaluar sistemáticamente el efecto de la obstrucción corporal en un sistema de posicionamiento UWB operando en la banda de 6.5 GHz. Los datos recolectados abarcan múltiples ubicaciones de portación del dispositivo, orientaciones relativas del cuerpo, y participantes con características antropométricas diversas.

Los resultados evidencian que [COMPLETAR: resumen cualitativo de los hallazgos principales sin interpretación profunda, ej. "el error de distancia varía significativamente según la ubicación corporal del dispositivo y la orientación relativa, con valores que van desde X cm en condiciones óptimas hasta X metros en condiciones de obstrucción severa"].

El siguiente capítulo profundizará en la interpretación de estos resultados, comparándolos con el estado del arte y discutiendo las implicaciones prácticas y teóricas de los hallazgos.
