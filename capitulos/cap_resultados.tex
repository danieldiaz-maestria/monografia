\chapter{FASE 1: VALIDACIÓN DE ESTIMACIÓN DE DISTANCIA}
\label{ch:resultados}

Este capítulo presenta los resultados obtenidos durante la Fase 1 de la investigación experimental, correspondiente a la validación del sistema de estimación de distancia UWB. Se exponen de manera sistemática los datos recolectados y procesados según la metodología descrita en el capítulo anterior, organizados por categorías que facilitan su interpretación posterior.

Los resultados presentados en este capítulo se enfocan en caracterizar la exactitud de las mediciones de distancia punto a punto bajo diferentes condiciones de obstrucción corporal y propagación. La Fase 2, correspondiente al sistema completo de posicionamiento 2D con cuatro nodos ancla, será presentada en secciones subsiguientes una vez completada la experimentación.

\section{VERIFICACIÓN Y CALIBRACIÓN DEL SISTEMA}
\label{sec:verificacion_sistema}

\subsection{Calibración de Antenna Delays}

Previo a la recolección de datos experimentales, se realizó un procedimiento de calibración de los retardos de antena (\textit{antenna delays}) del transceptor DW1000 presente en los módulos DWM1001. Estos parámetros compensan los retardos intrínsecos del hardware entre el instante temporal registrado por el transceptor y la emisión/recepción física de la señal en la interfaz radioeléctrica.

El DW1000 utiliza dos constantes de retardo que deben calibrarse para garantizar la exactitud de las mediciones de distancia mediante Two-Way Ranging (TWR):

\begin{itemize}
\item \texttt{TX\_ANT\_DLY}: retardo efectivo desde que el DW1000 genera el instante temporal de transmisión hasta que la señal electromagnética es emitida físicamente por la antena. 
\item \texttt{RX\_ANT\_DLY}: retardo desde que la señal llega físicamente a la antena hasta que el DW1000 registra el instante temporal de recepción.
\end{itemize}

Los valores por defecto proporcionados por el fabricante son aproximaciones generales que no consideran las particularidades de cada implementación de hardware (diseño de PCB, tipo de antena, trazas de RF). Por tanto, es necesario calibrarlos empíricamente para cada sistema específico.

\subsubsection{Procedimiento de Calibración}

El método de calibración implementado se basa en la comparación entre distancias conocidas y distancias medidas, ajustando iterativamente los \textit{antenna delays} hasta minimizar el error sistemático. El procedimiento consistió en los siguientes pasos:

\textbf{1. Configuración inicial:} Se configuraron dos módulos DWM1001 en modo TWR unilateral (SS-TWR), con valores iniciales de \texttt{TX\_ANT\_DLY} = 16380 y \texttt{RX\_ANT\_DLY} = 16450 (unidades de tiempo internas del DW1000, equivalentes a $\approx$15.65 ps por unidad).

\textbf{2. Posicionamiento de referencia:} Los módulos fueron ubicados a una distancia fija conocida $D_{\text{real}}$ = 1.00 m, medida con cinta métrica láser de precisión $\pm$1 mm, en ambiente exterior despejado para minimizar efectos de multitrayecto.

\textbf{3. Medición no calibrada:} Se realizaron 250 mediciones continuas de distancia, registrando el valor promedio $D_{\text{medida}}$ y calculando el error sistemático:

\begin{equation}
E = D_{\text{medida}} - D_{\text{real}}
\end{equation}

\textbf{4. Ajuste de retardos:} El error sistemático se traduce a un incremento/decremento en las constantes de retardo mediante el factor de conversión empírico para el DW1000 ($\approx$0.015 m por unidad de delay):

\begin{equation}
\Delta = \frac{E}{0.015}
\end{equation}

El ajuste se distribuyó simétricamente entre ambos retardos para mantener el balance TX/RX:

\begin{align}
\texttt{TX\_ANT\_DLY}_{\text{nuevo}} &= \texttt{TX\_ANT\_DLY}_{\text{anterior}} + \frac{\Delta}{2} \\
\texttt{RX\_ANT\_DLY}_{\text{nuevo}} &= \texttt{RX\_ANT\_DLY}_{\text{anterior}} + \frac{\Delta}{2}
\end{align}

\textbf{5. Iteración:} El proceso se repitió hasta lograr convergencia, definida como error sistemático $|E| < 5$ mm. La calibración convergió en tres iteraciones.

\subsubsection{Resultados de Calibración}

Los valores finales calibrados fueron \texttt{TX\_ANT\_DLY} = 16388 y \texttt{RX\_ANT\_DLY} = 16458. En la validación final a 1.00 m de distancia, el sistema registró MAE = 3.64 cm ($\sigma$ = 3.09 cm) en condiciones exteriores con línea de vista despejada, correspondiente al mejor caso de referencia (dispositivo en pecho, exterior, LOS). Este nivel de exactitud se considera adecuado para las aplicaciones de localización wearable objetivo de este trabajo.

Es importante destacar que la calibración fue realizada para la configuración específica de hardware utilizada (módulos DWM1001 con antenas integradas). Diferentes configuraciones de antena o modificaciones en el diseño de PCB requerirían repetir el procedimiento de calibración para mantener la trazabilidad metrológica del sistema.


\section{CARACTERIZACIÓN ANTROPOMÉTRICA DE LOS PARTICIPANTES}
\label{sec:caracteristicas_participantes}

Tal como se indicó en el paper de validación, este trabajo fue conducido como un caso de estudio fundacional con un único sujeto de prueba. Este enfoque permitió una caracterización detallada de los efectos de la ubicación del dispositivo y las condiciones de canal, minimizando la variabilidad inter-sujeto. Se reconoce que para garantizar la generalización de estos hallazgos para aplicaciones wearables robustas, futuros trabajos deberán enfocarse en un plan de pruebas exhaustivo involucrando múltiples sujetos con diversas condiciones corporales y características antropométricas.


\section{ANÁLISIS DEL ERROR DE DISTANCIA POR UBICACIÓN CORPORAL}
\label{sec:error_por_ubicacion}

En esta sección se presentan los resultados de la Fase 1, correspondientes a las mediciones de distancia entre un nodo móvil (ubicado en diferentes partes del cuerpo) y un nodo fijo de referencia. Estas mediciones constituyen la base fundamental para el sistema de posicionamiento 2D, ya que la exactitud de la localización depende directamente de la calidad de las estimaciones de distancia hacia cada nodo ancla.

Se evaluaron siete ubicaciones corporales del dispositivo móvil: cabeza (head), cadera (hip), mano (hand), muñeca (wrist), pecho (chest), rodilla (knee) y tobillo (ankle), bajo dos condiciones de propagación (LOS y NLOS) en dos escenarios diferentes (exterior e interior). A continuación se presentan los resultados organizados por ubicación corporal.

\subsection{Dispositivo en el Pecho (Chest)}

El pecho (chest) representa una de las ubicaciones más críticas para la evaluación de la obstrucción corporal, ya que el torso constituye la mayor masa de tejido que puede obstruir la señal directa entre el nodo móvil y el nodo fijo.

\begin{table}[ht]
\centering
\caption{Estadísticas de Error de Distancia - Dispositivo en Pecho}
\label{tab:error_pecho}
\begin{tabular}{lccc}
\toprule
\textbf{Escenario} & \textbf{Condición} & \textbf{MAE (m)} & \textbf{Desv. Est. (m)} \\
\midrule
Exterior & LOS & 0.0364 & 0.0309 \\
Exterior & NLOS & 0.5932 & -- \\
Interior & LOS & 0.0266 & -- \\
Interior & NLOS & 1.0480 & -- \\
\bottomrule
\end{tabular}
\end{table}

Los resultados muestran que el pecho alcanza la exactitud más alta en condiciones LOS, particularmente en el escenario interior con un MAE de solo 2.66 cm. Sin embargo, experimenta la degradación más severa al pasar a condiciones NLOS en el escenario exterior, con un factor de degradación de 16.3×, el más alto registrado entre todas las configuraciones evaluadas. Esta degradación extrema se debe a que el torso bloquea completamente la línea de vista directa, forzando al sistema a estimar la distancia basándose en señales difractadas alrededor del cuerpo o reflejadas por el entorno.

\subsection{Otras Ubicaciones Corporales Evaluadas}

Además del pecho, se evaluaron seis ubicaciones adicionales. La Tabla \ref{tab:resumen_todas_ubicaciones} presenta un resumen comparativo de los resultados obtenidos.

\begin{table}[ht]
\centering
\caption{Resumen de MAE por Ubicación Corporal y Condición - Fase 1}
\label{tab:resumen_todas_ubicaciones}
\begin{tabular}{lcccc}
\toprule
\multirow{2}{*}{\textbf{Ubicación}} & \multicolumn{2}{c}{\textbf{Exterior}} & \multicolumn{2}{c}{\textbf{Interior}} \\
\cmidrule(lr){2-3} \cmidrule(lr){4-5}
 & \textbf{LOS (m)} & \textbf{NLOS (m)} & \textbf{LOS (m)} & \textbf{NLOS (m)} \\
\midrule
Cabeza (Head) & 0.0464 & 0.0610 & 0.1750 & 0.3110 \\
Cadera (Hip) & 0.0817 & -- & -- & -- \\
Mano (Hand) & 0.3672 & 0.6100 & 0.9020 & 0.2471 \\
Muñeca (Wrist) & 0.0720 & -- & 1.0599 & 0.1777 \\
Pecho (Chest) & 0.0364 & 0.5932 & 0.0266 & 1.0480 \\
Rodilla (Knee) & 0.1513 & 2.2312 & 0.1270 & 0.5000 \\
Tobillo (Ankle) & 0.3274 & -- & -- & -- \\
\bottomrule
\end{tabular}
\end{table}

\textbf{Hallazgos principales:}

\begin{itemize}
\item \textbf{Cabeza:} Muestra comportamiento consistente y robusto, con degradación mínima entre LOS y NLOS (factor 1.3× en exterior). La forma curva de la cabeza favorece la difracción de señales UWB.

\item \textbf{Rodilla:} Presenta el peor caso registrado con MAE de 2.23 m en condición NLOS exterior (error máximo de 5.48 m), con factor de degradación de 14.7× y alta variabilidad ($\sigma$ = 1.80 m).

\item \textbf{Mano y Muñeca en Interior:} Exhiben el fenómeno contraintuitivo donde el desempeño en NLOS es superior al de LOS. La mano mejora de 0.902 m (LOS) a 0.247 m (NLOS) con factor 0.36×, y la muñeca de 1.060 m (LOS) a 0.178 m (NLOS) con factor 0.60×. Esto sugiere que las reflexiones especulares de las paredes del corredor actúan como guía de ondas, proveyendo caminos de propagación más favorables que la señal directa obstruida.
\end{itemize}


\section{ANÁLISIS ESTADÍSTICO GLOBAL - FASE 1}
\label{sec:distribuciones_error}

\subsection{Métricas Globales del Sistema}

Considerando las 28 configuraciones evaluadas (7 ubicaciones × 2 condiciones × 2 escenarios), el análisis global del sistema de estimación de distancia arroja las siguientes métricas:

\begin{itemize}
\item \textbf{MAE Global:} 0.3944 m
\item \textbf{RMSE Global:} 0.6690 m  
\item \textbf{Percentil 95:} 1.1424 m
\end{itemize}

El percentil 95 indica que el 95\% de todas las mediciones presentan errores inferiores a 1.14 m, proporcionando una medida de la variabilidad extrema del sistema. Se observó una tendencia sistemática a sobreestimar las distancias, fenómeno atribuible a efectos de multitrayecto y tiempos de procesamiento de señal en los dispositivos UWB.

\subsection{Distribución Probabilística de Errores}

La distribución probabilística de errores mostró asimetría positiva con cola extendida hacia errores mayores, indicando naturaleza no gaussiana en configuraciones específicas, particularmente bajo condiciones NLOS severas. Este comportamiento sugiere que modelos de error tradicionales basados en distribuciones gaussianas pueden ser inadecuados para caracterizar sistemas UWB con obstrucción corporal.



\section{FASE 2: SISTEMA DE POSICIONAMIENTO 2D COMPLETO}
\label{sec:desempeno_posicionamiento}

\textbf{Nota:} La Fase 2 de esta investigación corresponde a la implementación y evaluación del sistema completo de posicionamiento 2D utilizando cuatro nodos ancla en configuración geométrica. Esta fase integrará las mediciones de distancia hacia múltiples anclas para estimar posiciones 2D del nodo móvil mediante algoritmos de trilateración.

Los resultados de la Fase 1 presentados en este capítulo establecen la línea base de desempeño del sistema de estimación de distancia bajo diferentes condiciones de obstrucción corporal. La Fase 2 utilizará estos hallazgos para:

\begin{itemize}
\item Implementar el sistema completo de localización con cuatro nodos ancla en el escenario experimental.
\item Evaluar el error de posicionamiento 2D (distancia euclidiana entre posición estimada y real).
\item Analizar métricas como CEP (Error Circular Probable), exactitud al 95\%, y RMSE de posición.
\item Comparar algoritmos de trilateración (mínimos cuadrados ponderados, Filtro de Kalman).
\item Generar mapas de calor de error de posición en el área del escenario experimental.
\end{itemize}

Los resultados de la Fase 2 serán presentados en sección subsiguiente una vez completada la experimentación correspondiente.

[COMPLETAR: Incluir figuras mostrando mapas de calor del error de posición en el área del escenario experimental para cada ubicación corporal]

\begin{figure}[ht]
    \centering
    % \includegraphics[width=1.0\textwidth]{imagenes/mapa_error_pecho.pdf}
    \caption{Mapa de Calor del Error de Posición - Dispositivo en Pecho}
    \label{fig:mapa_error_pecho}
\end{figure}


\section{ANÁLISIS DE CASOS EXTREMOS}
\label{sec:casos_extremos}

\subsection{Condiciones de Peor Caso}

Se identificaron las combinaciones de ubicación corporal, orientación del cuerpo y posición en el escenario que resultaron en los mayores errores de estimación de distancia y posición.

[COMPLETAR: Describir los escenarios de peor caso observados, con datos cuantitativos]

\subsection{Condiciones de Mejor Caso}

[COMPLETAR: Similar al anterior, para condiciones óptimas]


\section{EFECTOS DE SEGUNDA ORDEN}
\label{sec:efectos_segunda_orden}

\subsection{Influencia de la Distancia}

[COMPLETAR: Analizar si el error de BS depende de la distancia entre el nodo móvil y los nodos ancla]

\subsection{Efecto del Multitrayecto del Entorno}

[COMPLETAR: Si es posible distinguir, analizar la contribución del multitrayecto del entorno vs. el efecto directo de la obstrucción corporal]


\section{SÍNTESIS DE RESULTADOS}

Este capítulo ha presentado los resultados experimentales obtenidos al evaluar sistemáticamente el efecto de la obstrucción corporal en un sistema de posicionamiento UWB operando en la banda de 6.5 GHz. Los datos recolectados abarcan múltiples ubicaciones de portación del dispositivo, orientaciones relativas del cuerpo, y participantes con características antropométricas diversas.

Los resultados evidencian que [COMPLETAR: resumen cualitativo de los hallazgos principales sin interpretación profunda, ej. "el error de distancia varía significativamente según la ubicación corporal del dispositivo y la orientación relativa, con valores que van desde X cm en condiciones óptimas hasta X metros en condiciones de obstrucción severa"].

El siguiente capítulo profundizará en la interpretación de estos resultados, comparándolos con el estado del arte y discutiendo las implicaciones prácticas y teóricas de los hallazgos.
