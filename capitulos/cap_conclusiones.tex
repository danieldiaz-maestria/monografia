\chapter{CONCLUSIONES Y RECOMENDACIONES}
\label{ch:conclusiones}

Este capítulo final presenta de manera integrada las conclusiones derivadas de la investigación realizada, evalúa el grado de cumplimiento de los objetivos planteados, sintetiza las principales contribuciones del trabajo y plantea líneas de investigación futura orientadas a profundizar y extender los resultados obtenidos.

El objetivo general de este trabajo de maestría consistió en analizar el efecto de la obstrucción corporal sobre el desempeño de un sistema de posicionamiento en interiores basado en UWB, operando en la banda de 6.5 GHz, cuando el dispositivo móvil se ubica en diferentes partes del cuerpo humano. Dicho objetivo fue cumplido satisfactoriamente mediante el diseño e implementación de un sistema experimental completo que permitió caracterizar de forma sistemática el fenómeno de la obstrucción corporal bajo múltiples configuraciones. El conjunto de datos recolectados y su posterior análisis proporcionan una comprensión cuantitativa del impacto de la body shadowing sobre la exactitud de las mediciones de distancia y sobre el error final de localización del nodo móvil, con especial énfasis en la banda de 6.5 GHz, la cual ha sido escasamente abordada en la literatura previa.

Con cada una de las fases del proyecto, en primer lugar se evaluó el efecto de la obstrucción corporal sobre las medidas de señal y la estimación de distancias en un enlace de comunicación UWB, considerando distintas ubicaciones del dispositivo sobre el cuerpo humano. Para ello se analizaron sistemáticamente siete posiciones corporales —cabeza, pecho, cadera, mano, muñeca, rodilla y tobillo—, recolectando mediciones de tiempo de vuelo (ToF) bajo condiciones LOS y NLOS en dos escenarios distintos (exterior e interior). Los resultados obtenidos permitieron cuantificar cómo cada ubicación presenta patrones característicos de error, con variaciones que abarcan desde 4.62 cm (pecho en exterior LOS) hasta 97.76 cm (cadera en interior NLOS), evidenciando la fuerte influencia de la geometría cuerpo–dispositivo sobre el desempeño del enlace. La cabeza demostró ser la ubicación más consistente con factores de degradación LOS-NLOS de 6.7× en exterior y solo 2.0× en interior, mientras que la cadera presentó la mayor variabilidad con factores de hasta 13.2×.

En segundo lugar, se abordó el análisis estadístico del efecto de la obstrucción corporal sobre el enlace de comunicación basado en UWB. Este análisis incluyó la caracterización de distribuciones de error mediante pruebas no paramétricas (Mann-Whitney), análisis de tamaños de efecto, e intervalos de confianza bootstrap. Se determinó que, bajo condiciones de no línea de vista (NLOS) inducidas por el cuerpo humano, las distribuciones de error se apartan significativamente del comportamiento gaussiano, presentando asimetría positiva pronunciada, colas extendidas hacia valores extremos (hasta 2.90 m de error máximo), y coeficientes de variación superiores al 100\% en configuraciones críticas como cadera NLOS. El análisis reveló que ubicaciones del torso (cadera, pecho) presentan obstrucción severa dominante, mientras que ubicaciones en extremidades (cabeza, mano, muñeca) se benefician del fenómeno de multitrayecto constructivo en entornos interiores, exhibiendo errores menores en interior NLOS que en exterior NLOS. Estos hallazgos tienen implicaciones directas para el diseño y la selección de algoritmos de localización robustos, especialmente aquellos que asumen modelos estadísticos simplificados del error.

Adicionalmente, se realizó un análisis exhaustivo del Tiempo de Vuelo (ToF) que reveló que la variabilidad temporal constituye el indicador más confiable para distinguir condiciones LOS/NLOS. Se documentó un incremento de 251.64\% en la desviación estándar promedio al pasar de LOS (0.18 ns) a NLOS (0.63 ns), con el rango de valores ToF expandiéndose 50.31\% (de 28.52 ns en LOS a 42.86 ns en NLOS). El análisis por configuración mostró que el tag de referencia (sin obstrucción corporal) presenta variabilidad mínima ($\Delta$ std = 0.57 ns), mientras que el pecho del sujeto 2 exhibió la mayor degradación con desviación estándar de 9.08 ns en NLOS y rango de 42.86 ns. Se estableció un umbral de 0.49 ns de desviación estándar como criterio para detección NLOS y se recomendó aplicar un factor de peso de 0.28 a mediciones NLOS en algoritmos de trilateración.

Finalmente, se evaluó el desempeño del sistema completo de posicionamiento 2D en interiores, conformado por cuatro nodos ancla ubicados en las esquinas de un salón de 10.4 m × 7.4 m y un dispositivo móvil evaluado en 18 posiciones distribuidas en una cuadrícula 3×6. Se comparó el desempeño de multilateración directa versus filtro de Kalman para tres configuraciones: tag de referencia, cadera en tres sujetos, y pecho en tres sujetos. Los resultados muestran que el sistema alcanza exactitudes del orden de 48 cm (tag), 55-95 cm (cadera), y 70-91 cm (pecho) con filtro de Kalman. El filtro demostró ser especialmente efectivo en reducir errores máximos (41-49\% de reducción) y desviación estándar (41-48\% de reducción) en presencia de condiciones NLOS severas, aunque para el tag sin obstrucción corporal, la multilateración directa presentó mejor desempeño (error 0.48 m vs 0.80 m del Kalman). La variabilidad inter-sujeto fue significativa, con diferencias de hasta 0.40 m en error promedio entre sujetos para la misma ubicación corporal, confirmando la relevancia de considerar explícitamente la obstrucción corporal en el diseño y evaluación de sistemas UWB de posicionamiento en interiores.


\section{CONCLUSIONES PRINCIPALES}
\label{sec:conclusiones_principales}

Basándose en los resultados experimentales y el análisis realizado, se derivan las siguientes conclusiones:

\begin{enumerate}
\item \textbf{La ubicación corporal del dispositivo es un factor crítico para el desempeño del sistema.} Los resultados demuestran que no todas las ubicaciones de portación son equivalentes. La cabeza presenta el comportamiento más consistente y predecible, con MAE de 4.87 cm en exterior LOS y degradación controlada hacia 18.16 cm en interior NLOS (factor 2.0×). En contraste, la cadera exhibe el peor desempeño con MAE de 8.31 cm en LOS que se degrada hasta 97.76 cm en interior NLOS (factor 13.2×). El pecho muestra excelente exactitud en LOS (4.62 cm) pero degradación severa en NLOS (hasta 83.98 cm). Esta conclusión tiene implicaciones directas para el diseño de sistemas comerciales de seguimiento de personas: la elección de dónde portar el dispositivo debe considerar el compromiso entre comodidad del usuario y exactitud de localización requerida por la aplicación. Para aplicaciones que toleren errores de hasta 1 m, cadera y pecho son viables; para exactitud subdecimétrica, se recomienda cabeza o muñeca.

\item \textbf{El efecto de la obstrucción corporal presenta una fuerte dependencia con la condición de propagación.} La transición entre condiciones LOS y NLOS se manifiesta claramente en la variación del error, con factores de degradación que varían dramáticamente según la ubicación: desde 2.0× para cabeza en interior hasta 13.8× para pecho en exterior. Los errores en NLOS pueden alcanzar valores de hasta 2.90 m en casos extremos (cadera sujeto 3), confirmando que la body shadowing constituye uno de los limitantes principales de la exactitud en aplicaciones de seguimiento de personas. El análisis de ToF reveló que la variabilidad temporal (desviación estándar) aumenta 251.64\% en NLOS, permitiendo establecer un umbral de detección de 0.49 ns para clasificación automática de condiciones de propagación.

\item \textbf{La banda de 6.5 GHz presenta características mixtas para mitigación de body shadowing.} Los resultados obtenidos en este trabajo, comparados con estudios previos en frecuencias inferiores (3.5-4.5 GHz), muestran que la banda de 6.5 GHz experimenta mayor atenuación en tejido biológico pero se compensa parcialmente por mejor resolución temporal y capacidades de difracción en geometrías curvas como la cabeza. La comparación con Otim et al. (2019, simulaciones FDTD) revela que los errores experimentales en 6.5 GHz son comparables a las predicciones teóricas, mientras que estudios como Tanghe et al. (2023) en frecuencias más bajas reportan exactitudes similares (0.35-0.50 m), sugiriendo que el desempeño es más dependiente de las técnicas de mitigación implementadas que de la banda específica dentro del rango UWB.

\item \textbf{Las distribuciones de error en NLOS son marcadamente no-gaussianas.} El análisis estadístico riguroso mediante pruebas de Shapiro-Wilk ($p \ll 0.001$) y Levene ($p \ll 0.001$) rechazó categóricamente los supuestos de normalidad y homocedasticidad. Las distribuciones presentan asimetría positiva pronunciada con colas extendidas hacia valores extremos, especialmente para ubicaciones del torso en NLOS. Esta no-gaussianidad tiene implicaciones críticas para el diseño de algoritmos de localización: los enfoques tradicionales que asumen ruido gaussiano (como mínimos cuadrados ordinarios) producen estimaciones subóptimas. El filtro de Kalman implementado en este estudio logró reducir el error máximo en 41-49\% y la desviación estándar en 41-48\%, demostrando la efectividad de técnicas robustas para manejo de outliers y mediciones degradadas.

\item \textbf{El fenómeno de multitrayecto constructivo favorece el desempeño en interiores para ubicaciones específicas.} El análisis comparativo entre escenarios reveló que, contrario a la intuición, varias ubicaciones corporales presentan menor error en interior NLOS que en exterior NLOS: cabeza (18.66 cm vs 32.05 cm, $p \ll 0.001$, $r = -0.660$), mano (26.01 cm vs 41.60 cm, $p \ll 0.001$, $r = -0.693$), y tobillo (29.96 cm vs 38.45 cm, $p \ll 0.001$, $r = -0.488$). Este fenómeno se explica por las reflexiones en paredes, suelo y techo que generan trayectorias alternativas compensando la obstrucción de la línea de vista directa. En contraste, ubicaciones del torso (cadera, pecho) muestran diferencias triviales entre escenarios ($r < 0.05$), indicando que la obstrucción severa domina independientemente del entorno.

\item \textbf{La variabilidad inter-sujeto es significativa y debe considerarse en el diseño del sistema.} En la Fase 2 con tres sujetos evaluados, se observaron diferencias sustanciales en el desempeño para la misma ubicación corporal: cadera varió de 0.60 m (sujeto 1) a 0.95 m (sujeto 2), y pecho de 0.70 m (sujeto 3) a 0.91 m (sujeto 2). El análisis de ToF mostró que el pecho del sujeto 2 presentó la máxima desviación estándar (9.08 ns en NLOS) y rango (42.86 ns), mientras que el sujeto 3 en cadera mostró mejor estabilidad. Esto sugiere que características antropométricas individuales (constitución corporal, densidad de tejidos, postura) influyen significativamente en la propagación, y que sistemas operacionales podrían beneficiarse de una fase de calibración personalizada por usuario o de modelos adaptativos que aprendan patrones individuales.

\item \textbf{La exactitud de localización alcanzable en condiciones reales con body shadowing es subdecimétrica a métrica.} El sistema completo de posicionamiento 2D demostró exactitudes de 0.48 m para el tag de referencia (sin obstrucción), 0.55-0.95 m para cadera, y 0.70-0.91 m para pecho, con el filtro de Kalman. Esta exactitud es suficiente para aplicaciones de seguimiento de personas en manufactura inteligente, monitoreo de pacientes en hospitales, seguimiento de deportistas en entrenamiento, y sistemas de evacuación de emergencia. Para aplicaciones que requieren exactitud centimétrica (cirugía asistida, ensamblaje de precisión), será necesario implementar estrategias de mitigación adicionales como fusión con IMU, detección y compensación NLOS basada en el umbral ToF identificado (0.49 ns), o configuraciones multi-ancla con redundancia que permitan descartar mediciones degradadas. El percentil 95 de error de 1.14 m en la Fase 1 establece un límite superior realista para el 95\% de mediciones bajo condiciones adversas.
\end{enumerate}


\section{CONTRIBUCIONES DE LA INVESTIGACIÓN}
\label{sec:contribuciones}

Esta investigación ha generado las siguientes contribuciones al campo de los sistemas de posicionamiento en interiores:

\subsection{Contribuciones Científicas}

\begin{itemize}
\item \textbf{Caracterización experimental en 6.5 GHz:} Se proporciona la primera caracterización sistemática (según revisión de literatura realizada) del efecto de BS en sistemas UWB operando en 6.5 GHz, llenando una brecha en el conocimiento científico que se concentraba en frecuencias inferiores.

\item \textbf{Modelos estadísticos de error:} Se han desarrollado modelos de distribución de probabilidad del error de distancia para diferentes condiciones de propagación y ubicaciones corporales, que pueden ser utilizados para simulaciones realistas de desempeño de IPS.

\item \textbf{Protocolo experimental reproducible:} Se ha diseñado y documentado un protocolo experimental sistemático que puede ser adoptado por otros investigadores para estudios comparativos, facilitando la reproducibilidad y la comparación de resultados entre estudios.
\end{itemize}

\subsection{Contribuciones Prácticas}

\begin{itemize}
\item \textbf{Guías de diseño para sistemas comerciales:} Las recomendaciones derivadas de este estudio pueden informar el diseño de sistemas IPS UWB comerciales. Se estableció que para aplicaciones que requieran exactitud subdecimétrica, la ubicación óptima del dispositivo es cabeza o muñeca, mientras que para aplicaciones que toleren errores de hasta 1 m, cadera y pecho son aceptables priorizando la comodidad del usuario. Se recomienda implementar detección NLOS basada en el umbral de variabilidad ToF de 0.49 ns y aplicar ponderación diferencial (peso 0.28 para mediciones NLOS) en algoritmos de trilateración. La configuración óptima de anclas debe considerar altura mínima de 1.5 m para minimizar obstrucción por mobiliario y maximizar líneas de vista directas.

\item \textbf{Benchmarks de desempeño:} Los resultados cuantitativos obtenidos establecen valores de referencia contra los cuales se pueden comparar desarrollos tecnológicos futuros: tag de referencia 0.48 m, cadera 0.55-0.95 m, pecho 0.70-0.91 m para sistemas 2D con cuatro anclas. En estimación de distancia punto-a-punto, los benchmarks son: cabeza 4.87 cm (exterior LOS) / 18.16 cm (interior NLOS), pecho 4.62 cm (exterior LOS) / 83.98 cm (interior NLOS), cadera 8.31 cm (exterior LOS) / 97.76 cm (interior NLOS). El filtro de Kalman demostró capacidad de reducción de errores máximos de 41-49\% y desviación estándar de 41-48\%, estableciendo una línea base para evaluación de técnicas de mitigación alternativas.

\item \textbf{Validación de viabilidad tecnológica:} Los resultados confirman que la tecnología UWB en 6.5 GHz con módulos DWM1001 es viable para aplicaciones de seguimiento de personas en interiores con exactitudes de 0.5-1.0 m, suficientes para casos de uso como monitoreo de trabajadores en manufactura, seguimiento de pacientes en hospitales, análisis de movimiento deportivo, y sistemas de evacuación de emergencia. Para aplicaciones críticas que requieren exactitud centimétrica (robótica quirúrgica, ensamblaje de precisión), se requieren técnicas de mitigación adicionales como fusión con IMU, configuraciones multi-ancla con redundancia, o calibración personalizada por usuario. El sistema demostró robustez aceptable bajo condiciones NLOS moderadas (ubicaciones en extremidades) pero degradación significativa bajo NLOS severa (ubicaciones en torso), confirmando la necesidad de considerar body shadowing explícitamente en el diseño de sistemas operacionales.
\end{itemize}


\section{TRABAJOS FUTUROS}
\label{sec:trabajos_futuros}

Los hallazgos de esta investigación abren múltiples direcciones para trabajos futuros que pueden profundizar o extender el conocimiento generado:

\subsection{Extensiones Experimentales}

\begin{enumerate}
\item \textbf{Evaluación en múltiples escenarios:} Replicar el estudio en entornos con características arquitectónicas y de propagación diversas (espacios abiertos tipo almacén, oficinas con particiones, áreas con alta densidad de mobiliario metálico) para evaluar la generalización de los resultados.

\item \textbf{Aumento del tamaño muestral:} Incrementar el número de participantes para reforzar el análisis estadístico de la influencia de características antropométricas y explorar efectos de segunda orden (edad, género, composición corporal).

\item \textbf{Mediciones dinámicas exhaustivas:} Extender la caracterización a escenarios dinámicos donde el participante realiza actividades naturales (caminar en diferentes velocidades, subir escaleras, movimientos complejos), capturando la variabilidad adicional introducida por el movimiento.

\item \textbf{Evaluación de BS múltiple:} Analizar el efecto de múltiples personas presentes simultáneamente en el escenario, donde un nodo móvil puede experimentar obstrucción no solo por el cuerpo que lo porta sino también por otras personas cercanas.
\end{enumerate}

\subsection{Desarrollo de Técnicas de Mitigación}

\begin{enumerate}
\item \textbf{Algoritmos de detección NLOS:} Desarrollar y validar algoritmos de aprendizaje automático que clasifiquen cada medición de ToF como LOS o NLOS basándose en características de la señal. Este trabajo identificó que la variabilidad temporal (desviación estándar del ToF) constituye el indicador más confiable, con un incremento de 251.64\% en NLOS y un umbral sugerido de 0.49 ns. Trabajos futuros deberían explorar clasificadores supervisados (SVM, Random Forest, redes neuronales) entrenados con características adicionales como potencia recibida, relación entre primer trayecto y trayectos subsecuentes, parámetros del canal estimados (CIR), y kurtosis de la distribución de ToF. La integración de estos clasificadores permitiría descartar o ponderar diferencialmente mediciones según su condición de propagación, mejorando la robustez del sistema.

\item \textbf{Filtros adaptativos avanzados:} Si bien el filtro de Kalman estándar demostró efectividad en reducir errores máximos (41-49\%) y desviación estándar (41-48\%), trabajos futuros deberían implementar y comparar esquemas más sofisticados como Extended Kalman Filter (EKF) para manejo de no-linealidades en trilateración, Unscented Kalman Filter (UKF) para propagación más precisa de incertidumbre, y filtros de partículas para representación completa de distribuciones no-gaussianas observadas en NLOS. Los parámetros del filtro (matrices de covarianza de proceso y medición) deberían ajustarse dinámicamente según la condición LOS/NLOS detectada y la ubicación corporal identificada, aprovechando los modelos estadísticos de error desarrollados en este estudio.

\item \textbf{Fusión con sensores inerciales:} Integrar datos de IMU (acelerómetros, giroscopios, magnetómetros) para estimar la orientación del cuerpo en tiempo real y aplicar modelos de corrección de error dependientes de la orientación. Los resultados de este trabajo muestran que la degradación LOS-NLOS varía dramáticamente según la ubicación (factor 2.0× para cabeza vs 13.2× para cadera), sugiriendo que conocer la orientación del usuario relativa a cada ancla permitiría predecir qué mediciones estarán degradadas por body shadowing y aplicar correcciones específicas. Técnicas de fusión sensorial mediante filtro de Kalman extendido o algoritmos SLAM (Simultaneous Localization and Mapping) podrían mejorar la robustez y continuidad de las estimaciones, especialmente durante períodos transitorios de obstrucción.

\item \textbf{Corrección basada en modelos de propagación:} Desarrollar modelos fenomenológicos o basados en datos (machine learning) que predigan el sesgo de error esperado dada la configuración geométrica, la ubicación corporal del dispositivo, y la orientación estimada del usuario. Los modelos de regresión podrían entrenarse con los datos experimentales de este estudio, aprendiendo la relación entre características observables (ToF, RSSI, estadísticas de canal) y el error real de distancia. Alternativamente, modelos físicos basados en teoría de difracción (Geometrical Theory of Diffraction - GTD) parametrizados con las características dieléctricas del tejido humano a 6.5 GHz podrían proporcionar estimaciones de la atenuación y retardo esperados. Estas predicciones se aplicarían como correcciones antes del algoritmo de trilateración, potencialmente reduciendo el sesgo sistemático observado en NLOS.

\item \textbf{Configuraciones multi-ancla con redundancia y selección adaptativa:} Ampliar la configuración de cuatro anclas utilizada en este estudio hacia sistemas con 6-8 anclas distribuidas no solo en las esquinas sino también en posiciones intermedias y a diferentes alturas. Esta redundancia permitiría implementar algoritmos de selección que, en cada instante, utilicen únicamente el subconjunto de mediciones con mejor calidad (detectadas como LOS o con baja variabilidad ToF). Técnicas como RANSAC (Random Sample Consensus) podrían identificar y descartar outliers, mientras que ponderación por calidad (inverse variance weighting) priorizaría automáticamente las mediciones más confiables, mitigando el impacto de body shadowing sin requerir calibración específica por usuario.
\end{enumerate}

\subsection{Estudios Comparativos}

\begin{enumerate}
\item \textbf{Comparación multi-frecuencia:} Realizar estudios experimentales controlados donde se comparen directamente múltiples bandas de frecuencia UWB (3.5, 4.5, 6.5, 8 GHz) bajo condiciones idénticas para cuantificar el compromiso entre atenuación, difracción y resolución temporal.

\item \textbf{Comparación multi-tecnología:} Contrastar el desempeño de UWB con tecnologías alternativas (WiFi RTT, Bluetooth AoA, sistemas híbridos) en presencia de BS para identificar las fortalezas y debilidades relativas de cada tecnología.
\end{enumerate}

\subsection{Modelado y Simulación}

\begin{enumerate}
\item \textbf{Modelos electromagnéticos refinados:} Desarrollar simulaciones FDTD de alta fidelidad que incorporen modelos antropomórficos detallados del cuerpo humano con propiedades dieléctricas diferenciadas por tejido, validándolos contra los datos experimentales de este estudio.

\item \textbf{Simuladores de nivel de sistema:} Crear simuladores que integren los modelos estadísticos de error derivados de este estudio, permitiendo evaluar rápidamente diferentes arquitecturas de sistemas IPS sin necesidad de despliegues experimentales costosos.
\end{enumerate}

\subsection{Aplicaciones Especializadas}

\begin{enumerate}
\item \textbf{Sistemas para grupos específicos de usuarios:} Adaptar y optimizar los sistemas IPS para aplicaciones específicas como seguimiento de pacientes geriátricos en hospitales, tracking de bomberos en emergencias (considerando equipo de protección adicional que afecta la propagación), o atletas de alto rendimiento con requerimientos de exactitud centimétrica.

\item \textbf{Integración con arquitecturas IoT:} Desarrollar arquitecturas completas que integren los IPS UWB con infraestructura IoT, plataformas de analítica de datos y sistemas de toma de decisiones para aplicaciones industriales 4.0.
\end{enumerate}

