\chapter{ANÁLISIS Y DISCUSIÓN}
\label{ch:analisis}

En este capítulo se interpretan y discuten los resultados experimentales presentados en el capítulo anterior. Se contrastan los hallazgos con el estado del arte revisado en el capítulo de revisión sistemática de literatura, se explican los fenómenos físicos subyacentes que dan cuenta de los patrones observados, y se analizan las implicaciones prácticas de los resultados para el diseño e implementación de sistemas de posicionamiento UWB en escenarios reales.

\section{INTERPRETACIÓN DE LOS PATRONES DE ERROR OBSERVADOS}
\label{sec:interpretacion_patrones}

\subsection{Efecto de la Ubicación Corporal del Dispositivo}

Los resultados experimentales revelan que la ubicación donde se porta el dispositivo UWB tiene un impacto crítico sobre la exactitud de la estimación de distancia. La Fase 1 de validación proporcionó caracterización detallada de siete ubicaciones corporales bajo diferentes condiciones de propagación.

La ubicación en el pecho presentó la mejor exactitud en condiciones LOS, alcanzando MAE de 4.62 cm en exterior y 13.74 cm en interior, siendo una de las configuraciones más precisas del estudio. Sin embargo, experimenta la mayor degradación en NLOS, alcanzando MAE de 63.69 cm (exterior) y 83.98 cm (interior), con factores de degradación de 13.8× y 6.1× respectivamente. Esta degradación severa se debe a que el torso, cuando está alejado del nodo ancla, actúa como obstáculo completo.

La ubicación en la cabeza mostró el comportamiento más consistente. En condiciones LOS logró exactitud de 4.87 cm (exterior) y 8.91 cm (interior). En NLOS, la degradación fue moderada: 32.43 cm (exterior) y 18.16 cm (interior) con factor de 6.7× en exterior y 2.0× en interior. La geometría esférica del cráneo favorece la difracción de señales, manteniendo buena exactitud incluso con obstrucción.

La muñeca presentó excelente consistencia con MAE de 7.78 cm (exterior LOS) y 6.42 cm (interior LOS), degradándose a 24.21 cm y 22.00 cm en NLOS respectivamente. El brazo tiene menor masa corporal y mayor movilidad, permitiendo exactitud superior en la mayoría de casos.

La cadera exhibió el peor desempeño: excelente exactitud en LOS (8.31 cm exterior, 7.19 cm interior) pero degradación extrema en NLOS (94.78 cm exterior, 97.76 cm interior) con factores de 11.4× y 13.6×, los mayores registrados. Esta ubicación sufre obstrucción severa por la pelvis y el torso.


\section{ANÁLISIS DE LA VARIABILIDAD ESTADÍSTICA}
\label{sec:analisis_variabilidad}

\subsection{Naturaleza de las Distribuciones de Error}

El análisis de distribuciones de error reveló que en condiciones LOS, la exactitud presenta distribuciones aproximadamente gaussianas, con desviaciones estándar en el rango de 4-10 cm según ubicación corporal. Esto es consistente con ruido de medición gaussiano dominado por efectos de cuantización temporal en el DW1000 y ruido térmico de los receptores. La exactitud relativa (error relativo respecto a distancia medida) en LOS es aproximadamente 0.5-1.0% para distancias de 1-5 m.

En condiciones NLOS, las distribuciones exhiben asimetría positiva pronunciada y colas pesadas, con desviaciones estándar que alcanzan 36-71 cm en ubicaciones críticas como cadera y pecho. Esta no-gaussianidad refleja la naturaleza determinista del fenómeno de obstrucción corporal: cuando el cuerpo bloquea el trayecto directo, el error no es una fluctuación aleatoria simétrica, sino un sesgo sistemático positivo y altamente variable. El percentil 95 del error en NLOS alcanza 1.14 m globalmente, indicando que eventos de error extremo son más frecuentes que lo esperado en una distribución normal, lo cual tiene implicaciones significativas para algoritmos de localización que asuman gaussianidad.

\subsection{Implicaciones para Algoritmos de Localización}

La no-gaussianidad del error en NLOS tiene implicaciones importantes para el diseño de algoritmos de localización y filtrado:

Los algoritmos tradicionales de trilateración por mínimos cuadrados asumen ruido gaussiano en las mediciones de distancia. Nuestros resultados experimentales demuestran que cuando este supuesto se viola severamente, como ocurre en presencia de obstrucción corporal, el estimador de mínimos cuadrados produce estimaciones de posición significativamente sesgadas, con errores máximos hasta 3 veces superiores al promedio.

En la Fase 2, la aplicación del Filtro de Kalman demostró mitigar parcialmente este problema. Para el tag (sin obstrucción), la multilateración directa mostró mejor exactitud (MAE 0.4802 m) que el filtro de Kalman (MAE 0.7985 m), confirmando que sin obstrucción corporal, el algoritmo simple es óptimo. Sin embargo, en presencia de cuerpo humano, los resultados variaron significativamente:

\begin{itemize}
\item \textbf{Cadera (peor caso):} Exactitud con multilateración de 0.55-0.81 m versus Kalman de 0.77-0.95 m. Multilateración mantiene ligera ventaja incluso en este caso crítico de alta obstrucción.
\item \textbf{Pecho:} Resultados mixtos según sujeto. Sujeto 1: Kalman mejor (0.7454 m vs 0.8471 m). Sujeto 2: Multilateración mejor (0.7761 m vs 0.9108 m). Sujeto 3: Kalman marginalmente mejor (0.7027 m vs 0.7175 m).
\end{itemize}

El valor agregado del Filtro de Kalman se manifiesta principalmente en la reducción de errores máximos y variabilidad: en pecho del sujeto 1, Kalman reduce error máximo de 2.67 m a 1.37 m (49\% reducción) y desviación estándar de 0.58 m a 0.34 m (41\% reducción). Esto demuestra que aunque el error promedio puede no mejorar significativamente, la estabilidad de las estimaciones es sustancialmente mejorada, lo cual es crítico para aplicaciones que requieren exactitud consistente.

Enfoques adicionales que podrían mejorar la exactitud incluyen: (1) mínimos cuadrados ponderados iterativamente para descartar mediciones con errores atípicos, (2) detección NLOS mediante análisis de características de señal para aplicar modelos de error específicos, (3) Filtros de Partículas para manejar distribuciones no gaussianas, (4) fusión con sensores inerciales para mantener exactitud durante períodos de degradación de mediciones de distancia.


\section{DESEMPEÑO DEL SISTEMA DE POSICIONAMIENTO COMPLETO}
\label{sec:discusion_posicionamiento}

\subsection{Exactitud de Localización Alcanzada}

El análisis del error de posición 2D (que integra las mediciones de distancia hacia los cuatro nodos ancla) en la Fase 2 muestra que el sistema alcanza exactitudes variadas dependiendo de la ubicación corporal:

\begin{itemize}
\item \textbf{Tag (referencia sin cuerpo):} Multilateración logra exactitud de 0.48 m promedio, confirmando que la geometría de los nodos ancla y el algoritmo base proporcionan precisión razonable en ausencia de obstrucción.
\item \textbf{Cadera:} Exactitud de 0.55-0.95 m, siendo la ubicación que presenta peor desempeño en el sistema 2D. La obstrucción severa del torso y pelvis degrada la exactitud de todas las distancias simultáneamente.
\item \textbf{Pecho:} Exactitud de 0.70-0.91 m, comparable a cadera en algunos sujetos pero con variabilidad inter-sujeto importante, sugiriendo dependencia de características antropométricas individuales.
\end{itemize}

Considerando que el escenario experimental tiene dimensiones 10.4 m × 7.4 m (área de 77 m²), una exactitud de posicionamiento de 0.70-0.95 m representa un error relativo de 6.7-9.1\% respecto a la diagonal máxima del escenario (12.8 m). Esta exactitud es comparable a sistemas comerciales de localización UWB reportados en escenarios con obstrucción corporal presente (típicamente 0.5-1.5 m), pero significativamente inferior a la exactitud en condiciones LOS puro (10-30 cm) reportada en la literatura para sistemas UWB sin obstrucción.

\subsection{Dependencia de la Geometría del Despliegue}

La configuración geométrica de los nodos ancla influye en la exactitud alcanzable mediante el factor de Dilución de Precisión Geométrica (GDOP). En nuestro escenario experimental con cuatro nodos ancla distribuidos en las esquinas de un rectángulo de 10.4 m × 7.4 m, el GDOP varía significativamente según la posición dentro del escenario. Las zonas del escenario donde el nodo móvil está aproximadamente equidistante de los cuatro nodos ancla presentan mejor exactitud debido a que la geometría favorece la trilateración. 

En contraste, las posiciones próximas a uno de los nodos ancla (distancia < 1.5 m) presentan degradación de exactitud debido a GDOP desfavorable, fenómeno que se observó en varios puntos de medición donde el error de posición fue mayor de lo esperado por degradación de exactitud de distancia sola.

Adicionalmente, la propagación de exactitud de distancia hacia exactitud de posición depende de si los errores en las cuatro mediciones de distancia están correlacionados o son independientes. En nuestros resultados, la obstrucción corporal tiende a afectar principalmente la(s) medición(es) hacia ancla(s) en determinada(s) dirección(es) relativa(s) al cuerpo, introduciendo correlación que amplifica el error de posición cuando múltiples nodos sufren obstrucción simultáneamente.


\section{COMPARACIÓN CON EL ESTADO DEL ARTE}
\label{sec:comparacion_estado_arte}

\subsection{Concordancia con Estudios Experimentales}

Varios estudios experimentales previos han caracterizado el efecto de la obstrucción corporal en sistemas UWB, permitiendo comparaciones directas con nuestros resultados.

El trabajo de Tanghe et al. (2023) reportó errores de distancia promedio de 15-25 cm en condiciones LOS y 40-120 cm en NLOS para un dispositivo UWB operando en banda similar (6-8 GHz) con nodo en muñeca. Nuestro estudio obtuvo MAE de 6.42-7.78 cm (LOS) y 22-24 cm (NLOS) para muñeca, mostrando exactitud superior en ambas condiciones. Esta mejora puede atribuirse a: (1) calibración rigurosa de antenna delays implementada en nuestro estudio, (2) mayor número de mediciones por punto (250 vs 50-100 en Tanghe), permitiendo caracterización estadística más robusta, (3) condiciones de laboratorio más controladas en nuestros experimentos de validación.

El estudio de Schimtt et al. (2019) analizó el impacto de BS en IPS basados en RF, reportando degradación de exactitud de 12 cm (LOS) a 85 cm (NLOS promedio) para dispositivos en torso. Nuestros resultados para pecho muestran concordancia notable: 4.62-13.74 cm (LOS) degradando a 63.69-83.98 cm (NLOS). La similitud cuantitativa confirma que la magnitud del efecto de BS en la banda 6-7 GHz es consistente entre diferentes implementaciones de hardware UWB.

Pradabphon et al. (2019) evaluaron experimentalmente la propagación UWB con cuerpo humano, reportando que la ubicación en cabeza presentaba menor variabilidad de error que ubicaciones en torso. Este hallazgo es directamente confirmado por nuestros datos: la desviación estándar del error en cabeza (3.72-11.20 cm) es consistentemente menor que en cadera (5.34-71.00 cm) y pecho (3.09-58.35 cm), validando la recomendación de cabeza como ubicación óptima para aplicaciones que requieren exactitud consistente.

\subsection{Aporte Diferencial de este Estudio}

Este trabajo aporta evidencia experimental específica para la banda de 6.5 GHz, que ha sido menos estudiada que bandas inferiores (3-4 GHz) o superiores (7-10 GHz). Los resultados sugieren que esta banda representa un punto óptimo de compromiso: (1) Exactitud en LOS comparable a bandas superiores (4.6-8.3 cm), (2) Degradación moderada en NLOS para ubicaciones anatómicas favorables como cabeza (factor 2.0-6.7×), significativamente mejor que lo reportado para bandas inferiores donde el mayor tamaño de la zona de Fresnel reduce la resolución de trayectos múltiples, (3) Penetración de materiales constructivos suficiente para mantener conectividad en escenarios de interiores complejos.

Adicionalmente, la caracterización sistemática de \textbf{siete ubicaciones corporales} (cabeza, cadera, mano, muñeca, pecho, rodilla, tobillo) bajo un protocolo experimental uniforme (mismo hardware, mismo escenario, mismo sujeto en Fase 1, mismas condiciones ambientales) permite comparaciones directas que no estaban disponibles en la literatura previa. Estudios anteriores evaluaban típicamente 1-3 ubicaciones en configuraciones experimentales heterogéneas, imposibilitando conclusiones robustas sobre la ubicación óptima. Nuestro enfoque sistemático establece claramente que: (1) cabeza minimiza error y variabilidad en todas las condiciones, (2) muñeca constituye alternativa viable con exactitud solo ligeramente degradada, (3) cadera y pecho deben evitarse en aplicaciones críticas debido a degradación extrema en NLOS.

Finalmente, este estudio es uno de los primeros en evaluar cuantitativamente el desempeño del Filtro de Kalman en sistemas UWB con obstrucción corporal real. Mientras que trabajos previos propusieron KF teóricamente o lo evaluaron en simulación, nuestros resultados experimentales en Fase 2 demuestran que: (1) KF no necesariamente mejora exactitud promedio en presencia de BS severo, (2) el valor agregado del KF radica en reducción de errores máximos (41-49\%) y estabilización de varianza (41-48\% reducción de desviación estándar), (3) la efectividad del KF depende críticamente de la ubicación corporal y características del sujeto, sugiriendo necesidad de filtrado adaptativo con detección online de condición de propagación.


\section{IMPLICACIONES PARA APLICACIONES PRÁCTICAS}
\label{sec:implicaciones_practicas}

\subsection{Recomendaciones de Diseño}

Basándose en los resultados obtenidos, se pueden derivar recomendaciones prácticas para el diseño de sistemas IPS basados en UWB cuando se anticipa obstrucción corporal:

\begin{enumerate}
\item \textbf{Ubicación óptima del dispositivo:} Los resultados de la Fase 1 indican que portar el dispositivo en la cabeza minimiza el error medio y la variabilidad en ambas condiciones LOS y NLOS, alcanzando exactitud de 4.87 cm en LOS exterior y manteniendo 18.16 cm en NLOS interior (factor degradación 2.0×, el más bajo registrado). Para aplicaciones donde la exactitud es crítica, se recomienda esta ubicación. La muñeca constituye segunda opción con exactitud similar en LOS (6.42-7.78 cm) pero mayor degradación en NLOS (22-24 cm). Deben evitarse cadera y pecho para aplicaciones que requieran exactitud consistente bajo orientaciones variables, dado su factor de degradación LOS/NLOS de 11-14×.

\item \textbf{Número y distribución de nodos ancla:} En escenarios con obstrucción corporal inevitable, se recomienda configuraciones con más de cuatro nodos ancla distribuidos para maximizar cobertura de línea de vista desde múltiples ángulos. Con cuatro nodos en esquinas de un rectángulo (configuración usada en Fase 2), la probabilidad de que todos cuatro enlaces estén simultáneamente en NLOS severo es alta en posiciones donde el cuerpo actúa de barrera central. Aumentar a 6-8 nodos distribuidos en puntos altos (paredes, techo) reduce esta probabilidad.

\item \textbf{Estrategias de mitigación:} La aplicación del Filtro de Kalman redujo el error máximo de posición en 41-49\% y la desviación estándar en 41-48\% en los casos críticos (pecho), mejorando estabilidad aunque no necesariamente exactitud promedio. Se recomienda implementar Kalman filtering como técnica estándar. Complementariamente, se recomienda implementar algoritmos de detección NLOS mediante análisis de características de señal (amplitud, ancho de pulso) para descartar o ponderar diferencialmente mediciones con error predicho > 50 cm, mejorando robustez.

\item \textbf{Expectativas realistas de desempeño:} En aplicaciones con movilidad libre del usuario donde la orientación relativa varía continuamente, debe anticiparse exactitud de localización del orden de 70-95 cm (promedio), significativamente mayor que exactitud nominal en LOS puro (48 cm sin cuerpo). El percentil 95 del error de posición 2D alcanza aproximadamente 1.2-1.5 m, reflejando la variabilidad inherente a obstrucción corporal. Para aplicaciones de seguimiento de personal en interiores, esta exactitud es funcional; para análisis biomecánico detallado, se requeriría sensores inerciales complementarios.
\end{enumerate}

\subsection{Aplicabilidad a Casos de Uso Específicos}

\textbf{Seguimiento de personal en entornos industriales:} La exactitud alcanzada de 70-95 cm es suficiente para aplicaciones de seguimiento grueso de personal (ej. verificar en qué zona amplia se encuentra un trabajador, geofencing de áreas restringidas). Sin embargo, para aplicaciones que requieren exactitud inferior a 50 cm, como prevención de colisiones con maquinaria o localización precisa de herramientas, se requeriría complementar UWB con sensores inerciales o detectores de proximidad de corto rango.

\textbf{Localización en hospitales:} La exactitud de 70-95 cm es comparable a requisitos típicos de seguimiento de pacientes y personal médico en plantas hospitalarias. Permite identificar en qué habitación o corredor se encuentra una persona, siendo funcional para aplicaciones de alertas de eventos adversos (paciente fuera de cama, personal en zona no autorizada). La ubicación óptima en cabeza es prácticamente implementable mediante dispositivos wearables tipo audífonos o diademas.

\textbf{Deportes y análisis biomecánico:} La exactitud de 70-95 cm en posicionamiento 2D horizontal es insuficiente para análisis detallado de cinemática de movimiento, que típicamente requiere exactitud de 5-10 cm. UWB sería complementario a sistemas de captura de movimiento ópticos o inerciales, proporcionando contexto espacial cuando se requiere tracking en espacios amplios.


\section{LIMITACIONES DEL ESTUDIO}
\label{sec:limitaciones}

Es importante reconocer las limitaciones inherentes a este estudio experimental:

\begin{enumerate}
\item \textbf{Tamaño muestral limitado:} Con 3 participantes en la Fase 2, la generalización de conclusiones sobre influencia de características antropométricas debe tomarse con cautela. Sin embargo, la Fase 1 incluyo múltiples mediciones (250+ por configuración) de un único sujeto, proporcionando caracterización estadística robusta de ubicaciones corporales. Estudios futuros con mayor número de participantes permitirían análisis de variabilidad inter-sujeto.

\item \textbf{Escenario experimental único:} Los experimentos se realizaron en un salón cerrado de dimensiones 10.4 m × 7.4 m con paredes de hormigón. Las características específicas de este entorno (propiedades dieléctricas de materiales, dimensiones, ausencia de mobiliario complejo) influyen en los resultados. La replicación en múltiples escenarios con geometrías, tamaños y características de propagación diversas (hospitales, fábricas, oficinas) fortalecería la validez externa de conclusiones.

\item \textbf{Mediciones predominantemente estáticas:} Aunque se incluyeron transiciones entre posiciones, la mayoría de datos corresponden a condiciones estáticas (persona manteniendo postura en cada punto). En aplicaciones reales con el usuario caminando continuamente, efectos adicionales como aceleración corporal, variaciones posturales dinámicas y mayor multitrayecto podrían influir en desempeño.

\item \textbf{Configuración de hardware específica:} Resultados obtenidos con módulos DWM1001 (transceptor DW1000 a 6.5 GHz). Diferentes implementaciones UWB (otras bandas de frecuencia como 3.5 GHz o 7.25 GHz, diferentes tipos de antena, procesadores de señal distintos) presentarían variaciones en exactitud.

\item \textbf{Condiciones controladas:} Experimentos realizados con actividad humana externa minimizada. En escenarios operacionales reales con múltiples personas en movimiento, el efecto de obstrucción corporal múltiple simultánea (varias personas bloqueando diferentes enlaces) e interferencia inter-usuario degradaría adicionalemente exactitud.
\end{enumerate}
