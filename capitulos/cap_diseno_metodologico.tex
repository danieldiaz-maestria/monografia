\chapter{DISEÑO METODOLÓGICO}
\label{ch:metodologia_experimental}

En este capítulo se describe detalladamente la metodología empleada para llevar a cabo la investigación experimental, estructurada en dos fases secuenciales:

\textbf{Fase 1 - Plan de Pruebas de Validación:} Evaluación exhaustiva de la exactitud en la estimación de distancia UWB analizando el impacto de la ubicación del dispositivo móvil en diferentes partes del cuerpo humano bajo condiciones de propagación LOS y NLOS en dos escenarios representativos (exterior e interior). Esta fase, ya ejecutada y documentada en el artículo de validación, establece la línea base de desempeño del sistema.

\textbf{Fase 2 - Sistema de Posicionamiento 2D:} Implementación y evaluación del sistema completo de localización utilizando cuatro nodos ancla para determinar posiciones 2D mediante trilateración. Esta fase integrará los hallazgos de la Fase 1 para evaluar el error de posicionamiento en el plano.

Este capítulo se enfoca en describir la metodología de la Fase 1, que constituye el plan de pruebas de validación completo del sistema de estimación de distancia a 6.5 GHz.

\section{CONFIGURACIÓN DEL HARDWARE}
\label{sec:configuracion_hardware}

\subsection{Dispositivos UWB}

\subsubsection{Selección de Dispositivos}

Para este estudio se han seleccionado módulos de desarrollo Qorvo DWM1001, que integran un transceptor UWB DW1000, un microcontrolador Nordic Semiconductor nRF52832 y un sensor de movimiento. Los módulos operan con el firmware PANS (Positioning and Networking Stack) de Qorvo, el cual implementa la técnica TWR para estimación de distancia. Estos dispositivos están configurados para operar en la banda de frecuencia central de 6.5 GHz. La elección de estos dispositivos se fundamenta en:

\begin{itemize}
\item Capacidad de operar en la banda de 6.5 GHz con el ancho de banda necesario para lograr resolución temporal adecuada.
\item Soporte nativo para mediciones TWR con registro de timestamps de alta precisión.
\item Interfaz de comunicación accesible que permite la extracción de datos crudos de ToF y otras métricas de señal.
\item Consumo energético compatible con operación portátil prolongada.
\item Disponibilidad de documentación técnica detallada y herramientas de desarrollo.
\end{itemize}

\subsubsection{Parámetros de Configuración}

Los dispositivos UWB se configuran con los siguientes parámetros operacionales:

\begin{itemize}
\item \textbf{Frecuencia central:} 6489.6 MHz (Canal 5)
\item \textbf{Ancho de banda:} 499.2 MHz
\item \textbf{Tasa de transmisión de datos:} 6.8 Mbps
\item \textbf{Frecuencia de repetición de pulsos (PRF):} 64 MHz
\item \textbf{Longitud del preámbulo:} 128 símbolos
\item \textbf{Código del preámbulo (TX/RX):} 10
\item \textbf{Potencia de transmisión:} -17 dBm
\item \textbf{Sensibilidad del receptor:} -93 dBm
\end{itemize}

Estos parámetros se seleccionan para optimizar el compromiso entre exactitud de medición, alcance efectivo y consumo energético, siguiendo las recomendaciones del fabricante y las mejores prácticas identificadas en la revisión de literatura.


\section{DISEÑO DEL ESCENARIO EXPERIMENTAL - FASE 1}
\label{sec:escenario_experimental}

\subsection{Descripción General}

El plan de pruebas de validación (Fase 1) se implementó en dos escenarios principales diseñados para evaluar diferentes condiciones de propagación características de aplicaciones reales de sistemas de posicionamiento indoor y outdoor. La selección de estos escenarios permite caracterizar el comportamiento del sistema UWB tanto en ambientes con mínima reflexión (exterior) como en entornos con alta dispersión multitrayecto (interior tipo corredor).

\subsection{Escenario Exterior}

Se utilizó un campo abierto con dimensiones similares a las de un campo de fútbol. Este escenario fue diseñado para:

\begin{itemize}
\item Minimizar las reflexiones provenientes de estructuras verticales como paredes y edificios.
\item Proporcionar un entorno base para analizar los efectos primarios de propagación.
\item Aislar el impacto de la obstrucción corporal (BS) sobre la señal directa.
\item Considerar fenómenos fundamentales de propagación como las Pérdidas en Espacio Libre (FSPL) y el multitrayecto generado por reflexión en el suelo.
\end{itemize}

\textbf{Características del Escenario Exterior:}
\begin{itemize}
\item \textbf{Tipo:} Campo abierto (dimensiones ~100 m × 60 m)
\item \textbf{Superficie:} Césped natural
\item \textbf{Obstrucciones:} Mínimas (ausencia de paredes y estructuras verticales cercanas)
\item \textbf{Condiciones climáticas:} Mediciones realizadas en días despejados para minimizar variables ambientales
\end{itemize}

\subsection{Escenario Interior}

Se empleó un corredor de edificio universitario para representar condiciones típicas de aplicaciones indoor:

\textbf{Características del Escenario Interior:}
\begin{itemize}
\item \textbf{Dimensiones:} 20 m de largo × 3 m de ancho
\item \textbf{Materiales constructivos:} Paredes de concreto, piso de baldosa cerámica, techo de placa de concreto
\item \textbf{Mobiliario:} Corredor típico con mobiliario mínimo
\item \textbf{Características de propagación:} Introduce efectos de multitrayecto controlados resultantes de reflexiones en paredes, techo y piso
\item \textbf{Fuentes de interferencia:} Redes WiFi institucionales activas en bandas de 2.4 GHz y 5 GHz, equipos electrónicos en oficinas adyacentes
\end{itemize}

Este escenario permite evaluar cómo las reflexiones especulares en superficies confinadas afectan la estimación de distancia, representando condiciones realistas para sistemas de localización indoor.

\subsection{Configuración de Nodos - Fase 1}

En la Fase 1 del plan de pruebas de validación, la configuración experimental consistió en:

\begin{itemize}
\item \textbf{Nodo Fijo de Referencia:} Un módulo DWM1001 montado en un trípode a una altura estándar de 1.5 m sobre el nivel del piso, actuando como ancla de referencia para las mediciones de distancia.
\item \textbf{Nodo Móvil:} Un módulo DWM1001 portado por el sujeto de prueba en diferentes ubicaciones corporales.
\end{itemize}

Esta configuración punto a punto permite evaluar la exactitud fundamental de la estimación de distancia antes de implementar el sistema completo de posicionamiento 2D. El nodo fijo permanece en una ubicación constante durante cada sesión de medición, mientras el sujeto se desplaza a distancias conocidas siguiendo un protocolo sistemático.

\textbf{Nota:} La Fase 2 implementará un sistema con cuatro nodos ancla en configuración geométrica optimizada para trilateración y posicionamiento 2D.


\section{PROTOCOLO DE RECOLECCIÓN DE DATOS}
\label{sec:protocolo_recoleccion}

\subsection{Sujetos de Prueba}

La Fase 1 de este estudio fue conducida como un caso de estudio fundacional con el Sujeto de Prueba 2. Este enfoque metodológico permitió una caracterización detallada de los efectos de la ubicación del dispositivo en el cuerpo sin variabilidad inter-sujeto, estableciendo una línea base de desempeño del sistema y un control riguroso de variables experimentales.

Para la Fase 2, se contó con la participación de los tres sujetos de prueba (Sujeto 1, Sujeto 2 y Sujeto 3, siendo este último el mismo participante de la Fase 1). Este plan de pruebas exhaustivo, que involucró a múltiples sujetos con diversas características antropométricas, ha permitido una evaluación más robusta y generalizable del sistema de posicionamiento. Los sujetos de prueba participaron bajo consentimiento informado, con conocimiento completo del propósito del estudio, los procedimientos a realizar y el uso de los datos recolectados.

A continuación, se presentan las características antropométricas detalladas de los tres sujetos participantes en la Fase 2, así como sus respectivas fotografías:

\begin{figure}[hbt]
    \centering
    \includegraphics[width=0.3\textwidth]{imagenes/sujeto1.jpg}
    \caption{Sujeto de prueba 1}
    \label{fig:sujeto1}
\end{figure}

\begin{table}[hbt]
    \centering
    \caption{Características Antropométricas - Sujeto 1}
    \label{tab:antropometricas_sujeto1}
    \begin{tabular}{lcc}
        \toprule
        \rowcolor{headerblue}
        \textbf{Característica} & \textbf{Valor} & \textbf{Unidad} \\
        \midrule
        Estatura                & 1.75           & m                \\
        Peso                    & 70             & kg               \\
        Índice de Masa Corporal & 22.86          & kg/m²            \\
        Género                  & Masculino      & -                \\
        Edad                    & 25             & años             \\
        \bottomrule
    \end{tabular}
\end{table}

\begin{figure}[hbt]
    \centering
    \includegraphics[width=0.3\textwidth]{imagenes/sujeto2.jpg}
    \caption{Sujeto de prueba 2}
    \label{fig:sujeto2}
\end{figure}

\begin{table}[hbt]
    \centering
    \caption{Características Antropométricas - Sujeto 2}
    \label{tab:antropometricas_sujeto2}
    \begin{tabular}{lcc}
        \toprule
        \rowcolor{headerblue}
        \textbf{Característica} & \textbf{Valor} & \textbf{Unidad} \\
        \midrule
        Estatura                & 1.62           & m                \\
        Peso                    & 60             & kg               \\
        Índice de Masa Corporal & 22.86          & kg/m²            \\
        Género                  & Masculino      & -                \\
        Edad                    & 28             & años             \\
        \bottomrule
    \end{tabular}
\end{table}

\begin{figure}[hbt]
    \centering
    \includegraphics[width=0.3\textwidth]{imagenes/sujeto3.jpg}
    \caption{Sujeto de prueba 3}
    \label{fig:sujeto3}
\end{figure}

\begin{table}[hbt]
    \centering
    \caption{Características Antropométricas - Sujeto 3}
    \label{tab:antropometricas_sujeto3}
    \begin{tabular}{lcc}
        \toprule
        \rowcolor{headerblue}
        \textbf{Característica} & \textbf{Valor} & \textbf{Unidad} \\
        \midrule
        Estatura                & 1.80           & m                \\
        Peso                    & 85             & kg               \\
        Índice de Masa Corporal & 26.23          & kg/m²            \\
        Género                  & Masculino      & -                \\
        Edad                    & 30             & años             \\
        \bottomrule
    \end{tabular}
\end{table}

\subsection{Ubicaciones del Dispositivo Móvil en el Cuerpo}

Se evaluaron siete ubicaciones corporales del dispositivo móvil UWB, seleccionadas estratégicamente por su relevancia en aplicaciones prácticas de seguimiento de personas y dispositivos wearables:

\begin{enumerate}
\item \textbf{Cabeza:} Dispositivo ubicado en la parte superior craneal. Esta posición es relevante para aplicaciones de realidad aumentada y cascos inteligentes.

\item \textbf{Pecho:} Dispositivo colocado en el centro del pecho, aproximadamente a la altura del esternón, orientado hacia adelante. Representa la ubicación típica de badges de identificación y monitores cardíacos.

\item \textbf{Cadera:} Dispositivo posicionado lateralmente a nivel de la cintura/cadera. Común en dispositivos clip-on y sistemas de rastreo industrial.

\item \textbf{Mano:} Dispositivo sostenido directamente en la mano dominante. Simula el escenario de un usuario portando activamente un dispositivo móvil o herramienta.
\item \textbf{Muñeca:} Dispositivo colocado en la muñeca como si fuera un reloj inteligente (smartwatch), representando uno de los wearables más populares.

\item \textbf{Rodilla:} Dispositivo ubicado en la superficie frontal o lateral de la rodilla. Relevante para aplicaciones de análisis biomecánico y monitoreo deportivo.

\item \textbf{Tobillo:} Dispositivo posicionado en la superficie externa del tobillo. Utilizado en sistemas de monitoreo de marcha y rastreo en aplicaciones de seguridad.
\end{enumerate}

Para cada ubicación corporal, el dispositivo fue fijado de manera segura para minimizar el movimiento relativo durante las mediciones. Se documentó fotográficamente cada configuración para asegurar reproducibilidad en futuras réplicas del experimento.

\subsection{Condiciones de Propagación: LOS y NLOS}

Para cada combinación de ubicación corporal del dispositivo y escenario (exterior/interior), se evaluaron dos condiciones de propagación fundamentales definidas por la orientación del cuerpo del sujeto respecto al nodo fijo de referencia:

\textbf{Condición LOS (Line of Sight):} Se aseguró la existencia de una trayectoria visual directa entre el nodo fijo y el dispositivo móvil. El sujeto se orientó de manera que la parte del cuerpo que portaba el dispositivo mantuviera visibilidad directa con el nodo de referencia. Esta condición representa el escenario óptimo donde la señal UWB puede propagarse sin obstrucción corporal significativa.

\textbf{Condición NLOS (Non-Line of Sight):} Se obstruyó deliberadamente la línea de visión directa entre el nodo fijo y el dispositivo móvil mediante el cuerpo humano. Esta condición se logró orientando al sujeto de espaldas al nodo fijo (rotación de 180°), asegurando que el torso y la parte del cuerpo con el dispositivo bloquearan la propagación directa de la señal.

\textbf{Control de Postura:} Para garantizar la consistencia y reproducibilidad en todas las mediciones NLOS, se instruyó al sujeto a mantener una postura corporal estandarizada:
\begin{itemize}
\item Postura erguida y vertical
\item Orientación directamente opuesta al nodo fijo (180°)
\item Brazos descansando naturalmente a los costados del cuerpo
\item Evitar movimientos durante la adquisición de cada conjunto de mediciones
\end{itemize}

Esta dicotomía LOS/NLOS permite caracterizar el efecto máximo de la obstrucción corporal (BS) sobre la exactitud de la estimación de distancia, aislando este factor de otras fuentes de error.

\subsection{Protocolo de Medición - Fase 1}

\subsubsection{Configuración del Grid de Medición}

Las mediciones se realizaron siguiendo un protocolo sistemático de posiciones estáticas donde el sujeto permanece inmóvil durante la adquisición de datos:

\begin{itemize}
\item \textbf{Distancias evaluadas:} 1 m, 2 m, 3 m, ..., 13 m (incrementos de 1 metro)
\item \textbf{Rango total:} 1 m a 13 m (13 puntos de medición por configuración)
\item \textbf{Método de marcación:} Distancias reales marcadas con cinta métrica de alta precisión
\item \textbf{Alineación:} El sujeto se posiciona en línea recta desde el nodo fijo de referencia
\end{itemize}

\subsubsection{Muestras por Punto de Medición}

Para cada combinación de:
\begin{itemize}
\item Ubicación corporal (7 opciones: cabeza, pecho, cadera, mano, muñeca, rodilla, tobillo)
\item Condición de canal (2 opciones: LOS, NLOS)
\item Escenario (2 opciones: exterior, interior)
\item Distancia (13 puntos: 1-13 m)
\end{itemize}

Se registraron \textbf{250 mediciones de distancia estimada} por el sistema UWB. Este tamaño de muestra asegura análisis estadístico robusto y permite caracterizar adecuadamente la variabilidad de las mediciones.\\

\textbf{Total de mediciones - Fase 1:}
$$7 \text{ ubicaciones} \times 2 \text{ condiciones} \times 2 \text{ escenarios} \times 13 \text{ distancias} \times 250 \text{ muestras} = 91,000 \text{ mediciones}$$

\textbf{Nota:} Cabe aclarar que también se tomaron muestras en el dispositivo móvil tanto en interiores como en exteriores en los 13 puntos del experimento. 

\subsection{Condiciones de Control}

Para garantizar la reproducibilidad del experimento y minimizar la variabilidad no controlada:

\begin{itemize}
\item Las mediciones se realizan en horarios donde la actividad en las áreas circundantes es mínima, reduciendo interferencia de personas transitando.
\item Se verifica antes de cada sesión que no haya cambios significativos en el entorno (mobiliario movido, nuevos objetos metálicos, etc.).
\item Se registran las condiciones ambientales: temperatura, humedad relativa, presencia de dispositivos electrónicos activos.
\item Se realiza una medición de calibración en LOS (sin participante presente) al inicio de cada sesión experimental para verificar la estabilidad del sistema.
\end{itemize}

\subsection{Procedimiento Detallado de Medición}

El protocolo experimental implementado en cada sesión de medición se estructuró de manera sistemática para garantizar la integridad y reproducibilidad de los datos. Inicialmente, se procedió con la preparación del sistema, que abarcaba el encendido y la verificación funcional de los módulos DWM1001, así como el montaje del nodo fijo en un trípode a una altura estándar de 1.5 metros. Posteriormente, se realizó la inicialización del firmware PANS junto con la configuración de los parámetros UWB, finalizando esta etapa con la confirmación de la comunicación efectiva entre el nodo fijo y el nodo móvil.

De manera paralela, se llevó a cabo la marcación del grid de medición, utilizando una cinta métrica para delimitar con precisión las distancias desde 1 metro hasta 13 metros en línea recta desde el nodo fijo.. Antes de iniciar las pruebas con el participante, se ejecutó una calibración inicial en línea de vista (LOS) situando el nodo móvil en un trípode sin la presencia del sujeto; esto permitía verificar la exactitud del sistema en ausencia de obstrucciones e identificar posibles sesgos sistemáticos.

Una vez validado el sistema, se procedió a la instrumentación del sujeto, fijando de manera segura el nodo móvil en la ubicación corporal correspondiente y verificando su correcta orientación, proceso que era documentado fotográficamente. La adquisición de datos comenzó en condición LOS y en cada parte del cuerpo, donde el sujeto se posicionaba a 1 metro de distancia con orientación frontal hacia el nodo fijo. En esta posición, permaneciendo inmóvil durante aproximadamente 30 segundos, se registraban automáticamente 250 mediciones de distancia. Este procedimiento se repetía secuencialmente para cada punto marcado hasta alcanzar los 13 metros.

Posteriormente, se realizó la adquisición de datos en condición de no línea de vista (NLOS). Para ello, el sujeto se ubicó nuevamente en la primera marca de distancia y realizó una rotación de 180 grados para quedar de espaldas al nodo fijo, bloqueando la señal con con el nodo en cada una de las 7 posiciones corporales. Se adquirieron entonces otras 250 mediciones automáticas, repitiendo el proceso en cada una de las distancias establecidas.

Al completar las mediciones en una posición específica, se efectuó el cambio de ubicación corporal del nodo móvil a la siguiente zona de interés, repitiendo los ciclos de adquisición LOS y NLOS para las siete ubicaciones definidas. Una vez cubiertas todas las ubicaciones corporales, se replicó el procedimiento completo en el segundo escenario experimental (Exterior).

\section{PROCESAMIENTO Y ANÁLISIS DE DATOS - FASE 1}
\label{sec:procesamiento_datos}

\subsection{Preprocesamiento de Datos Crudos}

Los datos crudos registrados por el sistema UWB incluyen timestamps de alta resolución, mediciones de ToF entre el nodo móvil y el nodo fijo, estimaciones de potencia de señal recibida y otros parámetros de diagnóstico. El preprocesamiento de la Fase 1 comprende:

\begin{itemize}
\item \textbf{Filtrado de valores atípicos:} Detección y remoción de mediciones claramente erróneas (ej. ToF negativo, distancias fuera del rango físico posible de 1-13 m) mediante criterios estadísticos robustos.

\item \textbf{Conversión de ToF a Distancia:} Aplicación de la ecuación $d = \text{ToF} \times c$, donde $c = 299{,}792{,}458$ m/s es la velocidad de la luz en el vacío.

\item \textbf{Corrección de sesgos sistemáticos:} Si se identifican sesgos consistentes en las mediciones de calibración LOS (sin sujeto), se aplican correcciones de offset al nodo de referencia.
\end{itemize}

\subsection{Cálculo de Métricas de Error}

Para cada medición de distancia entre el nodo móvil y el nodo fijo de referencia, se calcula el error de distancia:

\begin{equation}
e_i = d_{\text{medida},i} - d_{\text{real},i}
\end{equation}

donde $d_{\text{medida},i}$ es la distancia estimada por el sistema UWB y $d_{\text{real},i}$ es la distancia física real marcada con cinta métrica (1 m, 2 m, ..., 13 m).

Se definen las siguientes métricas estadísticas:

\begin{itemize}

\item \textbf{Error Absoluto Medio (Mean Absolute Error, MAE):}
\begin{equation}
\text{MAE} = \frac{1}{N} \sum_{j=1}^{N} |e_j|
\end{equation}

\item \textbf{Raíz del Error Cuadrático Medio (Root Mean Square Error, RMSE):}
\begin{equation}
\text{RMSE} = \sqrt{\frac{1}{N} \sum_{j=1}^{N} e_j^2}
\end{equation}


\item \textbf{Percentiles:} Se calculan los percentiles 50 (mediana), 90 y 95 del error absoluto para caracterizar la distribución completa del error.
\end{itemize}

\subsection{Análisis Estadístico - Fase 1}

\subsubsection{Caracterización de Distribuciones}

Para cada condición experimental (combinación de ubicación corporal, condición LOS/NLOS, escenario, distancia), se caracteriza la distribución estadística de los errores observados. Se evalúan distribuciones de probabilidad teóricas:

\begin{itemize}
\item Normal (Gaussiana)
\item Log-normal  
\item Gamma
\end{itemize}

La bondad de ajuste se evalúa mediante pruebas de Kolmogorov-Smirnov y criterios de información (AIC, BIC). Esta caracterización permite identificar si las distribuciones de error son simétricas (gaussianas) o presentan asimetría positiva con colas extendidas hacia errores mayores, lo cual tiene implicaciones para el diseño de algoritmos de mitigación.

\subsubsection{Factores Experimentales Analizados}

Los principales factores experimentales evaluados en la Fase 1 son:

\begin{itemize}
\item \textbf{Ubicación Corporal:} 7 niveles (cabeza, pecho, cadera, mano, muñeca, rodilla, tobillo)
\item \textbf{Condición de Canal:} 2 niveles (LOS, NLOS)
\item \textbf{Escenario:} 2 niveles (exterior, interior)
\item \textbf{Distancia:} Variable continua o discretizada (1-13 m)
\end{itemize}

Se analiza el efecto individual de cada factor sobre el error de distancia, así como las interacciones entre factores (e.g., si el efecto de la condición LOS/NLOS depende de la ubicación corporal).


\section{HERRAMIENTAS DE SOFTWARE}
\label{sec:herramientas_software}

\subsection{Software de Adquisición de Datos}

Para el desarrollo del presente proyecto se empleó un software provisto por la empresa Quorvo, el cual consiste en una librería integral que conforma un ecosistema completo de software. Dicho ecosistema incluye el PANS (Positioning and Navigation Software), encargado de los procesos de posicionamiento y navegación. Esta librería contiene todas las funciones y estructuras necesarias para controlar la comunicación entre los dispositivos denominados anclas (anchors) y el dispositivo etiqueta (tag).

El software está desarrollado en lenguaje C y se utiliza el entorno de desarrollo SEGGER Embedded Studio para la validación del algoritmo, la compilación del código y la carga del binario en el dispositivo. Este entorno emplea J-Link como controlador (driver) para la programación del hardware.

Las funciones de comunicación y adquisición de datos se encuentran predefinidas dentro de la librería. El trabajo realizado en esta tesis de maestría consiste en la modificación de una parte del algoritmo con el objetivo de establecer la conexión de la etiqueta con cuatro anclas y habilitar la transmisión de los datos obtenidos a través de una interfaz serial. Los algoritmos desarrollados y modificados se encuentran disponibles en el siguiente repositorio de GitHub, tanto para el dispositivo anchor como para el tag \cite{DiazAlgoritmosUWB}.


\subsection{Software de Procesamiento y Análisis}

El procesamiento de datos y el análisis estadístico se llevan a cabo utilizando Python 3.x, apoyado en un conjunto de bibliotecas científicas. NumPy se emplea para operaciones numéricas y álgebra lineal, Pandas para la manipulación y análisis de datos estructurados, y SciPy para la aplicación de funciones estadísticas avanzadas, ajuste de distribuciones y análisis ANOVA. La visualización de los resultados se realiza mediante Matplotlib y Seaborn, mientras que scikit-learn se utiliza para la implementación de filtros y algoritmos de localización.

\subsection{Control de Versiones y Reproducibilidad}

Todo el código desarrollado se gestiona mediante Git y se almacena en un repositorio público. Los datos experimentales se organizan en formato personalizado con metadatos descriptivos. Se proporciona un entorno computacional reproducible mediante un archivo requirements.txt que especifica las versiones exactas de todas las dependencias de software que se instalan en un entorno virtual.


\section{CONSIDERACIONES ÉTICAS}
\label{sec:consideraciones_eticas}

El protocolo experimental ha sido aprobado por el Comité de Ética correspondiente y diseñado siguiendo los lineamientos éticos vigentes. Los participantes son informados detalladamente sobre el propósito del estudio y los procedimientos involucrados, firman un consentimiento informado antes de participar y pueden retirarse del estudio en cualquier momento sin consecuencias. Asimismo, sus datos personales (nombre, edad, entre otros) son anonimizados y protegidos conforme a las normativas de protección de datos aplicables, tales como la Ley 1581 de 2012 en Colombia o el GDPR en Europa. Finalmente, se garantiza que los participantes no son sometidos a ningún procedimiento invasivo ni a riesgos para su salud o integridad física.