\chapter{DISEÑO METODOLÓGICO}
\label{ch:metodologia_experimental}

En este capítulo se describe detalladamente la metodología empleada para llevar a cabo la investigación experimental, estructurada en dos fases secuenciales:

\textbf{Fase 1 - Plan de Pruebas de Validación:} Evaluación exhaustiva de la exactitud en la estimación de distancia UWB analizando el impacto de la ubicación del dispositivo móvil en diferentes partes del cuerpo humano bajo condiciones de propagación LOS y NLOS en dos escenarios representativos (exterior e interior). Esta fase, ya ejecutada y documentada en el artículo de validación, establece la línea base de desempeño del sistema.

\textbf{Fase 2 - Sistema de Posicionamiento 2D:} Implementación y evaluación del sistema completo de localización utilizando cuatro nodos ancla para determinar posiciones 2D mediante trilateración. Esta fase integrará los hallazgos de la Fase 1 para evaluar el error de posicionamiento en el plano.

Este capítulo se enfoca en describir la metodología de la Fase 1, que constituye el plan de pruebas de validación completo del sistema de estimación de distancia a 6.5 GHz.

\section{CONFIGURACIÓN DEL HARDWARE}
\label{sec:configuracion_hardware}

\subsection{Dispositivos UWB}

\subsubsection{Selección de Dispositivos}

Para este estudio se han seleccionado módulos de desarrollo Qorvo DWM1001, que integran un transceptor UWB DW1000, un microcontrolador Nordic Semiconductor nRF52832 y un sensor de movimiento. Los módulos operan con el firmware PANS (Positioning and Networking Stack) de Qorvo, el cual implementa la técnica TWR para estimación de distancia. Estos dispositivos están configurados para operar en la banda de frecuencia central de 6.5 GHz. La elección de estos dispositivos se fundamenta en:

\begin{itemize}
\item Capacidad de operar en la banda de 6.5 GHz con el ancho de banda necesario para lograr resolución temporal adecuada.
\item Soporte nativo para mediciones TWR con registro de timestamps de alta precisión.
\item Interfaz de comunicación accesible que permite la extracción de datos crudos de ToF y otras métricas de señal.
\item Consumo energético compatible con operación portátil prolongada.
\item Disponibilidad de documentación técnica detallada y herramientas de desarrollo.
\end{itemize}

\subsubsection{Parámetros de Configuración}

Los dispositivos UWB se configuran con los siguientes parámetros operacionales:

\begin{itemize}
\item \textbf{Frecuencia central:} 6489.6 MHz (Canal 5)
\item \textbf{Ancho de banda:} 499.2 MHz
\item \textbf{Tasa de transmisión de datos:} 6.8 Mbps
\item \textbf{Frecuencia de repetición de pulsos (PRF):} 64 MHz
\item \textbf{Longitud del preámbulo:} 128 símbolos
\item \textbf{Código del preámbulo (TX/RX):} 10
\item \textbf{Potencia de transmisión:} -17 dBm
\item \textbf{Sensibilidad del receptor:} -93 dBm
\end{itemize}

Estos parámetros se seleccionan para optimizar el compromiso entre exactitud de medición, alcance efectivo y consumo energético, siguiendo las recomendaciones del fabricante y las mejores prácticas identificadas en la revisión de literatura.


\section{DISEÑO DEL ESCENARIO EXPERIMENTAL - FASE 1}
\label{sec:escenario_experimental}

\subsection{Descripción General}

El plan de pruebas de validación (Fase 1) se implementó en dos escenarios principales diseñados para evaluar diferentes condiciones de propagación características de aplicaciones reales de sistemas de posicionamiento indoor y outdoor. La selección de estos escenarios permite caracterizar el comportamiento del sistema UWB tanto en ambientes con mínima reflexión (exterior) como en entornos con alta dispersión multitrayecto (interior tipo corredor).

\subsection{Escenario Exterior}

Se utilizó un campo abierto con dimensiones similares a las de un campo de fútbol. Este escenario fue diseñado para:

\begin{itemize}
\item Minimizar las reflexiones provenientes de estructuras verticales como paredes y edificios.
\item Proporcionar un entorno base para analizar los efectos primarios de propagación.
\item Aislar el impacto de la obstrucción corporal (BS) sobre la señal directa.
\item Considerar fenómenos fundamentales de propagación como las Pérdidas en Espacio Libre (FSPL) y el multitrayecto generado por reflexión en el suelo.
\end{itemize}

\textbf{Características del Escenario Exterior:}
\begin{itemize}
\item \textbf{Tipo:} Campo abierto (dimensiones ~100 m × 60 m)
\item \textbf{Superficie:} Césped natural
\item \textbf{Obstrucciones:} Mínimas (ausencia de paredes y estructuras verticales cercanas)
\item \textbf{Condiciones climáticas:} Mediciones realizadas en días despejados para minimizar variables ambientales
\end{itemize}

\subsection{Escenario Interior}

Se empleó un corredor de edificio universitario para representar condiciones típicas de aplicaciones indoor:

\textbf{Características del Escenario Interior:}
\begin{itemize}
\item \textbf{Dimensiones:} 20 m de largo × 3 m de ancho
\item \textbf{Materiales constructivos:} Paredes de concreto, piso de baldosa cerámica, techo de placa de concreto
\item \textbf{Mobiliario:} Corredor típico con mobiliario mínimo
\item \textbf{Características de propagación:} Introduce efectos de multitrayecto controlados resultantes de reflexiones en paredes, techo y piso
\item \textbf{Fuentes de interferencia:} Redes WiFi institucionales activas en bandas de 2.4 GHz y 5 GHz, equipos electrónicos en oficinas adyacentes
\end{itemize}

Este escenario permite evaluar cómo las reflexiones especulares en superficies confinadas afectan la estimación de distancia, representando condiciones realistas para sistemas de localización indoor.

\subsection{Configuración de Nodos - Fase 1}

En la Fase 1 del plan de pruebas de validación, la configuración experimental consistió en:

\begin{itemize}
\item \textbf{Nodo Fijo de Referencia:} Un módulo DWM1001 montado en un trípode a una altura estándar de 1.5 m sobre el nivel del piso, actuando como ancla de referencia para las mediciones de distancia.
\item \textbf{Nodo Móvil:} Un módulo DWM1001 portado por el sujeto de prueba en diferentes ubicaciones corporales.
\end{itemize}

Esta configuración punto a punto permite evaluar la exactitud fundamental de la estimación de distancia antes de implementar el sistema completo de posicionamiento 2D. El nodo fijo permanece en una ubicación constante durante cada sesión de medición, mientras el sujeto se desplaza a distancias conocidas siguiendo un protocolo sistemático.

\textbf{Nota:} La Fase 2 implementará un sistema con cuatro nodos ancla en configuración geométrica optimizada para trilateración y posicionamiento 2D.


\section{PROTOCOLO DE RECOLECCIÓN DE DATOS}
\label{sec:protocolo_recoleccion}

\subsection{Sujeto de Prueba}

El plan de pruebas de validación (Fase 1) fue conducido como un caso de estudio fundacional con un único sujeto de prueba. Este enfoque metodológico permite:

\begin{itemize}
\item Caracterización detallada de los efectos de la ubicación del dispositivo en el cuerpo sin variabilidad inter-sujeto.
\item Evaluación sistemática de todas las combinaciones de ubicación corporal, condición de canal (LOS/NLOS) y escenario bajo condiciones antropométricas constantes.
\item Establecimiento de la línea base de desempeño del sistema antes de considerar la variabilidad antropométrica.
\item Control riguroso de variables experimentales para aislar el efecto de la obstrucción corporal.
\end{itemize}

\textbf{Justificación del Enfoque:} Como se reconoce en el artículo de validación, este trabajo constituye un caso de estudio fundacional que prioriza la exhaustividad en la evaluación de configuraciones (7 ubicaciones × 2 condiciones × 2 escenarios = 28 configuraciones) sobre la cantidad de sujetos. Para garantizar la generalización de estos hallazgos para aplicaciones wearables robustas, se reconoce que futuros trabajos deberán incluir un plan de pruebas exhaustivo con múltiples sujetos que presenten diversas características antropométricas (altura, peso, Índice de Masa Corporal, género).

El sujeto de prueba participó bajo consentimiento informado, con conocimiento completo del propósito del estudio, los procedimientos a realizar y el uso de los datos recolectados.

\subsection{Ubicaciones del Dispositivo Móvil en el Cuerpo}

Se evaluaron siete ubicaciones corporales del dispositivo móvil UWB, seleccionadas estratégicamente por su relevancia en aplicaciones prácticas de seguimiento de personas y dispositivos wearables:

\begin{enumerate}
\item \textbf{Cabeza (Head):} Dispositivo ubicado en la parte superior craneal. Esta posición es relevante para aplicaciones de realidad aumentada y cascos inteligentes.

\item \textbf{Pecho (Chest):} Dispositivo colocado en el centro del pecho, aproximadamente a la altura del esternón, orientado hacia adelante. Representa la ubicación típica de badges de identificación y monitores cardíacos.

\item \textbf{Cadera (Hip):} Dispositivo posicionado lateralmente a nivel de la cintura/cadera. Común en dispositivos clip-on y sistemas de rastreo industrial.

\item \textbf{Mano (Hand):} Dispositivo sostenido directamente en la mano dominante. Simula el escenario de un usuario portando activamente un dispositivo móvil o herramienta.

\item \textbf{Muñeca (Wrist):} Dispositivo colocado en la muñeca como si fuera un reloj inteligente (smartwatch), representando uno de los wearables más populares.

\item \textbf{Rodilla (Knee):} Dispositivo ubicado en la superficie frontal o lateral de la rodilla. Relevante para aplicaciones de análisis biomecánico y monitoreo deportivo.

\item \textbf{Tobillo (Ankle):} Dispositivo posicionado en la superficie externa del tobillo. Utilizado en sistemas de monitoreo de marcha y rastreo en aplicaciones de seguridad.
\end{enumerate}

Para cada ubicación corporal, el dispositivo fue fijado de manera segura para minimizar el movimiento relativo durante las mediciones. Se documentó fotográficamente cada configuración para asegurar reproducibilidad en futuras réplicas del experimento.

\subsection{Condiciones de Propagación: LOS y NLOS}

Para cada combinación de ubicación corporal del dispositivo y escenario (exterior/interior), se evaluaron dos condiciones de propagación fundamentales definidas por la orientación del cuerpo del sujeto respecto al nodo fijo de referencia:

\textbf{Condición LOS (Line of Sight):} Se aseguró la existencia de una trayectoria visual directa entre el nodo fijo y el dispositivo móvil. El sujeto se orientó de manera que la parte del cuerpo que portaba el dispositivo mantuviera visibilidad directa con el nodo de referencia. Esta condición representa el escenario óptimo donde la señal UWB puede propagarse sin obstrucción corporal significativa.

\textbf{Condición NLOS (Non-Line of Sight):} Se obstruyó deliberadamente la línea de visión directa entre el nodo fijo y el dispositivo móvil mediante el cuerpo humano. Esta condición se logró orientando al sujeto de espaldas al nodo fijo (rotación de 180°), asegurando que el torso y la parte del cuerpo con el dispositivo bloquearan la propagación directa de la señal.

\textbf{Control de Postura:} Para garantizar la consistencia y reproducibilidad en todas las mediciones NLOS, se instruyó al sujeto a mantener una postura corporal estandarizada:
\begin{itemize}
\item Postura erguida y vertical
\item Orientación directamente opuesta al nodo fijo (180°)
\item Brazos descansando naturalmente a los costados del cuerpo
\item Evitar movimientos durante la adquisición de cada conjunto de mediciones
\end{itemize}

Esta dicotomía LOS/NLOS permite caracterizar el efecto máximo de la obstrucción corporal (BS) sobre la exactitud de la estimación de distancia, aislando este factor de otras fuentes de error.

\subsection{Protocolo de Medición - Fase 1}

\subsubsection{Configuración del Grid de Medición}

Las mediciones se realizaron siguiendo un protocolo sistemático de posiciones estáticas donde el sujeto permanece inmóvil durante la adquisición de datos:

\begin{itemize}
\item \textbf{Distancias evaluadas:} 1 m, 2 m, 3 m, ..., 13 m (incrementos de 1 metro)
\item \textbf{Rango total:} 1 m a 13 m (13 puntos de medición por configuración)
\item \textbf{Método de marcación:} Distancias reales marcadas con cinta métrica de alta precisión
\item \textbf{Alineación:} El sujeto se posiciona en línea recta desde el nodo fijo de referencia
\end{itemize}

\subsubsection{Muestras por Punto de Medición}

Para cada combinación de:
\begin{itemize}
\item Ubicación corporal (7 opciones: cabeza, pecho, cadera, mano, muñeca, rodilla, tobillo)
\item Condición de canal (2 opciones: LOS, NLOS)
\item Escenario (2 opciones: exterior, interior)
\item Distancia (13 puntos: 1-13 m)
\end{itemize}

Se registraron \textbf{250 mediciones de distancia estimada} por el sistema UWB. Este tamaño de muestra asegura análisis estadístico robusto y permite caracterizar adecuadamente la variabilidad de las mediciones.

\textbf{Total de mediciones - Fase 1:}
$$7 \text{ ubicaciones} \times 2 \text{ condiciones} \times 2 \text{ escenarios} \times 13 \text{ distancias} \times 250 \text{ muestras} = 91,000 \text{ mediciones}$$

\textbf{Nota:} No todas las combinaciones fueron evaluadas en el plan de validación (algunas ubicaciones solo se probaron en ciertos escenarios), pero se lograron 28 configuraciones completas documentadas en los resultados.

\subsection{Condiciones de Control}

Para garantizar la reproducibilidad del experimento y minimizar la variabilidad no controlada:

\begin{itemize}
\item Las mediciones se realizan en horarios donde la actividad en las áreas circundantes es mínima, reduciendo interferencia de personas transitando.
\item Se verifica antes de cada sesión que no haya cambios significativos en el entorno (mobiliario movido, nuevos objetos metálicos, etc.).
\item Se registran las condiciones ambientales: temperatura, humedad relativa, presencia de dispositivos electrónicos activos.
\item Se realiza una medición de calibración en LOS (sin participante presente) al inicio de cada sesión experimental para verificar la estabilidad del sistema.
\end{itemize}

\subsection{Procedimiento Detallado de Medición}

El protocolo experimental para cada sesión de medición siguió los siguientes pasos sistemáticos:

\begin{enumerate}
\item \textbf{Preparación del Sistema:}
\begin{itemize}
    \item Encendido y verificación de funcionamiento de los módulos DWM1001
    \item Montaje del nodo fijo en trípode a altura de 1.5 m
    \item Inicialización del firmware PANS y configuración de parámetros UWB
    \item Verificación de comunicación entre nodo fijo y nodo móvil
\end{itemize}

\item \textbf{Marcación del Grid de Medición:}
\begin{itemize}
    \item Marcación precisa de distancias desde 1 m hasta 13 m con cinta métrica
    \item Alineación recta desde el nodo fijo
    \item Verificación visual de las marcas
\end{itemize}

\item \textbf{Calibración LOS Inicial:}
\begin{itemize}
    \item Mediciones de referencia sin sujeto presente (nodo móvil en trípode)
    \item Verificación de exactitud del sistema en ausencia de obstrucción
    \item Identificación de posibles sesgos sistemáticos
\end{itemize}

\item \textbf{Instrumentación del Sujeto:}
\begin{itemize}
    \item Fijación segura del nodo móvil en la ubicación corporal correspondiente
    \item Verificación de orientación correcta del dispositivo
    \item Documentación fotográfica de la configuración
\end{itemize}

\item \textbf{Adquisición de Datos - Condición LOS:}
\begin{itemize}
    \item Sujeto se posiciona en la primera distancia (1 m)
    \item Orientación frontal hacia el nodo fijo (condición LOS)
    \item Adquisición automática de 250 mediciones de distancia
    \item Sujeto permanece inmóvil durante adquisición (~30 segundos)
    \item Repetir para cada distancia (2 m, 3 m, ..., 13 m)
\end{itemize}

\item \textbf{Adquisición de Datos - Condición NLOS:}
\begin{itemize}
    \item Sujeto se posiciona en la primera distancia (1 m)
    \item Rotación 180° (de espaldas al nodo fijo, condición NLOS)
    \item Adquisición automática de 250 mediciones
    \item Repetir para cada distancia
\end{itemize}

\item \textbf{Cambio de Ubicación Corporal:}
\begin{itemize}
    \item Cambio del nodo móvil a la siguiente ubicación corporal
    \item Repetir pasos 5-6 para las 7 ubicaciones
\end{itemize}

\item \textbf{Cambio de Escenario:}
\begin{itemize}
    \item Repetir procedimiento completo en el segundo escenario
\end{itemize}

\item \textbf{Verificación y Respaldo:}
\begin{itemize}
    \item Verificación de integridad de datos (ausencia de pérdidas o corrupciones)
    \item Descarga y organización de datos recolectados
    \item Respaldo en ubicaciones redundantes
\end{itemize}
\end{enumerate}


\section{PROCESAMIENTO Y ANÁLISIS DE DATOS - FASE 1}
\label{sec:procesamiento_datos}

\subsection{Preprocesamiento de Datos Crudos}

Los datos crudos registrados por el sistema UWB incluyen timestamps de alta resolución, mediciones de ToF entre el nodo móvil y el nodo fijo, estimaciones de potencia de señal recibida y otros parámetros de diagnóstico. El preprocesamiento de la Fase 1 comprende:

\begin{itemize}
\item \textbf{Filtrado de valores atípicos:} Detección y remoción de mediciones claramente erróneas (ej. ToF negativo, distancias fuera del rango físico posible de 1-13 m) mediante criterios estadísticos robustos.

\item \textbf{Conversión de ToF a Distancia:} Aplicación de la ecuación $d = \text{ToF} \times c$, donde $c = 299{,}792{,}458$ m/s es la velocidad de la luz en el vacío.

\item \textbf{Corrección de sesgos sistemáticos:} Si se identifican sesgos consistentes en las mediciones de calibración LOS (sin sujeto), se aplican correcciones de offset al nodo de referencia.
\end{itemize}

\subsection{Cálculo de Métricas de Error}

Para cada medición de distancia entre el nodo móvil y el nodo fijo de referencia, se calcula el error de distancia:

\begin{equation}
e_i = d_{\text{medida},i} - d_{\text{real},i}
\end{equation}

donde $d_{\text{medida},i}$ es la distancia estimada por el sistema UWB y $d_{\text{real},i}$ es la distancia física real marcada con cinta métrica (1 m, 2 m, ..., 13 m).

Se definen las siguientes métricas estadísticas:

\begin{itemize}

\item \textbf{Error Absoluto Medio (Mean Absolute Error, MAE):}
\begin{equation}
\text{MAE} = \frac{1}{N} \sum_{j=1}^{N} |e_j|
\end{equation}

\item \textbf{Raíz del Error Cuadrático Medio (Root Mean Square Error, RMSE):}
\begin{equation}
\text{RMSE} = \sqrt{\frac{1}{N} \sum_{j=1}^{N} e_j^2}
\end{equation}


\item \textbf{Percentiles:} Se calculan los percentiles 50 (mediana), 90 y 95 del error absoluto para caracterizar la distribución completa del error.
\end{itemize}

\subsection{Análisis Estadístico - Fase 1}

\subsubsection{Caracterización de Distribuciones}

Para cada condición experimental (combinación de ubicación corporal, condición LOS/NLOS, escenario, distancia), se caracteriza la distribución estadística de los errores observados. Se evalúan distribuciones de probabilidad teóricas:

\begin{itemize}
\item Normal (Gaussiana)
\item Log-normal  
\item Gamma
\end{itemize}

La bondad de ajuste se evalúa mediante pruebas de Kolmogorov-Smirnov y criterios de información (AIC, BIC). Esta caracterización permite identificar si las distribuciones de error son simétricas (gaussianas) o presentan asimetría positiva con colas extendidas hacia errores mayores, lo cual tiene implicaciones para el diseño de algoritmos de mitigación.

\subsubsection{Factores Experimentales Analizados}

Los principales factores experimentales evaluados en la Fase 1 son:

\begin{itemize}
\item \textbf{Ubicación Corporal:} 7 niveles (cabeza, pecho, cadera, mano, muñeca, rodilla, tobillo)
\item \textbf{Condición de Canal:} 2 niveles (LOS, NLOS)
\item \textbf{Escenario:} 2 niveles (exterior, interior)
\item \textbf{Distancia:} Variable continua o discretizada (1-13 m)
\end{itemize}

Se analiza el efecto individual de cada factor sobre el error de distancia, así como las interacciones entre factores (e.g., si el efecto de la condición LOS/NLOS depende de la ubicación corporal).


\section{HERRAMIENTAS DE SOFTWARE}
\label{sec:herramientas_software}

\subsection{Software de Adquisición de Datos}

[ESPECIFICAR: Lenguaje de programación y bibliotecas utilizadas para la interfaz con los dispositivos UWB, ej. Python con biblioteca DWM1001-API, C++ con SDK del fabricante, etc.]

\subsection{Software de Procesamiento y Análisis}

El procesamiento de datos y análisis estadístico se realiza utilizando:

\begin{itemize}
\item \textbf{Python 3.x} con las bibliotecas científicas:
    \begin{itemize}
    \item \texttt{NumPy}: Operaciones numéricas y álgebra lineal
    \item \texttt{Pandas}: Manipulación y análisis de datos estructurados
    \item \texttt{SciPy}: Funciones estadísticas avanzadas, ajuste de distribuciones, ANOVA
    \item \texttt{Matplotlib} y \texttt{Seaborn}: Visualización de datos
    \item \texttt{scikit-learn}: Implementación de filtros y algoritmos de localización
    \end{itemize}
\item [ESPECIFICAR alternativas si aplica: MATLAB, R, etc.]
\end{itemize}

\subsection{Control de Versiones y Reproducibilidad}

Todo el código desarrollado se gestiona mediante Git y se almacena en un repositorio [ESPECIFICAR: privado/público, plataforma]. Los datos experimentales se organizan en formato estándar [ESPECIFICAR: CSV, HDF5, etc.] con metadatos descriptivos. Se proporciona un entorno computacional reproducible mediante [ESPECIFICAR: archivo requirements.txt, contenedor Docker, etc.] que especifica las versiones exactas de todas las dependencias de software.


\section{CONSIDERACIONES ÉTICAS}
\label{sec:consideraciones_eticas}

El protocolo experimental ha sido [ESPECIFICAR: "aprobado por el Comité de Ética de..." o "diseñado siguiendo los lineamientos éticos de..."]. Los participantes:

\begin{itemize}
\item Son informados detalladamente sobre el propósito del estudio y los procedimientos.
\item Firman un consentimiento informado antes de participar.
\item Pueden retirarse del estudio en cualquier momento sin consecuencias.
\item Sus datos personales (nombre, edad, etc.) son anonimizados y protegidos según normativas de protección de datos [ESPECIFICAR normativa aplicable: Ley 1581 de 2012 en Colombia, GDPR en Europa, etc.].
\item No se someten a ningún procedimiento invasivo o de riesgo para su salud o integridad física.
\end{itemize}